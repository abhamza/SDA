\documentclass[a4paper,10pt]{article}

% Packages
\usepackage[utf8]{inputenc}
\usepackage[frenchb]{babel}
\usepackage{graphicx}
\usepackage{url}
\usepackage{hyperref}
\usepackage{a4wide}
\usepackage{amsmath}
%\usepackage{clrscode3epg}
%\usepackage{fullpage}
\renewcommand{\labelenumi}{(\alph{enumi})}

% Style
\parskip=\smallskipamount

% En-têtes
\title{
    \textbf{Structures de données et algorithmes}\\
    Projet 2: Arbre binaire de recherche
}

\author{Gilles \textsc{Louppe} - Thomas \textsc{Désir} - Pierre \textsc{Geurts}}
\date{21 mars 2012}

% Corps
\begin{document}
\maketitle

\section*{\'Enoncé}

L'objectif du projet est d'implémenter une structure d'arbre binaire
de recherche et de l'utiliser pour implémenter un (vrai) dictionnaire,
c'est-à-dire une liste de mots accompagnés de leur définition. Nous
vous demandons de remettre le code source correspondant à votre
implémentation ainsi qu'un bref rapport répondant aux questions
ci-dessous.

Le projet est à réaliser {\bf individuellement} pour le {\bf 20 avril
  2012} à {\bf 23h59} au plus tard. Le projet est à
remettre via une interface web disponible sur
\href{http://www.montefiore.ulg.ac.be/~glouppe/2011-2012/students.info0902.php}{la
  page des TPs}.

Un projet non rendu à temps recevra automatiquement une cote nulle. En
cas de plagiat avéré, l'étudiant se verra affecter une cote nulle à
l'ensemble du projet. Soyez bref mais précis dans votre rapport, qui
fera au maximum 3 pages, et respectez bien la numérotation des
sous-questions de l'énoncé.

Les critères de correction sont précisés sur la page web des projets.

\section{Implémentation de la structure de base}

La structure à implémenter est un dictionnaire permettant de stocker
des paires clé-valeur, où on supposera que les clés et les valeurs
sont des chaînes de caractères. Ce dictionnaire doit être implémenté à
partir d'un arbre binaire de recherche (simple) dont l'interface est
décrite dans le fichier \texttt{BinarySeachTree.h}. Une structure de
donnée opaque \texttt{BinarySearchTree} ainsi que huits méthodes
réalisant diverses fonctions sur cette structure y sont définies.

Nous vous demandons d'implémenter les opérations de base suivantes:
\begin{description}
\item[\texttt{createEmptyBinarySearchTree}] qui crée un nouvel arbre de recherche binaire vide.
\item[\texttt{freeBinarySearchTree}]  qui libère la mémoire allouée par la structure de l'arbre.
\item[\texttt{getNumWords}] qui retourne le nombre d'éléments dans l'arbre.
\item[\texttt{insertWord}] qui permet d'ajouter une nouvelle paire clé-valeur dans l'arbre. Si la clé existe déjà, sa valeur est remplacée par la nouvelle.
\item[\texttt{findWord}] qui retourne la valeur associée à une clé de l'arbre.
\item[\texttt{removeWord}] qui supprime une paire clé-valeur de l'arbre.
\end{description}
La définition du contenu de la structure \texttt{BinarySearchTree\_t}
est libre d'implémentation. Les fonctions de modification de la
structure ne doivent pas nécessairement maintenir l'arbre équilibré.

\section{Recherche sur base d'un préfixe}

On vous demande d'implémenter une fonction \texttt{findWordsByPrefix}
renvoyant la liste des mots du dictionnaire commençant par un préfixe
donné.

Dans votre rapport:
\begin{enumerate}
\item Expliquez brièvement le principe de votre implémentation
\item \'Etudiez la complexité en temps dans le meilleur et le pire
  cas. Faites l'analyse dans le cas où l'arbre est équilibré et dans
  le cas où il ne le serait pas nécessairement et formulez ces
  complexités en fonction du nombre de mots à retrouver (noté $r$) et
  du nombre total de mots dans l'arbre (noté $n$)
\end{enumerate}

\section{Ré-équilibrage de l'arbre}

Plutôt que d'implémenter les opérations compliquées d'insertion et de
suppression avec ré-équilibrage, on se propose d'implémenter une
fonction permettant de vérifier que l'arbre est équilibré et une
fonction permettant de ré-équilibrer l'arbre si nécessaire:
\begin{itemize}
\item[\texttt{isBalanced}] renverra 1 si l'arbre est équilibré, 0
  sinon. On dira qu'un arbre est équilibré si ses sous-arbres gauche
  et droit sont équilibrés et si la différence de hauteur entre ces
  sous-arbres est inférieure ou égale à 1. Un arbre vide est supposé
  être équilibré.
\item[\texttt{rebalanceTree}] renvoie un nouvel arbre équilibré à
  partir d'un arbre existant (équilibré ou non). Une façon simple
  d'implémenter le ré-équilibrage est de convertir l'arbre en un
  tableau trié et ensuite de convertir le tableau trié en un arbre
  équilibré.
\end{itemize}

Dans votre rapport:
\begin{enumerate}
\item Expliquez brièvement le principe de votre implémentation de ces deux fonctions
\item \'Etudiez la complexité en temps dans le meilleur et pire cas de
  ces deux fonctions en fonction du nombre de mots stockés dans
  l'arbre
\end{enumerate}

\section*{Annexe: Liste des fichiers fournis et à rendre}

\subsection*{Fichier fourni}
\begin{description}
\item[\texttt{BinarySearchTree.h}:] contient l'interface à implémenter
\item[\texttt{dico.txt}:] une liste de mots avec leur définition
\item[\texttt{main.c}:] contient une fonction main chargeant les définitions du fichier dico.txt et les stockant dans le dictionnaire. Ce fichier vous est fourni pour vous permettre de tester votre implémentation.
\end{description}
\subsection*{Fichiers à rendre}
Deux fichiers sont à rendre dans une archive \texttt{.tar.gz}. Le nom de l'archive n'a pas d'importance.
\begin{description}
\item[\texttt{BinarySearchTree.c}] implémente les fonctions de l'interface \texttt{BinarySeachTree.h}
\item[\texttt{Rapport.pdf}] contient vos réponses aux questions
\end{description}

{\em Note}: Les noms de fichiers font partie de l'énoncé. Tout fichier ne
correspondant pas à ceux demandés dans l'énoncé ne sera pas pris en compte.

\end{document}
