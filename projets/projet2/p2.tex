\documentclass[a4paper,10pt]{article}

% Packages
\usepackage[utf8]{inputenc}
\usepackage[frenchb]{babel}
\usepackage{graphicx}
\usepackage{url}
\usepackage{hyperref}
\usepackage{a4wide}
\usepackage{amsmath}
\usepackage{../../clrscode3epg}
\usepackage{color}
%\usepackage{fullpage}
\renewcommand{\labelenumi}{(\alph{enumi})}

% Style
\parskip=\smallskipamount

% En-têtes
\title{
    \textbf{Structures de données et algorithmes}\\
    Projet 2: Tables de hachage
}

\author{Gilles \textsc{Louppe} -- Julien \textsc{Becker}}
\date{15 avril 2013}

% Corps
\begin{document}
\maketitle

L'objectif du projet est d'écrire un algorithme efficace permettant de
calculer l'intersection de deux fichiers (en terme de lignes). On
étudiera d'abord théoriquement différentes solutions et la plus
efficace sera ensuite implémentée.

Le projet est à réaliser {\bf individuellement ou par deux ?} pour le {\bf 15 avril 2013} à
{\bf 05h00} (du matin) au plus tard. Le projet est à remettre via une interface
web disponible sur \href{http://www.montefiore.ulg.ac.be/~glouppe/2012-2013/students.info0902.php}{la page des TPs}.

Un projet non rendu à temps recevra automatiquement une cote nulle. En
cas de plagiat avéré, l'étudiant se verra affecter une cote nulle à
l'ensemble du projet. Soyez bref mais précis dans votre rapport, qui
fera au maximum 5 pages, et respectez bien la numérotation des
sous-questions de l'énoncé.

Les critères de correction sont précisés sur la page web des projets.

\section{Enoncé}

On aimerait implémenter un programme permettant d'afficher les lignes
communes à deux fichiers passés en argument. Par exemple, si les
fichiers \texttt{A.txt} et \texttt{B.txt} ont les contenus suivants:

\begin{verbatim}
      A.txt:            B.txt:
      aaa               bbb
      bbb               eee
      ccc               fff
      ddd
      eee
\end{verbatim}

Le résultat de l'application du programme affichera les lignes
\texttt{bbb} et \texttt{eee}. (L'ordre d'affichage des lignes communes n'a pas
d'importance.)

\section{Analyse théorique}

Sans perte de généralité, supposons que les fichiers aient été chargés
dans deux tableaux $A$ et $B$, respectivement de longueurs $N_A$ et $N_B$ et que
$N_A\leq N_B$.
\begin{enumerate}
\item Donnez le pseudo-code d'une fonction calculant l'intersection de $A$ et $B$.
  Les éléments communs doivent être placés dans un tableau auxiliaire $C$ et
  ce tableau doit être retourné par cette fonction. Notez que la longueur du tableau $C$ n'est a priori pas connue.
\item Quelle est la complexité théorique minimale, en fonction de $N_A$ et $N_B$, quelle que soit la
  solution adoptée ? (sans faire d'autres hypothèses sur le contenu
  des fichiers). Justifiez votre réponse.
\item Une solution générale à ce problème consiste à lire et à stocker
  dans une structure de données les éléments de $A$ et à
  ensuite parcourir les éléments de $B$ en affichant ceux qui
  se trouvent dans la structure. Donnez et justifiez la complexité au pire cas de cette solution
  lorsqu'on utilise les structures de données suivantes:
\begin{itemize}
\item liste
\item vecteur
\item arbre binaire de recherche
\item table de hachage
\end{itemize}
\item Si on pouvait supposer que $A$ et $B$ étaient
  triés par ordre alphabétique, comme procéderiez-vous et quelle
  serait la complexité de votre solution aux meilleur et pire cas ?
\end{enumerate}

\section{Implémentation}

On vous demande d'implémenter la solution basée sur une table de
hachage.

\begin{itemize}
\item Implémentez dans un fichier \texttt{HashTable.c} l'interface de table de hachage définie dans le fichier \texttt{HashTable.h}.\\
  \textbf{Remarque}: Vous pouvez choisir le système de gestion de collision
  que vous voulez mais votre programme doit rester {\it efficace} quelle
  que soit $N_A$ et $N_B$.
\item Implémentez dans un fichier \texttt{Intersection.c} la fonction d'intersection définie dans le fichier \texttt{Intersection.h}.
\end{itemize}

Nous vous fournissons un programme de test (\texttt{main.c}) calculant l'intersection de deux fichiers texte.
Pour compiler le programme, vous pouvez utiliser la commande suivante:

{\small \texttt{gcc main.c Array.c Intersection.c HashTable.c --std=c99 --pedantic -Wall -Wextra -Wmissing-prototypes -o intersection}}

Ce programme prend comme argument deux fichiers texte et imprime à l'écran les lignes communes aux deux fichiers.

{\small \texttt{./intersection A.txt B.txt}}

\end{document}
