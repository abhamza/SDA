\documentclass[a4paper,10pt]{article}

% Packages
\usepackage[utf8]{inputenc}
\usepackage[frenchb]{babel}
\usepackage{graphicx}
\usepackage{url}
\usepackage{hyperref}
\usepackage{a4wide}
\usepackage{amsmath}
\usepackage{../../clrscode3epg}
\usepackage{color}
%\usepackage{fullpage}
\renewcommand{\labelenumi}{(\alph{enumi})}

% Style
\parskip=\smallskipamount

% En-têtes
\title{
    \textbf{Structures de données et algorithmes}\\
    Projet 2: Tables de hachage
}

\author{Gilles \textsc{Louppe} -- Julien \textsc{Becker} -- Pierre \textsc{Geurts}}
\date{15 avril 2013}

% Corps
\begin{document}
\maketitle

L'objectif du projet est d'écrire un algorithme efficace permettant de
calculer l'intersection de deux fichiers (en terme de lignes). On
étudiera d'abord théoriquement différentes solutions et la plus
efficace sera ensuite implémentée.

Le projet est à réaliser individuellement pour le {\bf 15 avril 2013}
à {\bf 05h00} (du matin) au plus tard. Le projet est à remettre via
une interface web disponible sur
\href{http://www.montefiore.ulg.ac.be/~glouppe/2012-2013/students.info0902.php}{la
  page des TPs}.

Un projet non rendu à temps recevra automatiquement une cote nulle. En
cas de plagiat avéré, l'étudiant se verra affecter une cote nulle à
l'ensemble du projet. Soyez bref mais précis dans votre rapport, qui
fera au maximum 5 pages, et respectez bien la numérotation des
sous-questions de l'énoncé.

Les critères de correction sont précisés sur la page web des projets.

\section{Enoncé}

On aimerait implémenter un programme permettant d'afficher les lignes
communes à deux fichiers passés en argument. Par exemple, si les
fichiers \texttt{A.txt} et \texttt{B.txt} ont les contenus suivants:

\begin{verbatim}
      A.txt:            B.txt:
      aaa               bbb
      bbb               eee
      ccc               fff
      ddd
      eee
\end{verbatim}

Le résultat de l'application du programme affichera les lignes
\texttt{bbb} et \texttt{eee}. (L'ordre d'affichage des lignes communes n'a pas
d'importance.)

\section{Analyse théorique}

Sans perte de généralité, supposons que les fichiers aient été chargés
dans deux tableaux $A$ et $B$, respectivement de longueurs $N_A$ et $N_B$ et que
$N_A\leq N_B$.
\begin{enumerate}
\item Quelle est la complexité théorique minimale pour calculer
  l'intersection de $A$ et $B$, en fonction de $N_A$ et $N_B$, quelle
  que soit la solution adoptée ? (sans faire d'autres hypothèses sur
  le contenu de $A$ et $B$). Justifiez votre réponse.
\item Une solution générale à ce problème consiste à lire et à stocker
  dans une structure de données de type dictionnaire les éléments de $A$ et à ensuite
  parcourir les éléments de $B$ en ne retenant que ceux qui se
  trouvent dans la structure.

  \begin{enumerate}
    \item[1.] Donnez le pseudo-code d'une fonction calculant
      l'intersection de $A$ et $B$ selon l'algorithme décrit
      précédemment, en supposant que vous disposez des fonctions
      $\proc{search}$, $\proc{insert}$ et $\proc{delete}$ d'accès au
      dictionnaire (voir le transparent 198 du cours théorique).  Les
      éléments communs doivent être placés dans un tableau auxiliaire
      $C$ et ce tableau doit être retourné par cette fonction. Notez
      que la longueur du tableau $C$ n'est a priori pas connue.

    \item[2.] Donnez et justifiez la complexité au pire cas de cette
      solution lorsqu'on utilise les structures de données suivantes:
    \begin{itemize}
    \item liste
    \item vecteur trié
    \item arbre binaire de recherche
    \item table de hachage
    \end{itemize}
  \end{enumerate}
\item Si on pouvait supposer que $A$ et $B$ étaient
  triés par ordre alphabétique, comme procéderiez-vous et quelle
  serait la complexité de votre solution aux meilleur et pire cas ?
\end{enumerate}

\section{Implémentation}

On vous demande d'implémenter la solution basée sur une table de
hachage.

\begin{enumerate}
\item Implémentez dans un fichier \texttt{HashTable.c} l'interface générique de table de hachage définie dans le fichier \texttt{HashTable.h}.
  Vous pouvez choisir le système de gestion de collision
  que vous voulez mais votre programme doit rester {\it efficace} quelle
  que soit $N_A$ et $N_B$.
\item Implémentez dans un fichier \texttt{Intersection.c} la fonction d'intersection définie dans le fichier \texttt{Intersection.h}. Cette fonction doit implémenter l'algorithme décrit au point 2 (b) en utilisant comme structure de données une \texttt{HashTable} telle que définie dans \texttt{HashTable.h}.
\end{enumerate}

Nous vous fournissons un programme de test (\texttt{main.c}) calculant l'intersection de deux fichiers texte.
Pour compiler le programme, vous pouvez utiliser la commande suivante:

\begin{verbatim}
gcc main.c \
    StringArray.c \
    Intersection.c \
    HashTable.c --std=c99 --pedantic -Wall -Wextra -Wmissing-prototypes -o intersection
\end{verbatim}

Ce programme prend comme argument deux fichiers texte et imprime à l'écran les lignes communes aux deux fichiers.

\begin{verbatim}
./intersection A.txt B.txt
\end{verbatim}

\section{Analyse}

On vous demande de vérifier expérimentalement l'efficacité de votre
implémentation de la table de hachage, en particulier l'insertion.
\begin{enumerate}
\item Calculez les temps de calcul nécessaires à l'insertion de $n$
  valeurs dans votre table pour des valeurs de $n$
  croissantes. Reportez les résultats de cette expérience dans une
  table ou une figure. Comme pour le premier projet, il peut être
  utile de calculer des moyennes sur plusieurs expériences.
\item Discutez ces résultats par rapport aux résultats théoriques attendus.
\end{enumerate}

Pour répondre à cette question, vous devrez écrire un fichier
$\texttt{main.c}$ séparé, que vous ne devez pas nous rendre. Pour
générer des clés aléatoires à placer dans la table de hachage, vous
pouvez utiliser la fonction \texttt{rand()} définie dans la librairie \texttt{stdlib.h}. Choisissez la taille
initiale de votre table ainsi que les valeurs de $n$ de manière à ce
que votre analyse prenne en compte le rehachage.

\end{document}
