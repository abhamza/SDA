\documentclass[a4paper,10pt]{article}

% Packages
\usepackage[utf8]{inputenc}
\usepackage[frenchb]{babel}
\usepackage{graphicx}
\usepackage{url}
\usepackage{hyperref}
\usepackage{a4wide}
\usepackage{amsmath}
\usepackage{../../clrscode3epg}
%\usepackage{fullpage}
\renewcommand{\labelenumi}{(\alph{enumi})}

% Style
\parskip=\smallskipamount

% En-têtes
\title{
    \textbf{Structures de données et algorithmes}\\
    Projet de seconde session 2011-2012:\\Résolution de problème
}

%\author{Julien \textsc{Becker} - Gilles \textsc{Louppe} - Pierre \textsc{Geurts}}
\date{16 mars 2012}

% Corps
\begin{document}
\maketitle

\section*{\'Enoncé}

L'objectif du projet est de vous exercer à l'utiliser des techniques de résolutions de problèmes.

Le projet est à réaliser {\bf individuellement} pour le {\bf 31 août 2012} à
{\bf 05h00} (du matin) au plus tard. Le projet est à remettre via une interface
web disponible sur \href{http://www.montefiore.ulg.ac.be/~glouppe/2011-2012/students.info0902.php}{la page des TPs}.

Un projet non rendu à temps recevra automatiquement une cote nulle. En cas de
plagiat avéré, l'étudiant se verra affecter une cote nulle à l'ensemble du
projet. Les mêmes critères de correction que ceux utilisés dans le cadre du
cours \href{http://www.montefiore.ulg.ac.be/~info0030/}{INFO0030 - Projet de programmation}
seront utilisés pour évaluer l'implémentation des algorithmes.

\subsection*{Fichier fourni}
\begin{description}
\item[\texttt{blablabla.h}] contient l'interface à implémenter.
\end{description}
\subsection*{Fichiers à rendre}
Deux fichiers sont à rendre dans une archive \texttt{.tar.gz}. Le nom de l'archive n'a pas d'importance.
\begin{description}
\item[\texttt{blablabla.c}] implémente les fonctions de l'interface \texttt{blablabla.h}
\item[\texttt{Rapport.pdf}] contient vos réponses aux questions
\end{description}

{\em Note}: Les noms de fichiers font partie de l'énoncé. Tout fichier ne
correspondant pas à ceux demandés dans l'énoncé ne sera pas pris en compte.

\subsection*{Implémentation}

Soit une séquence $\langle x_1,\ldots,x_m\rangle$. Une sous-séquence
de $\langle x_1,\ldots,x_m\rangle$ est une séquence $\langle
x_{i_1},\ldots,x_{i_k}\rangle$ où $1\leq i_1<i_2<\ldots<i_k\leq m$. On
dira qu'une sous-séquence $\langle x_{i_1},\ldots,x_{i_k}\rangle$ est
contiguë si $i_{j+1}=i_{j}+1, \forall j=1,\ldots,k-1$.

On vous demande d'implémenter au total quatre fonctions:
\begin{itemize}
\item Une fonction renvoyant les bornes de la plus longue
  sous-séquence contiguë de valeur constantes dans une séquence. Par
  exemple, si la séquence d'entrée est $\langle
  1,6,3,3,3,4,4,3,4\rangle$, la plus longue sous-séquence contiguë est
  $\langle 3,3,3\rangle$ et la fonction renverra $(2,4)$. Cette
  fonction devra être implémentée de deux manières:
\begin{itemize}
\item Par une recherche exhaustive
\item Par une approche diviser-pour-régner
\end{itemize}
\item Une fonction renvoyant la longueur (et la valeur correspondante)
  de la plus longue sous-séquence non nécessairement contiguë de
  valeurs constantes dans une séquence. Par exemple, si la séquence
  d'entrée est $\langle 1,6,3,3,3,4,4,3,4\rangle$, la plus longue
  sous-séquence constante non contiguë est $\langle 3,3,3,3\rangle$ et la
  fonction renverra $(4,3)$. Cette fonction devra être implémentée par
  programmation dynamique.
\item Une fonction renvoyant la longueur (et la valeur correspondante)
  de la plus longue sous-séquence commune (non nécessairement
  contiguë) à deux séquences. Par exemple, si les deux séquences
  d'entrée sont $\langle 1,6,3,3,3,4,4,3,4\rangle$ et $\langle
  3,6,2,5,3,4,4,4,4\rangle$, la plus longue sous-séquence commune
  constante est $\langle 4,4,4 \rangle$ et la fonction renverra
  $(3,4)$. Cette fonction devra également être implémentée par
  programmation dynamique.
\end{itemize}

\subsection*{Rapport}

Questions:
\begin{itemize}
\item Pour chacune des quatre fonctions à implémenter, nous vous demandons
d'expliquer le principe de votre solution et d'analyser sa complexité
au pire cas. Pour les deux fonctions à implémenter par programmation
dynamique, on vous demande d'expliciter la récurrence à la
base de votre solution.
\item En supposant que les séquences contiennent toutes des entiers
  positifs inférieurs à 255, expliquez le principe et donnez la
  complexité d'une solution plus efficace au dernier problème
  n'utilisant pas la programmation dynamique.
\end{itemize}

Vos réponses à ces questions doivent être rédigées dans un bref rapport (au format
PDF) à joindre avec votre fichier source.

\end{document}
