\documentclass[a4paper,10pt]{article}

% Packages
\usepackage[utf8]{inputenc}
\usepackage[frenchb]{babel}
\usepackage{graphicx}
\usepackage{url}
\usepackage{hyperref}
\usepackage{a4wide}
\usepackage{amsmath}
\usepackage{../../clrscode3epg}
\usepackage{color}
%\usepackage{fullpage}
\renewcommand{\labelenumi}{(\alph{enumi})}

% Style
\parskip=\smallskipamount

% En-têtes
\title{
    \textbf{Structures de données et algorithmes}\\
    Projet 1: Algorithmes de tri
}

\author{Gilles \textsc{Louppe} -- Julien \textsc{Becker}}
\date{16 mars 2013}

% Corps
\begin{document}
\maketitle

L'objectif du projet est d'implémenter, de comparer et d'analyser empiriquement
trois algorithmes de tri.

Le projet est à réaliser {\bf individuellement} pour le {\bf 16 mars 2013} à
{\bf 05h00} (du matin) au plus tard. Le projet est à remettre via une interface
web disponible sur \href{http://www.montefiore.ulg.ac.be/~glouppe/2012-2013/students.info0902.php}{la page des TPs}.

Un projet non rendu à temps recevra automatiquement une cote nulle. En
cas de plagiat avéré, l'étudiant se verra affecter une cote nulle à
l'ensemble du projet. Soyez bref mais précis dans votre rapport, qui
fera au maximum 5 pages, et respectez bien la numérotation des
sous-questions de l'énoncé.

Les critères de correction seront précisés sur la page web des
projets.

\section{Enoncé}

Dans beaucoup d'applications réelles, les tableaux qu'on a à trier
sont constitués de sous-tableaux déjà triés. Il serait donc
intéressant de mettre au point un algorithme de tri tirant parti de
cette propriété. Sur base de cette idée, on se propose d'étudier un
nouvel algorithme de tri itératif basé sur le principe suivant (en
supposant le tableau déjà trié par l'algorithme jusqu'à l'indice $i$):
\begin{enumerate}
%\item On suppose que le tableau est trié jusqu'à l'indice $i$
\item On recherche l'indice $j$ maximal tel que le sous-tableau $A[i+1\twodots j]$ soit trié.
\item On fusionne $A[1\twodots i]$ et $A[i+1 \twodots j]$ (comme le
  fait la fonction $\proc{merge}$ vue au cours théorique, transparent
  45).
\item Tant que le tableau n'est pas complètement trié, on recommence
  en 2 en prenant $i=j+1$.
\end{enumerate}

\bigskip

Exemple: Soit le tableau suivant: $[1,5,2,6,4,3,9]$. Les étapes de
fusion seront les suivantes (la partie verte est fusionnée avec la
partie rouge pour donner la partie bleue):
\begin{itemize}
\item $[{\color{green} 1,5},{\color{red} 2,6},4,3,9]$ $\Rightarrow$ $[{\color{blue} 1,2,5,6},4,3,9]$
\item $[{\color{green} 1,2,5,6},{\color{red}4},3,9]$ $\Rightarrow$ $[{\color{blue} 1,2,4,5,6},3,9]$
\item $[{\color{green} 1,2,4,5,6},{\color{red} 3,9}]$ $\Rightarrow$ $[{\color{blue} 1,2,3,4,5,6,9}]$
\end{itemize}

\bigskip

\section{Analyse théorique}
\begin{enumerate}
\item Enoncez l'invariant associé à la boucle principale de cet
  algorithme (dont le corps est constitué des étapes 1 et 2 décrites
  ci-dessus).
\item Ecrivez en pseudo-code une fonction $\proc{NewSort}$
  implémentant l'algorithme décrit ci-dessus. Vous pouvez supposer que la fonction
  $\proc{merge}$ telle que définie au cours théorique est connue.
\item Etudier sa complexité en temps dans le meilleur cas et dans le
  pire cas en fonction de la taille $n$ du tableau à trier. Expliquez
  à quoi correspondent les meilleur et pire cas et justifiez la
  complexité que vous obtenez dans ces deux cas.
%\item Discutez les avantages et inconvénients de cet algorithme:
%\begin{itemize}
%\item Par rapport au tri par insertion ($\proc{InsertionSort}$)
%\item Par rapport au tri rapide ($\proc{QuickSort}$)
%\end{itemize}
\item Est-ce que cet algorithme de tri est stable ? Justifiez brièvement votre réponse.
\item Quelle est
  la complexité au pire cas de l'algorithme si le tableau de taille $n$ est
  constitué de $k (\leq n)$ blocs prétriés de {\it taille identique} ? Par exemple, le tableau
  suivant est constitué de 4 blocs prétriés de taille 3:
$$[{\color{green} 5,6,7},{\color{blue}1,8,9},{\color{green}2,3,11},{\color{blue}12,15,16}]$$
\end{enumerate}

\section{Analyse expérimentale}

\subsection{Implémentation}

\begin{enumerate}
\item Implémenter l'algorithme $\proc{NewSort}$ dans un fichier \texttt{NewSort.c}.
\item Implémenter l'algorithme $\proc{QuickSort}$ dans un fichier \texttt{QuickSort.c}.
\item Implémenter l'algorithme $\proc{InsertionSort}$ dans un fichier \texttt{InsertionSort.c}.
\end{enumerate}

Les trois implémentations doivent respecter l'interface de tri décrite dans le
fichier \texttt{Sort.h}. Chaque tri doit être implémenté dans un fichier qui lui
est propre.

\subsubsection*{Exemple avec le BubbleSort}

Afin de vous aider à prendre rapidement en main le code mis en place, nous avons
implémenté à titre d'exemple l'algorithme BubbleSort dans le fichier
\texttt{BubbleSort.c} ainsi qu'un petit programme d'essai \texttt{main.c}.

Pour compiler le programme, vous pouvez utiliser la commande suivante:

{\small \texttt{gcc main.c Array.c BubbleSort.c --std=c99 --pedantic -Wall -Wextra -Wmissing-prototypes -o test}}

Ce programme ne prend aucun argument.

%% \subsubsection*{Robustesse}

%% à définir.

\subsection{Temps d'exécution sur des tableaux aléatoires}

\begin{enumerate}
\item Soit $n$ le nombre de données à trier dans un tableau. Calculez
empiriquement le temps d'exécution moyen des trois algorithmes
($\proc{QuickSort}$, $\proc{InsertionSort}$ et $\proc{NewSort}$) pour
différentes valeurs de $n$ (10, 100, 1.000, 10.000, 100.000 et
1.000.000) lorsque les tableaux sont générés de manière complètement
aléatoires. La moyenne doit être obtenue sur un ensemble de 20
expériences. Reportez ces résultats dans une table au format donné ci-dessous.

\begin{center}
\begin{tabular}{cccc}
	\hline
	n & $\proc{QuickSort}$ & $\proc{InsertionSort}$ & $\proc{NewSort}$ \\
	\hline
	10 & & & \\
	100 & & &\\
	1.000 & & &\\
	10.000 & & &\\
	100.000 & & &\\
	1.000.000 & & &\\
\end{tabular}
\end{center}

\item Commentez ces résultats en comparant les algorithmes:
\begin{itemize}
\item les uns par rapport aux autres;
\item par rapport à leur complexité théorique.
\end{itemize}
\end{enumerate}

{\em Notes:}
\begin{itemize}
\item Une fonction \texttt{createRandomArray} vous est fournie pour générer un tableau aléatoire de taille $n$.
\item Le temps d'exécution est une valeur peu précise qui dépend fortement
des capacités de l'ordinateur mais également de l'état d'utilisation de celui-ci
au moment des expériences. Pour limiter cet effet, il vous est conseillé
de réaliser toutes vos mesures de manière séquentielle sur la même machine.
\end{itemize}

\subsection{Temps d'exécution sur des tableaux de blocs pré-triés}

\begin{enumerate}
\item Pour une taille de tableau $n=5000$, calculez empiriquement le temps
d'exécution moyen des trois algorithmes ($\proc{QuickSort}$,
$\proc{InsertionSort}$ et $\proc{NewSort}$) lorsque le tableau à trier
contient $k$ blocs pré-triés, pour des valeurs de $k$
croissantes. Comme pour le tableau précédent, on reportera dans le
tableau ci-dessus les temps moyens sur 20 expériences.

\begin{center}
\begin{tabular}{cccc}
	\hline
	k & $\proc{QuickSort}$ & $\proc{InsertionSort}$ & $\proc{NewSort}$ \\
	\hline
	1 & & &  \\
	20 & & &\\
	100 & & &\\
	500 & & &\\
	1000 & & &\\
	5000 & & &\\
\end{tabular}
\end{center}

{\em Note:} Une fonction \texttt{createRandomBlockArray} vous est fournie pour générer un tableau aléatoire de blocs pré-triés.

\item Commentez ces résultats en comparant les algorithmes les uns par rapport aux autres.
\item Concluez sur l'intérêt ou non de $\proc{newSort}$ par rapport aux deux autres algorithmes de tri.

\end{enumerate}



\end{document}
