\documentclass[a4paper,10pt]{article}

% Packages
\usepackage[utf8]{inputenc}
\usepackage[frenchb]{babel}
\usepackage{graphicx}
\usepackage{url}
\usepackage{hyperref}
\usepackage{a4wide}
\usepackage{amsmath}
\usepackage{../../clrscode3epg}
%\usepackage{fullpage}
\renewcommand{\labelenumi}{(\alph{enumi})}

% Style
\parskip=\smallskipamount

% En-têtes
\title{
    \textbf{Structures de données et algorithmes}\\
    Projet de seconde session 2012-2013:\\Résolution de problème
}

%\author{Julien \textsc{Becker} - Gilles \textsc{Louppe} - Pierre \textsc{Geurts}}
\date{}

% Corps
\begin{document}
\maketitle

\section*{\'Enoncé}

L'objectif du projet est de vous exercer à l'utilisation des techniques de résolutions de problèmes.

Le projet est à réaliser {\bf individuellement} pour le {\bf 25 août
  2013} à {\bf 23h59} au plus tard. Le projet est à remettre via une
interface web disponible sur
\href{http://www.montefiore.ulg.ac.be/~glouppe/2012-2013/students.info0902.php}{la
  page des TPs}.

Un projet non rendu à temps recevra automatiquement une cote nulle. En cas de
plagiat avéré, l'étudiant se verra affecter une cote nulle à l'ensemble du
projet. Les critères de correction sont précisés sur la page web des projets.

\subsection*{Fichier fourni}
\begin{description}
\item[\texttt{SubSequence.h}] contient l'interface à implémenter.
\end{description}
\subsection*{Fichiers à rendre}
Deux fichiers sont à rendre dans une archive \texttt{.tar.gz}. Le nom de l'archive n'a pas d'importance.
\begin{description}
\item[\texttt{SubSequence.c}] implémente les fonctions de l'interface \texttt{SubSequence.h}
\item[\texttt{Rapport.pdf}] contient vos réponses aux questions
\end{description}

{\em Note}: Les noms de fichiers font partie de l'énoncé. Tout fichier ne
correspondant pas à ceux demandés dans l'énoncé ne sera pas pris en compte.

\subsection*{Implémentation}

Une séquence $S=\langle x_0,\ldots,x_{m-1}\rangle$ est une liste
ordonnée de valeurs entières. Une sous-séquence d'une séquence
$S=\langle x_0,\ldots,x_{m-1}\rangle$ est une séquence $\langle
x_{i_1},\ldots,x_{i_k}\rangle$ d'éléments de $S$ où $0\leq i_1<i_2<\ldots<i_k<m$. On
dira qu'une sous-séquence $\langle x_{i_1},\ldots,x_{i_k}\rangle$ est
{\bf contiguë} si $i_{j+1}=i_{j}+1, \forall j=1,\ldots,k-1$ et qu'une
sous-séquence est constante si tous ses éléments sont {\bf identiques},
$x_{i_1}=x_{i_2}=\ldots=x_{i_k}$.

On vous demande d'implémenter au total quatre fonctions:

\begin{itemize}
\item Une fonction renvoyant les indices délimitant une plus longue
  sous-séquence contiguë constante dans une séquence. Par exemple, si
  la séquence d'entrée est $\langle 1,6,3,3,3,4,4,3,4\rangle$, la plus
  longue sous-séquence contiguë est $\langle 3,3,3\rangle$ et la
  fonction renverra $(2,4)$. Cette fonction devra être implémentée de
  deux manières différentes:
\begin{itemize}
\item en utilisant une recherche exhaustive (\texttt{getCstContSubSeqES})
\item en utilisant une approche diviser-pour-régner (\texttt{getCstContSubSeqDC})
\end{itemize}

\item Une fonction \texttt{getCstSubSeqDP} renvoyant la longueur (et la valeur correspondante)
  d'une plus longue sous-séquence constante non nécessairement
  contiguë dans une séquence. Par exemple, si la séquence d'entrée est
  $\langle 1,6,3,3,3,4,4,3,4\rangle$, la plus longue sous-séquence
  constante non contiguë est $\langle 3,3,3,3\rangle$ et la fonction
  renverra $(4,3)$. Cette fonction devra être implémentée par
 programmation dynamique.

\item Une fonction \texttt{getLongestCommonCstSubSeqDP} renvoyant la longueur (et la valeur correspondante)
  d'une plus longue sous-séquence constante (non nécessairement
  contiguë) commune à deux séquences. Par exemple, si les deux
  séquences d'entrée sont $\langle 1,6,3,3,3,4,4,3,4\rangle$ et
  $\langle 3,4,6,4,5,3,4,4\rangle$, la plus longue sous-séquence
  commune constante est $\langle 4,4,4 \rangle$ et la fonction
  renverra $(3,4)$. Cette fonction devra également être implémentée
  par programmation dynamique.\\

{\it Suggestion: Soit $X$ et $Y$ les séquences d'entrée. Calculez pour
  tout $i$ et $j$, la longueur $c[i,j]$ de la plus longue
  sous-séquence constante commune à $X$ et $Y$ qui se {\bf termine} à
  l'élément $i$ de $X$ et à l'élément $j$ de $Y$. Dans l'exemple
  précédent, on aura $c[3,0]=1$ et $c[5,2]=0$.}
\end{itemize}

\subsection*{Rapport}


Pour \textbf{chacune} des quatre fonctions à implémenter, on vous demande:
\begin{itemize}
\item d'expliquer le principe de votre solution;
\item d'analyser sa complexité au pire cas.
\end{itemize}

Pour les deux fonctions à implémenter par programmation
dynamique, on vous demande d'expliciter la récurrence à la
base de votre solution.

Vos réponses à ces questions doivent être rédigées dans un bref rapport (au format
PDF) à joindre avec votre fichier source.

\end{document}
