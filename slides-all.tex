% A FAIRE VERIFIER ETC !!!

% - regarder à ce que les définitions d'arbres soient cohérente entre le chapitre 3 et le chapitre 5 (notamment, arbre ordonné, arbre complet, etc.)
% - ajouter les slides ``ce qu'on a vu'', ``ce qu'on n'a pas vu'' (ce qu'on aurait pu voir)

% - ajouter des exemples d'application: table hash -> recherche genome, arbre binaire: scheduling des avions (voir le cours du MIT de Kellis et autres)

% - ajouter le tas binomial dans le cours 4 ? Pour la concaténation de tas ?

%\documentclass[12pt,handout]{beamer}
\documentclass{beamer}
\usepackage[utf8]{inputenc}
\usepackage[french]{babel}

% PACKAGES
% --------

\usetheme{default}

\usepackage{pst-node}
\usepackage{clrscode3epg}
\usepackage{multido}
\usepackage{fancybox}
\usepackage{graphics}
\usepackage{verbatim}

%\usepackage{pgfpages}
%\pgfpagesuselayout{4 on 1}[a4paper,border shrink=5mm,landscape]
%\pgfpagesuselayout{2 on 1}[a4paper,border shrink=5mm,portrait]
%\pgfpagesuselayout{resize to}[a4paper,border shrink=5mm,landscape]

\usepackage{multirow}
\usepackage{comment}

% BEAMER OPTION
% -------------
% Dérivé du style de Julien Brusten et de Antonio Carzaniga

\setbeamertemplate{navigation symbols}{}

\usepackage[footheight=1em]{beamerthemeboxes}

\addfootboxtemplate{\color{white}}{~~~~~\color{lightgray}\insertpart
     \hfill\insertframenumber~~~}

\defbeamertemplate*{part page}{mypartpage}[1][]
{
\begin{centering}
{\usebeamerfont{part name}\usebeamercolor[fg]{part name}
Partie~\insertpartnumber}
\vskip1em\par
\begin{beamercolorbox}[rounded=true,shadow=true,sep=8pt,center,#1]{part title}
\usebeamerfont{part title}\insertpart\par
\end{beamercolorbox}
\end{centering}
}

\setbeamertemplate{part page}[mypartpage][]

\setbeamertemplate{itemize item}{\vbox{\hrule width1ex height1ex depth0pt}}

\newhsbcolor{Lightgray}{0 0 .95}

\newcommand{\semitransp}[2][35]{\color{fg!#1}#2}

\setbeamersize{text margin left=.5cm,text margin right=.5cm,sidebar width left=0pt,sidebar width right=0pt}

\makeatletter

\numberwithin{section}{part}

\AtBeginPart{%
\beamer@tocsectionnumber=0\relax\addtocontents{toc}{\protect\beamer@partintoc{\the\c@part}{\beamer@partnameshort}{\the\c@page}}
\frame{\partpage}
}%


%% number, shortname, page.
\providecommand\beamer@partintoc[3]{%
  \ifnum\c@tocdepth=-1\relax
    % requesting onlyparts.
    \makebox[6em]{Partie #1:} #2
    \par
  \fi
}

\define@key{beamertoc}{onlyparts}[]{%
  \c@tocdepth=-1\relax

}

\makeatother%


%\AtBeginPart{\frame{\partpage}}

% debut du document

\definecolor{darkgreen}{rgb}{0,0.5,0}
\definecolor{darkred}{rgb}{0.6,0,0}
\definecolor{gold}{rgb}{1,0.84,0}

\setbeamercolor{alerted text}{fg=beamer@blendedblue}

\newcommand{\pgcd}{\mbox{pgcd}}
\newcommand{\arete}[2]{#1\mbox{---}#2}

\title{Structures de données et algorithmes}
\author{Pierre Geurts}
\date{
Version du 5 février 2013

\bigskip

\bigskip

{\small
\begin{tabular}{lcl}
E-mail & : & {\tt p.geurts@ulg.ac.be}\\
URL    & : & {\tt http://www.montefiore.ulg.ac.be/}\\
       &   & {\tt ~~~~\~{}geurts/sda.html}\\
Bureau & : & R 141 (Montefiore)\\
Téléphone & : & 04.366.48.15 | 04.366.99.64
\end{tabular}
}
}

\newcommand{\nats}{\mathbb{N}}
\newcommand{\ints}{\mathbb{Z}}
\newcommand{\reals}{\mathbb{R}}
\newcommand{\rationals}{\mathbb{Q}}

\setbeamertemplate{section in toc}[sections numbered]
\setbeamertemplate{subsection in toc}[sections numbered]

\begin{document}

\begin{frame}
\titlepage
\end{frame}


\begin{frame}{Contact}
\begin{itemize}
\item Chargé de cours:
\begin{itemize}
\item Pierre Geurts, \url{p.geurts@ulg.ac.be}, I141 Montefiore, 04/3664815
\end{itemize}
\item Assistants:
\begin{itemize}
\item Gilles Louppe, \url{g.louppe@ulg.ac.be}, GIGA-R (B34,+1), CHU, 04/3662766
\item Julien Becker, \url{j.becker@ulg.ac.be}, GIGA-R (B34,+1), CHU, 04/3669805
\end{itemize}
\item Sites web du cours:
\begin{itemize}
\item Cours théorique: \url{http://www.montefiore.ulg.ac.be/~geurts/sda.html}
\item Répétitions et projets: \url{http://www.montefiore.ulg.ac.be/~glouppe/2011-2012/students.info0902.php}
\end{itemize}
\end{itemize}
\end{frame}

\begin{frame}{Objectif du cours}

\begin{itemize}
\item Introduction à l'étude systématique des algorithmes et des
  structures de données
\item Vous fournir une boîte à outils contenant:
\begin{itemize}
\item Des structures de données permettant d'organiser et d'accéder efficacement
  aux données
\item Les algorithmes les plus populaires
\item Des méthodes génériques pour la modélisation, l'analyse et la résolution de problèmes algorithmiques
\end{itemize}
\item On insistera sur la généralité des algorithmes et structures de données et on les étudiera de manière formelle
\item Les projets visent à vous familiariser à la résolution de problèmes
\end{itemize}

\note{
Maintenant qu'on est débarassé de l'apprentissage, on va pouvoir se focaliser sur l'algorithmique. Vous avez appris à écrire des programmes, on va apprendre à écrire des algorithmes: procédure de résolution de problèmes.

Suite du cours d'introduction à l'informatique et précède le cours de techniques de programmation que vous aurez pour certains l'an prochain.
}
\end{frame}

\begin{frame}{Organisation du cours}
\begin{itemize}
\item Cours théoriques:
\begin{itemize}
\item Les vendredis de 14h à 16h, S94, Bâtiment B4 (Europe).
\item 10-12 cours
%\item Transparents disponibles sur la page web du cours avant chaque cours
\end{itemize}
\item Répétitions:
\begin{itemize}
\item Certains vendredis de 16h à 18h, S94, Bâtiment B4 (Europe)
\item $\pm$5 répétitions (+ debriefing des projets)
\end{itemize}
\item Projets:
\begin{itemize}
\item Trois projets tout au long de l'année, de difficulté croissante
\item Les deux premiers individuels, le troisième en binôme
\item En C
\end{itemize}

\bigskip
%\item Assistance au cours théorique n'est pas requise mais fortement conseillée.
\item Evaluation sur base des projets (30\%) et d'un examen écrit (70\%).
\end{itemize}
\note{
Le cours théorique sera totalement indépendant du langage de programmation. On utilisera ce qu'on appelle le pseudo-code

%\bigskip

Demander s'ils connaissant tous bien le C.

%\bigskip

Si certains points vus au cours théorique ne sont pas clairs, on peut
organiser une répétition mais il est préférable que vous posiez vos
questions directement au cours. Avant chaque cours, séance de
questions-réponses sur la séance précédente.

%\bigkskip

En plus de ça, il est recommandé d'essayer d'implémenter et de tester
les algorithmes et structures de données vues au cours.  }
\end{frame}


\begin{frame}{Notes de cours}
\begin{itemize}
\item Transparents disponibles sur la page web du cours avant chaque cours
\item Pas de livre de référence obligatoire mais les ouvrages suivants ont été utilisés pour préparer le cours:
{\small
\begin{itemize}
\item {\bf Introduction to algorithms, Cormen, Leiserson, Rivest, Stein, MIT press, Third edition, 2009.}
\begin{itemize}
\item \url{http://mitpress.mit.edu/algorithms/}
\end{itemize}
\item Algorithms, Sedgewick and Wayne, Addison Wesley, Fourth edition, 2011.
\begin{itemize}
\item \url{http://algs4.cs.princeton.edu/home/}
\end{itemize}
\item Data structures and algorithms in Java, Goodrich and Tamassia, Fifth edition, 2010.
\begin{itemize}
\item \url{http://ww0.java4.datastructures.net/}
\end{itemize}
\item Algorithms, Dasgupta, Papadimitriou, and Vazirani, McGraw-Hill, 2006.
\begin{itemize}
\item \url{http://cseweb.ucsd.edu/users/dasgupta/book/index.html}
\item \url{http://www.cs.berkeley.edu/~vazirani/algorithms/all.pdf}
\end{itemize}
\end{itemize}
}
\end{itemize}
\note{Bouquin de référence: CLRS (pas obligatoire, mais vous avez peu de chance de regretter votre achat). Plus un ouvrage de réference qu'un ouvrage pédagogique. On ne verra qu'une toute petite partie. Une version française existe mais je ne l'ai pas et je me baserai sur la version anglaise
}
\end{frame}

\begin{frame}{Cours sur le web}
Ce cours s'inspire également de plusieurs cours disponibles sur le web:
\begin{itemize}
\item Antonio Carzaniga, Faculty of Informatics, University of Lugano
\begin{itemize}
\item \url{http://www.inf.usi.ch/carzaniga/edu/algo/index.html}
\end{itemize}
\item Marc Gaetano, Polytechnique, Nice-Sophia Antipolis
\begin{itemize}
\item \url{http://users.polytech.unice.fr/~gaetano/asd/}
\end{itemize}
\item Robert Sedgewick, Princeton University
\begin{itemize}
\item \url{http://www.cs.princeton.edu/courses/archive/spr10/cos226/lectures.html}
\end{itemize}
\item Charles Leiserson and Erik Demaine, MIT.
\begin{itemize}
\item \url{http://ocw.mit.edu/courses/electrical-engineering-and-computer-science/6-046j-introduction-to-algorithms-sma-5503-fall-2005/index.htm}
\end{itemize}
\item Le cours de 2009-2010 de Bernard Boigelot
\end{itemize}
\end{frame}


\begin{frame}{Contenu du cours}

\tableofcontents[onlyparts]

\end{frame}


\part{Introduction}

% Prendre l'intro de ``intro-tres-bien.pdf'' -> 1h

% Prendre l'intro à Fibonacci de ... pour introduire la récurrence et
% montrer la différence entre un algorithme efficace et un pas
% efficace. Prendre aussi l'exemple de la factorielle de Boigelot. ->
% 1h

% A dire au premier cours -> prendre le cours de

% - Motiver avec des exemples concrets (google) -> voir les slides de Bernard ou autres. Internet, etc.
%   (voir intro-pas-mal.ppt dans le répertoire resources)

% - Définir algorithme
% - Donner un exemple: le tri par insertion -> permet de décrire le pseudo-code (dire que je le ferai en anglais et que je me permettrai d'utiliser des termes anglais aussi lors de la présentation. En préparant le cours, j'ai eu beaucoup de mal à retrouver les correspondances avec certains termes anglais)

% - Parler des structures de données, parler de l'interface, donner un exemple (liste liée !! Dire que c'est à la fois un type de données abstraits et aussi une implémentation particulière en C -> implémentation avec un pointeur)
%
% Passer en revue ce qu'on va voir au cours

% Introduire analyse de complexité + récurrence.

% Fibonacci -> introduction aux récurrences (Carzinaga)
%
% - Prendre un exemple pour montrer qu'on peut gagner énormément en utilisant une implémentation non naïve et
% - Passer en revue toutes les structures de données et les algorithmes qu'on verra dans la suite
% - Détailler et expliquer la structure du cours

% - Voir le bouquin allemand qui revisite la multiplication !!!

% - Voir les slides dans le répertoire resources


\section{Algorithms $+$ Data structures $=$ Programs (\footnotesize{Niklaus Wirth})}

\begin{frame}{Plan}

\tableofcontents

\note{la citation vient du titre d'un bouquin de 1976 dont le message principal était que les structures de données et les algorithmes étaient intimmement liés}
\end{frame}

\begin{frame}{Introduction}
% Ulrich Klehmet
\begin{itemize}
\item Qu'est-ce qu'une structure de données ?
\item Qu'est-ce qu'un algorithme ?
\item Pourquoi étudier les deux ensemble ?
\item Comment peut-on déterminer l'utilité d'une certaine combinaison de structures de données et d'algorithmes ?
\end{itemize}
\end{frame}

\begin{frame}{Algorithmes}

\begin{itemize}
\item Un {\bf algorithme} est une suite {\em finie} et {\em non-ambiguë} d'opérations ou d'instructions permettant de résoudre un {\em problème}
\item Provient du nom du mathématicien persan {\em Al-Khawarizmi}
  ($\pm 820$), le père de l'algèbre
\item Un problème algorithmique est souvent formulé comme la
  transformation d'un ensemble de valeurs, {\bf d'entrée}, en un
  nouvel ensemble de valeurs, {\bf de sortie}.
\item Exemple d'algorithmes:
\begin{itemize}
\item Une recette de cuisine (ingrédients $\rightarrow$ plat préparé)
\item La recherche dans un dictionnaire (mot $\rightarrow$ définition)
\item La division entière (deux entiers $\rightarrow$ leur quotient)
\item Le tri d'une séquence (séquence $\rightarrow$ séquence ordonnée)
%% \begin{itemize}
%% \item Entrée = une séquence de $n$ nombres $(a_1,a_2,\ldots,a_n)$
%% \item Sortie = une permutation (réarrangement) $(a'_1,a'_2,\ldots,a'_n)$ de la séquence d'entrée telle que $a'_1\leq a'_2\leq \ldots\leq a'_n$.
%% \end{itemize}
\end{itemize}
\end{itemize}

\note{dictionnaire: structure de donnée: le dictionnaire: mots écrits
  en français et surtout, classés par ordre alphabétique}
\end{frame}

\begin{frame}{Algorithmes}
\begin{itemize}
\item On étudiera essentiellement les algorithmes {\bf corrects}.
\begin{itemize}
\item Un algorithme est (totalement) {\em correct} lorsque pour chaque instance, il se termine en produisant la bonne sortie.
\item Il existe également des algorithmes {\em partiellement corrects} dont la terminaison n'est pas assurée mais qui fournissent la bonne sortie lorsqu'il se termine.
\item Il existe également des algorithmes {\em approximatifs} qui fournissent une sortie inexacte mais néanmoins proche de l'optimum.% Un algorithme incorrect peut ne pas se terminer ou renvoyer une réponse incorrecte.
%\item Un algorithme incorrect peut être parfois utile, si son taux d'erreurs est contrôlé.
\end{itemize}
\bigskip
\item Les algorithmes seront évalués en termes d'{\em utilisation de resources}, essentiellement {\bf temps de calcul} mais aussi utilisation de la {\bf mémoire}.
\end{itemize}
\end{frame}

\begin{frame}{Algorithmes}

Un algorithme peut être spécifié de différentes manières:
\begin{itemize}
\item en langage naturel,
\item graphiquement,
\item en pseudo-code,
\item par un programme écrit dans un langage informatique
\item ...
\end{itemize}
La seule condition est que la description soit précise.

%% \item doit formulé de manière si précise qu'il peut être exécuté sur une machine.
%% \item doit être formulé de manière si préciseIl doit finalement se terminer
%% \item Complet. Toujours donner une solution lorsqu'il en existe une.
%% \item Correct. Toujours donner une solution ``correcte''.
%% \end{itemize}
%% \item Pour qu'un algorithme soit une solution envisageable à un
%%   problème, il doit aussi être efficace, c'est-à-dire donner une
%%   solution en un temps raisonnable.
%% \item Il peut y avoir beaucoup algorithmes pour résoudre le même problème.
%% \end{itemize}

%\item Généralement décrit en utilisant un langage de programmation
%  mais cela pourrait être fait en français directement

\end{frame}

% objectif: présenter le pseudo-code

\begin{frame}{Exemple: le tri}
\begin{itemize}
\item Le problème de tri:
\begin{itemize}
\item Entrée: une séquence de $n$ nombres $\langle a_1,a_2,\ldots,a_n\rangle$
\item Sortie: une permutation de la séquence de départ $\langle a_1',a_2',\ldots,a_n'\rangle$ telle que $a_1'\leq a_2'\leq\ldots\leq a_n'$
\end{itemize}
\bigskip
\item Exemple:
\begin{itemize}
\item Entrée: $\langle 31,41,59,26,41,58\rangle$
\item Sortie: $\langle 26,31,41,41,58,59\rangle$
\end{itemize}
\end{itemize}

\end{frame}

\begin{frame}{Tri par insertion}

\hfill\includegraphics[height=3cm]{Figures/CLRS/Fig-2-1.pdf}

Description en langage naturel:

% PRENDRE LA DESCRIPTION DANS LE BOUQUIN CLRS

\bigskip

On parcourt la séquence de gauche à droite

\bigskip

Pour chaque élément $a_j$:
\begin{itemize}
\item On l'{\alert{insère}} à sa position dans une nouvelle séquence ordonnée contenant les éléments le précédant dans la séquence.
\end{itemize}
On s'arrête dès que le dernier élément a été inséré à sa place dans la séquence.

\end{frame}

\begin{frame}
\frametitle{Tri par insertion}
\centerline{\includegraphics[width=8cm]{Figures/01-insertionsort.pdf}}
\end{frame}

\begin{frame}[fragile]
\frametitle{Tri par insertion}

Description en C (sur des tableaux d'entiers):

%\fcolorbox{white}{Lightgray}{
{\small
\begin{verbatim}
void InsertionSort (int *a, int length) {
  int key;
  for(int j = 1; i < length; j++) {
    key = a[j];
    /* Insert a[j] into the sorted sequence a[0...j-1] */
    i = j-1;
    while (i>=0 && a[i]>key) {
      a[i+1] = a[i];
      i = i-1;
    }
    a[i+1] = key;
  }
}
\end{verbatim}
}

\end{frame}

\begin{frame}{Insertion sort}
Description en {\bf pseudo-code} (sur des tableaux d'entiers):

\bigskip

\fcolorbox{white}{Lightgray}{
\begin{codebox}
\Procname{$\proc{Insertion-Sort}(A)$}
\li \For $j \gets 2$ \To $\attrib{A}{length}$
\li     \Do
$\id{key} \gets A[j]$
\li \Comment Insert $A[j]$ into the sorted sequence
    $A[1 \twodots j-1]$.
\li $i \gets j-1$
\li \While $i > 0$ and $A[i] > \id{key}$
\li   \Do
        $A[i+1] \gets A[i]$
\li        $i \gets i-1$
    \End
\li $A[i+1] \gets \id{key}$
\End
\end{codebox}
}

\end{frame}

\begin{frame}{Pseudo-code}

Objectifs:
\begin{itemize}
\item Décrire les algorithmes de manière à ce qu'ils soient compris
par des humains.

~\\

\item Rendre la description indépendante de l'implémentation

~\\

\item S'affranchir de détails tels que la gestion d'erreurs, les déclarations de type, etc.

\end{itemize}

~\\

Très proche du C (langage procédural plutôt qu'orienté objet)

~\\

Peut contenir certaines instructions en langage naturel si nécessaire

% donner quelques règles de pseudo-code
\end{frame}


\begin{frame}{Pseudo-code}
Quelques règles
\begin{itemize}
\item Structures de blocs indiquées par l'indentation
\item Boucles ($\For$, $\While$, $\Repeat$) et conditions ($\If$,
  $\Then$, $\Else$, $\ElseIf$) comme en C.
\item Le compteur de boucle garde sa valeur à la sortie de la boucle
\item En sortie d'un \texttt{for}, le compteur a la valeur de la borne max+1.
\begin{columns}
\begin{column}{3cm}
\begin{codebox}
\zi \For $i \gets 1$ \To $Max$
\zi     \Do $Code$
\End
\end{codebox}
\end{column}
$\Leftrightarrow$
\begin{column}{5cm}
\begin{codebox}
\zi i=1
\zi \While $i \leq Max$
\zi     \Do $Code$
\zi     $i\gets i+1$
\End
\end{codebox}
\end{column}
\end{columns}
\item Commentaires indiqués par $\Comment$
\item Affectation ($\gets$) et test d'égalité ($\isequal$) comme en C.
\end{itemize}
\end{frame}

\begin{frame}{Pseudo-code}

\begin{itemize}
\item Les variables ($i$, $j$ et $key$ par exemple) sont locales à la fonction.
\item $A[i]$ désigne l'élément $i$ du tableau $A$. $A[i..j]$ désigne
  un intervalle de valeurs dans un tableau. $A.length$ est la taille du tableau.
\item L'indexation des tableaux commence à 1.
\item Les types de données composés sont organisés en {\it objets}, qui sont composés d'attributs. On accède à la valeur de l'attribut $attr$ pour un objet $x$ par $x.attr$.
\item Un variable représentant un tableau ou un objet est considérée comme un pointeur vers ce tableau ou cet objet. 
\item Paramètres passés par valeur comme en C.
\item ...
\end{itemize}
\note{Dire que for i=1 to max est equivalent à i=1, while (i<=max) \{... i++\}}
\end{frame}

\begin{frame}{Trois questions récurrentes face à un algorithme}

\begin{enumerate}
\item Mon algorithme est-il correct, se termine-t-il ? %\textcolor{darkred}{oui}

\bigskip

\item Quelle est sa vitesse d'exécution ? %\textcolor{darkred}{$o(n^2)$}

\bigskip

\item Y-a-t'il moyen de faire mieux ? %\textcolor{darkred}{oui}

\end{enumerate}

\bigskip

Exemple du \textcolor{darkred}{tri par insertion}
\begin{enumerate}
\item Oui $\rightarrow$ technique des invariants (partie 2)
\item $O(n^2)$ $\rightarrow$ analyse de complexité (partie 2)
\item Oui $\rightarrow$ il existe un algorithme $O(n\log n)$ (partie 1)
\end{enumerate}

% commenter sur la facilité de répondre à ces questions
\note{On va passer en revue rapidement ces trois points pour l'insertion-sort. On reviendra sur ça la semaine prochaine}
\end{frame}

\begin{frame}{Correction de $\proc{Insertion-Sort}$}

\begin{center}
\fcolorbox{white}{Lightgray}{
\begin{codebox}
\Procname{$\proc{Insertion-Sort}(A)$}
\li \For $j \gets 2$ \To $\attrib{A}{length}$
\li     \Do
$\id{key} \gets A[j]$
\li $i \gets j-1$
\li \While $i > 0$ and $A[i] > \id{key}$
\li   \Do
        $A[i+1] \gets A[i]$
\li        $i \gets i-1$
    \End
\li $A[i+1] \gets \id{key}$
\End
\end{codebox}
}
\end{center}

\bigskip
\begin{itemize}
\item \alert{Invariant:} (pour la boucle externe) le sous-tableau $A[1\twodots j-1]$ contient les éléments du tableau original $A[1\twodots j-1]$ ordonnés.
\item On doit montrer que
\begin{itemize}
\item l'invariant est vrai avant la première itération %({\em intialisation})
\item l'invariant est vrai avant chaque itération suivante %({\em maintenance})
\item En sortie de boucle, l'invariant implique que l'algorithme est correct %({\em terminaison})
\end{itemize}
\end{itemize}
\note{Si c'est vrai à la terminaison, alors on aura montrer que le tableau est trié à la fin}
\end{frame}

\begin{frame}{Correction de $\proc{Insertion-Sort}$}
\begin{itemize}
\item Avant la première itération:
\begin{itemize}
\item $j=2 \Rightarrow A[1]$ est trivialement ordonné.
\end{itemize}
\bigskip
\item Avant la $j$ème itération:
\begin{itemize}
\item Informellement, la boucle interne déplace $A[j-1]$, $A[j-2]$, $A[j-3]\ldots$ d'une position vers la droite jusqu'à la bonne position pour $key$ ($A[j]$).
\end{itemize}
\bigskip
\item En sortie de boucle:
\begin{itemize}
\item A la sortie de boucle, $j=A.length+1$. L'invariant implique que $A[1\twodots A.length]$ est ordonné.
\end{itemize}
\end{itemize}
\end{frame}

\begin{frame}{Complexité de $\proc{Insertion-Sort}$}
\begin{center}
\fcolorbox{white}{Lightgray}{
\begin{codebox}
\Procname{$\proc{Insertion-Sort}(A)$}
\li \For $j \gets 2$ \To $\attrib{A}{length}$
\li     \Do
$\id{key} \gets A[j]$
\li $i \gets j-1$
\li \While $i > 0$ and $A[i] > \id{key}$
\li   \Do
        $A[i+1] \gets A[i]$
\li        $i \gets i-1$
    \End
\li $A[i+1] \gets \id{key}$
\End
\end{codebox}
}
\end{center}

\bigskip

\begin{itemize}
\item Nombre de comparaisons $T(n)$ pour trier un tableau de taille $n$?
\item Dans le pire des cas:
\begin{itemize}
\item La boucle $\For$ est exécutée $n-1$ fois ($n=A.length$).
\item La boucle $\While$ est exécutée $j-1$ fois
\end{itemize}
\end{itemize}

\end{frame}

\begin{frame}{Complexité de $\proc{Insertion-Sort}$}

\begin{itemize}
\item Le nombre de comparaisons est borné par:
$$T(n)\leq \sum_{j=2}^n (j-1)$$
\item Puisque $\sum_{i=1}^n i=i(i+1)/2$, on a:
$$T(n)\leq\frac{n(n-1)}{2}$$
\item Finalement, $T(n)=O(n^2)$
\end{itemize}

\bigskip

(borne inférieure ?)

\end{frame}

%% \begin{frame}{pourquoi les deux ?}
%% coucou
%% \end{frame}

%% mettre à dans le chapitre sur les structures élémentaires (ou bien lors de l'introductuion du heap, ou bien à la fin de cette section

%% dire qu'on va reprendre le problème de l'insertion sort et du merge
%% sort pour illustrer les concepts suivants:
% - amélioration des performances
% - récurrence
% - divide and conquer
% - complexité en nb d'opération
% -

%% \begin{frame}{type de données abstraits (tda)}

%% un type de données abstrait (tda) représente l'interface d'une structure de données.

%% un tda spécifie précisément :
%% \begin{itemize}
%% \item la nature et les propriétés des données gérées;
%% \item les modalités d utilisation des opérations pouvant être effectuées.
%% en particulier, un tda décrit généralement :
%% \begin{itemize}
%% \item de quelle(s) façon(s) un nouvel exemplaire de la structure de
%% données peut être créé (instanciation de la structure);
%% \item pour chaque opération: quels en sont les param`etres, les effets sur les données gérées, les valeurs de retour éventuelles, et leur comportement en cas d erreur.
%% \end{itemize}
%% \end{itemize}

%% en général, un tda admet différentes implémentations (plusieurs
%% représentations possibles des données, plusieurs algorithmes pour les
%% opérations).

%% \end{frame}

\begin{frame}{Structures de données}

\begin{itemize}
\item Méthode pour stoquer et organiser les données pour en faciliter
  l'accés et la modification
\item Une structure de données regroupe:
\begin{itemize}
\item un certain nombre de données à gérer, et
\item un ensemble d'opérations pouvant être appliquées à ces données
\end{itemize}
\item Dans la plupart des cas, il existe
\begin{itemize}
\item plusieurs manières de représenter les données et
\item différents algorithmes de manipulation.
\end{itemize}
\item On distingue généralement l'\alert{interface} des structures de
  leur \alert{implémentation}.
\end{itemize}

\end{frame}

\begin{frame}{Types de données abstraits}

\begin{itemize}
\item Un type de données abstrait (TDA) représente l'interface d'une structure de données.
\item Un TDA spécifie précisément:
\begin{itemize}
\item la nature et les propriétés des données
\item les modalités d'utilisation des opérations pouvant être effectuées
\end{itemize}
\item En général, un TDA admet différentes implémentations (plusieurs représentations possibles des données, plusieurs algorithmes pour les opérations).
\end{itemize}

\end{frame}

\begin{frame}{Exemple: file à priorités}

\begin{itemize}
\item Données gérées: des objets avec comme attributs:
\begin{itemize}
\item une clé, munie d'un opérateur de comparaison selon un ordre total
\item une valeur quelconque
\end{itemize}
\medskip
\item Opérations:
\begin{itemize}
\item Création d'une file vide
\item $\proc{Insert}(S,x)$: insère l'élément $x$ dans la file $S$.
%\item $\proc{Maximum}(S)$: renvoie l'élément de $S$ avec la clé la plus grande.
\item $\proc{Extract-Max}(S)$: retire et renvoie l'élément de $S$ avec
  la clé la plus grande.
\end{itemize}
\medskip
\item Il existe de nombreuses façons d'implémenter ce TDA:
\begin{itemize}
\item Tableau non trié;
\item Liste triée;
\item Structure de tas;
\item $\ldots$
\end{itemize}
Chacune mène à des complexités différentes des opérations $\proc{Insert}$ et $\proc{Extract-Max}$
\end{itemize}

\end{frame}

\begin{frame}{Exemples de problèmes algorithmiques réels}
% voir slides de sedgewick, intro-pas-mal.ppt

Dans les applications pratiques, algorithmes et structures de données sont indisociables.
\begin{itemize}
\item google: accéder: très friand en structure de données. index inversé pour arriver à une page
\item routage: trouver le plus court chemin entre deux noeuds dans un réseau
\item bioinformatique: un des problèmes cruciaux: trouver une sous-séquence dans une masse incroyable de séquences: problèmes de programmation dynamique sur des strings
\item Machine learning: kNN 
\end{itemize}

\end{frame}

\section{Introduction à la récurrence}

% dans cette section, on va parler de fibonacci et proposer une version récurrente
% on va parler de la tail-récurrence
% commencer par la factorielle
% décrire ensuite le merge sort
% ne pas détailler le coût associé au merge (on le fera dans la chapitre suivant)
% juste conclure sur une comparaison entre insertion sort et merge sort
% (pour montrer qu'il y a une énorme différence entre les deux)

% (bouquin de Sedgewick)

\begin{frame}{Algorithmes récursifs}

Une procédure est {\bf récursive} si elle s'invoque elle-même
directement ou indirectement.

\bigskip

Motivation: Simplicité d'expression de certains algorithmes

\bigskip

Exemple: Fonction factorielle:

\[n!=\left\{\begin{array}{ll}
1 &\mbox{si }n=0\\
n \cdot (n-1)! &\mbox{si }n>0$$
\end{array}
\right.
\]

\begin{center}
\fcolorbox{white}{Lightgray}{
\begin{codebox}
\Procname{$\proc{Factorial}(n)$}
\li \If $n\leq 1$
\li \Then \Return 1 \End
\li \Return $n \cdot \proc{Factorial}(n-1)$
\end{codebox}
}
\end{center}

\end{frame}

\begin{frame}{Algorithmes récursifs}

\begin{center}
\fcolorbox{white}{Lightgray}{
\begin{codebox}
\Procname{$\proc{Factorial}(n)$}
\li \If $n\leq 1$
\li \Then \Return 1 \End
\li \Return $n \cdot \proc{Factorial}(n-1)$
\end{codebox}
}
\end{center}

\bigskip

Règles pour développer une solution récursive:
\bigskip
\begin{itemize}
\item On doit définir un cas de base ($n\leq1$)
\item On doit diminuer la ``taille'' du problème à chaque étape ($n\rightarrow n-1$)
\item Quand les appels récursifs se partagent la même structure de données, les sous-problèmes ne doivent pas se superposer (pour éviter les effets de bord)
\end{itemize}
\end{frame}

\begin{frame}{Exemple de récursion multiple}

Calcul du $k$ième nombre de Fibonacci:
\begin{eqnarray*}
F_0&=&0\\
F_1&=&1\\
\forall i\geq 2: F_i& = &F_{i-2}+F_{i-1}
\end{eqnarray*}

Algorithme:
\begin{center}
\fcolorbox{white}{Lightgray}{
\begin{codebox}
\Procname{$\proc{Fibonacci}(n)$}
\li \If $n \leq 1$
\li \Then \Return n \End
\li \Return $\proc{Fibonacci}(n-2)+\proc{Fibonacci}(n-1)$
\end{codebox}
}
\end{center}
% Prendre le début du cours de Carnagazi

\end{frame}

\begin{frame}{Exemple de récursion multiple}

\begin{center}
\fcolorbox{white}{Lightgray}{
\begin{codebox}
\Procname{$\proc{Fibonacci}(n)$}
\li \If $n \leq 1$
\li \Then \Return n \End
\li \Return $\proc{Fibonacci}(n-2)+\proc{Fibonacci}(n-1)$
\end{codebox}
}
\end{center}
% Prendre le début du cours de Carnagazi

\bigskip

\begin{enumerate}
\item L'algorithme est correct?
\item Quelle est sa vitesse d'exécution?
\item Y-a-t'il moyen de faire mieux?
\end{enumerate}

\end{frame}

\begin{frame}{Exemple de récursion multiple}

\begin{center}
\fcolorbox{white}{Lightgray}{
\begin{codebox}
\Procname{$\proc{Fibonacci}(n)$}
\li \If $n \leq 1$
\li \Then \Return n \End
\li \Return $\proc{Fibonacci}(n-2)+\proc{Fibonacci}(n-1)$
\end{codebox}
}
\end{center}
% Prendre le début du cours de Carnagazi

\bigskip

\begin{enumerate}
\item L'algorithme est correct?
\begin{itemize}
\item Clairement, l'algorithme est correct.
\item En général, la correction d'un algorithme récursif se démontre par induction.
\end{itemize}
\item Quelle est sa vitesse d'exécution?
\item Y-a-t'il moyen de faire mieux?
\end{enumerate}

\note{faire la démonstration au tableau:
Pour n<=1, l'algo renvoie n, ce qui est correct
Si l'algorithme est correct pour tout n<n', on doit montrer qu'il est correct pour n+1. C'est évident.
}
\end{frame}

\begin{frame}{Vitesse d'exécution}
\begin{itemize}
\item Nombre d'opérations pour calculer $\proc{Fibonacci}(n)$ en fonction de $n$
\item Empiriquement:

\begin{center}
\includegraphics[width=7.5cm]{Figures/cpu-fibonacci.pdf}\\
~\hfill\scriptsize(Carzaniga)
\end{center}

\item Toutes les implémentations atteignent leur limite, plus ou moins loin
\end{itemize}
\end{frame}

\begin{frame}{Trace d'exécution}
\centerline{\includegraphics[width=8cm]{Figures/trace-fibonacci.pdf}}
~\hfill\scriptsize(Boigelot)
\end{frame}

\begin{frame}{Complexité}
\begin{center}
\fcolorbox{white}{Lightgray}{
\begin{codebox}
\Procname{$\proc{Fibonacci}(n)$}
\li \If $n \leq 1$
\li \Then \Return n \End
\li \Return $\proc{Fibonacci}(n-2)+\proc{Fibonacci}(n-1)$
\end{codebox}
}
\end{center}

\bigskip

\begin{itemize}
\item Soit $T(n)$ le nombre d'opérations de base pour calculer $\proc{Fibonacci}(n)$:
\begin{eqnarray*}
T(0) & = & 2, T(1)=2\\
T(n) & = & T(n-1)+T(n-2)+2\\
\end{eqnarray*}
\item On a donc $T(n)\geq F_n$ (= le $n$ème nombre de Fibonacci).
\end{itemize}

\end{frame}

\begin{frame}{Complexité}

\begin{itemize}
\item Comment croît $F_n$ avec $n$ ?
$$T(n)\geq F_n=F_{n-1}+F_{n-2}$$
Puisque $F_n\geq F_{n-1}\geq F_{n-3}\geq\ldots$
$$F_n\geq 2 F_{n-2}\geq 2(2 F_{n-4})\geq 2(2(2 F_{n-6})) \geq 2^{\frac{n}{2}}$$
Et donc
$$T(n)\geq (\sqrt{2})^n \approx (1.4)^n$$
\item $T(n)$ croît \alert{exponentiellement} avec $n$

\bigskip

\item Peut-on faire mieux ?
\end{itemize}
\note{Dire que le problème vient du fait qu'on calcule plusieurs fois les mêmes choses

pprev va stoquer la valeur de $F_{n-2}$ et prev va stoquer la valeur de $F_{n-1}$.
}
\end{frame}

\begin{frame}{Solution itérative}

\begin{center}
\fcolorbox{white}{Lightgray}{
\begin{codebox}
\Procname{$\proc{Fibonacci-Iter}(n)$}
\li \If $n \leq 1$
\li \Then \Return n \End
\li \Else $pprev\gets 0$
\li \Then $prev\gets 1$
\li \For $i\gets 2 \To n$
\li \Do $f\gets prev+pprev$
\li $pprev\gets prev$
\li $prev\gets f$\End
\li \Return f \End
\end{codebox}
}
\end{center}

%Complexité: $O(n)$

\end{frame}

\begin{frame}{Vitesse d'exécution}
Complexité: $O(n)$

\bigskip

\centerline{\includegraphics[width=8cm]{Figures/cpu-fibonacci-iter.pdf}}

\end{frame}

\begin{frame}{Tri par fusion}

Idée d'un tri basé sur la récursion:
\begin{itemize}
\item on sépare le tableau en deux sous-tableaux de la même taille
\item on trie (récursivement) chacun des sous-tableaux
\item on fusionne les deux sous-tableaux triés en maintenant l'ordre
\end{itemize}
Le cas de base correspond à un tableau d'un seul élément.

\bigskip

\begin{center}
\fcolorbox{white}{Lightgray}{%
    \begin{codebox}
      \Procname{$\proc{merge-sort}(A,p,r)$}
      \li \If $\id{p}<\id{r}$
      \li \Then $q \gets \lfloor \frac{p+r}{2} \rfloor$
      \li       $\proc{merge-sort}(A,p,q)$
      \li       $\proc{merge-sort}(A,q+1,r)$
      \li       $\proc{merge}(A,p,q,r)$ \End
    \end{codebox}}
\end{center}

\centerline{Appel initial: $\proc{merge-sort}(A,1,A.length)$}

\bigskip

Exemple d'application du principe général de ``\alert{diviser pour régner}''

\note{ce principe reviendra souvent dans ce cours}

\end{frame}

\begin{frame}{Tri par fusion: illustration}

\centerline{\includegraphics[width=6cm]{Figures/CLRS/mergesort-power2.pdf}}

\note{\centerline{\includegraphics[width=6cm]{Figures/CLRS/mergesort-notpower2.pdf}}}

\end{frame}

\begin{frame}{Fonction $\proc{merge}$}

$\proc{Merge}(A,p,q,r)$:
\begin{itemize}
\item {\bf Entrée:} tableau $A$ et indice $p$, $q$, $r$ tels que:
\begin{itemize}
\item $p\leq q<r$ (pas de tableaux vides)
\item Les sous-tableaux $A[p\twodots q]$ et $A[q+1\twodots r]$ sont ordonnés
\end{itemize}
\item {\bf Sortie:} Les deux sous-tableaux sont fusionnés en seul sous-tableau ordonné dans $A[p\twodots r]$
\end{itemize}

\bigskip

Idée:
\begin{itemize}
\item Utiliser un pointeur vers le début de chacune des listes;
\item Déterminer le plus petit des deux éléments pointés;
\item Déplacer cet élément vers le tableau fusionné;
\item Avancer le pointeur correspondant
\end{itemize}

\note{Faire l'analogie avec deux jeux de cartes triés qu'on veut rassembler. prendre un jeu de carte.}

\end{frame}

\begin{frame}{Fusion: algorithme}

\begin{center}\small
   \fcolorbox{white}{Lightgray}{
    \begin{codebox}
      \Procname{$\proc{Merge}(A,p,q,r)$}
      \li $n_1 = q-p+1$; $n2 = r-q$
      \li Soit $L[1..n_1+1]$ and $R[1..n_2+1]$ deux nouveaux tableaux
      \li \For $i=1$ \To $n_1$
      \li \Do $L[i]=A[p+i-1]$ \End
      \li \For $j=1$ \To $n_2$
      \li \Do $R[j]=A[q+j]$ \End
      \li $L[n_1+1]=\infty$; $R[n_2+1]=\infty$ \RComment \textcolor{red}{Sentinels}
      \li i=1;j=1
      \li \For $k\gets p$ \To $r$
      \li \Do \If $L[i]\leq R[j]$
      \li \Then $A[k]\gets L[i]$
      \li       $i\gets i+1$
      \li \Else $A[k]=R[j]$
      \li       $j\gets j+1$
          \End
    \end{codebox}
}
\end{center}

\end{frame}

\begin{frame}{Fusion: illustration}

\centerline{\includegraphics[width=8cm]{Figures/mergeillustration.pdf}}

\bigskip

Complexité: $O(n)$ (où $n=r-p+1$)

\end{frame}

\begin{frame}{Vitesse d'exécution}

Complexité de $\proc{merge-sort}$: $O(n\log n)$ (voir partie 2)

\bigskip

\centerline{\includegraphics[width=8cm]{Figures/compare-sort.pdf}}

\note{Dans un prochain cours, on verra que le tri par fusion est
  optimal en terme de temps de calcul (N log N)}

\end{frame}

\begin{frame}{Remarques}

\begin{itemize}
\item La fonction $\proc{merge}$ nécessite d'allouer deux tableaux $L$
  et $R$ (dont la taille est $O(n)$). Exercice (difficile): écrire une
  fonction $\proc{merge}$ qui ne nécessite pas d'allocation
  supplémentaire.
\item On pourrait réécrire $\proc{merge-sort}$ de manière itérative (au prix de la simplicité)
\item Version récursive du tri par insertion:
\begin{center}
\fcolorbox{white}{Lightgray}{
\begin{codebox}
\Procname{$\proc{Insertion-Sort-rec}(A,n)$}
\li \If $n>1$
\li \Then $\proc{Insertion-Sort-rec}(A,n-1)$
\li       $\proc{Merge}(A,1,n-1,n)$ \End
\end{codebox}
}
\end{center}
\end{itemize}
\note{Attirer l'attention sur le fait qu'il faut aussi tenir compte de
  la taille mémoire requise par un algorithme.

\bigskip

Dire que l'insertion sort est une version particulière du merge sort où au lieu de diviser en deux, on divise en 1-n-1 et n. C'est donc nettement moins efficace. On obtiendrait l'insertion-sort à partir du mergesort en changeant le q=p+r/2 en q=r-1.
}
\end{frame}

\begin{frame}{Note sur l'implémentation de la récursivité}
\begin{itemize}
\item Trace d'exécution de la factorielle
\centerline{\includegraphics[width=2cm]{Figures/trace-factorielle.pdf}}
\item Chaque appel récursif nécessite de mémoriser le \alert{contexte d'invocation}
\item L'espace mémoire utilisé est donc $O(n)$ ($n$ appels récursifs)
\end{itemize}
\note{!! Il est important de garder ça en mémoire: les algos récursifs ont un coût en terme d'espace mémoire}
\end{frame}

\begin{frame}{Récursivité terminale}

\begin{itemize}
\item Définition: Une procédure est \alert{récursive terminale} (tail récursive) si elle n'effectue plus aucune opération après s'être invoquée récursivement.

\bigskip

\item Avantages:
\begin{itemize}
\item Le contexte d'invocation ne doit pas être mémorisé et donc l'espace mémoire nécessaire est réduit
\item Les procédures récursives terminales peuvent facilement être converties en procédures itératives
\end{itemize}
\end{itemize}

\end{frame}

\begin{frame}{Version récursive terminale de la factorielle}

\begin{center}
\fcolorbox{white}{Lightgray}{
\begin{codebox}
\Procname{$\proc{Factorial2}(n)$}
\li \Return $\proc{Factorial2-rec}(n,2,1)$
\end{codebox}}
\end{center}

\begin{center}
\fcolorbox{white}{Lightgray}{
\begin{codebox}
\Procname{$\proc{Factorial2-rec}(n,i,f)$}
\li \If $i>n$
\li \Then \Return $f$\End
\li \Return $\proc{Factorial2-rec}(n,i+1,f\cdot i)$
\end{codebox}}
\end{center}

\bigskip

Espace mémoire utilisé: $O(1)$ (si la récursion terminale est implémentée efficacement)

\end{frame}

\begin{frame}{Ce qu'on a vu}

\begin{itemize}
\item Définitions générales: algorithmes, structures de données, structures de données abstraites...
\item Analyse d'un algorithme itératif ($\proc{Insertion-Sort}$)
\item Notions de récursivité
\item Analyse d'un algorithme récursif ($\proc{Fibonacci}$)
\item Tri par fusion ($\proc{MergeSort}$)
\end{itemize}

\end{frame}

\part{Outils d'analyse}

\begin{frame}{Plan}

\tableofcontents

\end{frame}

\section{Correction d'un algorithme}

\subsection{Algorithme itératif}

\begin{frame}{Analyse d'algorithmes}

Questions à se poser lors de la définition d'un algorithme:
\begin{itemize}
\item Mon algorithme est-il correct ?
\item Mon algorithme est-il efficace ? %en termes d'utilisation des
%  resources, temps CPU et/ou espace mémoire ?
\end{itemize}

\bigskip

Autre question importante seulement marginalement abordée dans ce cours:
\begin{itemize}
\item Modularité,fonctionnalité, robustesse, facilité d'utilisation, temps
  de programmation, simplicité, extensibilité, fiabilité,
  existence d'une solution algorithmique
\end{itemize}

\end{frame}

\begin{frame}{Correction d'un algorithme}%, complétude, terminaison}

\begin{itemize}
\item La correction d'un algorithme s'étudie par rapport à un problème donné
\item Un problème est une collection d'instances de ce problème.
\begin{itemize}
\item Exemple de problème: trier un tableau
\item Exemple d'instance de ce problème: trier le tableau $[8,4,15,3]$
\end{itemize}
\item Un algorithme est correct pour une instance d'un problème s'il
  produit une solution correcte pour cette intance
\item Un algorithme est correct pour un problème s'il est correct pour
  toutes ses instances
\item On s'intéressera ici à la correction d'un algorithme pour un
  problème, par pour seulement certaines de ses instances
\end{itemize}

\note{}

\end{frame}

\begin{frame}{Comment vérifier la correction?}
\begin{itemize}
\item Première solution: en testant concrètement l'algorithme:
\begin{itemize}
\item Suppose d'implémenter l'algorithme dans un langage (programme)
  et de le faire tourner
\item Supppose qu'on peut déterminer les instances du problème à vérifier
\item Il est très difficile de prouver empiriquement qu'on n'a pas de bug %On peut prouver qu'il y a un bug, pas qu'il n'y en a pas.
\end{itemize}
\item Deuxième solution: en dérivant une preuve mathématique formelle:
\begin{itemize}
\item Pas besoin d'implémenter et de tester toutes les instances du problème
\item Sujet à des ``bugs'' également
\end{itemize}
\item En pratique, on combinera les deux

\bigskip

\item Preuves de correction:
\begin{itemize}
\item Précondition, post-conditions
\item Algorithmes itératifs: assertions, invariants de boucle et induction
\item Algorithmes récursifs: preuve par induction directement
\end{itemize}
\end{itemize}
\end{frame}

\begin{frame}{Correction: cas itératif}

Pour prouver qu'un algorithme itératif est correct:
\begin{itemize}
\item On analyse chaque boucle de l'algorithme séparément, en
  démarrant avec la boucle la plus interne s'il y a plusieurs
  boucles imbriquées.
\item Pour chaque boucle, on met en évidence un invariant de boucle
\begin{itemize}
\item Ensemble de propriétés qui relient les variables du programme
\item Ces propriétés doivent être vraies avant, pendant et après la boucle
\end{itemize}
\item On prouve que l'invariant est vérifié.
\item On utilise l'invariant pour prouver que l'algorithme se termine.
\item On utilise l'invariant pour prouver que l'algorithme calcule le résultat correct.
\end{itemize}

\end{frame}

\begin{frame}{Assertion}

\begin{itemize}
\item Relation entre les variables qui est vraies à un moment donné dans l'exécution
\item Assertions particulières:
\begin{itemize}
\item Pre-condition $P$: conditions que doivent remplir les entrées de l'algorithme
\item Post-condition $Q$: conditions qui expriment que le résultat de l'algorithme est correcte
%\item Invariant: condition que doivent remplir
\end{itemize}
\item $P$ et $Q$ définissent les instances et solutions valides du problème
\item Un algorithme est correct si $P$ \{code\} $Q$ est vrai.
\item Pour vérifier la correction d'une boucle, on introduit la notion d'invariant.%On suppose $P$ vérifié, on montre que $I$ est un invariant valide et on en déduit que $R$ est vrai également.
\end{itemize}

\end{frame}

\begin{frame}{Invariant}
\begin{itemize}
\item \alert{Invariant:} Une assertion qui définit ce qui est vrai avant chaque itération de la boucle
\item \alert{Initialisation:} Prouver que l'invariant est vrai avant la
  première itération (sous l'hypothèse que la pré-condition $P$ est
  vérifiée)
\item \alert{Maintenance:} Prouver que si l'invariant est vrai avant la
  $i$-ième itération, il l'est également après celle-ci, et par
  conséquent aussi avant la $i+1$-ième itération
\item \alert{Terminaison:} Prouver que si l'invariant est vrai après la
  dernière itération, la post-condition $Q$ est vérifiée
\end{itemize}

\centerline{\includegraphics[width=8cm]{Figures/02-invariant.pdf}}

\end{frame}

\begin{frame}{Mise en \oe uvre}
\begin{itemize}
\item Trouver un invariant de boucle une fois l'algorithme mis au point peut être assez complexe
\item Idéalement, l'invariant devrait être la propriété clé qui définit l'algorithme
\item Dans la suite du cours, on ne fournira un invariant que dans quelques cas
\end{itemize}
\end{frame}
\begin{frame}{Exemples: exponentielle}
\end{frame}

\begin{frame}{Exemples: fibonacci itératif}
\end{frame}

\begin{frame}{Exemples: insertion sort}
\end{frame}

\begin{frame}{Preuve de terminaison d'un algorithme itératif}

Pour chaque boucle: on définit une fonction de terminaison $t$ qui est telle que $t$ décroit strictement à chaque itération de la boucle et que sa valeur est bornée vers le bas par le gardien de la boucle.

\end{frame}

\begin{frame}{Correction: cas récursif}
\begin{itemize}
\item Preuve par induction: donner le schéma général
\end{itemize}
\end{frame}

\begin{frame}{Exemple: Fibonacci}
\end{frame}

\begin{frame}{Exemple: merge sort}
\end{frame}

\begin{frame}{Preuve de terminaison}
\begin{itemize}
\item itératif: Définir une fonction $t$ montrer que $t$ décroit lors des itérations et 
\item récursif: montrer que le problème est réduit à chaque appel récursif
\end{itemize}
\end{frame}

% Pre-condition=assertion qui est vrai avant le programme
% Post-condition=assertion qui est vrai après l'exécution du programme
% Boucle: invariante:
% Tel que invariant & non G (gardien de boucle) => Post-condition

% Terminaison: on définit une fonction naturel, on montre qu'elle
% décroit à chaque itération et que le gardien


\subsection{Algorithme récursif}

\begin{frame}{Preuve par induction}

\end{frame}

\begin{frame}{Conclusion sur la correction}
Dans la suite, on ne présentera des invariants ou des preuves par induction que lorsque ce sera nécessaire (cas non triviaux)
\end{frame}

\section{Complexité algorithmique}

\begin{frame}{Plan}

\tableofcontents

\end{frame}

\subsection{Itératif}

\begin{frame}{Efficacité}
Pourquoi est-ce important?
\begin{itemize}
\item Un algorithme trop lent (dans certaines applications, un algorithm quadratique est déjà trop lent)
\item Il faut pouvoir s'adapter à la croissance des données actuelles
\end{itemize}

\end{frame}

%% Définition mathématique des trois types de complexité

%% Soit Dn
%% l’ensemble des données de taille n. Soit I un sous ensemble de
%% Dn et soit t(I) le nombre d’opérations élémentaires pour exécuter I.

%% Complexité dans le meilleur des cas
%% meilleur(n) = min {t(I), I Dn}
%% }

%% Complexité dans le pire des cas
%% pire(n) = max {t(I), I  Dn
%% }

%% Complexité en moyenne
%% moyenne(n) = S Pr (I) t(I) où Pr(I) est la probabilité de I

%% Note sur la complexité amortie
%% En moyenne sur plusieurs instance...

\begin{frame}{RAM model}

\end{frame}

\begin{frame}{Application à l'insertion sort}
\end{frame}

\subsection{Combinaison de fonctions}

\begin{frame}{règle de la somme et du produit}
Voir slides ``intro-pas-mal.ppt''
\end{frame}

\subsection{Algorithmes récursifs}

\begin{frame}{Algorithmes récursifs}

\begin{itemize}
\item Fibonacci
\item Merge sort
\end{itemize}

\end{frame}

\begin{frame}
Complexité en espace:
\begin{itemize}
\item ...
\end{itemize}
\end{frame}

\begin{frame}{Pour en savoir plus (ou ce qu'on n'a pas vu)}

\begin{itemize}
\item Méthode systématique pour l'analyse de la complexité d'algorithmes récursifs (l'an prochain)
\item Etude au cas moyen
\end{itemize}

\end{frame}

%A la fin de chaque cours, je vais dire toutes les simplifications
%qu'on a prise et encourager les étudiants à se poser certaines
%questions. Exemple:



\part{Algorithmes de tri}

%% % http://www.sorting-algorithms.com/

\begin{frame}{Plan}

\tableofcontents

\end{frame}

\section{Algorithmes de tri}

\begin{frame}{Tri}

\begin{itemize}
\item Un des problèmes algorithmiques les plus fondamentaux.
\item En général, on veut trier des enregistrements avec une clé et des
  données attachées.
\medskip
\begin{center}
\begin{tabular}{ccccccc}
Record$_1$ & & Record$_2$ & & Record$_3$ & & Record$_n$\\
\cline{1-1} \cline{3-3} \cline{5-5} \cline{7-7}
\multicolumn{1}{|c|}{Key$_1$} & & \multicolumn{1}{|c|}{Key$_2$} & & \multicolumn{1}{|c|}{Key$_3$} & \ldots & \multicolumn{1}{|c|}{Key$_n$}\\
 \cline{1-1} \cline{3-3} \cline{5-5} \cline{7-7}
\multicolumn{1}{|c|}{Data$_1$} & & \multicolumn{1}{|c|}{Data$_2$} & & \multicolumn{1}{|c|}{Data$_3$} & & \multicolumn{1}{|c|}{Data$_n$}\\
 \cline{1-1} \cline{3-3} \cline{5-5} \cline{7-7}
\end{tabular}
\end{center}
\medskip
\item Ici, on va ignorer ces données satellites et se focaliser sur
  les algorithmes de tri

\bigskip

\item Le problème de tri:
\begin{itemize}
\item Entrée: une séquence de $n$ nombres $\langle a_1,a_2,\ldots,a_n\rangle$
\item Sortie: une permutation de la séquence de départ $\langle a_1',a_2',\ldots,a_n'\rangle$ telle que $a_1'\leq a_2'\leq\ldots\leq a_n'$
\end{itemize}
\end{itemize}

\end{frame}

\begin{frame}{Applications}

Applications innombrables:
\begin{itemize}
\item Tri des mails selon leur ancienneté
\item Tri des résultats de requête dans un moteur de recherche
\item Tri des facettes des objets pour l'affichage dans les jeux 3D
\item Gestion des opérations bancaires
\item ...
\end{itemize}

\bigskip

Le tri sert aussi de brique de base pour d'autres algorithmes:
\begin{itemize}
\item Recherche binaire dans un tableau trié
\item Recherche des éléments dupliqués dans une liste
\item Recherche du $k$ème élément le plus grand dans une liste
\item ...
\end{itemize}

\bigskip

Des études montrent qu'environ 25\% du temps CPU des ordinateurs est utilisé pour trier
\end{frame}

\begin{frame}{Différents types de tri}

\begin{itemize}
\item {\bf Tri interne:} tri en mémoire centrale. {\bf Tris externes:} données sur un disque externe.
\item {\bf Tri de tableau:} tri qui trie un tableau. Extensible à toutes structures de données offrant un accès en temps (quasi) constant à ses éléments.
\item {\bf Tri générique:} peut trier n'importe quel type d'objets pour autant qu'on puisse comparer ces objets.
\item {\bf Tri comparatif:} basé sur la comparaison entre les éléments (clés)

\end{itemize}

\end{frame}

\begin{frame}{Différents types de tri}
\begin{itemize}
\item {\bf Tri itératif:} basé sur un ou plusieurs parcours itératifs du tableau
\item {\bf Tri récursif:} basé sur une procédure récursive
\item {\bf Tri en place:} modifie directement la structure qu'il est en train de trier. Ne nécessite qu'une quantité très limitée de mémoire supplémentaire.
\item {\bf Tri  stable:} conserve l'ordre relatif des éléments égaux (au sens de la méthode de comparaison).
\end{itemize}

\note{Stable: expliquer ce que ca veut dire en illustrant avec les colonnes dans excel: tri sur deux colonnes: un tri stable va maintenir l'ordre sur la deuxième colonne

\bigskip

Stable si $$(2,1),(1,2),(2,3),(2,4),(6,5)$$
est trié en $$(1,2),(2,1),(2,3),(2,4),(6,5)$$
}
\end{frame}

\begin{frame}{Jusqu'ici}

  \begin{center}
    \def\arraystretch{1.5}
  \begin{tabular}{@{}lccc@{}c@{}}
    \emph{Algorithme}&\multicolumn{3}{c}{\emph{Complexité}}&\emph{En place?}\\
    & \emph{\small Pire} & \emph{\small Moyenne} & \emph{Meilleure} & \\
    \hline\hline
    \proc{Insertion-Sort}&$\Theta(n^2)$&$\Theta(n^2)$&$\Theta(n)$&oui\\
    \hline
    \proc{Selection-Sort}&$\Theta(n^2)$&$\Theta(n^2)$&$\Theta(n^2)$&oui\\
    \hline
    \proc{Bubble-Sort}&$\Theta(n^2)$&$\Theta(n^2)$&$\Theta(n)$&oui\\
    \hline
    \proc{Merge-Sort}&$\Theta(n\log{n})$&$\Theta(n\log{n})$&$\Theta(n\log{n})$&non\\
    \hline\hline
    \hspace{1em}\alert{??}& &\alert{$\Theta(n\log{n})$}& &\alert{oui}\\
    \hline
    \hspace{1em}\alert{??}&\alert{$\Theta(n\log{n})$}& & &\alert{oui}\\
    \hline\hline
  \end{tabular}
  \end{center}

\note{Expliquer le principe des algorithmes

\bigskip

Demander lesquels sont stables: selection: non, insertion: oui, merge: oui, bubble: oui.
}
\end{frame}

\section{Tri rapide}

\begin{frame}{Tri rapide}
\begin{itemize}
\item {\em Quicksort} en anglais
\item Inventé par Hoare en 1960
\item Dans le top 10 des algorithmes du 20-ième siècle (SIAM)
\item L'exemple le plus célèbre de la technique du ``diviser pour régner''
\item Tri en place, comme tri par insertion, et contrairement au tri par fusion
\item Complexité: $\Theta(n^2)$ dans le pire des cas, $\theta(n\log n)$ en moyenne
\end{itemize}
\end{frame}

\begin{frame}{\proc{QuickSort}: principe}
Pour trier un sous-tableau $A[p\twodots r]$:
\begin{itemize}
\item Partitionner $A[p\twodots r]$ en deux sous-tableaux: $A[p..q-1]$ et $A[q+1\twodots r]$ tels que tout élément de $A[p\twodots q-1]$ est $\leq A[q]$ et $A[q]<$ à tout élément de $A[q+1\twodots r]$.\\~\hfill\alert{\emph{(diviser)}}

\item Appeler récursivement l'algorithme pour trier $A[p\twodots q-1]$ et $A[q+1\twodots r]$\\~\hfill\alert{\emph{(régner)}}
\end{itemize}

\bigskip

Remarques:
\begin{itemize}
\item $A[q]$ est appelé le ``\alert{pivot}''
\item Par rapport au tri par fusion, il n'y a pas d'opération de combinaison
\end{itemize}

\end{frame}

\begin{frame}{\proc{QuickSort}: principe}

\centerline{\includegraphics[width=8cm]{Figures/03-quicksort-illustration3.pdf}}

% Gaetano -> à modifier pour mettre les q,r, etc.

\end{frame}

\begin{frame}\frametitle{\proc{QuickSort} Algorithm}

  \begin{center}
%    \begin{small}

    %% \fcolorbox{white}{Lightgray}{%
    %% \begin{codebox}
    %%   \Procname{$\proc{Partition}(A,\id{begin},\id{end})$}
    %%   \li $q\gets \id{begin}$
    %%   \li $v\gets A[\id{end}]$
    %%   \li \For $i\gets \id{begin}+1$ \To $\id{end}-1$
    %%   \li \Do \If $A[i]\le v$
    %%   \li     \Then $\id{swap}(A[i], A[q])$
    %%   \li           $q\gets q+1$
    %%           \End
    %%       \End
    %%   \li $\id{swap}(A[\id{end}], A[q])$
    %%   \li \Return $q$
    %% \end{codebox}}

    %% \bigskip
    \fcolorbox{white}{Lightgray}{
    \begin{codebox}
      \Procname{$\proc{QuickSort}(A,\id{p},\id{r})$}
      \li \If $\id{p}<\id{r}$
      \li \Then $q\gets\proc{Partition}(A,\id{p},\id{r})$
      \li $\proc{QuickSort}(A,\id{p},q-1)$
      \li $\proc{QuickSort}(A,q+1,\id{r})$
      \End
    \end{codebox}}
%    \end{small}

\bigskip

Appel initial: $\proc{Quicksort}(A,1,\attrib{A}{length})$
  \end{center}
\end{frame}

\begin{frame}{Partition: principe}

\centerline{\includegraphics[width=6cm]{Figures/03-partition-illustration.pdf}}

\bigskip

\begin{itemize}
\item On sélectionne le dernier élément $A[r]$ comme le {\bf pivot}
\item On initialise un indice $i$ à $p-1$
\item On parcourt le tableau de gauche à droite avec un indice $j=p \To r-1$
\item Si $A[j]\le A[r]$, on incrémente $i$ et on échange $A[j]$ et $A[i]$
\item En sortie de boucle, on échange $A[i+1]$ et $A[r]$ et on renvoie $i+1$
\end{itemize}
\end{frame}

\begin{frame}{Partition: illustration}

\begin{columns}
\begin{column}{4cm}
\centerline{\includegraphics[width=3.5cm]{Figures/03-partition2.pdf}}
\end{column}
\begin{column}{6cm}\small
$A[r]$ est le pivot\\
$A[p\twodots i]$ contient des éléments $\leq$ au pivot\\
$A[i+1\twodots j-1]$ contient des éléments $>$ que le pivot\\
$A[j\twodots r-1]$ est la partie du tableau non encore examinée
\end{column}
\end{columns}

\end{frame}

\begin{frame}{Partition: pseudo-code}

\begin{center}
    \fcolorbox{white}{Lightgray}{%
    \begin{codebox}
      \Procname{$\proc{Partition}(A,\id{p},\id{r})$}
      \li $x\gets A[\id{r}]$
      \li $i\gets p-1$
      \li \For $j\gets \id{p}$ \To $\id{r}-1$
      \li \Do \If $A[j]\le x$
      \li     \Then $i\gets i+1$
      \li           $\id{swap}(A[i], A[j])$
              \End
          \End
      \li $\id{swap}(A[\id{i}+1], A[r])$
      \li \Return $i+1$
    \end{codebox}}
\end{center}

\end{frame}

\begin{frame}{Partition: correction}

\begin{center}\small
    \fcolorbox{white}{Lightgray}{%
    \begin{codebox}
      \Procname{$\proc{Partition}(A,\id{p},\id{r})$}
      \li $x\gets A[\id{r}]$
      \li $i\gets p-1$
      \li \For $j\gets \id{p}$ \To $\id{r}-1$
      \li \Do \If $A[j]\le x$
      \li     \Then $i\gets i+1$
      \li           $\id{swap}(A[i], A[j])$
              \End
          \End
      \li $\id{swap}(A[\id{i}+1], A[r])$
      \li \Return $i+1$
    \end{codebox}}
\end{center}

{\bf Pré-condition:} $\{A[p\twodots r],\mbox{ un tableau de nombres}\}$\\
{\bf Post-condition:} $\{A[p\twodots i]\le A[i+1]< A[i+2\twodots r]\}$\\
{\bf Invariant:}
\begin{enumerate}
\item Les valeurs dans $A[p\twodots i]$ sont $\leq$ au pivot
\item Les valeurs dans $A[i+1\twodots j-1]$ sont $>$ que le pivot
\item $A[r]=pivot$
\end{enumerate}

\end{frame}


\begin{frame}{Partition: correction}

{\bf Avant la boucle:} $i=p-1$ et $j=p$ $\Rightarrow$ $A[p\twodots i]$ et $A[i+1\twodots j-1]$ sont vides

\medskip

{\bf Pendant la boucle:}
\centerline{\includegraphics[width=5cm]{Figures/03-partition-illustration.pdf}}

\begin{itemize}
\item Si $A[j]>x$, on incrémente juste $j$. Donc si l'invariant était vrai avant l'exécution du corps, il reste vrai après.
\item Si $A[j]\le x$, on échange $A[j]$ et $A[i+1]$ et $i$ et $j$ sont incrémentés. L'invariant reste donc vérifié également
\end{itemize}

\medskip

{\bf Après la boucle:} 
\centerline{\includegraphics[width=5cm]{Figures/03-fininvariant.pdf}}

En sortie de boucle, l'invariant est vérifié et
on a $j=r$. Echanger $A[i+1]$ et $A[r]$ établit la post-condition.

\end{frame}

\begin{frame}{Algorithme complet}
  \begin{center}
   \begin{small}

\fcolorbox{white}{Lightgray}{%
    \begin{codebox}
      \Procname{$\proc{Partition}(A,\id{p},\id{r})$}
      \li $x\gets A[\id{r}]$
      \li $i\gets p-1$
      \li \For $j\gets \id{p}$ \To $\id{r}-1$
      \li \Do \If $A[j]\le x$
      \li     \Then $i\gets i+1$
      \li           $\id{swap}(A[i], A[j])$
              \End
          \End
      \li $\id{swap}(A[\id{i}+1], A[r])$
      \li \Return $i+1$
    \end{codebox}}

\bigskip

    \fcolorbox{white}{Lightgray}{
    \begin{codebox}
      \Procname{$\proc{QuickSort}(A,\id{p},\id{r})$}
      \li \If $\id{p}<\id{r}$
      \li \Then $q\gets\proc{Partition}(A,\id{p},\id{r})$
      \li $\proc{QuickSort}(A,\id{p},q-1)$
      \li $\proc{QuickSort}(A,q+1,\id{r})$
      \End
    \end{codebox}}
    \end{small}
  \end{center}
\note{\centerline{\includegraphics[width=5cm]{Figures/03-partition-altern.pdf}}}
\end{frame}

\begin{frame}{Illustration}

\centerline{\includegraphics[width=8cm]{Figures/03-quicksort-illustration2-2.pdf}}

\end{frame}

\begin{frame}{Complexité de $\proc{Partition}$}

\begin{center}
    \fcolorbox{white}{Lightgray}{%
    \begin{codebox}
      \Procname{$\proc{Partition}(A,\id{p},\id{r})$}
      \li $x\gets A[\id{r}]$
      \li $i\gets p-1$
      \li \For $j\gets \id{p}$ \To $\id{r}-1$
      \li \Do \If $A[j]\le x$
      \li     \Then $i\gets i+1$
      \li           $\id{swap}(A[i], A[j])$
              \End
          \End
      \li $\id{swap}(A[\id{i}+1], A[r])$
      \li \Return $i+1$
    \end{codebox}}

\end{center}

$$T(n)=\Theta(n)$$

\end{frame}

\begin{frame}{Complexité de $\proc{QuickSort}$}

\begin{center}
    \fcolorbox{white}{Lightgray}{
    \begin{codebox}
      \Procname{$\proc{QuickSort}(A,\id{p},\id{r})$}
      \li \If $\id{p}<\id{r}$
      \li \Then $q\gets\proc{Partition}(A,\id{p},\id{r})$
      \li $\proc{QuickSort}(A,\id{p},q-1)$
      \li $\proc{QuickSort}(A,q+1,\id{r})$
      \End
    \end{codebox}}
\end{center}

\begin{itemize}
\item Pire cas:\hfill{\it (quand se produit-il ?)}

\begin{itemize}
\item $q=p$ ou $q=r$ 
\item Le partitionnement transforme un problème de taille $n$ en un problème de taille $n-1$
$$T(n)=T(n-1)+\Theta(n)$$
\item Même complexité que le tri par insertion:
$$T(n)=\Theta(n^2)$$
\end{itemize}
\end{itemize}
\note{complexité$=\sum_{i=1}^n i$

\bigskip

Ca se produit quand ? (tableau déjà trié). Tri par insertion: tableau trié en ordre inverse.
}
\end{frame}

\begin{frame}{Complexité de \proc{QuickSort}}

\begin{center}
    \fcolorbox{white}{Lightgray}{
    \begin{codebox}
      \Procname{$\proc{QuickSort}(A,\id{begin},\id{end})$}
      \li \If $\id{begin}<\id{end}$
      \li \Then $q\gets\proc{Partition}(A,\id{begin},\id{end})$
      \li $\proc{QuickSort}(A,\id{begin},q-1)$
      \li $\proc{QuickSort}(A,q+1,\id{end})$
      \End
    \end{codebox}}
\end{center}

\begin{itemize}
\item Meilleur cas:
\begin{itemize}
\item $q=\lfloor n/2 \rfloor$
\item Le partitionnement transforme un problème de taille $n$ en deux
  problèmes de taille $\lceil n/2\rceil$ et $\lfloor n/2 \rfloor-1$ respectivement
$$T(n)=2 T(n/2)+\Theta(n)$$

\item Même complexité que le tri par fusion:
$$T(n)=\Theta(n\log n)$$
\end{itemize}
\end{itemize}
\end{frame}

\begin{frame}{Complexité moyenne de \proc{QuickSort}}
\begin{itemize}
\item Complexité moyenne identique à la complexité du meilleur cas
$$T(n)=\Theta(n\log n)$$
\item Intuitivement:
\begin{itemize}
\item En moyenne, on s'attend à une alternance de ``bons'' et de
  ``mauvais'' partitionnements
\item La complexité d'un mauvais partitionnement suivi d'un bon est
  identique à la complexité d'une bon partitionnement directement
  (seule la constante est modifiée).
\end{itemize}
\end{itemize}
\centerline{\includegraphics[width=10cm]{Figures/03-quicksort-casmoyen.pdf}}

\note{Peut-être dire aussi qu'un partitionnement très déséquilibré (1/10 - 9/10) garde une complexité de $n\log n$}

\end{frame}

\begin{frame}{Variantes de \proc{Quicksort}}

\begin{itemize}
\item Choix du pivot:
\begin{itemize}
\item Prendre un élément au hasard plutôt que le dernier.
\item Prendre la médiane de 3 éléments
\item Diminue drastiquement les chances d'être dans le pire cas
\end{itemize}
\end{itemize}

\begin{small}
\begin{center}
    \fcolorbox{white}{Lightgray}{%
     \begin{codebox}
       \Procname{$\proc{Randomized-Partition}(A,\id{p},\id{r})$}
       \li $i=\proc{Random}(p,r)$
       \li $\id{swap}(A[end],A[i])$
       \li \Return $\proc{Partition}(A,p,r)$
     \end{codebox}}

\bigskip

    \fcolorbox{white}{Lightgray}{%
     \begin{codebox}
       \Procname{$\proc{Median-Of-3-Partition}(A,\id{p},\id{r})$}
       \li $i=\proc{median}(A,p,\lfloor (p+r)/2\rfloor,r)$
       \li $\id{swap}(A[r],A[i])$
       \li \Return $\proc{Partition}(A,p,r)$
     \end{codebox}}
\end{center}
\end{small}

\end{frame}

\begin{frame}{Variantes de \proc{Quicksort}}

\begin{itemize}
\item Petits sous-tableaux
\begin{itemize}
\item \proc{Quicksort} est trop lourd pour des petits tableaux
\item Utiliser un tri naïf (par ex., par insertion) sur les sous-tableaux de longueur inférieure à $k$ ($k\approx 20$).
\end{itemize}
\end{itemize}

\begin{small}
\begin{center}
    \fcolorbox{white}{Lightgray}{
    \begin{codebox}
      \Procname{$\proc{QuickSort}(A,\id{p},\id{r})$}
      \li \If $r-p+1 \leq \const{CUTOFF}$
      \li \Then $\proc{InsertionSort}(A,\id{p},\id{r})$
      \li \Return \End
      %\li \If $\id{begin}<\id{end}$
      \li $q\gets\proc{Partition}(A,p,r)$
      \li $\proc{QuickSort}(A,p,q-1)$
      \li $\proc{QuickSort}(A,q+1,r)$
      \End
    \end{codebox}}

\end{center}
\end{small}

\end{frame}

\begin{frame}{Conclusion sur $\proc{QuickSort}$}

\begin{itemize}
\item Rapide en moyenne $\Theta(n\log n)$
\item Pire cas en $\Theta(n^2)$ mais très improbable avec choix du pivot bien fait
\item Bonne performance au niveau du cache
\item Tri \alert{en place} (mais utilise de la mémoire pour la trace récursive)
\item \alert{Pas stable}
\item En pratique souvent un peu plus rapide que \proc{Merge-Sort}
\item Complexité en espace $O(\log n)$ si bien implémenté (récursif terminal, en développant d'abord la partition la plus petite)
\end{itemize}

\note{
Récursif terminal sur le deuxième appel: car rien à faire une fois que c'est terminé. Itératif: on itere tant que end-start>1, on splitte entre start et end, on appelle récursivement sur la partie la plus petite (pour limiter la profondeur d'appel) et puis on modifie start et end pour qu'ils pointent sur la partie la plus grande.

Log(n) parce qu'au pire point de vue mémoire, on coupe en 2 à chaque fois et la profondeur vaut log2n
}
\end{frame}

\begin{frame}{Jusqu'ici}

  \begin{center}
    \def\arraystretch{1.5}
  \begin{tabular}{@{}lccc@{}c@{}}
    \emph{Algorithme}&\multicolumn{3}{c}{\emph{Complexité}}&\emph{En place?}\\
    & \emph{\small Pire} & \emph{\small Moyenne} & \emph{Meilleure} & \\
    \hline\hline
    \proc{Insertion-Sort}&$\Theta(n^2)$&$\Theta(n^2)$&$\Theta(n)$&oui\\
    \hline
    \proc{Selection-Sort}&$\Theta(n^2)$&$\Theta(n^2)$&$\Theta(n^2)$&oui\\
    \hline
    \proc{Bubble-Sort}&$\Theta(n^2)$&$\Theta(n^2)$&$\Theta(n)$&oui\\
    \hline
    \proc{Merge-Sort}&$\Theta(n\log{n})$&$\Theta(n\log{n})$&$\Theta(n\log{n})$&non\\
    \hline
    \proc{QuickSort} & $\Theta(n^2)$ & $\Theta(n\log{n})$ & $\Theta(n\log{n})$ & oui\\
    \hline\hline
    \hspace{1em}\alert{??}&\alert{$\Theta(n\log{n})$}& & &\alert{oui}\\
    \hline\hline
  \end{tabular}
  \end{center}

\end{frame}

\section{Tri par tas}

\begin{frame}{Tri par tas: introduction}

\begin{itemize}
\item \emph{Heapsort} en anglais
\item inventé par Williams en 1964
\item basé sur une structure de données très utile, le \emph{tas}
\item complexité bornée par $\Theta(n\log n)$ (dans tous les cas)
\item tri en place
\item mise en oeuvre très simple

\bigskip

\item Suite du cours:
\begin{itemize}
\item Introduction aux arbres
\item Tas
\item Tri par tas
\end{itemize}
\end{itemize}

% Voir slides français.

\note{On va d'abord voir le tas, puis le heapsort, On verra la semaine
  prochaine comment utiliser le tas pour faire une file à priorités}
\end{frame}

\subsection{Introduction aux arbres}

\begin{frame}{Arbres: définition}
\begin{itemize}
\item Définition: Un arbre (\emph{tree}) $T$ est un graphe dirigé $(N,E)$, où:
\begin{itemize}
\item $N$ est un ensemble de n\oe uds, et
\item $E\subset N\times N$ est un ensemble d'arcs,
\end{itemize}
possédant les propriétés suivantes:
\begin{itemize}
\item T est connexe et acyclique
\item Si $T$ n'est pas vide, alors il possède un n\oe ud distingué appelé racine (\emph{root node}). Cette racine est unique.
\item Pour tout arc $(n_1,n_2)\in E$, le n\oe ud $n_1$ est le \alert{parent} de $n_2$.
\begin{itemize}
\item La racine de $T$ ne possède pas de parent.
\item Les autres n\oe uds de $T$ possèdent un et un seul parent.
\end{itemize}
\end{itemize}
\end{itemize}

\centerline{\includegraphics[width=5cm]{Figures/03-exemple-arbre2.pdf}}

\end{frame}

\begin{frame}{Arbres: terminologie}
\begin{itemize}
\item Si $n_2$ est le parent de $n_1$, alors $n_1$ est le \alert{fils} (\emph{child}) de $n_2$.
\item Deux n\oe uds $n_1$ et $n_2$ qui possèdent le même parent sont
  des \alert{frères} (\emph{siblings}).
\item Un n\oe ud qui possède au moins un fils est un n\oe ud \alert{interne}.
\item Un n\oe ud externe (c'est-à-dire, non interne) est une \alert{feuille}
  (\emph{leaf}) de l'arbre.
\item Un n\oe ud $n_2$ est un \alert{ancêtre} (\emph{ancestor}) d'un n\oe ud
  $n_1$ si $n_2$ est le parent de $n_1$ ou un ancêtre du parent de $n_1$.
\item Un n\oe ud $n_2$ est un \alert{descendant} d'un n\oe ud $n_1$ si $n_1$ est un ancêtre de $n_2$.
\end{itemize}

\centerline{\includegraphics[width=5cm]{Figures/03-exemple-arbre2.pdf}}

\end{frame}

\begin{frame}{Arbres: terminologie}

\begin{itemize}
\item Un \alert{chemin} est une séquence de n\oe uds $n_1$, $n_2$, \ldots, $n_m$ telle que pour tout $i\in [1,m-1]$, $(n_i,n_{i+1})$ est un arc de l'arbre.\\
Remarque: Il n'existe jamais de chemin reliant deux feuilles distinctes.
\item La \alert{hauteur} (\emph{height}) d'un n\oe ud $n$ est le nombre d'arcs d'un plus long chemin de ce n\oe ud vers une feuille. La \emph{hauteur de l'arbre} est la hauteur de sa racine.
\item La \alert{profondeur} (\emph{depth}) d'un n\oe ud $n$ est le nombre d'arcs sur le chemin qui le relie à la racine.
\end{itemize}

\centerline{\includegraphics[width=5cm]{Figures/03-exemple-arbre2.pdf}}

\end{frame}

\begin{frame}{Arbre binaire}

\begin{itemize}
\item Un arbre \alert{ordonné} est un arbre dans lequel les ensembles de fils de chacun de ses n\oe uds sont ordonnés.
\item Un arbre \alert{binaire} est un arbre ordonné possédant les propriétés suivantes:
\begin{itemize}
\item Chacun de ses n\oe uds possède au plus deux fils.
\item Chaque n\oe ud fils est soit un fils gauche, soit un fils droit.
\item Le fils gauche précède le fils droit dans l'ordre des fils d'un n\oe ud.
\end{itemize}
\item Un arbre \alert{binaire entier} ou {propre} (\emph{full} or \emph{proper}) est un arbre binaire dans lequel tous les n\oe uds internes possèdent exactement deux fils.
\item Un arbre \alert{binaire parfait} est un arbre binaire entier dans lequel toutes les feuilles sont à la même profondeur.
\end{itemize}

\end{frame}

\begin{frame}{Propriétés des arbres binaires entiers}
\label{sec3:proparbres}
% fig: Dupont
\centerline{\includegraphics[width=4.5cm]{Figures/03-arbrebinaireentier.pdf}}

% text: Dupont
\begin{itemize}
\item Le nombre de n\oe uds externes est égal au nombre de n\oe uds internes plus 1.
\item Le nombre de n\oe uds internes est égal à $\frac{n-1}{2}$, où
  $n$ désigne le nombre de n\oe uds.
\item Le nombre de n\oe uds à la profondeur (ou niveau) $i$ est $\leq 2^i$.
\item La hauteur $h$ de l'arbre est $\leq$ au nombre de n\oe uds internes.
\item Le lien entre hauteur et nombre de n\oe uds peut être résumé comme suit:
\begin{center}
$n\in \Omega(h)\mbox{ et }n\in O(2^h)$ (ou $h\in O(n)\mbox{ et }h\in \Omega(\log n)$)
\end{center}
\end{itemize}

\note{
\begin{itemize}
\item Par récursion: vrai pour un seul arbre. Vrai pour gauche et droite, vrai pour l'arbre
\item $n=N_I+N_E=N_I+N_I+1=2 N_I$
\item On multiplie le nombre de noeuds au plus par deux par niveau
\item pire cas est une chaine
\item Prop 4: Chaine $\Rightarrow h\leq N_I=n-1/2\Rightarrow n\geq 2h+1=\Omega(h)$\\
Prop 3: Arbre parfait: $N_E=N_I+1=(n-1)/2\leq 2^h \Rightarrow n=O(h)$
\end{itemize}
}
\end{frame}

\subsection{Tas}

\begin{frame}{Tas: définition}\label{sec:03tas}

Un arbre \alert{binaire complet} est un arbre binaire tel que:
\begin{itemize}
\item Si $h$ dénote la hauteur de l'arbre:
\begin{itemize}
\item Pour tout $i\in [0,h-1]$, il y a exactement $2^i$ n\oe uds à la profondeur $i$.
\item Une feuille a une profondeur $h$ ou $h-1$.
\item Les feuilles de profondeur maximale ($h$) sont ``tassées'' sur la gauche.
\end{itemize}
\end{itemize}
\medskip

Un \alert{tas binaire} (binary heap) est un arbre binaire complet tel que:
\begin{itemize}
\item Chacun de ses n\oe uds est associé à une clé.
\item La clé de chaque n\oe ud est supérieure ou égale à celle de ses
  fils (\alert{propriété d'ordre du tas}).
\end{itemize}

% fig: CLRS
\centerline{\includegraphics[width=5cm]{Figures/03-tas-binaire.pdf}}

\end{frame}

\begin{frame}{Propriété d'un tas}
\label{03:hauteurtas}
\begin{itemize}
\item Soit $T$ un arbre binaire complet contenant $n$ entrées et de hauteur $h$:
\begin{itemize}
\item $n$ est supérieur ou égal à la taille de l'arbre parfait de hauteur $h-1$ plus un, soit $2^{h-1+1}-1+1=2^h$
\item $n$ est inférieur ou égal à la taille de l'arbre parfait de hauteur $h$, soit $2^{h+1}-1$
\begin{eqnarray*}
2^h\leq n\leq 2^{h+1}-1 & \Leftrightarrow & 2^h\leq n < 2^{h+1}\\
& \Leftrightarrow & h \leq \log_2 n < h+1 \\
& \Leftrightarrow & h=\lfloor \log_2 n\rfloor
\end{eqnarray*}
\end{itemize}
\end{itemize}

% fig: CLRS
\centerline{\includegraphics[width=5cm]{Figures/03-tas-binaire.pdf}}

\note{
\begin{itemize}
\item Point 1: hauteur de l'arbre de la figure=3: arbre complet de hauteur
  $h-1\Rightarrow$ $N_E=2^{h-1}$ feuilles. $n=N_E+N_E-1=2*2^{h-1}-1=2^h-1$.\\
$+1$ parce que sinon l'arbre binaire complet serait de profondeur $h-1$.
\item Point 2: idem.
\end{itemize}


}
\end{frame}

%% \begin{frame}{Tas: interface}

%% Opérations définies sur un tas $H$:
%% \begin{itemize}
%% \item $\attrib{H}{heap-size}$: le nombre de clés dans $H$.
%% \item $\proc{Build-Max-Heap}(A)$ construit un tas à partir du tableau $A$.
%% \item $\proc{Heap-Insert}(H,key)$ insère $key$ dans le tas.
%% \item $\proc{Heap-Extract-Max}(H)$ extrait la clé maximale.
%\end{itemize}

%\bigskip

%\centerline{\includegraphics[width=5cm]{Figures/03-tas-binaire.pdf}}

%\end{frame}

\begin{frame}{Implémentation par un tableau}

\centerline{\includegraphics[width=10cm]{Figures/03-tas-binaire-tableau.pdf}}

\bigskip

Un tas peut être représenté de manière compacte à l'aide d'un tableau $A$.

\begin{itemize}
\item La racine de l'arbre est le premier élément du tableau.
\item $\proc{Parent}(i)=\lfloor i/2\rfloor$
\item $\proc{Left}(i)=2i$
\item $\proc{Right}(i)=2i+1$
\end{itemize}

Propriété d'ordre du tas: $\forall i, A[\proc{Parent}(i)]\geq A[i]$

\end{frame}

\begin{frame}{Principe du tri par tas}
\begin{itemize}
\item On construit un tas à partir du tableau à trier $\rightarrow$ $\proc{Build-max-heap}(A)$.
\item Tant que le tas contient des éléments:
\begin{itemize}
\item On extrait l'élément au sommet du tas qu'on place dans le
  tableau trié et on le remplace par l'élément le plus à droite

\centerline{\includegraphics[width=5cm]{Figures/03-heap-extract-max2.pdf}~~~~\includegraphics[width=5cm]{Figures/03-heap-extract-max2-2.pdf}}

\item On rétablit la propriété de tas en tenant compte du fait que les sous-arbres de droite et de gauche sont des tas $\rightarrow$ $\proc{Max-Heapify}(A,1)$ 
\end{itemize}
\end{itemize}
(Tout se fait dans le tableau initial $\rightarrow$ en place)
\end{frame}

%% \begin{frame}{$\proc{Heap-Extract-Max}$}

%% \begin{itemize}
%% \item Procédure $\proc{Heap-Extract-Max}$:
%% \begin{itemize}
%% \item Extrait la clé maximale. Elle est toujours à la racine \emph{(Pourquoi ?)}
%% \item Réarrange le tas pour maintenir la propriété d'ordre du tas
%% \end{itemize}

%% % FIG: Gaetano -> a refaire
%% \centerline{\includegraphics[width=5cm]{Figures/03-heap-extract-max2.pdf}~~~~\includegraphics[width=5cm]{Figures/03-heap-extract-max2-2.pdf}}

%% \item Principe:
%% \begin{itemize}
%% \item On remplace la racine par la feuille la plus à droite ($A[1]=A[N]$)
%% \item Réarrange le tas en tenant compte du fait que les sous-arbres de droite et de gauche sont des tas (appel de $\proc{Max-Heapify}(A,1)$)
%% \end{itemize}
%% \end{itemize}
%% \end{frame}

\begin{frame}{$\proc{Max-Heapify}$}

\begin{itemize}
\item Procédure $\proc{Max-Heapify}(A,i)$:
\begin{itemize}
\item Suppose que le sous-arbre de gauche du n\oe ud $i$ est un tas
\item Suppose que le sous-arbre de droite du n\oe ud $i$ est un tas
\item But: réarranger le tas pour maintenir la propriété d'ordre du tas
\end{itemize}
\item Ex: $\proc{Max-Heapify}(A,2)$
\centerline{\includegraphics[width=8cm]{Figures/03-max-heapify.pdf}}
\end{itemize}

\note{
\begin{itemize}
\item On prend le plus grand des voisins de gauche et de droite
\item On le swap avec la racine.
\item On appelle la procédure récursivement sur le fils modifié
\end{itemize}
}
\end{frame}

\begin{frame}{$\proc{Max-Heapify}$}

\begin{center}
\begin{small}
\fcolorbox{white}{Lightgray}{
      \begin{codebox}
        \Procname{$\proc{Max-Heapify}(A,i)$}
        \li $l\gets\proc{Left}(i)$
        \li $r\gets\proc{Right}(i)$
        \li \If $l\le \attrib{A}{heap-size} \wedge A[l]>A[i]$
        \li \Then $\id{largest}\gets l$
        \li \Else $\id{largest}\gets i$
        \End
        \li \If $r\le \attrib{A}{heap-size} \wedge A[r]>A[\id{largest}]$
        \li \Then $\id{largest}\gets r$
        \End
        \li \If $\id{largest}\ne i$
        \li \Then $\id{swap}(A[i],A[\id{largest}])$
        \li       $\proc{Max-Heapify}(A,\id{largest})$
        \End
      \end{codebox}}
\end{small}
\end{center}

\bigskip

\begin{itemize}
\item Complexité ? La hauteur du n\oe ud: $T(n)=O(\log n)$ (Transp. \pageref{03:hauteurtas})
\end{itemize}

\note{On vérifie qu'il y a bien un fils gauche et un fils de droite}

\end{frame}

\begin{frame}{Construction d'un tas}

\begin{center}\small
\fcolorbox{white}{Lightgray}{
      \begin{codebox}
        \Procname{$\proc{Build-Max-Heap}(A)$}
        \li $\attrib{A}{heap-size}\gets\attrib{A}{length}$
        \li \For $i\gets\lfloor \attrib{A}{length}/2\rfloor$ \Downto $1$
        \li \Do $\proc{Max-Heapify}(A,i)$
        \End
      \end{codebox}}
\end{center}
{\footnotesize (Invariant: chaque n\oe ud i, i+1,\ldots, n est la racine d'un tas)}
\bigskip

\centerline{\includegraphics[width=10cm]{Figures/03-build-max-heap.pdf}}

\bigskip

\begin{itemize}
\item La tableau initial est interprété comme un arbre binaire complet
\item On tasse les n\oe uds internes de bas en haut et de droite à gauche
\end{itemize}

\note{Montrer au tableau comment ça marche

\bigskip

borne à $length(A)/2$ car fils gauche d'un n\oe ud $i$ est donné par $2*i$. Pas de n\oe ud si $2i>A.length \Leftrightarrow i>A.length/2$.


}

\end{frame}

\begin{frame}{Complexité de $\proc{Build-Max-Heap}$}

\begin{itemize}
\item Borne simple:
\begin{itemize}
\item $O(n)$ appels à $\proc{Max-Heapify}$, chacun étant $O(\log n)$ $\Rightarrow O(n\log n)$.
\end{itemize}

\bigskip


\item Analyse plus fine:
\begin{itemize}
\item Pour simplifier l'analyse, on suppose que l'arbre binaire est parfait.
\item On a donc $n=2^{h+1}-1$ pour un $h\geq 0$, qui est aussi la hauteur de l'arbre résultant
\end{itemize}
\end{itemize}

\note{ $2^{h+1}-1$ $\Rightarrow$ car le nombre de feuille est $2^h$ et
  donc le nombre total de n\oe ud est $2 2^{h}-1=2^{h+1}-1$ }

\end{frame}

\begin{frame}{Complexité de $\proc{Build-Max-Heap}$}
\centerline{\includegraphics[width=10cm]{Figures/03-max-heapify-complex2.pdf}}

\begin{itemize}
\item Il y a $2^i$ n\oe uds à la profondeur $i$ (= hauteur $h-i$).
\item On doit appeler $\proc{Max-Heapify}$ sur chacun d'eux $\Rightarrow O(h-i)$.
\item Nombre d'opérations en fonction de $h$:
$$T(h)=\sum_{i=0}^{h-1} 2^i O(h-i)=O(\sum_{i=0}^{h-1} 2^i (h-i))$$
\end{itemize}
\end{frame}

\begin{frame}{Complexité de $\proc{Build-Max-Heap}$}
\begin{itemize}
\item $\sum_{i=0}^{h-1} 2^i (h-i)=1\cdot 2^{h-1}+2\cdot 2^{h-2}+\ldots+(h-1)\cdot 2^1+h\cdot 2^0$

\bigskip

\begin{small}
\begin{tabular}{ccccccccccl}
$2^{h-1}$ & + & $2^{h-2}$ & + & $2^{h-3}$ & + & \ldots & +& $2^0$ & = & $2^{h}-1$\\
 & + & $2^{h-2}$ & + & $2^{h-3}$ & + & \ldots & +& $2^0$ & = & $2^{h-1}-1$\\
 & & & + & $2^{h-3}$ & + & \ldots & +& $2^0$ & = & $2^{h-2}-1$\\
 &   &        &   &      &   & \ldots &  &    & \ldots & \\
 &   &        &   &     &    &        & & $2^0$ & = & $2^1-1$\\
\hline
 &   &        &   &     &    &        & &     & = & $\left(\sum_{i=1}^h 2^i\right)-h$
\end{tabular}
\end{small}

\medskip

(En utilisant $\sum_{i=0}^n x^i=\frac{x^{n+1}-1}{x-1}$)

\bigskip

\item On obtient donc $T(h)=O(2^{h+1}-h-2).$
\item Puisque $h=log_2(n+1)-1=O(\log n)$, on a $$T(n)=O(n).$$
\end{itemize}
\end{frame}

%\subsection{File à priorités}

\subsection{Tri par tas}

\begin{frame}{Tri par tas: algorithme}

\begin{center}
\fcolorbox{white}{Lightgray}{
        \begin{codebox}
          \Procname{$\proc{Heap-Sort}(A)$}
          \li $\proc{Build-Max-Heap}(A)$
          \li \For $i\gets\attrib{A}{length}$ \Downto $2$
          \li \Do $\id{swap}(A[i],A[1])$
          \li     $\attrib{A}{heap-size} \gets\attrib{A}{heap-size}-1$
          \li     $\proc{Max-Heapify}(A,1)$
          \End
        \end{codebox}}
\end{center}

\bigskip

{\small
Invariant:\\
$A[1\twodots i]$ est un tas contenant les $i$ éléments les plus petits de $A[1\twodots \attrib{A}{length}]$ et $A[i+1\twodots \attrib{A}{length}]$ contient les $n-i$ éléments les plus grands de $A[1\twodots \attrib{A}{length}]$ triés.
}
\end{frame}

\begin{frame}{Tri par tas: illustration}
Tableau initial: $A=[7,4,3,1,2]$

\bigskip

%Fig:CLRS
\centerline{\includegraphics[width=7cm]{Figures/03-heapsort-exemple.pdf}}

\end{frame}

\begin{frame}{Complexité de \proc{Heap-Sort}}
\begin{center}\small
\fcolorbox{white}{Lightgray}{
        \begin{codebox}
          \Procname{$\proc{Heap-Sort}(A)$}
          \li $\proc{Build-Max-Heap}(A)$
          \li \For $i\gets\attrib{A}{length}$ \Downto $2$
          \li \Do $\id{swap}(A[i],A[1])$
          \li     $\attrib{A}{heap-size} \gets\attrib{A}{heap-size}-1$
          \li     $\proc{Max-Heapify}(A,1)$
          \End
        \end{codebox}}
\end{center}

\begin{itemize}
\item $\proc{Build-Max-Heap}$: $O(n)$
\item Boucle $\For$: $n-1$ fois
\item Echange d'éléments: $O(1)$
\item $\proc{Max-Heapify}$: $O(\log n)$
\end{itemize}
Total: $O(n\log n)$ \alert{(pour le pire cas et le cas moyen)}

\bigskip

Le tri par tas est cependant généralement battu par le tri rapide
\end{frame}

\section{Synthèse}

\begin{frame}{Résumé}

 \begin{center}
    \def\arraystretch{1.5}
  \begin{tabular}{@{}lccc@{}c@{}}
    \emph{Algorithme}&\multicolumn{3}{c}{\emph{Complexité}}&\emph{En place?}\\
    & \emph{\small Pire} & \emph{\small Moyenne} & \emph{Meilleure} & \\
    \hline\hline
    \proc{Insertion-Sort}&$\Theta(n^2)$&$\Theta(n^2)$&$\Theta(n)$&oui\\
    \hline
    \proc{Selection-Sort}&$\Theta(n^2)$&$\Theta(n^2)$&$\Theta(n^2)$&oui\\
    \hline
    \proc{Bubble-Sort}&$\Theta(n^2)$&$\Theta(n^2)$&$\Theta(n)$&oui\\
    \hline
    \proc{Merge-Sort}&$\Theta(n\log{n})$&$\Theta(n\log{n})$&$\Theta(n\log{n})$&non\\
    \hline
    \proc{Quick-Sort} & $\Theta(n^2)$ & $\Theta(n\log{n})$ & $\Theta(n\log{n})$ & oui\\
    \hline
    \proc{Heap-Sort} & $\Theta(n\log{n})$ & $\Theta(n\log{n})$ & $\Theta(n\log{n})^*$ & oui\\
    \hline\hline
  \end{tabular}\\
\medskip
{~\hfill\footnotesize $^*$ pas montré dans ce cours}
  \end{center}

\end{frame}

\begin{frame}{Peut-on faire mieux que $O(n\log n)$?}

% Gaetano

\begin{itemize}
\item Non, si on se restreint aux tri \emph{comparatifs}, c'est-à-dire:
\begin{itemize}
\item Aucune hypothèse sur les éléments à trier
\item Nécessité de les comparer entre eux
\end{itemize}
\item Complexité d'un problème algorithmique versus complexité d'un algorithme
\item Dans ce cas, un algorithme de tri est:
\begin{itemize}
\item Une suite de comparaisons d'éléments suivant une certaine méthode
\item Un processus qui transforme un tableau $[e_0,e_1,\ldots,e_{n-1}]$ en un autre tableau $[e_{\sigma_0},e_{\sigma_1},\ldots,e_{\sigma_{n-1}}]$ où $(\sigma_0,\sigma_1,\ldots,\sigma_{n-1})$ est une permutation de $(0,1,\ldots,n-1)$.
\end{itemize}
\end{itemize}

\end{frame}

\begin{frame}{Arbre de décision: exemple}

Un algorithme de tri = un arbre binaire de décision (entier)

\bigskip

\centerline{\includegraphics[width=8cm]{Figures/03-decisioninsertion.pdf}}

\bigskip

(arbre de décision pour le tri par insertion du tableau $[e_0,e_1,e_2]$)

\bigskip

\emph{Exercice: construire l'arbre pour le tri par fusion}

\end{frame}

\begin{frame}{Arbre de décision: définition}

Un algorithme de tri = un arbre binaire de décision

\begin{itemize}
\item feuille de l'arbre: une permutation des éléments du tableau initial
\item tri: le chemin de la racine à la feuille correspondant au tableau trié
\item hauteur de l'arbre: le pire cas pour le tri
\item branche la plus courte: le meilleur cas pour le tri
\item hauteur moyenne de l'arbre: la complexité en moyenne du tri
\end{itemize}

\end{frame}

\begin{frame}{Arbre de décision: propriété}
\begin{itemize}
\item Un arbre binaire de hauteur $h$ a au plus $2^h$ feuilles (cf transp. \pageref{sec3:proparbres})
\item Le nombre de feuilles de l'arbre de décision est  exactement $n!$ où  $n$ est la taille du tableau à trier\\
\emph{(par l'absurde: si moins que $n!$ certains tableaux ne seraient pas correctement triés)}
% 
\item On a donc:
$$n!\leq 2^h \Rightarrow \log(n!)\leq h$$
\item Formule de Stirling:
$$n!=\sqrt{2\pi n} (\frac{n}{e})^n (1+\Theta(\frac{1}{n}))\Rightarrow n!\geq (\frac{n}{e})^n$$

$$h\geq \log(n!)>\log((\frac{n}{e})^n)=n\log n - n\log e\Rightarrow h=\Omega(n\log n)$$

\end{itemize}
\bigskip

\centerline{\bf Le problème du tri comparatif est $\Omega(n\log n)$}

\note{Dire qu'il ne peut pas y en avoir moins que $n!$ feuilles sinon, un cas ne serait pas traiter}

\end{frame}

\begin{frame}{Ce qu'on a vu}

\begin{itemize}
\item Catégorisation des algorithmes de tri
\item \proc{QuickSort} (tri en place en $O(n\log n))$
\item Analyse du cas moyen d'un algorithme
\item Notre première structure de données: le tas
\item \proc{HeapSort} (tri en place en $\Theta(n\log n)$)
\item Borne inférieure sur les tris comparatifs

\bigskip

\item Illustration:
\begin{itemize}
\item \url{http://www.sorting-algorithms.com/}
\item \url{http://www.youtube.com/user/algorythmics}
\end{itemize}

\end{itemize}

\end{frame}

\begin{frame}{Ce qu'on n'a pas vu}

\begin{itemize}
\item Analyse formelle de la complexité moyenne du $\proc{QuickSort}$.
\item Invariant pour le tri par tas
\item Méthodes de tri linéaire
\item Méthodes de sélection: trouver l'élément de rang $i$
\end{itemize}

\note{Demander comment il ferait ça de manière naive

\bigskip

Tri par comptage: suppose qu'on trie des entiers et qu'on connait le min et le max: on fait un histogramme et puis on parcourt cet histogramme}
\end{frame}


\part{Structures de données élémentaires}

% Structures de pile, file, liste liée (notion de liste liée comme une implémentation), arbre (?), files à priorité
% complexité amortie ? ajout d'élément à une structure de ce type.

\begin{frame}{Plan}

\tableofcontents

\end{frame}

\section{Introduction}

\begin{frame}{Concept}
\begin{itemize}
\item Une \alert{structure de données} est une manière d'organiser et de stocker l'information
\begin{itemize}
\item Pour en faciliter l'accès ou dans d'autres buts
\end{itemize}
\item Une structure de données a une \alert{interface} qui consiste en un ensemble de procédures pour ajouter, effacer, accéder, réorganiser, etc. les données.
\item Une structure de données conserve des \alert{données} et éventuellement des \alert{méta-données}
\begin{itemize}
\item Par exemple: un tas utilise un tableau pour stocker les clés et une variable $\attrib{A}{heap-size}$ pour retenir le nombre d'éléments qui sont dans le tas.
\end{itemize}
\item Un type de données abstrait (TDA) = définition des propriétés
  de la structure et de son interface (``cahier des charges'')
\end{itemize}
\end{frame}

\begin{frame}{Structures de données}

Dans ce cours:
\begin{itemize}
\item Principalement des \alert{ensembles dynamiques} (dynamic sets), amenés à croître, se
  rétrécir et à changer au cours du temps.
\item Les objets de ces ensembles comportent des attributs.
\item Un de ces attributs est une \alert{clé} qui permet d'identifier
  l'objet, les autres attributs sont la plupart du temps non
  pertinents pour l'implémentation de la structure.
\item Certains ensembles supposent qu'il existe un \alert{ordre total}
  entre les clés.
\end{itemize}

\end{frame}

\begin{frame}{Opérations standards sur les structures}

\begin{itemize}
\item Deux types: opérations de recherche/accès aux données et opérations de modifications
\item Recherche: exemples:
\begin{itemize}
\item $\proc{Search}(S,k)$: retourne un pointeur $x$ vers un élément dans $S$ tel que $\attrib{x}{key}=k$, ou $\const{NIL}$ si un tel élément n'appartient pas à $S$.
\item $\proc{Minimum}(S)$, $\proc{Maximum}(S)$: retourne un pointeur
  vers l'élément avec la plus petite (resp. grande) clé.
\item $\proc{Successor}(S,x)$,$\proc{Predecessor}(S,x)$ retourne un pointeur vers l'élément tout juste plus grand (resp. petit) que $x$ dans $S$, $\const{NIL}$ si $x$ est le maximum (resp. minimum).
\end{itemize}
\item Modification: exemples:
\begin{itemize}
\item $\proc{Insert}(S,x)$: insère l'élément $x$ dans $S$.
\item $\proc{Delete}(S,x)$: retire l'élément $x$ de $S$.
\end{itemize}
\end{itemize}

\end{frame}

\begin{frame}{Implémentation d'une structure de données}
\begin{itemize}
\item Etant donné un TDA (interface), plusieurs implémentations sont généralement possibles
\item La complexité des opérations dépend de l'implémentation, \alert{pas du TDA}.

\bigskip

\item Les briques de base pour implémenter une structure de données
  dépendent du langage d'implémentation
\begin{itemize}
\item Dans ce cours, les principaux outils du C: tableaux, structures
  à la C (objets avec attributs), liste liées (simples, doubles,
  circulaires), etc.
\end{itemize}
\item Une structure de données peut être implémentée à l'aide d'une
  autre structure de données (de base ou non)
\end{itemize}
\end{frame}

\begin{frame}{Quelques structures de données standards}

\begin{itemize}
\item Pile: collection d'objets accessible selon une politique LIFO
\item File: collection d'objets accessible selon une politique FIFO
\item File double: combine accès LIFO et FIFO
\item Liste: collection d'objets ordonnés accessible à partir de leur position
\item Vecteur: collection d'objets ordonnés accessible à partir de leur rang
\item File à priorité: accès uniquement à l'élément de clé (priorité) maximale

\bigskip

\item Dictionnaire: structure qui implémente les 3 opérations
  recherche, insertion, suppression (cf. partie 5)
\end{itemize}

\end{frame}

\section{Pile}

\begin{frame}{Pile}
\begin{itemize}
\item Ensemble dynamique d'objets accessibles selon une discipline
  \alert{LIFO} (``Last-in first-out'').
\item Interface
\begin{itemize}
\item $\proc{Stack-Empty}(S)$ renvoie vrai si et seulement si la pile est vide
\item $\proc{Push}(S,x)$ pousse la valeur $x$ sur la pile $S$
\item $\proc{Pop}(S)$ extrait et renvoie la valeur sur le sommet de la pile $S$
\end{itemize}
\item Applications:
\begin{itemize}
\item Option 'undo' dans un traitement de texte
\item Langage postscript
\item Appel de fonctions dans un compilateur
\item \ldots
\end{itemize}
\item Implémentations:
\begin{itemize}
\item avec un tableau (taille fixée a priori)
\item au moyen d'une liste liée (allouée de manière dynamique)
\item \ldots
\end{itemize}
\end{itemize}
\end{frame}

\begin{frame}{Implémentation par un tableau}

\begin{itemize}
\item $S$ est un tableau qui contient les éléments de la pile
\item $\attrib{S}{top}$ est la position courante de l'élément au sommet de $S$

\medskip

\begin{columns}
\begin{column}{5cm}
\centerline{\includegraphics[width=3cm]{Figures/04-piletableau.pdf}}

\bigskip

\begin{center}
\begin{small}
\fcolorbox{white}{Lightgray}{%
        \begin{codebox}
          \Procname{$\proc{Push}(S,x)$}
          \li \If $\attrib{S}{top}\isequal \attrib{S}{length}$
          \li \Then \Error ``overflow''\End
          \li $\attrib{S}{top}\gets \attrib{S}{top}+1$
          \li $S[\attrib{S}{top}]\gets x$
        \end{codebox}}
\end{small}
\end{center}

\end{column}
\begin{column}{5cm}
    \begin{center}
      \begin{small}
      \fcolorbox{white}{Lightgray}{%
        \begin{codebox}
          \Procname{$\proc{Stack-Empty}(S)$}
          \li \Return $\attrib{S}{top}\isequal 0$
          \End
        \end{codebox}}

\bigskip

      \fcolorbox{white}{Lightgray}{%
        \begin{codebox}
          \Procname{$\proc{Pop(S)}$}
          \li \If $\proc{Stack-Empty}(S)$
          \li \Then \Error ``underflow''
          \li \Else $\attrib{S}{top}\gets \attrib{S}{top}-1$
          \li       \Return $S[top(S)+1]$
          \End
        \end{codebox}}

      \end{small}
    \end{center}
\end{column}
\end{columns}

\item Complexité en temps \alert{et en espace}: $O(1)$\\
(Inconvénient: L'espace occupé ne dépend pas du nombre d'objets)
\end{itemize}

\end{frame}

\begin{frame}{Rappel: liste simplement et doublement liée}

\centerline{\includegraphics[width=8cm]{Figures/04-listeliee.pdf}}

\bigskip

\begin{itemize}
\item Structure de données composée d'une séquence d'éléments de liste.
\item Chaque élément $x$ de la liste est composé:
\begin{itemize}
\item d'un contenu utile $\attrib{x}{data}$ de type arbitraire (par exemple une clé),
\item d'un pointeur $\attrib{x}{next}$ vers l'élément suivant dans la séquence
\item \emph{Doublement liée: }d'une pointeur $\attrib{x}{prev}$ vers l'élément précédent dans la séquence
\end{itemize}
\item Soit $L$ une liste liée
\begin{itemize}
\item $\attrib{L}{head}$ pointe vers le premier élément de la liste
\item \emph{Doublement liée:} $\attrib{L}{tail}$ pointe vers le dernier élément de la liste
\end{itemize}
\item Le dernier élément possède un pointeur $\attrib{x}{next}$ vide (noté $\const{NIL}$)
\item \emph{Doublement liée:} Le premier élément possède un pointeur $\attrib{x}{prev}$ vide
%% \item Avantage
%% \begin{itemize}
%% \item L'insertion et la suppression d'éléments est réalisable en temps $O(1)$ en tête de liste, ainsi qu'à la suite d'un élément donné (à n'importe quel endroit pour une liste doublement liée).
%% \item Les éléments de la liste peuvent être alloués dynamiquement
%% \end{itemize}
\end{itemize}
\note{Mais l'espace mémoire est deux fois plus important qu'avec un tableau classique}
\end{frame}

%% \begin{frame}{Opérations sur une liste doublement liée}

%% %Interface:
%% {\small
%%     \begin{itemize}
%% %      \medskip
%%     \item $\proc{List-Insert}(L,x)$ ajoute l'élément $x$ au début de la liste $L$
%% %      \medskip
%%     \item $\proc{List-Delete}(L,x)$ supprime l'élément $x$ de la liste $L$
%% %      \medskip
%%     \item $\proc{List-Search}(L,k)$ trouve une élément dont la clé $k$
%%       est dans la liste $L$
%%     \end{itemize}
%% }

%% %% \fcolorbox{white}{Lightgray}{%
%% %%       \begin{codebox}
%% %%         \Procname{$\proc{List-Init}(L)$}
%% %%         \li $\id{prev}(\id{nil}(L))\gets\id{nil}(L)$
%% %%         \li $\id{next}(\id{nil}(L))\gets\id{nil}(L)$
%% %%       \end{codebox}}

%% \begin{center}
%% \begin{footnotesize}
%% \fcolorbox{white}{Lightgray}{%
%%       \begin{codebox}
%%         \Procname{$\proc{List-Insert}(L,x)$}
%%         \li $\attrib{x}{next}\gets \attrib{L}{head}$
%%         \li \If $\attrib{L}{head}\ne \const{NIL}$
%%         \li \Then $\attrib{L}{head}.\id{prev} \gets x$ \End
%%         \li $\attrib{L}{head}\gets x$
%%         \li $\attrib{x}{prev}\gets \const{NIL}$
%%       \end{codebox}}
%% ~~~~~
%% \fcolorbox{white}{Lightgray}{%
%%       \begin{codebox}
%%         \Procname{$\proc{List-Delete}(L,x)$}
%%         \li \If $\attrib{x}{prev}\ne \const{NIL}$
%%         \li \Then $\attrib{x}{prev}.\id{next}\gets \attrib{x}{next}$
%%         \li \Else $\attrib{L}{head}\gets \attrib{x}{next}$\End
%%         \li \If $\attrib{x}{next}\ne \const{NIL}$
%%         \li \Then $\attrib{x}{next}.\id{prev}\gets \attrib{x}{prev}$ \End
%%       \end{codebox}}

%% \bigskip

%% \fcolorbox{white}{Lightgray}{%
%%       \begin{codebox}
%%         \Procname{$\proc{List-Search}(L,k)$}
%%         \li $x\gets \attrib{L}{head}$
%%         \li \While $x\ne \const{NIL}\wedge\attrib{x}{key}\ne k$
%%         \li \Do $x\gets \attrib{x}{next}$
%%             \End
%%         \li \Return $x$
%%       \end{codebox}}
%% \end{footnotesize}
%% \end{center}

%% Complexité: $O(1)$ pour l'insertion et la suppression, $O(n)$ pour la recherche.
%% \end{frame}


%% \begin{frame}{Sentinelle}
%% \begin{itemize}\small
%% \item On peut simplifier le code en ajoutant une \alert{sentinelle} $L.nil$ qui représente le $\const{NIL}$
%% \item $L.nil.next$ pointe vers le premier élément et $L.nil.prev$ pointe vers le dernier élément (, $L.nil.next=L.nil$ et $L.nil.prev=L.nil$)
%% \end{itemize}

%% \centerline{\includegraphics[width=8cm]{Figures/04-sentinel.pdf}}

%% \begin{center}
%% \begin{footnotesize}
%% \fcolorbox{white}{Lightgray}{%
%%       \begin{codebox}
%%         \Procname{$\proc{List-Insert'}(L,x)$}
%%         \li $\attrib{x}{next}\gets \attrib{L}{nil}.\id{next}$
%%         \li $\attrib{L}{nil}.\id{next}.\id{prev} \gets x$
%%         \li $\attrib{L}{nil}.\id{next} \gets x$
%%         \li $\attrib{x}{prev}\gets \attrib{L}{nil}$
%%       \end{codebox}}
%% ~~~~~
%% \fcolorbox{white}{Lightgray}{%
%%       \begin{codebox}
%%         \Procname{$\proc{List-Delete'}(L,x)$}
%%         \li $\attrib{x}{prev}.\id{next}\gets \attrib{x}{next}$
%%         \li $\attrib{x}{next}.\id{prev}\gets \attrib{x}{prev}$
%%       \end{codebox}}

%% \bigskip

%% \fcolorbox{white}{Lightgray}{%
%%       \begin{codebox}
%%         \Procname{$\proc{List-Search'}(L,k)$}
%%         \li $x\gets \attrib{L}{nil}.\id{next}$
%%         \li \While $x\ne \attrib{L}{nil}\wedge\attrib{x}{key}\ne k$
%%         \li \Do $x\gets \attrib{x}{next}$
%%             \End
%%         \li \Return $x$
%%       \end{codebox}}
%% \end{footnotesize}
%% \end{center}

%% \note{
%% Toutes les pointeurs vers NIL sont remplacés par un pointeur vers $L.nil$

%% Ne permet de gagner qu'un facteur constant

%% \bigskip

%% Fait perdre de la place si on a beaucoup de petites listes


%% \bigskip

%% Principal intérêt est le clareté du code
%% }
%% \end{frame}


\begin{frame}{Implémentation d'une pile à l'aide d'une liste liée}

\begin{itemize}
\item $S$ est une liste simplement liée ($S.head$ pointe vers le premier élément de la liste)

\medskip

\begin{columns}
\begin{column}{5cm}
\begin{center}
\includegraphics[width=5cm]{Figures/04-pilelisteliee.pdf}
\end{center}

\bigskip

\fcolorbox{white}{Lightgray}{%
        \begin{codebox}
          \Procname{$\proc{Push}(S,x)$}
          \li $\attrib{x}{next}\gets \attrib{S}{head}$
          \li $\attrib{S}{head}\gets x$
        \end{codebox}}
\end{column}
\begin{column}{5cm}
\begin{center}
  \footnotesize
  \fcolorbox{white}{Lightgray}{%
    \begin{codebox}
      \Procname{$\proc{Stack-Empty}(S)$}
      \li \If $\attrib{S}{head}\isequal \const{NIL}$
      \li \Then \Return \const{true}
      \li \Else \Return \const{false}
        \End
  \end{codebox}}\hfill

\bigskip

  \fcolorbox{white}{Lightgray}{%
    \begin{codebox}
      \Procname{$\proc{Pop(S)}$}
      \li \If $\proc{Stack-Empty}(S)$
          \li \Then \Error ``underflow''
          \li \Else $x = S.head$
          \li       $\attrib{S}{head}\gets \attrib{S}{head}.\id{next}$
          \li       \Return $x$
          \End
        \end{codebox}}
    \end{center}
\end{column}
\end{columns}

\item Complexité en temps $O(1)$, complexité en espace $O(n)$ pour $n$ opérations
\end{itemize}

\end{frame}

\begin{frame}{Application}
\begin{itemize}
\item Vérifier l'appariement de parenthèses ($[]$,$()$ ou $\{\}$) dans une chaîne de caractères
\begin{itemize}
\item Exemples: $((x)+(y)]/2\rightarrow$ non, $[- (b) + \mbox{sqrt}(4*(a)*c)]/ (2*a) \rightarrow$ oui
\end{itemize}
\item Solution basée sur une pile:
\begin{center}
\begin{small}
\fcolorbox{white}{Lightgray}{%
        \begin{codebox}
          \Procname{$\proc{ParenthesesMatch}(A)$}
          \li $S\gets$ pile vide
          \li \For $i\gets 1 \To \attrib{A}{length}$
          \li \Do \If $A[i]$ est une parenthèse gauche
          \li \Then $\proc{Push}(S,A[i])$
          \li \ElseIf $A[i]$ est une parenthèse droite
          \li \Then \If $\proc{Stack-Empty}(S)$
          \li \Then \Return \const{False}

          \li \ElseIf  $\proc{Pop}(S)$ n'est pas du même type que $A[i]$
          \li \Then \Return \const{False}
          \End\End\End
          \li \End \Return $\proc{Stack-Empty}(S)$
          \End
        \end{codebox}}
\end{small}
\end{center}
\end{itemize}
\end{frame}

\section{Files simple et double}

\begin{frame}{File}
\begin{itemize}
\item Ensemble dynamique d'objets accessibles selon une discipline \alert{FIFO} (``First-in first-out'').
\item Interface
\begin{itemize}
\item $\proc{Enqueue}(Q,s)$ ajoute l'élément $x$ à la fin de la file $Q$
\item $\proc{Dequeue}(Q)$  retire l'élément à la tête de la file $Q$
\end{itemize}

\bigskip

\item Implémentation à l'aide d'un tableau circulaire
\begin{itemize}
\item $Q$ est un tableau de taille fixe $\attrib{Q}{length}$
\begin{itemize}
\item Mettre plus de $\attrib{Q}{length}$ éléments dans la file provoque une erreur de dépassement
\end{itemize}
\item $\attrib{Q}{head}$  est la position à la tête de la file
\item $\attrib{Q}{tail}$ est la première position vide à la fin de la file
\item Initialement: $\attrib{Q}{head}=\attrib{Q}{tail}=1$
\end{itemize}
\end{itemize}
\end{frame}

\begin{frame}{\proc{Enqueue} et \proc{Dequeue}}
\begin{small}
Etat initial:
\bigskip
\centerline{\includegraphics[width=6cm]{Figures/04-file1.pdf}}
$\proc{Enqueue}(Q,17)$, $\proc{Enqueue}(Q,3)$, $\proc{Enqueue}(Q,5)$
\bigskip
\centerline{\includegraphics[width=6cm]{Figures/04-file2.pdf}}
$\proc{Dequeue}(Q) \rightarrow 15$
\bigskip
\centerline{\includegraphics[width=6cm]{Figures/04-file3.pdf}}
\end{small}
\end{frame}

\begin{frame}{\proc{Enqueue} et \proc{Dequeue}}

\begin{center}
\begin{small}
\fcolorbox{white}{Lightgray}{%
  \begin{codebox}
    \Procname{$\proc{Enqueue(Q,x)}$}
    \li $Q[\attrib{Q}{tail}]\gets x$
    \li \If $\attrib{Q}{tail}\isequal \attrib{Q}{length}$
    \li       \Then $\attrib{Q}{tail}\gets 1$
    \li       \Else $\attrib{Q}{tail}\gets\attrib{Q}{tail}+1$
  \end{codebox}}
~~~~~
\fcolorbox{white}{Lightgray}{%
  \begin{codebox}
    \Procname{$\proc{Dequeue(Q)}$}
    \li $x\gets Q[\attrib{Q}{head}]$
    \li \If $\attrib{Q}{head}\isequal \attrib{Q}{length}$
    \li       \Then $\attrib{Q}{head}\gets 1$
    \li       \Else $\attrib{Q}{head}\gets\attrib{Q}{head}+1$\End
    \li \Return $x$
  \end{codebox}}
\end{small}
\end{center}

\bigskip

\begin{itemize}
\item Complexité en temps $O(1)$, complexité en espace $O(1)$.
\item {\it Exercice: ajouter la gestion d'erreur}
\end{itemize}
\end{frame}

\begin{frame}{Implémentation à l'aide d'une liste liée}
\centerline{\includegraphics[width=6cm]{Figures/04-filelisteliee.pdf}}
\begin{itemize}
\item $Q$ est une liste simplement liée
\item $\attrib{Q}{head}$ (resp. $\attrib{Q}{tail}$) pointe vers la tête (resp. la queue) de la liste

\begin{center}
\begin{small}
\fcolorbox{white}{Lightgray}{%
  \begin{codebox}
    \Procname{$\proc{Enqueue(Q,x)}$}
    \li $\attrib{x}{next}\gets \const{NIL}$
    \li \If $\attrib{Q}{head}\isequal \const{NIL}$
    \li \Then $\attrib{Q}{head}\gets x$
    \li \Else $\attrib{Q}{tail}.\id{next} \gets x$
    \End
    \li $\attrib{Q}{tail}\gets x$
  \end{codebox}}
~~~~~
\fcolorbox{white}{Lightgray}{%
  \begin{codebox}
    \Procname{$\proc{Dequeue(Q)}$}
    \li \If $\attrib{Q}{head}\isequal \const{NIL}$
    \li \Then \Error ``underflow''
    \End
    \li $x\gets \attrib{Q}{head}$
    \li $\attrib{Q}{head}\gets\attrib{Q}{head}.\id{next}$
    \li \If $\attrib{Q}{head}\isequal \const{NIL}$
    \li \Then $\attrib{Q}{tail}\gets \const{NIL}$ \End
    \li \Return $x$
  \end{codebox}}
\end{small}
\end{center}

\item Complexité en temps $O(1)$, complexité en espace $O(n)$ pour $n$ opérations
\end{itemize}

\end{frame}

\begin{frame}{File double}
Double ended-queue (deque)

\bigskip

\begin{itemize}
\item Généralisation de la pile et de la file
\item Collection ordonnée d'objets offrant la possibilité
\begin{itemize}
\item d'insérer un nouvel objet \alert{avant le premier} ou \alert{après le dernier}
\item d'extraire le \alert{premier} ou le \alert{dernier} objet
\end{itemize}
\item Interface:
\begin{itemize}
\item $\proc{insert-first}(Q,x)$: ajoute $x$ au début de la file double
\item $\proc{insert-last}(Q,x)$: ajoute $x$ à la fin de la file double
\item $\proc{remove-first}(Q)$: extrait l'objet situé en première position
\item $\proc{remove-last}(Q)$: extrait l'objet situé en dernière position
\item \ldots
\end{itemize}
\item Application: équilibrage de la charge d'un serveur
\end{itemize}

\end{frame}

\begin{frame}{Implémentation à l'aide d'une liste doublement liée}
\begin{itemize}
\item A l'aide d'une liste double liée
\item Soit la file double $Q$:
\begin{itemize}
\item $Q.head$ pointe vers un élément \alert{sentinelle} en début de liste
\item $Q.tail$ pointe vers un élément \alert{sentinelle} en fin de liste
\item $Q.size$ est la taille courante de la liste
\end{itemize}

\centerline{\includegraphics[width=10cm]{Figures/04-dequeimplem.pdf}}

\item Les sentinelles ne contiennent pas de données. Elles permettent
  de simplifier le code (pour un coût en espace constant).
\end{itemize}

\bigskip

{\it Exercice: implémentation de la file double sans
  sentinelles, implémentation de la file simple avec sentinelle}

\end{frame}

\begin{frame}{Implémentation à l'aide d'une liste doublement liée}
\begin{center}
      \begin{small}\hfill
      \fcolorbox{white}{Lightgray}{%
        \begin{codebox}
          \Procname{$\proc{insert-first}(Q,x)$}
          \li $\attrib{x}{prev}\gets \attrib{Q}{head}$
          \li $\attrib{x}{next}\gets \attrib{Q}{head}.\id{next}$
          \li $\attrib{Q}{head}.\id{next}.\id{prev} \gets \id{x}$
          \li $\attrib{Q}{head}.\id{next}\gets \id{x}$
          \li $\attrib{Q}{size}\gets \attrib{Q}{size}+1$
        \end{codebox}}\hfill
\fcolorbox{white}{Lightgray}{%
        \begin{codebox}
          \Procname{$\proc{insert-last}(Q,x)$}
          \li $\attrib{x}{prev}\gets \attrib{Q}{tail}.\id{prev}$
          \li $\attrib{x}{next}\gets \attrib{Q}{tail}$
          \li $\attrib{Q}{tail}.\id{prev}.\id{next} \gets \id{x}$
          \li $\attrib{Q}{head}.\id{prev}\gets \id{x}$
          \li $\attrib{Q}{size}\gets \attrib{Q}{size}+1$
        \end{codebox}}\hfill~

      \bigskip
      \fcolorbox{white}{Lightgray}{%
        \begin{codebox}
          \Procname{$\proc{remove-first}(Q)$}
          \li \If $(\attrib{Q}{size}\isequal 0)$
          \li \Then \Error\End
          \li $x\gets \attrib{Q}{head}.\id{next}$
          \li $\attrib{Q}{head}.\id{next}\gets \attrib{Q}{head}.\id{next}.\id{next}$
          \li $\attrib{Q}{head}.\id{next}.\id{prev}\gets \attrib{Q}{head}$
          \li $\attrib{Q}{size}\gets \attrib{Q}{size}-1$
          \li \Return $x$
        \end{codebox}}
      \hfill
      \fcolorbox{white}{Lightgray}{%
        \begin{codebox}
          \Procname{$\proc{remove-last}(Q)$}
          \li \If $(\attrib{Q}{size}\isequal 0)$
          \li \Then \Error\End
          \li $x\gets \attrib{Q}{tail}.\id{prev}$
          \li $\attrib{Q}{tail}.\id{prev}\gets \attrib{Q}{tail}.\id{prev}.\id{prev}$
          \li $\attrib{Q}{tail}.\id{prev}.\id{next}\gets \attrib{Q}{head}$
          \li $\attrib{Q}{size}\gets \attrib{Q}{size}-1$
          \li \Return $x$
        \end{codebox}}
\end{small}
    \end{center}

Complexité $O(1)$ en temps et $O(n)$ en espace pour $n$ opérations.

\end{frame}

\section{Liste}

\begin{frame}{Liste}

\begin{itemize}
\item Ensemble dynamique d'objets ordonnés accessibles \alert{relativement}
  les uns aux autres, sur base de leur position
\item Généralise toutes les structures vues précédemment
\item Interface:
\begin{itemize}
\item Les fonctions d'une liste double (insertion et retrait en début et fin de liste)
\item $\proc{Insert-Before}(L,p,x)$: insére $x$ avant $p$ dans la liste
\item $\proc{Insert-After}(L,p,x)$: insère $x$ après $p$ dans la liste
\item $\proc{Remove}(L,p)$: retire l'élement à la position $p$
\end{itemize}

\bigskip

\item Implémentation similaire à la file double, à l'aide d'une liste doublement liée (avec sentinelles)
\end{itemize}

\end{frame}

\begin{frame}{Implémentation à l'aide d'une liste doublement liée}


\begin{columns}
\begin{column}{5cm}
\centerline{\includegraphics[width=6cm]{Figures/04-listelisteliee.pdf}}

\bigskip

\begin{center}
  \begin{small}\hfill
      \fcolorbox{white}{Lightgray}{%
        \begin{codebox}
          \Procname{$\proc{insert-before}(L,p,x)$}
          \li $\attrib{x}{prev}\gets \attrib{p}{prev}$
          \li $\attrib{x}{next}\gets \id{p}$
          \li $\attrib{p}{prev}.\id{next} \gets \id{x}$
          \li $\attrib{p}{prev} \gets \id{x}$
          \li $\attrib{L}{size}\gets \attrib{L}{size}+1$
        \end{codebox}}
  \end{small}
\end{center}
\end{column}
\begin{column}{5cm}
\begin{center}

\fcolorbox{white}{Lightgray}{%
        \begin{codebox}
          \Procname{$\proc{remove}(L,p)$}
          \li $\attrib{p}{prev}.\id{next} \gets \attrib{p}{next}$
          \li $\attrib{p}{next}.\id{prev} \gets \attrib{p}{prev}$
          \li $\attrib{L}{size}\gets \attrib{L}{size}-1$
          \li \Return $p$
        \end{codebox}}

\bigskip

\fcolorbox{white}{Lightgray}{%
        \begin{codebox}
          \Procname{$\proc{insert-after}(L,p,x)$}
          \li $\attrib{x}{prev}\gets \id{p}$
          \li $\attrib{x}{next}\gets \attrib{p}{next}$
          \li $\attrib{p}{next}.\id{prev} \gets \id{x}$
          \li $\attrib{p}{next} \gets \id{x}$
          \li $\attrib{L}{size}\gets \attrib{L}{size}+1$
        \end{codebox}}\hfill~
\end{center}
\end{column}
\end{columns}

Complexité $O(1)$ en temps et $O(n)$ en espace pour $n$ opérations.
%\note{\centerline{\includegraphics[width=8cm]{Figures/04-insertionliste.pdf}}}
\end{frame}

\section{Vecteur}

\begin{frame}{Vecteur}

\begin{itemize}
\item Ensemble dynamique d'objets occupant des rangs entiers
  successifs, permettant la consultation, le remplacement, l'insertion
  et la suppression d'éléments à des rangs arbitraires
\item Interface
\begin{itemize}
\item $\proc{Elem-At-Rank}(V,r)$ retourne l'élément au rang $r$ dans $V$.
\item $\proc{Replace-At-Rank}(V,r,x)$ remplace l'élément situé au rang $r$ par $x$ et retourne cet objet.
\item $\proc{Insert-At-Rank}(V,r,x)$ insère l'élément $x$ au rang $r$, en augmentant le rang des objets suivants.
\item $\proc{Remove-At-Rank}(V,r)$ extrait l'élément situé au rang $r$ et le retire de $r$, en diminuant le rang des objets suivants.
\item $\proc{Vector-Size}(V)$ renvoie la taille du vecteur.
\end{itemize}
\item Applications: tableau dynamique, gestion des éléments d'un menu,\ldots
\item Implémentation: liste liée, tableau extensible\ldots
\end{itemize}

\note{Inconvénient des structures précédentes: il faut parcourir la liste pour retrouver le kième élément ($O(1)$ dans un tableau\\

\bigskip

Avantage: espace mémoire $O(n)$.

\bigskip

Comment combiner les deux ?
}
\end{frame}

\begin{frame}{Implémentation par un tableau extensible}

\begin{itemize}
\item Les éléments sont stockés dans un tableau extensible $\attrib{V}{A}$ de taille initiale $\attrib{V}{c}$.
\item En cas de dépassement, la capacité du tableau est \alert{doublée}.
\item $\attrib{V}{n}$ retient le nombre de composantes.
\item Insertion et suppression:

\begin{center}
\begin{footnotesize}
\fcolorbox{white}{Lightgray}{%
  \begin{codebox}
    \Procname{$\proc{Insert-At-Rank}(V,r,x)$}
    \li \If $\attrib{V}{n}\isequal \attrib{V}{c}$
    \li \Then $\attrib{V}{c}\gets 2\cdot\attrib{V}{c}$
    \li  $W\gets $ ``Tableau de taille $\attrib{V}{c}$''
    \li  \For $i\gets 1 \To n$
    \li \Do $W[i]\gets V[i]$ \End
    \li $\attrib{V}{A}\gets W$\End
    \li \For $i\gets \attrib{V}{n} \Downto r$
    \li \Do $\attrib{V}{A}[i+1]\gets \attrib{V}{A}[i]$ \End
    \li $\attrib{V}{A}[r]\gets x$
    \li $\attrib{V}{n}\gets \attrib{V}{n}+1$
  \end{codebox}}
~~~~~
\fcolorbox{white}{Lightgray}{%
  \begin{codebox}
    \Procname{$\proc{Remove-At-Rank}(V,r)$}
    \li $tmp\gets \attrib{V}{A}[r]$
    \li \For $i\gets r \To \attrib{V}{n}-2$ \Do
    \li $\attrib{V}{A}[i]\gets \attrib{V}{A}[i+1]$
    \End
    \li $\attrib{V}{n}\gets \attrib{V}{n}-1$
    \li \Return $tmp$
  \end{codebox}}
\end{footnotesize}
\end{center}


\end{itemize}

\end{frame}

\begin{frame}{Complexité en temps}\label{sec05:amortie}
\begin{itemize}
%\item Accès à un élément, remplacement de sa valeur: $O(1)$
\item $\proc{Insert-At-Rank}$:
\begin{itemize}
\item $O(n)$ pour une opération individuelle, où $n$ est le nombre de composantes du vecteur
\item $O(n^2)$ pour $n$ opérations d'insertion en \alert{début} de vecteur
\item $O(n)$ pour $n$ opérations d'insertion en \alert{fin} de vecteur
\end{itemize}
\item Justification:
\begin{itemize}
\item Si la capacité du tableau passe de $c_0$ à $2^k c_0$ au cours des $n$ opérations, alors le coût des transferts entre tableaux s'élève à
$$c_0+2 c_0+\ldots+2^{k-1} c_0=(2^k-1) c_0.$$
Puisque $2^{k-1} c_0 < n \leq 2^k c_0$, ce coût est \alert{$O(n)$}.
\item On dit que le \alert{coût amorti} par opération est $O(1)$
\item Si on avait élargi le tableau avec un incrément constant $m$, le coût aurait été
$$c_0+(c_0+m)+(c_0+2m)+\ldots+(c_0+(k-1)m)=kc_0+\frac{k(k-1)}{2}m.$$
Puisque $c_0+(k-1)m<n\leq c_0+km$, ce coût aurait donc été \alert{$O(n^2)$}.
\end{itemize}
\end{itemize}
\end{frame}

\begin{frame}{Complexité en temps}
\begin{itemize}
%\item Accès à un élément, remplacement de sa valeur: $O(1)$
\item $\proc{Remove-At-Rank}$:
\begin{itemize}
\item $O(n)$ pour une opération individuelle, où $n$ est le nombre de composantes du vecteur
\item $O(n^2)$ pour $n$ opérations de retrait en \alert{début} de vecteur
\item $O(n)$ pour $n$ opérations de retrait en \alert{fin} de vecteur
\end{itemize}
\item Remarque: Un tableau circulaire permettrait d'améliorer l'efficacité des opérations d'ajout et de retrait en début de vecteur.
\end{itemize}

\end{frame}

\section{File à priorité}

% Implémentation par un tableau, implémentation par un tas.
% Tas de Fibonacci ou binomial pour la fusion de deux tas ?

\begin{frame}{File à priorité}
\begin{itemize}
\item Ensemble dynamique d'objets classés par ordre de \alert{priorité}
\begin{itemize}
\item Permet d'extraire un objet possédant la plus grande priorité
\item En pratique, on représente les priorités par les clés
\item Suppose un ordre total défini sur les clés
\end{itemize}
\item Interface:
\begin{itemize}
\item $\proc{Insert}(S,x)$: insère l'élément $x$ dans $S$.
\item $\proc{Maximum}(S)$: renvoie l'élément de $S$ avec la plus grande clé.
\item $\proc{Extract-Max}(S)$: supprime et renvoie l'élément de $S$ avec la plus grande clé.

\end{itemize}
\item Remarques:
\begin{itemize}
\item Extraire l'élément de clé maximale ou minimale sont des problèmes équivalents
\item La file FIFO est une file à priorité où la clé correspond à l'ordre d'arrivée des élements.
\end{itemize}
\item Application: gestion des jobs sur un ordinateur partagé
\end{itemize}
\note{Ordre total=relation binaire avec les propriétés suivantes:
\begin{itemize}
\item totalité: $\forall p_1, p_2: p_1\leq p_2 \mbox{ ou }p_2\leq p_1$
\item antisymétrie: $\forall p_1,p_2: p_1\leq p_2 \mbox{ et } p_2\leq p_1 \Rightarrow p_1=p_2$
\item Transitivité...
\end{itemize}

\bigskip

File FIFO est un cas particulier ou la priorité est représenté par l'ordre d'arrivé des éléments
}
\end{frame}

\begin{frame}{Implémentations}
\begin{itemize}
\item Implémentation à l'aide d'un tableau statique
\begin{itemize}
\item $Q$ est un tableau statique de taille fixée $\attrib{Q}{length}$.
\item Les éléments de $Q$ sont triés par ordre \alert{croissant} de clés. $\attrib{Q}{top}$ pointe vers le dernier élément.
\item Complexité en temps: extraction en $O(1)$ et insertion en $O(n)$ où $n$ est la taille de la file
\item Complexité en espace: $O(1)$
\end{itemize}
\item Implémentation à l'aide d'une liste liée
\begin{itemize}
\item $Q$ est une liste liée où les éléments sont triés par ordre \alert{décroissant} de clés
\item Complexité en temps: extraction en $O(1)$ et insertion en $O(n)$ où $n$ est la taille de la file
\item Complexité en espace: $O(n)$
\end{itemize}
\item Peut-on faire mieux?
\end{itemize}

{\it Exercice: comment obtenir $O(1)$ pour l'insertion et $O(n)$ pour l'extraction?}
 \note{Leur demander pourquoi on trie le tableau par ordre croissant de clé et pas par ordre décroissant: parce que retirer le premier élément veut dire qu'on doit décaler tous les autres et donc l'opération serait $O(n)$}
\end{frame}

\begin{frame}{Implémentation à l'aide d'un tas}
\begin{itemize}
\item La file est implémentée à l'aide d'un tas(-max) (voir slide \pageref{sec:03tas})
\item Accès et extraction du maximum:

\bigskip

\begin{center}
\begin{small}
\fcolorbox{white}{Lightgray}{
  \begin{codebox}
    \Procname{$\proc{Heap-Maximum}(A)$}
    \li \Return $A[1]$
\end{codebox}}~~~~~\fcolorbox{white}{Lightgray}{
  \begin{codebox}
    \Procname{$\proc{Heap-Extract-Max}(A)$}
    \li \If $\attrib{A}{heap-size}<1$
    \li \Then \Error ``heap underflow''
    \End
    \li $max \gets A[1]$
    \li $A[1]\gets A[\attrib{A}{heap-size}]$
    \li $\attrib{A}{heap-size}\gets \attrib{A}{heap-size}-1$
    \li $\proc{Max-heapify}(A,1)$ \Comment reconstruit le tas
    \li \Return $max$
\end{codebox}}
\end{small}
\end{center}

\bigskip

\item Complexité: $O(1)$ et $O(\log n)$ respectivement (voir chapitre 3)
\end{itemize}

\end{frame}

\begin{frame}{Implémentation à l'aide d'un tas}

\begin{center}
\includegraphics[width=8cm]{Figures/04-fileprio.pdf}
\end{center}

\end{frame}

\begin{frame}{Implémentation à l'aide d'un tas: insertion}
\begin{itemize}
\item $\proc{Increase-Key}(S,x,k)$ augmente la valeur de la clé de $x$
  à $k$ (on suppose que $k\geq$ à la valeur courante de la clé de $x$).

\bigskip

\begin{center}
\begin{small}
\fcolorbox{white}{Lightgray}{
  \begin{codebox}
    \Procname{$\proc{Heap-Increase-Key}(A,i)$}
    \li \If $key<A[i]$
    \li \Then \Error ``new key is smaller than current key'' \End
    \li $A[i]\gets key$
    \li \While $i>1$ and $A[\proc{Parent}(i)]<A[i]$
    \li \Do $\id{swap}(A[i],A[\proc{Parent}(i)])$
    \li     $i\gets \proc{Parent}(i)$
    \End
\end{codebox}}
\end{small}
\end{center}

\bigskip

\item Complexité: $O(\log n)$ (la longueur de la branche de la racine à $i$ étant $O(\log n)$ pour un tas de taille $n$).
\end{itemize}
\note{Montrer un exemple sur le slide précédent: augmenter le poids de $A[9]$ de 4 à 15: on échange les noeuds 4 et 9, puis les noeuds 4 et 2.}
\end{frame}

\begin{frame}{Implémentation à l'aide d'un tas: insertion}
\begin{itemize}
\item Pour insérer un élément avec une clé $key$:
\begin{itemize}
\item l'insérer à la dernière position sur le tas avec une clé $-\infty$,
\item augmenter sa clé de $-\infty$ à $key$ en utilisant la procédure précédente
\end{itemize}

\bigskip

\begin{center}
\begin{small}
\fcolorbox{white}{Lightgray}{
  \begin{codebox}
    \Procname{$\proc{Heap-Insert}(A,key)$}
    \li $\attrib{A}{heap-size}\gets \attrib{A}{heap-size}+1$
    \li $A[\attrib{A}{heap-size}]\gets -\infty$
    \li $\proc{Heap-Increase-Key}(A,\attrib{A}{heap-size},key)$
\end{codebox}}
\end{small}
\end{center}

\bigskip

\item Complexité: $O(\log n)$.

\bigskip

\item[$\Rightarrow$] Implémentation d'une file à priorité par un tas: $O(\log n)$ pour l'extraction et l'insertion.
\end{itemize}
\note{}
\end{frame}

\begin{frame}{Ce qu'on a vu}

\begin{itemize}
\item Quelques structures de données classiques et différentes implémentations pour chacune d'elles
\begin{itemize}
\item Pile
\item Liste
\item Files simples, doubles et à priorité
\item Vecteurs
\end{itemize}
\item Structures de type liste liée
\end{itemize}

\end{frame}

\begin{frame}{Ce qu'on n'a pas vu}

\begin{itemize}
\item Structure de type séquence: hybride entre le vecteur et la
  liste
\item Notion d'itérateur
\item Tas binomial: alternative au tas binaire qui permet la fusion
  rapide de deux tas
\item Evolution dynamique de la taille d'un tas (analyse amortie)
\item \ldots
\end{itemize}

\end{frame}


\part{Dictionnaires}

% Superbe site web avec les differents algorithmes

% http://www.sorting-algorithms.com/


% Plan

% recherche séquentielle
% recherche binaire (dichotomique) -> problème insertion reste lente
% binary search trees -> insertion et recherche rapide mais seulement en moyenne
% RB trees -> insertion et recherche rapide dans le pire cas
% table à accès direct: tout en $O(1)$ mais prend beaucoup de mémoire
% table hash: tout en ordre $O(1)$ mais structure non dynamique

\begin{frame}{Plan}

\tableofcontents[hideallsubsections]

\end{frame}

\section{Introduction}

\begin{frame}{Dictionnaires}
\begin{itemize}
\item Définition: un \alert{dictionnaire} est un ensemble dynamique
  d'objets avec des clés comparables qui supportent les opérations
  suivantes:
\begin{itemize}
\item $\proc{Search}(S,k)$ retourne un pointeur $x$ vers un élément dans $S$ tel que $\attrib{x}{key}=k$, ou $\const{NIL}$ si un tel élément n'appartient pas à $S$.
\medskip
\item $\proc{Insert}(S,x)$ insère l'élément $x$ dans le dictionnaire $S$. Si un élément de même clé se trouve déjà dans le dictionnaire, on met à jour sa valeur
\medskip
\item $\proc{Delete}(S,x)$ retire l'élément $x$ de $S$. Ne fait rien si l'élément n'est pas dans le dictionnaire.
\end{itemize}
\medskip
\item Pour faciliter la recherche, on peut supposer qu'il existe un ordre total sur les clés.
\end{itemize}
\end{frame}

\begin{frame}{Dictionnaires}
\begin{itemize}
\item Deux objectifs en général:
\begin{itemize}
\item minimiser le coût pour l'insertion et l'accès au données
\item minimiser l'espace mémoire pour le stockage des données
\end{itemize}
\item Exemples d'applications:
\begin{itemize}
\item Table de symboles dans un compilateur
\item Table de routage d'un DNS
\item \ldots
\end{itemize}
\item Beaucoup d'implémentations possibles
\end{itemize}
\note{Ici:
\begin{itemize}
\item Pas d'ordre sur les clés (du moins, on ne l'exploite pas)
\begin{itemize}
\item tableau à accès direct
\item table de hachage
\end{itemize}
\item Ordre sur les clés
\begin{itemize}
\item tableau trié
\item arbre binaire de recherche
\end{itemize}
\end{itemize}
}
\end{frame}

\begin{frame}{Liste liée}
Première solution:
\begin{itemize}
\item On stocke les paires clé-valeur dans une liste liée
\item Recherche:
\begin{center}
{\footnotesize
\fcolorbox{white}{Lightgray}{%
      \begin{codebox}
        \Procname{$\proc{List-Search}(L,k)$}
        \li $x\gets \attrib{L}{head}$
        \li \While $x\ne \const{NIL}\wedge\attrib{x}{key}\ne k$
        \li \Do $x\gets \attrib{x}{next}$
            \End
        \li \Return $x$
      \end{codebox}}
%% \fcolorbox{white}{Lightgray}{%
%%       \begin{codebox}
%%         \Procname{$\proc{List-Delete}(L,x)$}
%%         \li \If $\attrib{x}{prev}\ne \const{NIL}$
%%         \li \Then $\attrib{x}{prev}.\id{next}\gets \attrib{x}{next}$
%%         \li \Else $\attrib{L}{head}\gets \attrib{x}{next}$\End
%%         \li \If $\attrib{x}{next}\ne \const{NIL}$
%%         \li \Then $\attrib{x}{next}.\id{prev}\gets \attrib{x}{prev}$ \End
%%       \end{codebox}}
}
\end{center}
\item Insertion (resp. Suppression)
\begin{itemize}
\item On recherche la clé dans la liste
\item Si elle existe, on remplace la valeur (resp. on la supprime)
\item Si elle n'existe pas, on la place en début de liste (resp. on ne fait rien)
\end{itemize}
\item Complexité au pire cas\hfill{\it (meilleur cas ?)}
% remplacer ça par une table
\begin{itemize}
\item Insertion: $O(N)$
\item Recherche: $O(N)$
\item Suppression: $O(N)$
\end{itemize}
%\item Peut-on améliorer la recherche ?
\end{itemize}
\end{frame}

\begin{frame}{Vecteur trié}

Deuxième solution:
\begin{itemize}
\item On suppose qu'il existe un ordre total sur les clés
\item On stocke les éléments dans un \alert{vecteur} qu'on maintient trié
\item Recherche dichotomique (approche ``diviser-pour-régner'')

\begin{center}
\begin{small}
\fcolorbox{white}{Lightgray}{%
  \begin{codebox}
    \Procname{$\proc{Binary-Search}(V,k,low,high)$}
    \li \If $low>high$
    \li \Then \Return \const{NIL} \End
    \li $mid\gets \lfloor (low+high)/2\rfloor$
    \li $x\gets \proc{Elem-At-Rank}(V,mid)$
    \li \If $k\isequal x.key$
    \li \Then \Return $x$
    \li \ElseIf $k>x.key$
    \li \Then \Return $\proc{Binary-Search}(V,k,mid+1,high)$
    \li \Else \Return $\proc{Binary-Search}(V,k,low,mid-1)$
    \End
\end{codebox}}
\end{small}
\end{center}
Complexité: $O(\log n)$
\end{itemize}
\note{Comment est-ce que vous feriez l'insertion? comme insertion sort ?}
\end{frame}

\begin{frame}{Vecteur trié}
\begin{itemize}
\item Insertion: recherche de la position par $\proc{Binary-Search}$ puis insertion dans le vecteur par $\proc{Insert-At-Rank}$ (=décalage des éléments vers la droite).
\item Suppression: recherche puis suppression par $\proc{Remove-At-Rank}$ (=décalage des éléments vers la gauche).
\item Complexité au pire cas\hfill{\it (meilleur cas ?)}
\begin{itemize}
\item Insertion: $O(N)$ (on doit décaler les éléments à droite de la clé)
\item Recherche: $O(\log N)$ (recherche dichotomique)
\item Suppression: $O(N)$ (on doit décaler les éléments à gauche de la clé)
\end{itemize}
(Si le vecteur est implémenté par un tableau extensible !)
\end{itemize}

\note{Est-ce que ça ne vous donne pas une autre idée d'algorithme de tri ? Binary insertion sort: complexité ? Intéressant si le coût d'une comparaison est plus grand que le coût d'un swap: example si la comparaison nécessite de faire un calcul couteux (log(n) comparaison dans un cas, n dans l'aute)}

\end{frame}

\begin{frame}{Dictionnaires: jusqu'ici}

  \begin{center}\small
    \def\arraystretch{1.5}\renewcommand{\tabcolsep}{1mm}
    \begin{tabular}{@{}lcccccc@{}}
    &\multicolumn{3}{c}{\emph{Pire cas}} & \multicolumn{3}{c}{\emph{En moyenne}}\\
    \emph{Implémentation}& \proc{Search} & \proc{Insert} & \proc{Delete} & \proc{Search} & \proc{Insert} & \proc{Delete}\\
    \hline\hline
    Liste &$\Theta(n)$&$\Theta(n)$&$\Theta(n)$&$\Theta(n)$&$\Theta(n)$&$\Theta(n)$\\
    \hline
    Vecteur trié&$\Theta(\log n)$&$\Theta(n)$&$\Theta(n)$&$\Theta(\log n)$&$\Theta(n)$&$\Theta(n)$\\
    \hline\hline
  \end{tabular}
  \end{center}

\bigskip

Peut-on obtenir à la fois une insertion et une recherche ``efficaces''?

\end{frame}


\section{Arbres binaires de recherche}

\begin{frame}{Plan}

\tableofcontents[currentsection]

\end{frame}

\subsection{Type de données abstrait pour un arbre}

\begin{frame}{Type de données abstrait pour un arbre}
\begin{itemize}
\item Principe:
\begin{itemize}
\item Des données sont associées aux n\oe uds d'un arbre
\item Les n\oe uds sont accessibles les uns par rapport aux autres selon leur position dans l'arbre
\end{itemize}
\item Interface: Pour un arbre $T$ et un n\oe ud $n$
\begin{itemize}
\item $\proc{Parent}(T,n)$: renvoie le parent d'un n\oe ud $n$ (signale une erreur si $n$ est la racine)
\item $\proc{Children}(T,n)$: renvoie une structure de données (ordonnée
  ou non) contenant les fils du n\oe ud $n$ (exemple: une liste)
\item $\proc{isRoot}(T,n)$: renvoie vrai si $n$ est la racine de l'arbre
\item $\proc{isInternal}(T,n)$: renvoie vrai si $n$ est un n\oe ud interne
\item $\proc{isExternal}(T,n)$: renvoie vrai si $n$ est un n\oe ud externe
\item $\proc{GetData}(T,n)$: renvoie les données associées au n\oe ud $n$
\item $\proc{Left}(T,n)$, $\proc{Right}(T,n)$: renvoie les fils gauche et droit de $n$ (pour un arbre binaire)
\item $\proc{Root}(T)$: renvoie le n\oe ud racine de l'arbre
\item $\proc{Size}(T)$: renvoie le nombre de n\oe uds de l'arbre
\item \ldots
\end{itemize}
\end{itemize}
\end{frame}

\begin{frame}{Exemples d'opération sur un arbre}
\begin{itemize}
\item Calcul de la profondeur d'un n\oe ud

\begin{center}\small
\fcolorbox{white}{Lightgray}{%
  \begin{codebox}
    \Procname{$\proc{Depth-rec}(T,n)$}
    \li \If $\proc{isRoot}(T,n)$
    \li \Then \Return 0\End
    \li \Return $1+\proc{Depth-rec}(T,\proc{Parent}(T,n))$
\end{codebox}}
\end{center}
\item Version itérative

\begin{center}\small
\fcolorbox{white}{Lightgray}{%
  \begin{codebox}
    \Procname{$\proc{Depth-iter}(T,n)$}
    \li $d\gets 0$
    \li \While \textbf{not} $\proc{isRoot}(T,n)$ \Do
    \li $d\gets d+1$
    \li $n\gets \proc{Parent}(T,n)$ \End
    \li \Return $d$
\end{codebox}}
\end{center}
\item Complexité en temps: $O(n)$, où $n$ est la taille de l'arbre (si
  les opérations de l'interface sont $O(1)$)
\end{itemize}
\note{Complexité en espace: $O(n)$ pour la version récursive}
\end{frame}

\begin{frame}{Exemples d'opération sur un arbre}
\begin{itemize}
\item Calcul de la hauteur de l'arbre

\begin{center}\small
\fcolorbox{white}{Lightgray}{%
  \begin{codebox}
    \Procname{$\proc{Height}(T,n)$}
    \li \If $\proc{isExternal}(T,n)$
    \li \Then \Return 0 \End
    \li $h\gets 0$
    \li \For \textbf{each} $n2$ \textbf{in} $\proc{Children}(T,n)$
    \li \Do $h\gets \max(h,\proc{Height}(T,n2))$\End
    \li \Return $h+1$
\end{codebox}}
\end{center}

\item Complexité en temps: $O(n)$, où $n$ est la taille de l'arbre (si
  les opérations de l'interface sont $O(1)$)
\end{itemize}

\end{frame}

\begin{frame}{Implémentation d'un arbre binaire}

Première solution: \alert{numérotation de niveaux}
\begin{itemize}
\item L'arbre est représenté par un vecteur (ou un tableau)
\item Chaque position dans l'arbre est associée à un rang particulier:
\begin{itemize}
\item La racine est en position 1
\item Si un n\oe ud est au rang $r$, son successeur gauche est au rang $2r$, son successeur droit au rang $2r+1$
\end{itemize}
\item Si l'arbre binaire n'est pas un arbre binaire complet, le vecteur contiendra des trous (qu'il faudra pouvoir identifier)
\item Complexité en temps des opérations: $O(1)$
\item Complexité en espace: $O(2^n)$ pour un arbre de $n$ n\oe uds ($\Theta(n)$ pour un arbre binaire complet)
\end{itemize}

\note{Au pire, l'arbre aura une profondeur $h=n$, et donc il faudra pouvoir référencer $O(2^h)=O(2^n)$ éléments

\bigskip

On peut représenter un arbre k-aire avec le même truc. kr, kr+1, etc. pour les fils. Suppose qu'on connaisse le kmax}

\end{frame}

\begin{frame}{Implémentation d'un arbre binaire}

Deuxième solution: \alert{structure liée}
\begin{itemize}
\item Principe: on retient pour chaque n\oe ud $n$ de l'arbre:
\begin{itemize}
\item Un champ de données ($\attrib{n}{data}$)
\item Un pointeur vers son n\oe ud parent ($\attrib{n}{parent}$)
\item Un pointeur vers ses fils gauche et droit ($\attrib{n}{left}$ et $\attrib{n}{right}$)
\end{itemize}
\item Complexité en temps des opérations: $O(1)$
\item Complexité en espace: $\Theta(n)$ pour $n$ n\oe uds
\end{itemize}

\bigskip

{\it (généralise la notion de liste liée)}

\note{Représentation par un tableau (left, right, parent, etc.). Avantage: plus compact mais nécessite un tableau de taille fixée.}
\end{frame}

\begin{frame}{Implémentation d'un arbre quelconque}

\begin{itemize}
\item Même structure liée que pour un arbre binaire
\item Mais on remplace $\attrib{n}{left}$ et $\attrib{n}{right}$ par un pointeur $\attrib{n}{children}$ vers un ensemble dynamique
\item Le type d'ensemble dynamique (vecteur, liste, \ldots) dépendra des opérations devant être effectuées
\end{itemize}

\centerline{\includegraphics[width=9cm]{Figures/05-arbrequelconque.pdf}}

{\it (on peut aussi représenter un arbre quelconque par un arbre binaire)}
\note{Représentation fils-gauche, frère-droit}
\end{frame}

\begin{frame}{Implémentation des arbres binaires}

Implémentation des arbres binaires dans le reste de ce cours:
\begin{columns}
\begin{column}{7.5cm}
\begin{itemize}
\item $T$ représente l'arbre, qui consiste en un ensemble de n\oe uds
\item $T.root$ est le n\oe ud racine de l'arbre $T$
\item N\oe ud $x$
\begin{itemize}
\item $\attrib{x}{parent}$ est le parent du n\oe ud $x$
\item $\attrib{x}{key}$ est la clé stockée au n\oe ud $x$
\item $\attrib{x}{left}$ est le fils de  gauche du n\oe ud $x$
\item $\attrib{x}{right}$ est le fils de droite du n\oe ud $x$
\end{itemize}
\end{itemize}
\end{column}
\begin{column}{4cm}
\begin{center}
\includegraphics[width=3cm]{Figures/05-bstnode.pdf}
\end{center}
\end{column}
\end{columns}

\bigskip
Pour simplifier les notations, nos fonctions seront implémentées directement sur base de cette implémentation (et pas de l'interface générale)

\note{Le passage à une autre implémentation devrait être relativement aisé}

\end{frame}

\begin{frame}{Parcours d'arbres (binaire)}

\begin{itemize}
\item Un parcours d'arbre est une façon d'\alert{ordonner} les n\oe uds d'un arbre afin de les parcourir
\item Différents types de parcours:
\begin{itemize}
\item Parcours en profondeur:
\begin{itemize}
\item Infixe (en ordre)
\item Préfixe (en préordre)
\item Suffixe (en postordre)
\end{itemize}
\item Parcours en largeur
\end{itemize}
\end{itemize}
\note{Parcours en largeur d'un tas ? $\Rightarrow$ on passe simplement séquentiellement sur les éléments}
\end{frame}

\begin{frame}{Parcours infixe}

\begin{center}
\includegraphics[width=5cm]{Figures/05-onebst2.pdf}
\bigskip

$\Rightarrow \langle A, B, D, F, H, K\rangle$
\end{center}

\begin{itemize}
\item Parcours infixe (en ordre): Chaque n\oe ud est visité \alert{après} son fils gauche et \alert{avant} son fils droit

\bigskip
\begin{center}
\begin{small}
\fcolorbox{white}{Lightgray}{%
  \begin{codebox}
    \Procname{$\proc{Inorder-Tree-Walk}(x)$}
    \li \If $x\neq \const{NIL}$
    \li \Then $\proc{Inorder-Tree-Walk}(\attrib{x}{left})$
    \li print $\attrib{x}{key}$
    \li $\proc{Inorder-Tree-Walk}(\attrib{x}{right})$
    \End
\end{codebox}}
\end{small}
\end{center}

\end{itemize}

\end{frame}

\begin{frame}{Parcours préfixe}

\begin{center}
\includegraphics[width=6cm]{Figures/05-onebst2.pdf}
\bigskip

$\Rightarrow \langle F, B, A, D, H, K\rangle$
\end{center}

\begin{itemize}
\item Parcours préfixe (en préordre): chaque n\oe ud est visité \alert{avant} ses fils

\bigskip
\begin{center}
\begin{small}
\fcolorbox{white}{Lightgray}{%
  \begin{codebox}
    \Procname{$\proc{Preorder-Tree-Walk}(x)$}
    \li \If $x\neq \const{NIL}$
    \li \Then print $\attrib{x}{key}$
    \li $\proc{Preorder-Tree-Walk}(\attrib{x}{left})$
    \li $\proc{Preorder-Tree-Walk}(\attrib{x}{right})$
    \End
\end{codebox}}
\end{small}
\end{center}

\end{itemize}

\end{frame}

\begin{frame}{Parcours postfixe}

\begin{center}
\includegraphics[width=6cm]{Figures/05-onebst2.pdf}
\bigskip

$\Rightarrow \langle A, D, B, K, H, F\rangle$
\end{center}

\begin{itemize}
\item Parcours postfixe (en postordre): chaque n\oe ud est visité \alert{après} ses fils

\bigskip
\begin{center}
\begin{small}
\fcolorbox{white}{Lightgray}{%
  \begin{codebox}
    \Procname{$\proc{Postorder-Tree-Walk}(x)$}
    \li \If $x\neq \const{NIL}$
    \li \Then $\proc{Postorder-Tree-Walk}(\attrib{x}{left})$
    \li $\proc{Postorder-Tree-Walk}(\attrib{x}{right})$
    \li print $\attrib{x}{key}$
    \End
\end{codebox}}
\end{small}
\end{center}

\end{itemize}

\end{frame}

\begin{frame}{Complexité des parcours}


Tous les parcours en profondeur sont $\Theta(n)$ en temps
\begin{itemize}
\item Soit $T(n)$ le nombre d'opérations pour un arbre avec $n$ n\oe uds
\item On a $T(n)=\Omega(n)$ (on doit au moins parcourir chaque n\oe ud).
\item Etant donné la récurrence, on a:
$$T(n)\leq T(n_L)+T(n-n_L-1) + d$$ où $n_L$ est le nombre de n\oe uds du sous-arbre à gauche et $d$ une constante
\item On peut prouver par induction que $T(n)\leq (c+d) n +c$ où $c=T(0)$.
\item $T(n)=\Omega(n)$ et $T(n)=O(n)$ $\Rightarrow$ $T(n)=\Theta(n)$
\end{itemize}

\note{Preuve par induction:
\centerline{\includegraphics[width=10cm]{Figures/05-proofinorder.pdf}}
}
\end{frame}

\begin{frame}{Parcours en largeur}

\begin{itemize}
\item Parcours en largeur: on visite le n\oe ud le plus proche de la racine qui n'a pas déjà été visité. Correspond à une visite de n\oe ud de profondeur 1, puis 2, \ldots.
\item<2> Implémentation à l'aide d'une file en $\Theta(n)$
\end{itemize}

%\bigskip

\begin{columns}
\begin{column}{5cm}
\begin{center}
\includegraphics[width=6cm]{Figures/05-onebst2.pdf}
\bigskip

$\Rightarrow \langle F,B,H,A,D,K\rangle$
\end{center}
\end{column}~~~~~~~
\begin{column}{5cm}
\only<2>{
\begin{center}
\begin{footnotesize}
\fcolorbox{white}{Lightgray}{%
  \begin{codebox}
    \Procname{$\proc{Breadth-Tree-Walk}(x)$}
    \li $Q\gets$''Empty queue''
    \li $\proc{Enqueue}(Q,x)$
    \li \While \textbf{not} $\proc{Queue-Empty}(Q)$
    \li \Do $y\gets\proc{Dequeue}(Q)$
    \li print $y.key$
    \li \If $y.left\neq \const{NIL}$
    \li \Then  $\proc{Enqueue}(Q,y.left)$ \End
    \li \If $y.right\neq \const{NIL}$
    \li \Then  $\proc{Enqueue}(Q,y.right)$ \End
    \End
\end{codebox}}
\end{footnotesize}
\end{center}
}
\end{column}
\end{columns}

\medskip

\only<2>{\emph{(Exercice: Implémenter les parcours en profondeur de manière non récursive)}}
\note{Solution: il faut utiliser une pile mais une solution simple peut être trouvée avec une comparaison de pointeurs (voir les solutions du bouquin)

\bigskip

Quid de l'espace: taille max de la file ?? ($O(n/2)$)

}
\end{frame}

\subsection{Arbre binaire de recherche}

\begin{frame}{Plan}

\tableofcontents[currentsection]

\end{frame}


\begin{frame}{Arbres binaires de recherche}

\begin{itemize}
\item Une structure d'arbre binaire implémentant un dictionnaire, avec
  des opérations en $O(h)$ où $h$ est la hauteur de l'arbre


\bigskip

\item Chaque n\oe ud de l'arbre binaire est associé à une clé
\item L'arbre satisfait à la propriété d'arbre binaire de recherche
\begin{itemize}
\item Soient deux n\oe uds $x$ et $y$.
\item Si $y$ est dans le sous-arbre de gauche de $x$, alors $y.key<x.key$
\item Si $y$ est dans le sous-arbre de droite de $x$, alors $y.key\geq x.key$
\end{itemize}
\end{itemize}

\centerline{\includegraphics[width=9cm]{Figures/05-arbresbinaires.pdf}}

\note{Structure qui implémente la recherche binaire qu'on a vue la semaine passée}
\end{frame}

\begin{frame}{Parcours d'un arbre binaire de recherche}

\begin{center}
\includegraphics[width=5cm]{Figures/05-onebst.pdf}
\bigskip

$\Rightarrow \langle 2, 5, 5, 6, 7, 8\rangle$
\end{center}

\begin{itemize}
\item Le parcours infixe d'un arbre binaire de recherche permet d'afficher les clés par ordre croissant

\bigskip
\begin{center}
\begin{small}
\fcolorbox{white}{Lightgray}{%
  \begin{codebox}
    \Procname{$\proc{Inorder-Tree-Walk}(x)$}
    \li \If $x\neq \const{NIL}$
    \li \Then $\proc{Inorder-Tree-Walk}(\attrib{x}{left})$
    \li print $\attrib{x}{key}$
    \li $\proc{Inorder-Tree-Walk}(\attrib{x}{right})$
    \End
\end{codebox}}
\end{small}
\end{center}

\end{itemize}

\end{frame}

\begin{frame}{Recherche dans un arbre binaire}
\begin{itemize}
\item Recherche binaire
\begin{center}
\begin{small}
\fcolorbox{white}{Lightgray}{%
  \begin{codebox}
    \Procname{$\proc{Tree-Search}(x,k)$}
    \li \If $x\isequal \const{NIL}$ or $k\isequal \attrib{x}{key}$
    \li \Then \Return $x$\End
    \li \If $k<\attrib{x}{key}$
    \li \Then \Return $\proc{Tree-Search}(\attrib{x}{left},k)$
    \li \Else \Return $\proc{Tree-Search}(\attrib{x}{right},k)$
\end{codebox}}

\medskip

Appel initial (à partir d'un arbre $T$)\\
\fcolorbox{white}{Lightgray}{%
$\proc{Tree-Search}(\attrib{T}{root},k)$
}
\end{small}
\end{center}

\bigskip

\item Complexité ? $T(n)=O(h)$, où $h$ est la hauteur de l'arbre
\item Pire cas: $h=n$
\end{itemize}

\end{frame}

\begin{frame}{Recherche dans un arbre binaire}
\begin{itemize}
\item $\proc{Tree-Search}$ est récursive terminale.
\item Version itérative
\begin{center}
\begin{small}
\fcolorbox{white}{Lightgray}{%
  \begin{codebox}
    \Procname{$\proc{Iterative-Tree-Search}(T,k)$}
    \li $x\gets \attrib{T}{root}$
    \li \While $x\neq \const{NIL}$ and $k\neq \attrib{x}{key}$
    \li \Do \If $k<\attrib{x}{key}$
    \li \Then $x\gets \attrib{x}{left}$
    \li \Else $x\gets \attrib{x}{right}$
    \End\End
    \li \Return x
\end{codebox}}
\end{small}
\end{center}

\bigskip

\end{itemize}
\note{Invariant: Si $k$ est dans l'arbre, elle se trouve dans le sous-arbre dont la racine est $x$

\bigskip

Faire le lien avec la recherche binaire}

\end{frame}

\begin{frame}{Clés maximale et minimale}
\begin{itemize}
\item Etant donné la propriété d'arbre binaire
\begin{itemize}
\item La clé minimale se trouve dans le n\oe ud le plus à gauche
\item La clé maximale se trouve dans le no\oe ud le plus à droite
\end{itemize}

\bigskip

\begin{center}
\fcolorbox{white}{Lightgray}{%
\begin{codebox}
          \Procname{$\proc{Tree-Minimum}(x)$}
          \li \While $\attrib{x}{left}\ne\const{NIL}$
          \li \Do $x\gets\attrib{x}{left}$
              \End
          \li \Return $x$
        \end{codebox}}~~~~~\fcolorbox{white}{Lightgray}{%
\begin{codebox}
          \Procname{$\proc{Tree-Maximum}(x)$}
          \li \While $\attrib{x}{right}\ne\const{NIL}$
          \li \Do $x\gets\attrib{x}{right}$
              \End
          \li \Return $x$
        \end{codebox}}
\end{center}

\bigskip

\item Complexité: $O(h)$, où $h$ est la hauteur de l'arbre.
\end{itemize}
\end{frame}

\begin{frame}{Successeur et prédécesseur}
\begin{itemize}
\item Etant donné un n\oe ud $x$, trouver le n\oe ud contenant la valeur de clé suivante (dans l'ordre)

\begin{center}
\includegraphics[width=7cm]{Figures/05-treesuccessor.pdf}

\bigskip

Ex: successeur de 15 $\rightarrow 17$, successeur de 4 $\rightarrow 6$.
\end{center}


\item Le successeur de $x$ est le minimum du sous-arbre de droite s'il existe
\item Sinon, c'est le premier ancêtre $a$ de $x$ tel que $x$ tombe dans le sous-arbre de gauche de $a$.
\end{itemize}
\end{frame}

\begin{frame}{Successeur et prédécesseur}

\begin{columns}
\begin{column}{5cm}
\begin{center}
\begin{small}
\fcolorbox{white}{Lightgray}{
\begin{codebox}
      \Procname{$\proc{Tree-Successor}(x)$}
      \li \If $\attrib{x}{right}\ne\const{NIL}$\Then
      \li \Return $\proc{Tree-Minimum}(\attrib{x}{right})$\End
      \li $y\gets\attrib{x}{parent}$
      \li \While $y\ne\const{NIL}$ and $x\isequal\attrib{y}{right}$
      \li \Do $x\gets y$
      \li     $y\gets\attrib{y}{parent}$
      \End
      \li \Return $y$
    \end{codebox}}
\end{small}
\end{center}
\end{column}
\begin{column}{5cm}
\begin{center}

  \bigskip

\bigskip

\includegraphics[width=5cm]{Figures/05-treesuccessor.pdf}
\end{center}
\end{column}
\end{columns}

\bigskip

Complexité: $O(h)$, où $h$ est la hauteur de l'arbre

\bigskip

\emph{(Exercice: $\proc{Tree-Predecessor}$)}

\note{Attention, algo assez tordu}

\end{frame}

\begin{frame}{Insertion}

\begin{center}
\includegraphics[width=8cm]{Figures/05-bstinsertion.pdf}
\end{center}

\begin{itemize}
\item Pour insérer $x$, on recherche la clé $\attrib{x}{key}$ dans l'arbre
\item Si on ne la trouve pas, on l'ajoute à l'endroit où la recherche s'est arrêtée.
\end{itemize}
\end{frame}

\begin{frame}{Insertion}

  \begin{columns}
    \begin{column}{5cm}
\begin{center}
\begin{small}
\fcolorbox{white}{Lightgray}{
      \begin{codebox}
        \Procname{$\proc{Tree-Insert}(T, z)$}
      \li $y\gets\const{NIL}$
      \li $x\gets\attrib{T}{root}$
      \li \While $x\ne\const{nil}$
      \li \Do $y\gets x$
      \li     \If $\attrib{z}{key}<\attrib{x}{key}$
      \li     \Then $x\gets\attrib{x}{left}$
      \li     \Else $x\gets\attrib{x}{right}$
      \End
      \End
      \li $\attrib{z}{parent}\gets y$
      \li \If $y\isequal \const{NIL}$
      \li \Then \Comment Tree $T$ was empty
      \li        $\attrib{T}{root}\gets z$
      \li \ElseIf $\attrib{z}{key}<\attrib{y}{key}$
      \li       \Then $\attrib{y}{left}\gets z$
      \li       \Else $\attrib{y}{right}\gets z$
      \End
      \End
      \end{codebox}}
\end{small}
\end{center}
    \end{column}
    \begin{column}{5cm}
      \begin{center}
      \includegraphics[width=5cm]{Figures/05-bstinsertion.pdf}
      \end{center}

\bigskip

\bigskip

Complexité: $O(h)$ où $h$ est la hauteur de l'arbre
    \end{column}
  \end{columns}

\note{On suppose qu'on insère deux fois une clé qui y serait déjà. Sinon, on peut d'abord la chercher. Si elle est là, on update, sinon, on insère. $x$ trace le chemin, $y$ maintient le pointeur vers le parent de $x$.\\

\bigskip


\centerline{\includegraphics[width=8cm]{Figures/05-deletebstexample.pdf}}
}

\end{frame}

\begin{frame}{Suppression}

3 cas à considérer en fonction du n\oe ud $z$ à supprimer:
\begin{itemize}
\item $z$ n'a pas de fils gauche: remplacer $z$ par son fils droit
\centerline{\includegraphics[width=5cm]{Figures/05-bstdelete-1.pdf}}
\item $z$ n'a pas de fils droit: remplacer $z$ par son fils gauche
\centerline{\includegraphics[width=5cm]{Figures/05-bstdelete-2.pdf}}
\end{itemize}

\note{On pourrait aussi ne rien faire et garder les n\oe uds dans la
  structure et les marquer. Problème: beaucoup de place pour rien

\bigskip

Virer le minimum et le maximum est facile.



}

\end{frame}

\begin{frame}
\begin{itemize}
\item $z$ a deux fils: rechercher le successeur $y$ de $z$.\\\emph{NB: $y$ est dans le sous-arbre de droite et n'a pas de fils gauche.}
\begin{itemize}
\item Si $y$ est le fils droit de $z$, remplacer $z$ par $y$ et conserver le fils droit de $y$
\centerline{\includegraphics[width=5.5cm]{Figures/05-bstdelete-3.pdf}}
\item Sinon, $y$ est dans le sous-arbre droit de $z$ mais n'en est pas la racine. On remplace $y$ par son propre fils droit et on remplace $z$ par $y$.
\centerline{\includegraphics[width=8.5cm]{Figures/05-bstdelete-4.pdf}}
\end{itemize}
\end{itemize}
\note{$y$ n'a pas de fils gauche sinon, le successeur de $z$ se trouverait dans le sous-arbre de gauche qui correspond à des valeurs plus petites que $y$ et plus grande que $z$.
~\\
}
\end{frame}

\begin{frame}{Suppression}
\vspace{-1cm}
\begin{columns}
\begin{column}{6cm}
\begin{center}
\begin{small}
\fcolorbox{white}{Lightgray}{%
  \begin{codebox}
    \Procname{$\proc{Tree-Delete}(T,z)$}
    \li \If $\attrib{z}{left}\isequal \const{NIL}$
    \li \Then $\proc{Transplant}(T,z,z.right)$
    \li \ElseIf $\attrib{z}{right}\isequal\const{NIL}$
    \li \Then $\proc{Transplant}(T,z,z.left)$
    \li \Else \Comment $z$ has two children
    \li $y\gets\proc{Tree-Successor}(z)$
    \li \If $\attrib{y}{parent}\ne z$
    \li \Then $\proc{Transplant}(T,y,y.right)$
    \li $\attrib{y}{right}\gets\attrib{z}{right}$
    \li $\attrib{y}{right}.\id{parent} \gets y$
    \End
    \li \Comment Replace $z$ by $y$
    \li $\proc{Transplant}(T,z,y)$
    \li $\attrib{y}{left}\gets \attrib{z}{left}$
    \li $\attrib{y}{left}.\id{parent}\gets y$
    \End
\end{codebox}}
\end{small}
\end{center}
\end{column}
\begin{column}{6cm}

\bigskip

\bigskip

\begin{center}
\begin{small}
\fcolorbox{white}{Lightgray}{%
  \begin{codebox}
    \Procname{$\proc{Transplant}(T,u,v)$}
    \li\If $\attrib{u}{parent}\isequal \const{NIL}$
    \li\Then $\attrib{T}{root}\gets v$
    \li \ElseIf $u\isequal \attrib{u}{parent}.\id{left}$
    \li \Then $\attrib{u}{parent}.\id{left}\gets v$
    \li \Else $\attrib{u}{parent}.\id{right}\gets v$
    \End
    \li \If $v\ne\const{NIL}$
    \li \Then $\attrib{v}{parent}=\attrib{u}{parent}$
    \End
  \end{codebox}}
\end{small}
\end{center}

\end{column}
\end{columns}

Complexité: $O(h)$ pour un arbre de hauteur $h$\\(Tout est $O(1)$ sauf l'appel à $\proc{Tree-Successor}$).

\note{Transplant: remplace $u$ par $v$ dans $T$. On vérifie qu'il se trouve à gauche ou à la droite de son parent.\\
\bigskip

Pourquoi pas avec le prédécesseur ??? (Parce qu'on peut avoir la même clé que $z$ à droite ($\geq$)}
\end{frame}

\begin{frame}{Arbres binaires de recherche}

\begin{itemize}
\item Toutes les opérations sont $O(h)$ où $h$ est la hauteur de l'arbre
\item Si $n$ éléments ont été insérés dans l'arbre:
\begin{itemize}
\item Au pire, $h=n-1=O(n)$
\begin{itemize}
\item Elements insérés en ordre
\end{itemize}
\item Au mieux, $h=\lceil\log_2 n\rceil=O(\log n)$
\begin{itemize}
\item Pour un arbre binaire complet
\end{itemize}
\item En moyenne, on peut montrer que $h=O(\log n)$
\begin{itemize}
\item En supposant que les éléments ont été insérés en ordre aléatoire
\end{itemize}
\end{itemize}
\end{itemize}

\note{
Autre représentation: externe:

\centerline{\includegraphics[width=8cm]{Figures/03-external_bst.pdf}}
}
\end{frame}

\begin{frame}{Tri avec un arbre binaire de recherche}

\begin{center}
\begin{small}
\fcolorbox{white}{Lightgray}{%
  \begin{codebox}
    \Procname{$\proc{Binary-Search-Tree-Sort}(A)$}
    \li $T\gets$ ``Empty binary search tree''
    \li \For $i\gets 1$ \To $n$
    \li \Do $\proc{Tree-Insert}(T,A[i])$\End
    \li $\proc{Inorder-Tree-Walk}(\attrib{T}{root})$
    \End
  \end{codebox}
  }
\end{small}
\end{center}

\begin{itemize}
\item Exemple: $A=[6,5,7,2,5,8]$

~\hfill\includegraphics[width=5cm]{Figures/05-onebst.pdf}
\item Complexité en temps identique au quicksort
\begin{itemize}
\item Insertion: en moyenne, $n\cdot O(\log n)=O(n\log n)$, pire cas: $O(n^2)$
\item Parcours de l'arbre en ordre: $O(n)$
\item Total: $O(n\log n)$ en moyenne, $O(n^2)$ pour le pire cas
\end{itemize}
\item Complexité en espace cependant plus importante, pour le stockage de la structure d'arbres.
\end{itemize}

\note{File à priorité avec un arbre:
pire cas: $O(N)$ pour l'insertion et l'extraction. Cas moyen: $O(log n)$ dans les deux cas.
}

\end{frame}

\begin{frame}{Dictionnaires: jusqu'ici}

  \begin{center}\small
    \def\arraystretch{1.5}\renewcommand{\tabcolsep}{1mm}
    \begin{tabular}{@{}lcccccc@{}}
    &\multicolumn{3}{c}{\emph{Pire cas}} & \multicolumn{3}{c}{\emph{En moyenne}}\\
    \emph{Implémentation}& \proc{Search} & \proc{Insert} & \proc{Delete} & \proc{Search} & \proc{Insert} & \proc{Delete}\\
    \hline\hline
    Liste &$\Theta(n)$&$\Theta(n)$&$\Theta(n)$&$\Theta(n)$&$\Theta(n)$&$\Theta(n)$\\
    \hline
    Vecteur trié&$\Theta(\log n)$&$\Theta(n)$&$\Theta(n)$&$\Theta(\log n)$&$\Theta(n)$&$\Theta(n)$\\
\hline
ABR&$\Theta(n)$&$\Theta(n)$&$\Theta(n)$&$\Theta(\log n)$&$\Theta(\log n)$&$\Theta(\log n)$\\
    \hline\hline
  \end{tabular}
  \end{center}

\bigskip

\begin{itemize}
\item Peut-on obtenir $\Theta(\log n)$ dans le pire cas? Oui !
\item Deux solutions:
\begin{itemize}
\item Utiliser de la randomisation pour que le probabilité de
  rencontrer le pire cas soit négligeable
\item Maintenir les arbres équilibrés
\end{itemize}
\end{itemize}

\note{
randomization: n'assure pas qu'on ne sera jamais dans le pire cas. Suppose qu'on puisse jouer sur l'ordre d'insertion
}
\end{frame}


\subsection{Arbres équilibrés AVL}

\begin{frame}{Plan}

\tableofcontents[currentsection]

\end{frame}


\begin{frame}{Arbres équilibrés}
\begin{itemize}
\item Solution générale pour obtenir une complexité au pire cas en $O(\log n)$:% maintenir en permanence un arbre plus ou moins complet
\begin{itemize}
\item Définir un \alert{invariant} sur la structure d'arbre
\item Prouver que cet invariant garantit une hauteur $\Theta(\log n)$
\item Implémenter les opérations d'insertion et suppression de manière à maintenir l'invariant
\item Si ces opérations ne sont pas trop coûteuses (p.ex., $O(\log
  n)$), on aura gagné
\end{itemize}

\bigskip

\item Plusieurs types d'arbres équilibrés:
\begin{itemize}
\item \alert{Arbres AVL}
\begin{itemize}
\item \alert{Invariant: Arbres $H$-équilibrés}
\end{itemize}
\item Arbres 2-3-4
\item Arbres rouges et noirs
\item Splay trees, Scapegoat trees, treaps\ldots
\end{itemize}
\end{itemize}
\end{frame}

\begin{frame}{Arbres $H$-équilibrés}
\begin{itemize}
\item \alert{Définition:}

$$T\mbox{ est }H-\mbox{équilibré}\Leftrightarrow |h(g(T'))-h(d(T'))|\leq 1,$$
pour tout sous-arbre $T'$ de $T$, et où $g(X)$, $d(X)$ et $h(X)$ sont resp. le sous-arbre gauche, le sous-arbre droit et la hauteur de l'arbre $X$.

\medskip

\emph{(Les hauteurs des deux sous-arbres d'un même n\oe ud diffèrent au plus de un)}

\bigskip

\item \alert{Propriété:}

Pour tout arbre $H$-équilibré de taille $n$ et de hauteur $h$, on a 
$$h=\Theta(\log n)$$

Plus précisément, on peut prouver:
 $$\log(n+1)\leq h+1< 1,44 \log(n+2)$$
\end{itemize}

\end{frame}

\begin{frame}{Arbres $H$-équilibrés}
\alert{Démonstration}

Etant donné un arbre $H$-équilibré de taille $n$ et de hauteur $h\geq 1$, pour $h$ fixé, $n$ est
\begin{itemize}
\item Maximum: quand l'arbre est complet, soit quand
$n=2^{h+1}-1\Rightarrow n+1\leq 2^{h+1} \Rightarrow \log (n+1)\leq h+1 \Rightarrow h\in\Omega(\log n) $
\item Minimum: quand $n=N(h)$ où $N(h)$ est la taille d'un arbre $H$-équilibré de hauteur $h$ qui a le moins d'éléments.
\begin{itemize}
\item $N(h)$ peut être défini par récurrence par $N(h)=1+N(h-1)+N(h-2)$ avec $N(0)=1$ et $N(1)=2$.
\end{itemize}
\end{itemize}
\centerline{\includegraphics[width=5cm]{Figures/05-avlmaximum.pdf}}

\note{Pour que le nombre n\oe uds soit minimal, il faut que l'arbre soit déséquilibré (sinon, on pourrait enlever des n\oe uds et toujours satisfaire la propriété d'arbre $H$-équilibré)}

\end{frame}

\begin{frame}{~}%{Borne plus simple}
\begin{itemize}
\item[]
\begin{itemize}
\item On a donc\\
\begin{tabular}{cl}
& $N(h)=1+N(h-1)+N(h-2)$\\
$\Rightarrow$ & $N(h)>2 N(h-2)$ (car $N(h-1)>N(h-2)$)\\
$\Rightarrow$ & $N(h)>2^{h/2}$\\
$\Rightarrow$ & $h< 2\log N(h)$\\
\end{tabular}
\item dont on peut tirer que
$h\in O(\log n)$
\end{itemize}
\item On en déduit que
$$h=\Theta(\log n)$$\qed
\end{itemize}

\note{\begin{eqnarray*}
N(h)&<&2 N(h-2)\\
&<& 2^2 N(h-4)\\
&<& 2^3 N(h-6)\\
&<&\ldots\\
&<&2^i N(h-2i)\\
&<&2^(h/2) N(0)=2^(h/2)
\end{eqnarray*}

}
\end{frame}

\begin{frame}{Borne supérieure plus précise}
\begin{itemize}
\item[]
\begin{itemize}
\item En notant $F(h)=N(h)+1$, on a $F(h)=F(h-1)+F(h-2)$ avec $F(0)=2$, $F(1)=3$
\item $F$ est un récurrence de Fibonacci qui a pour solution
 $$F(h)=\frac{1}{\sqrt{5}} (\phi^{h+3}-\phi'^{h+3})\mbox{ avec }\phi=\frac{1+\sqrt{5}}{2}\mbox{ et }\phi'=\frac{1-\sqrt{5}}{2}$$
\item On a $$N(h)+1=\frac{1}{\sqrt{5}} (\phi^{h+3}-\phi'^{h+3})$$
\item ce qui donne
$$n+1\geq \frac{1}{\sqrt{5}} (\phi^{h+3}-\phi'^{h+3})> \frac{1}{\sqrt{5}} (\phi^{h+3}-1)$$
(car $|\phi'|<1$)
\item En prenant le $\log_{\phi}$ des deux membres:
$$h+1<1,44\log (n+2)$$
\end{itemize}
%A VERIFIER
\end{itemize}

\end{frame}

\begin{frame}{Arbres AVL}

\begin{itemize}
\item \alert{Définition:} Un arbre AVL est un arbre binaire de
  recherche $H$-équilibré
\item Inventé par Adelson-Velskii et Landis en 1960
\item Recherche:
\begin{itemize}
\item Par la fonction $\proc{Tree-Search}$ puisque c'est un arbre binaire
\item Complexité $\Theta(\log n)$ étant donné la propriété
\end{itemize}
\item Insertion:
\begin{itemize}
\item On insère l'élément comme dans un arbre binaire classique
\item On vérifie que l'invariant est respecté
\item Si ce n'est pas le cas, on modifie l'arbre
\end{itemize}
\end{itemize}
\end{frame}

\begin{frame}{Rotations}

\centerline{\includegraphics[width=10cm]{Figures/05-rotations.pdf}}

\bigskip

\begin{center}
\fcolorbox{white}{Lightgray}{%
    \begin{codebox}
      \Procname{$\proc{Left-Rotate}(x)$}
      \li $r\gets\attrib{x}{right}$
      \li $\attrib{x}{right}\gets\attrib{r}{left}$
      \li $\attrib{r}{left}\gets x$
      \li \Return $r$
    \end{codebox}}
~~~~~~~~~~~~~~~~~~~
\fcolorbox{white}{Lightgray}{%
    \begin{codebox}
      \Procname{$\proc{Right-Rotate}(x)$}
      \li $l\gets\attrib{x}{left}$
      \li $\attrib{x}{left}\gets\attrib{l}{right}$
      \li $\attrib{l}{right}\gets x$
      \li \Return $l$
    \end{codebox}}
\end{center}

Les rotations maintiennent la propriété d'arbre binaire  de recherche

\note{Deux types d'opération pour maintenir l'équilibre: rotations à gauche et à droite. Implémentée comme sur ce slide. Opération d'ordre $O(1)$.

\bigskip

Donner un exemple: 
\centerline{\includegraphics[width=10cm]{Figures/03-rotation.pdf}}
}
\end{frame}

\begin{frame}{Insertion dans un AVL}

\centerline{\includegraphics[width=10cm]{Figures/05-avlinsertion.pdf}}

\bigskip

\begin{itemize}
\item Insérer le nouvel élément comme dans un arbre binaire de recherche ordinaire
\item L'insertion peut créer un déséquilibre (l'arbre n'est plus $H$-équilibré)
\item Remonter depuis le nouveau n\oe ud jusqu'à la racine en
  restaurant l'équilibre des sous-arbres rencontrés si nécessaire
\end{itemize}

\note{Implémentation récursive:
%voir ici:http://www.enseignement.polytechnique.fr/profs/informatique/Luc.Maranget/421/poly/arbre-bin.html
}

\end{frame}

% Insertion:
% - si l'arbre est équilibré -> pas de risque de déséquilibrage
% - si déséquilibre à gauche ou à droite -> possibilité de violation dans le cas d'une insertion à droite ou à gauche
% - symétrique:
%   deux cas: 1) -> 1 rotation corrige le tir
%             2) -> 2 rotations corrigent le tir

% Implémentation: on doit maintenir la hauteur des noeuds (data augmentation)
% Au plus 2 rotations par insertion: Intuitivement, les opérations du slide précédent font que la hauteur du sous-arbre en $x$ n'est finalement pas augmentée suite à l'insertion. Tous les sous-arbres au dessus de $x$ sont maintenus équilibrés (et $x$ est le premier sous-arbre non $H$-équilibré).

% Deletion: idem mais plus de rotations sont possibles

\begin{frame}{Equilibrage}

\begin{itemize}
\item Soit $x$ le n\oe ud le plus bas violant l'invariant après l'insertion
  \begin{itemize}
  \item Tous ses sous-arbres sont $H$-équilibrés
  \item Il y a une différence d'au plus 2 niveaux entre ses
    sous-arbres gauche et droit
  \end{itemize}
\item Comment rétablir l'équilibre ?
\item Deux cas possibles (selon insertion à droite ou à gauche):
\end{itemize}
\begin{center}
Cas 1\hspace{4cm}Cas 2

\medskip

\includegraphics[width=8cm]{Figures/05-avlcas1-2.pdf}

\medskip
(Déséquilibre à droite)\hspace{1.3cm}(Déséquilibre à gauche)
\end{center}

\note{Tous ses sous-arbres sont $H$-équilibrés puisque c'est le plus bas qui viole l'invariant}

\end{frame}

\begin{frame}{Cas 1: déséquilibre à droite}
\begin{itemize}
\item Deux sous-cas possibles
\end{itemize}

\begin{center}
Cas 1.1\hspace{4cm}Cas 1.2

\medskip

\includegraphics[width=8cm]{Figures/05-avlcas1.pdf}

\medskip
Déséquilibre à l'extérieur\hspace{1.3cm}Déséquilibre à l'intérieur\\
(Cas droite-droite)\hspace{2cm}(Cas droite-gauche)
\end{center}

\bigskip

\emph{(Pourquoi le cas B et C de hauteur $h$ n'est pas possible ?)}

\note{Leur demander pourquoi ce sont les seuls deux cas.

Pourquoi pas B et C de hauteur h ? Parce que sinon, ça voudrait dire que la hauteur du sous-arbre en $y$ n'a pas augmenté (et donc il ne pourrait pas être devenu déséquilibré).

}
\end{frame}

\begin{frame}{Cas 1.1: déséquilibre à droite, extérieur (droite-droite)}
\begin{itemize}
\item Equilibre rétabli par une rotation à gauche de $x$
\end{itemize}

\begin{center}
\includegraphics[width=10cm]{Figures/05-avleqcas11.pdf}
\end{center}

\end{frame}

\begin{frame}{Cas 1.2: déséquilibre à droite, intérieur (droite-gauche)}
\begin{itemize}
\item Une rotation à gauche ne permet pas de rétablir l'équilibre
%\end{itemize}

\begin{center}
\includegraphics[width=10cm]{Figures/05-avlcas12-wrong.pdf}
\end{center}
\item Le sous-arbre $B$ contient au moins un élément (l'élément inséré)
\begin{center}
\includegraphics[width=7cm]{Figures/05-avlcas12-decomp.pdf}
\end{center}

\end{itemize}

\note{Pourquoi pas $B_l$ et $B_r$ de hauteur $h-1$ ? Parce qu'alors l'insertion n'aurait pas modifié la hauteur de $B$}
\end{frame}

\begin{frame}{Cas 1.2: déséquilibre à droite, intérieur (droite-gauche)}
\begin{itemize}
\item Equilibre rétabli par deux rotations
%\end{itemize}

\begin{center}
\includegraphics[width=10cm]{Figures/05-avlcas12-double.pdf}
\end{center}

\end{itemize}

\end{frame}

\begin{frame}{Cas 2: déséquilibre à gauche}
\begin{itemize}
\item Symétrique du cas 1
\item Deux sous-cas possibles
\end{itemize}

\begin{center}
Cas 2.1\hspace{4cm}Cas 2.2

\medskip

\includegraphics[width=8cm]{Figures/05-avlcas2.pdf}

\medskip
Déséquilibre à l'extérieur\hspace{1.3cm}Déséquilibre à l'intérieur\\
(Cas gauche-gauche)\hspace{2cm}(Cas gauche-droite)
\end{center}

\begin{itemize}
\item Résolus respectivement par une rotation (à droite) et une double rotation.
\end{itemize}

\end{frame}

\begin{frame}{Implémentation}
\begin{itemize}
\item Algorithme récursif: Pour insérer une clé dans un arbre $T$:
\begin{itemize}
\item On l'insère (récursivement) dans le sous-arbre approprié (gauche ou droit)
\item Si l'arbre résultant $T$ devient déséquilibré, on effectue une rotation simple ou double selon le cas dans lequel on se trouve
\end{itemize}
\item L'arbre après rééquilibrage étant de la même hauteur qu'avant l'insertion, on n'aura à faire qu'au plus une rotation (simple ou double).
\item L'implémentation est facilitée si on maintient en chaque n\oe ud $x$ un attribut $\attrib{x}{h}$ avec la hauteur du sous-arbre en $x$.
\item Complexité:
\begin{itemize}
\item $O(h)$ où $h$ est la hauteur de l'arbre,
\item c'est-à-dire $O(\log n)$ vu que l'arbre est $H$-équilibré.
\end{itemize}
\end{itemize}
\note{Exemple:
\centerline{\includegraphics[width=5cm]{Figures/03-avlexemple.pdf}}
}
\end{frame}

\begin{frame}{Suppression}
\begin{itemize}
\item Comme pour l'insertion, on doit rétablir l'équilibre suite à la suppression
\item La suppression d'un n\oe ud peut déséquilibrer le parent de ce n\oe ud
\item Contrairement à l'insertion, on peut devoir rééquilibrer plusieurs ancêtres du n\oe ud supprimé.
\item Chaque rotation étant d'ordre $O(1)$, la complexité d'une suppression reste cependant $O(h)$ pour un arbre de hauteur $h$ et donc $O(\log n)$ pour un AVL.
\end{itemize}
\end{frame}

\begin{frame}{Tri avec un AVL}

\begin{itemize}
\item Comme avec un arbre de binaire de recherche ordinaire, on peut trier avec un AVL
\begin{itemize}
\item On insère les éléments successivement dans l'arbre
\item On effectue un parcours en ordre de l'arbre
\end{itemize}
\item Complexité en temps: $\Theta(n\log n)$ (comme pour le tri par tas)
\item Complexité en espace: $\Theta(n)$ (pour la structure d'arbre temporaire) (versus $O(1)$ pour le heap-sort)
%\item Tas: optimisé pour retrouver et supprimer le minimum (ou le max)
%\item AVL: optimisé pour retrouver et supprimer un élément arbitraire 
%\item A ELABORER
\end{itemize}

\end{frame}


%%%%

\begin{frame}{Dictionnaires: jusqu'ici}

  \begin{center}\small
    \def\arraystretch{1.5}\renewcommand{\tabcolsep}{1mm}
    \begin{tabular}{@{}lcccccc@{}}
    &\multicolumn{3}{c}{\emph{Pire cas}} & \multicolumn{3}{c}{\emph{En moyenne}}\\
    \emph{Implémentation}& \proc{Search} & \proc{Insert} & \proc{Delete} & \proc{Search} & \proc{Insert} & \proc{Delete}\\
    \hline\hline
    Liste &$\Theta(n)$&$\Theta(n)$&$\Theta(n)$&$\Theta(n)$&$\Theta(n)$&$\Theta(n)$\\
    \hline
    Vecteur trié&$\Theta(\log n)$&$\Theta(n)$&$\Theta(n)$&$\Theta(\log n)$&$\Theta(n)$&$\Theta(n)$\\
\hline
ABR&$\Theta(n)$&$\Theta(n)$&$\Theta(n)$&$\Theta(\log n)$&$\Theta(\log n)$&$\Theta(\log n)$\\
\hline
AVL&$\Theta(\log n)$&$\Theta(\log n)$&$\Theta(\log n)$&$\Theta(\log n)$&$\Theta(\log n)$&$\Theta(\log n)$\\
    \hline\hline
  \end{tabular}
  \end{center}

\bigskip

\begin{itemize}
\item Peut-on faire mieux?
\item Oui, en changeant radicalement de philosophie
\end{itemize}

\note{Dire que les autres algos d'arbres ont la m\^eme complexit\'e.

\bigskip

Appli: système de réservation: on veut retrouver le prochain (min), }

\end{frame}


\begin{frame}{Demo}
Illustrations:
\begin{small}
\begin{itemize}
\item \url{http://people.ksp.sk/~kuko/bak/}
\item \url{http://www.csi.uottawa.ca/~stan/csi2514/applets/avl/BT.html}
\item \url{http://www.cs.jhu.edu/~goodrich/dsa/trees/avltree.html}
\item \url{http://www.cs.usfca.edu/~galles/visualization/flash.html}
\end{itemize}
\end{small}

\note{Montrer les heap min et max, le BST normal, les AVL et puis peut-être les skip-list

~\bigskip

Pour AVL, demander ce qui va se passer lorsqu'on insère un élément dans l'arbre
}
\end{frame}

\section{Tables de hachage}

\begin{frame}{Plan}

\tableofcontents[currentsection]

\end{frame}

\subsection{Principe}

\begin{frame}{Tableau à accès direct}

\begin{itemize}
\item On suppose:
\begin{itemize}
\item que chaque élément a une clé tirée d'un univers
  $U=\{0,1,\ldots,m-1\}$ où $m$ n'est pas trop grand
\item qu'il ne peut pas y avoir deux éléments avec la même clé.
\end{itemize}
\item Le dictionnaire est implémenté par un tableau $T[0\ldots m-1]$:
\begin{itemize}
\item Chaque position dans la table correspond à une clé de $U$.
\item S'il y a un élément $x$ avec la clé $k$, alors $T[k]$ contient un pointeur vers $x$.
\item Sinon, $T[k]$ est vide ($T[k]=\const{NIL}$).
\end{itemize}
\end{itemize}

\end{frame}

\begin{frame}{Tableau à accès direct}

\centerline{\includegraphics[width=9cm]{Figures/05-directtable.pdf}}

\begin{center}
\begin{small}
\fcolorbox{white}{Lightgray}{%
  \begin{codebox}
    \Procname{$\proc{Direct-Address-Search}(T,k)$}
    \li \Return $T[k]$
\end{codebox}}

\fcolorbox{white}{Lightgray}{%
  \begin{codebox}
    \Procname{$\proc{Direct-Address-Insert}(T,x)$}
    \li \Return $T[\attrib{x}{key}]=x$
\end{codebox}}

\fcolorbox{white}{Lightgray}{%
  \begin{codebox}
    \Procname{$\proc{Direct-Address-Delete}(T,x)$}
    \li \Return $T[\attrib{x}{key}]=\const{NIL}$
\end{codebox}}
\end{small}
\end{center}


\end{frame}

\begin{frame}{Tableau à accès direct}

\begin{itemize}
\item Complexité de toutes les opérations: $O(1)$ (dans tous les cas)
\item Problème:
\begin{itemize}
\item Complexité en espace: $\Theta(|U|)$
\item si l'univers de clés $U$ est grand, stocker une table de taille $|U|$ peut être peu pratique, voire impossible
\end{itemize}
\item Souvent l'ensemble des clés réellement stockées, noté $K$, est petit comparé à $U$ et donc l'espace alloué est gaspillé.

\bigskip\bigskip

\item Comment bénéficier de l'accès rapide d'une table à accès direct avec une table de taille raisonnable ?\\

\medskip

$\Rightarrow$ \alert{Table de hachage:}
\begin{itemize}
\item Réduit le stockage à $\Theta(|K|)$
\item Recherche en $O(1)$ (\alert{en moyenne} !)
\end{itemize}
\end{itemize}
\note{Le prix à payer est qu'on a plus une complexité en $O(1)$ dans le pire cas mais en moyenne. Ce qui n'est pas un problème}
\end{frame}

\begin{frame}{Table de hachage}

\begin{itemize}
\item Inventée en 1953 par Luhn
\item Idée:
\begin{itemize}
\item Utiliser une table $T$ de taille $m\ll|U|$
\item stocker $x$ à la position $h(\attrib{x}{key})$, où $h$ est une fonction de \alert{hachage}: $$h:U\rightarrow \{0,\ldots,m-1\}$$
\end{itemize}
\end{itemize}

\begin{center}
\begin{small}
\fcolorbox{white}{Lightgray}{%
  \begin{codebox}
    \Procname{$\proc{Hash-Insert}(T,x)$}
    \li $T[h(\attrib{x}{key})]\gets x$
\end{codebox}}
~~~~~~~~\fcolorbox{white}{Lightgray}{%
  \begin{codebox}
    \Procname{$\proc{Hash-Delete}(T,x)$}
    \li $T[h(\attrib{x}{key})]\gets \const{NIL}$
\end{codebox}}

\bigskip

\fcolorbox{white}{Lightgray}{%
  \begin{codebox}
    \Procname{$\proc{Hash-Search}(T,x)$}
    \li \Return $T[h(\attrib{x}{key})]$
\end{codebox}}
\end{small}
\end{center}

Est-ce que ces algorithmes sont corrects ?

\end{frame}

\begin{frame}{Table de hachage: collisions}

\centerline{\includegraphics[width=7cm]{Figures/05-hashtable1.pdf}}

\bigskip

\begin{itemize}
\item \alert{Collision:} lorsque deux clés distinctes $k_1$ et $k_2$ sont telles que $h(k_1)=h(k_2)$
\item Cela se produit toujours lorsque le nombre de clés observées est plus grand que la taille du tableau $T$ ($|K|>m$)
\item Très probable, même lorsque la fonction de hachage répartit les clés uniformément $\Rightarrow$ \alert{Paradoxe des anniversaire}
\end{itemize}
\note{Supposons qu'on veuille stocker de l'info sur vous dans un calendrier avec 365 cases}
\end{frame}

\begin{frame}{Paradoxe des anniversaires}

\begin{itemize}
\item Hypothèse:
\begin{itemize}
\item On néglige les années bissextiles
\item Les 365 jours présentent la même probabilité d'être un jour d'anniversaire
\end{itemize}
\item Si $p$ est la probabilité d'une collision d'anniversaires:
$$1-p = \frac{364}{365} \cdot \frac{363}{365} \cdot \frac{362}{365} \ldots \frac{365-(n-1)}{365} = \frac{365!}{(365-n)! 365^n}$$
Exemples:
\begin{itemize}
\item $n=23 \Rightarrow p>0,5$
\item $n=57 \Rightarrow p>0,99$
\item $n=70 \Rightarrow p>0,999$
\end{itemize}
\bigskip

\item Pour une table de hachage:
\begin{itemize}
\item $m=365$ et 57 clés $\Rightarrow$ plus de 99\% de chance de collision
\item $m=1000000$ et 2500 clés $\Rightarrow$ plus de 95\% de chance de collision
\end{itemize}
\end{itemize}
\note{Faire l'essai dans la classe: demander qui est ne en avril-mai}
\end{frame}

\begin{frame}{Collision}

\begin{itemize}
\item Pour éviter les collisions:
\begin{itemize}
\item on veille à utiliser une fonction de hachage qui disperse le
  plus possible les clés vers les différents compartiments.
\item on utilise un nombre de compartiments suffisamment grand
\end{itemize}
Cependant, même dans ce cas, la probabilité de collision peut
  être non négligeable.

\bigskip

\item Deux approches pour prendre en compte les collisions:
\begin{itemize}
\item Le chaînage (adressage fermé)
\item Le sondage (adressage ouvert)
\end{itemize}
\end{itemize}

\note{Voir plus loin pour la première propriété

\bigskip

On va d'abord voir la première solution. On verra ensuite la seconde}

\end{frame}


\begin{frame}{Résolution des collisions par chaînage}

Solution: mettre les éléments qui sont ``hachés'' vers la même
position dans une liste liée (simple ou double)

\bigskip

\centerline{\includegraphics[width=10cm]{Figures/05-hashtable2.pdf}}

\note{Une simple ou une double ? Une liste double est meilleure pour supprimer des éléments}
\end{frame}

\begin{frame}{Implémentation des opérations}

~\bigskip

\begin{center}
\begin{small}
\fcolorbox{white}{Lightgray}{%
  \begin{codebox}
    \Procname{$\proc{Chained-Hash-Insert}(T,x)$}
    \li $\proc{List-Insert}(T[h(\attrib{x}{key})],x)$
\end{codebox}}
~~~~~~~~\fcolorbox{white}{Lightgray}{%
  \begin{codebox}
    \Procname{$\proc{Chained-Hash-Delete}(T,x)$}
    \li $\proc{List-Delete}(T[h(\attrib{x}{key})],x)$
\end{codebox}}

\bigskip

\fcolorbox{white}{Lightgray}{%
  \begin{codebox}
    \Procname{$\proc{Chained-Hash-Search}(T,k)$}
    \li \Return $\proc{List-Search}(T[h(k)],k)$
\end{codebox}}
\end{small}
\end{center}

\bigskip
\bigskip

\begin{itemize}
\item Complexité:
\begin{itemize}
\item Insertion: $O(1)$
\item Suppression: $O(1)$ si liste doublement liée, $O(n)$ pour une liste de taille $n$ si liste simplement liée.
\item Recherche: $O(n)$ si liste de taille $n$.
\end{itemize}
\end{itemize}

\end{frame}

\begin{frame}{Analyse du cas moyen}

%\medskip

\begin{itemize}
\item Recherche d'une clé $k$ dans la table:
\begin{itemize}
\item recherche positive: la clé $k$ se trouve dans la table
\item recherche négative: la clé $k$ n'est pas dans la table
\end{itemize}
\medskip
\item Le \alert{facteur de charge} d'une table de hachage est donné par $\alpha=\frac{n}{m}$ où:
\begin{itemize}
\item $n$ est le nombre d'éléments dans la table
\item $m$ est la taille de la table (c'est-à-dire, le nombre de listes liées)
\end{itemize}
\medskip
\item \alert{hachage uniforme simple}: Pour toute clé $k\in U$,
$$Proba\{h(k)=i\}=\frac{1}{m}, \forall i\in\{0,\ldots,m-1\}$$

\bigskip

\centerline{\includegraphics[width=7cm]{Figures/05-hachageuniforme.pdf}}

\end{itemize}

\end{frame}

\begin{frame}{Analyse du cas moyen}
\begin{itemize}
\item Hypothèses:
\begin{itemize}
\item $h$ produit un hachage uniforme simple
\item le calcul de $h(k)$ est $\Theta(1)$
\item Insertion en début de liste
\end{itemize}
\item $\Rightarrow$ complexités moyennes:
\begin{itemize}
\item recherche négative: $\Theta(1+\alpha)$
\item recherche positive: $\Theta(1+\alpha)$
\end{itemize}
\item Si $n=O(m)$, \hfill {\it ($m$ croît au moins linéairement avec $n$)}, $$\alpha=\frac{O(m)}{m}=O(1)$$
\item Toutes les opérations sont donc $O(1)$ en moyenne
\end{itemize}

\note{
Pourquoi est-ce qu'on n'écrit pas directement $\Theta(\alpha)$ ? Parce que $\alpha=n/m=f(n)$ et pour que l'analyse ait un sens, il faut que $m$ grandisse avec $n$. Donc, on ne sait pas a priori comment $\alpha$ évolue avec $n$ et donc si on peut négliger 1 par rapport à $\alpha$

\bigskip


Si $n=O(m)$, on a $m=\Omega(n)$, ce qui veut dire que la taille de la table croît proportionnellement avec la taille des données. Si elle croît moins que linéairement, lorsque $n$ va croître le alpha va augmenter.}
\end{frame}

\begin{frame}{Analyse du cas moyen: recherche négative}
\begin{itemize}
\item La clé $k$ ne se trouve pas dans la table
\item Par la propriété de hachage uniforme simple, elle a la même probabilité d'être envoyée vers chaque position dans la table.
\item Recherche négative requière le parcours de la liste $T[h(k)]$ complète
\item Cette liste a une longueur moyenne $E[n_{h(k)}]=\alpha$.
\item Le nombre d'éléments à examiner lors d'une recherche négative est donc $\alpha$.
\item En ajoutant le temps de calcul de la fonction de hachage, on arrive à une complexité moyenne $\Theta(1+\alpha)$.
\end{itemize}
\end{frame}


\begin{frame}{Analyse du cas moyen: recherche positive}

\begin{itemize}
\item La clé $k$ se trouve dans la table
\item Supposons qu'elle ait été insérée à la $i$-ième étape (parmi $n$).
\begin{itemize}
\item Le nombre d'éléments à examiner pour trouver la clé est le
  nombre d'éléments insérés à la position $h(k)$ après $k$ plus 1 (la clé $k$ elle-même). 
\item En moyenne, sur les $n-i$ insertions après $k$, il y en aura
  $(n-i)/m$ qui correspondront à la position $h(k)$.
\end{itemize}
\item $k$ ayant pu être insérée à n'importe quelle étape parmi $n$ avec une probabilité $1/n$:
\begin{footnotesize}
$$\sum_{i=1}^n \frac{1}{n} (1+\frac{n-i}{m})=1+\frac{1}{n m}(\sum_{i=1}^n n-\sum_{i=1}^n i)=1+\frac{1}{n m} (n^2-\frac{n(n+1)}{2})=1+\frac{\alpha}{2}-\frac{\alpha}{2n}$$
\end{footnotesize}
\item En tenant compte du coût du hachage, la complexité en moyenne
  est donc $\Theta(2+\alpha/2-\alpha/2n)=\Theta(1+\alpha)$.
\end{itemize}

%% Intuitivement:
%% Supposons qu'on ait inséré $x$ à la $i$th étape (proba de chaque étape est $1/n$.

%% Nombre de comparaison est = au nombre d'éléments inséré après $x$ dans
%% la table, c'est-à-dire $(n-i+1)/m$ vu l'hypothèse d'uniform hashing

%% Proba totale=$\sum_{i=1}^N 1/N (n-i+1)/m$...

\note{ $2+\alpha/2-\alpha/2n\in \Theta(1+\alpha)$ car $2*(1+\alpha/4)<2+\alpha/2-\alpha/2n<2*(1+\alpha)$ pour $n$ grand}

\end{frame}

\subsection{Fonctions de hachage}

\begin{frame}{Plan}

\tableofcontents[currentsection,currentsubsection]

\end{frame}


\begin{frame}{Fonctions de hachage}
\begin{itemize}
\item Idéalement, la fonction de hachage
\begin{itemize}
\item devrait être facile à calculer ($O(1)$)
\item devrait satisfaire l'hypothèse de hachage uniforme simple
\end{itemize}
\item La deuxième propriété est très difficile à assurer en pratique:
\begin{itemize}
\item La distribution des clés est généralement inconnue
\item Les clés peuvent ne pas être indépendantes
\end{itemize}
\item En pratique, on utilise des heuristiques basées sur la nature
  attendue des clés
\item Si toutes les clés sont connues, il existe des algorithmes pour
  construire une fonction de hachage parfaite, sans collision
  (Exemple: le logiciel gperf)
\end{itemize}

\end{frame}

\begin{frame}{Fonctions de hachage: codage préalable}

\begin{itemize}
\item Les fonctions de hachage supposent que les clés sont des nombres naturels
\item Si ce n'est pas le cas, il faut préalablement utiliser une \alert{fonction de codage}
\item Exemple: codage des chaînes de caractères:
\begin{itemize}
\item On interprète la chaîne comme un entier dans une certaine base
\item Exemple pour ``SDA'': valeurs ASCII (128 possibles): $$S=83, D=68, A=65$$
\item Interprété comme l'entier:\hfill {\it (Pourquoi pas 83+68+65 ?)}
 $$(83\cdot 128^2)+(68\cdot 128^1)+(65\cdot 128^0)=1368641$$
\item Calculé efficacement par la méthode de Horner:
$$((83\cdot 128+68)\cdot 128 + 65)$$
\end{itemize}
\end{itemize}
\note{
Pourquoi ne pas prendre la somme des entiers (83+68+65) ? reste la même pour une permutation de l'entrée

\bigskip

Horner: $$C_n x^n+c_{n-1} x^{n-1}+\ldots+C_0 x^0$$
$$(((C_n x+c_{n-1}) x+c_{n-2})x+\ldots C_1)x+C_0$$
}
\end{frame}

\begin{frame}{Méthode de division}

La fonction de hachage calcule le reste de la division entière de la clé par la taille de la table
$$h(k)=k\bmod m.$$
Exemple: $m=20$ et $k=91$ $\Rightarrow h(k)=11$.

\bigskip

\alert{Avantages:} simple et rapide (juste une opération de division)

\bigskip

\alert{Inconvénients:} Le choix de $m$ est très sensible et certaines valeurs doivent être évitées

\bigskip

Exemples:
\begin{itemize}
\item Si $m=2^p$ pour un entier $p$, $h(k)$ ne dépend que des $p$ bits
  les moins significatifs de $k$\\
\begin{itemize}
\item Exemple: ``SDA'' $\bmod\ 128$= ``GAGA''$\bmod\ 128$=65
\end{itemize}
\item Si $k$ est une chaîne de caractères codée en base $2^p$ et $m=2^p-1$, permuter la chaîne ne modifie pas le valeur de hachage\\
\begin{itemize}
\item Exemple: ``SDA''=1368641, ``DSA''=1124801\\
$\Rightarrow 1368641\bmod 127=1124801\bmod 127=89$
\end{itemize}
\end{itemize}
\note{Théorie qui motive les clés de hachage est la théorie des nombres. On va donner le minimum ici.

\bigskip

Pourquoi pas: $k m/k_{max}$ ? Parce qu'on ne connaît pas $k_{max}$

\bigskip

%Dire que si $m$ est premier, alors la valeur hachée est une
%permutation de $\{0,1,\ldots,m-1\}$

On cherche souvent des nombres premiers sous la forme $2^p-1$ car il
existe un test de primalité efficace pour ces nombres. Dans le cas
général, tester la primalité d'une nombre est très complexe.

}
\end{frame}

\begin{frame}{Méthode de division}

\begin{itemize}
\item Si la fonction de hachage produit des séquences périodiques, il vaut mieux choisir $m$ premier
\item En effet, si $m$ est premier avec $b$, on a:
$$\{(a+b\cdot i) \bmod m|i=0,1,2,\ldots\}=\{0,1,2,\ldots,m-1\}$$
\item Exemple: hachage de $\{206, 211, 216, 221,\ldots\}$
\begin{itemize}
\item $m=100$: valeurs hachées possibles: 6, 11,\ldots, 96
\item $m=101$: toutes les entrées sont exploitées
\end{itemize}

\bigskip

\item[$\Rightarrow$] Bonne valeur de $m$: un nombre premier pas trop près d'une puissance exacte de 2
\end{itemize}
\end{frame}

\begin{frame}{Méthode de multiplication}
\begin{itemize}
\item Fonction de hachage:
$$h(k)=\lfloor m\cdot(k A \bmod 1)\rfloor$$
où
\begin{itemize}
%\item $m$ est la taille de la table de hachage
%\item $k$ est la clé
\item $A$ est une constante telle que $0<A<1$.
\item $k A \bmod 1=k A - \lfloor k A \rfloor$ est la partie
  fractionnaire de $kA$.
\end{itemize}
\item Inconvénient: plus lente que la méthode de division
\item Avantage: la valeur de $m$ n'est plus critique
\item La méthode marche mieux pour certaines valeurs de $A$. Par
  exemple:$$A=\frac{\sqrt{5}-1}{2}$$
\end{itemize}
\end{frame}

\begin{frame}{Méthode de multiplication: implémentation}
Calcul aisé si:
\begin{itemize}
\item $m=2^p$ pour un entier $p$
\item Les mots sont codés en $w$ bits et les clés $k$ peuvent être
  codées par un seul mot
\item $A$ de la forme $s/2^w$ pour $0<s<2^w$
\end{itemize}

\centerline{\includegraphics[width=8cm]{Figures/05-hashtablemultiplication.pdf}}

\bigskip

{\small\it Exemple: $m=2^3$, $w=5$ ($\Rightarrow 0<s<2^5$), $s=13$, $A=13/32$ $\Rightarrow h(21)=4$}
\end{frame}

\subsection{Adressage ouvert}

\begin{frame}{Plan}

\tableofcontents[currentsection,currentsubsection]

\end{frame}


\begin{frame}{Adressage ouvert: principe}

\begin{itemize}
\item Alternative au chaînage pour gérer les collisions
\item Tous les éléments sont stockés dans le tableau (pas de listes chaînées)
\item Ne fonctionne que si $\alpha\leq 1$
\item Pour insérer une clé $k$, on \alert{sonde} les cases
  systématiquement à partir de $h(k)$ jusqu'à en trouver une vide.
\item Différentes méthodes en fonction de la stratégie de sondage
\end{itemize}

\end{frame}

\begin{frame}{Adressage ouvert: stratégie de sondage}

On définit une nouvelle fonction de hachage qui dépend de la clé
  et du numéro du sondage:
$$h:U\times\{0,1,\ldots,m-1\} \rightarrow \{0,1,\ldots,m-1\}$$
et qui est telle que
$$\langle h(k,0), h(k,1), \ldots, h(k,m-1)\rangle$$
est une permutation de $\langle 0, 1, \ldots, m-1\rangle$.
%\item La table peut donc être totalement remplie et la suppression est difficile.


\centerline{\includegraphics[width=7cm]{Figures/05-hashopenaddressing.pdf}}

\end{frame}

\begin{frame}{Adressage ouvert: recherche et insertion}

\begin{center}
\fcolorbox{white}{Lightgray}{%
  \begin{codebox}
    \Procname{$\proc{Hash-Search}(T,k)$}
    \li $i\gets 0$
    \li \Repeat
    \li $j\gets h(k,i)$
    \li \If $T[j]\isequal k$
    \li \Then \Return j\End
    \li $i\gets i+1$
    \li \Until $T[j]\isequal \const{NIL}$ or $i\isequal m$
    \li \Return $\const{NIL}$
\end{codebox}}~~~\fcolorbox{white}{Lightgray}{%
  \begin{codebox}
    \Procname{$\proc{Hash-Insert}(T,k)$}
    \li $i\gets 0$
    \li \Repeat
    \li $j\gets h(k,i)$
    \li \If $T[j]\isequal \const{NIL}$
    \li \Then  $T[j]=k$
    \li \Return j
    \li \Else $i\gets i+1$ \End
    \li \Until $i\isequal m$
    \li \Error ``hash table overflow''
\end{codebox}}

\end{center}

\centerline{\includegraphics[width=7cm]{Figures/05-hashopenaddressing.pdf}}

\note{Suppression ??}

\end{frame}

\begin{frame}{Adressage ouvert: suppression}
\begin{itemize}
\item La suppression est possible mais pas aisée
\begin{itemize}
\item On évitera l'utilisation de l'adressage ouvert si on prévoit de nombreuses suppressions de clés dans le dictionnaire
\end{itemize}
\item On ne peut pas naïvement mettre $\const{NIL}$ dans la case contenant la clé $k$ qu'on désire effacer%\hfill (\emph{Pourquoi ?})
\item Solution:
\begin{itemize}
\item Utiliser une valeur spéciale $\const{DELETED}$ au lieu
  de $\const{NIL}$ pour signifier qu'on a effacé une valeur dans cette case
\item Lors d'une recherche: considérer un case contenant
  $\const{DELETED}$ comme une case contenant une clé
\item Lors d'une insertion: considérer une case contenant $\const{DELETED}$ comme une case vide.
\end{itemize}
\item Inconvénient: le temps de recherche ne dépend maintenant plus du facteur de charge $\alpha$ de la table%\hfill(\emph{Pourquoi ?})
\end{itemize}
\note{Demander pourquoi le temps de recherche ne dépend plus du facteur de charge (plutôt du facteur de charge maximum sur la durée de vie de la table)}
\end{frame}

\begin{frame}{Stratégies de sondage}
\begin{itemize}
\item Soit $h_k=\langle h(k,0), h(k,1), \ldots, h(k,m-1)\rangle$ la séquence de sondage correspondant à la clé $k$.
\item Hachage uniforme:
\begin{itemize}
\item chacun des $m!$ permutations de $\langle
  0,1,\ldots,m-1\rangle$ a la même probabilité d'être la séquence de
  sondage d'une clé $k$.
\item Difficile à implémenter.
\end{itemize}
\item En pratique, on se contente d'une garantie que la séquence de
  sondage soit une permutation de $\langle
  0,1,\ldots,m-1\rangle$.

\bigskip

\item Trois techniques pseudo-uniformes:
\begin{itemize}
\item sondage linéaire
\item sondage quadratique
\item double hachage
\end{itemize}
\end{itemize}

\end{frame}

\begin{frame}{Sondage linéaire}

\centerline{\includegraphics[width=7cm]{Figures/05-linearprobing.pdf}}

\bigskip

$$h(k,i)=(h'(k)+i) \bmod m,$$
où $h'(k)$ est une fonction de hachage ordinaire à valeurs dans $\{0,1,\ldots,m-1\}$.

%\bigskip

Propriétés:
\begin{itemize}
\item très facile à implémenter
\item effet de grappe fort: création de longues suites de cellules occupées
\begin{itemize}
\item La probabilité de remplir une cellule vide est $\frac{i+1}{m}$ où $i$ est le nombre de cellules pleines précédant la cellule vide
\end{itemize}
\item pas très uniforme
\end{itemize}

\note{On sonde les positions qui suivent directement la case trouvée en cyclant sur le tableau

\bigskip

longues suites augmentent les temps de calcul pour l'insertion et la recherche: on tombe souvent sur une collision}

\end{frame}

\begin{frame}{Sondage quadratique}

$$h(k,i)=(h'(k)+c_1 i+ c_2 i^2) \bmod m,$$
où $h'$ est une fonction de hachage ordinaire à valeurs dans $\{0,1,\ldots,m-1\}$, $c_1$ et $c_2$ sont deux constantes non nulles.

\bigskip

Propriétés:
\begin{itemize}
\item nécessité de bien choisir les constantes $c_1$ et $c_2$ (pour
  avoir une permutation de $\langle 0,1,\ldots,m-1\rangle$)
\item effet de grappe plus faible mais tout de même existant:
\begin{itemize}
\item  Deux clés de même valeur de hachage suivront le même chemin
$$h(k,0)=h(k',0)\Rightarrow h(k,i)=h(k',i)$$
\end{itemize}
\item meilleur que le sondage linéaire
\end{itemize}

\note{probleme: le saut ne depend pas de la clé $\rightarrow$ on crée des grappes malgré tout. Solution: rendre le saut dépendant de la clé (double hachage)}

\end{frame}

\begin{frame}{Double hachage}

\begin{columns}
\begin{column}{9cm}

$$h(k,i)=(h_1(k)+i h_2(k))\bmod m,$$
où $h_1$ et $h_2$ sont des fonctions de hachage ordinaires à valeurs dans $\{0,1,\ldots,m-1\}$.

\bigskip

Propriétés:
\begin{itemize}
\item difficile à implémenter à cause du choix de $h_1$ et $h_2$ ($h_2(k)$ doit être premier avec $m$ pour avoir une permutation de $\langle 0,1,\ldots,m-1\rangle$).
\item très proche du hachage uniforme
\item bien meilleur que les sondages linéaire et quadratique
\end{itemize}

\bigskip

\emph{Exemple: $h_1(k)=k\bmod 13$, $h_2(k)=1+(k\mod 11)$, insertion de la clé 14}

\end{column}
\begin{column}{2.5cm}
\begin{center}
\includegraphics[width=1.5cm]{Figures/05-doublehachage.pdf}
\end{center}
\end{column}
\end{columns}
\note{Pour la remarque $h_2(k)$ doit être premier avec $m$, voir le slide sur la périodicité}
\end{frame}

\begin{frame}{Adressage ouvert: élément d'analyse}
Pour une table de hachage à adressage ouvert de taille $m$ contenant $n$ éléments ($\alpha=n/m<1$) et en supposant le hachage uniforme
\begin{itemize}
\item Le nombre moyen de sondages pour une recherche négative ou un
  ajout est borné par $\frac{1}{1-\alpha}$
\item Le nombre moyen de sondages pour une recherche positive est borné par $\frac{1}{\alpha} \log \frac{1}{1-\alpha}$
\end{itemize}
%(\emph{pas démontré dans ce cours})

\bigskip

$\Rightarrow$ Si $\alpha$ est constant ($n=O(m)$), la recherche est $O(1)$.
\begin{itemize}
\item Si $\alpha=0.5$, une recherche nécessite en moyenne 2 sondages ($1/(1-0.5)$).
\item Si $\alpha=0.9$, une recherche nécessite en moyenne 10 sondages ($1/(1-0.9)$).
\end{itemize}

\note{Sans démonstration}

\end{frame}

\begin{frame}{Adressage ouvert versus chaînage}
\begin{itemize}
\item Chaînage:
\begin{itemize}
\item Peut gérer un nombre illimité d'éléments et de collisions
\item Performances plus stables
\item Surcoût lié à la gestion et le stockage en mémoire des listes liées
\end{itemize}
\item Adressage ouvert:
\begin{itemize}
\item Rapide et peu gourmand en mémoire
\item Choix de la fonction de hachage plus difficile (pour éviter les grappes)
\item On ne peut pas avoir $n>m$
\item Suppression problématique
\end{itemize}

\bigskip

\item D'autres alternatives existent:
\begin{itemize}
\item Two-probe hashing
\item Cuckoo hashing
\item \ldots
\end{itemize}
\end{itemize}

\note{Cuckoo hashing: on utilise plusieurs fonctions de hachage: si
  collision, on déplace l'élément à une nouvelle position en utilisant
  la fonction de hachage. Si re-collision, on prend la suivante, et
  ainsi de suite. Recherche: on utilise les fonctions de hachage en
  séquence.

\bigskip

Two-probe hashing: deux fonctions de hachage: on hache deux fois en cas de collision et on insère la clé dans la chaîne la plus courte. $\log\log n$ pour la longueur moyenne d'une chaîne.}

\end{frame}

\begin{frame}{Le rehachage}
\begin{itemize}
\item Lorsque $\alpha$ se rapproche de 1, les performances s'effondrent
\item Solution: \alert{rehachage}: création d'une table plus grande
\begin{itemize}
\item allocation d'une nouvelle table
\item détermination d'une nouvelle fonction de hachage, tenant compte du nouveau $m$
\item parcours des entrées de la table originale et insertion dans la nouvelle table
\end{itemize}
\item Si la taille est doublée, le coût asymptotique constant des opérations est conservé (voir slide \pageref{sec04:amortie}).
\end{itemize}

\note{Comment pourrait-on attaquer un système ?}

\end{frame}

\begin{frame}{Universal hashing}
\begin{itemize}
\item Les performances d'un table de hachage se dégrade fortement en
  cas de collisions multiples
\item Connaissant la fonction de hachage, un adversaire malintentionné pourrait s'amuser à entrer des clés créant des collisions. Exemples:
\begin{itemize}
\item Création de fichiers avec des noms bien choisis dans le kernel Linux 2.4.20
\item 28/12/2011: {\scriptsize \url{http://www.securityweek.com/hash-table-collision-attacks-could-trigger-ddos-massive-scale}}
\end{itemize}
\item C'est un exemple d'\alert{attaque par déni de service}

\bigskip

\item Parade: \alert{hachage universel}: choisir la fonction de hachage aléatoirement à chaque création d'une nouvelle instance de la table 
\item Exemple:
$$h(k)=((ak+b) \bmod p)\bmod m,$$
où $p$ est un premier très grand et $a$ et $b$ deux entiers choisis aléatoirement
\end{itemize}

\note{Si il utilise tout le temps la même clé, ça va écraser la valeur et ça ne posera pas de problème

\bigskip

fonction aléatoire $\Rightarrow$ l'utilisateur ne peut pas savoir a priori quelles clés vont créer des collisions
}

\end{frame}

\begin{frame}{Demo}

\begin{small}
\begin{itemize}
\item \url{http://groups.engin.umd.umich.edu/CIS/course.des/cis350/hashing/WEB/HashApplet.htm}
\end{itemize}
\end{small}

\end{frame}

\subsection{Comparaisons}

\begin{frame}{Plan}

\tableofcontents[currentsection,currentsubsection]

\end{frame}

\begin{frame}{Dictionnaires: résumé}

  \begin{center}\small
    \def\arraystretch{1.5}\renewcommand{\tabcolsep}{1mm}
    \begin{tabular}{@{}lcccccc@{}}
    &\multicolumn{3}{c}{\emph{Pire cas}} & \multicolumn{3}{c}{\emph{En moyenne}}\\
    \emph{Implémentation}& \proc{Search} & \proc{Insert} & \proc{Delete} & \proc{Search} & \proc{Insert} & \proc{Delete}\\
    \hline\hline
    Liste &$\Theta(n)$&$\Theta(n)$&$\Theta(n)$&$\Theta(n)$&$\Theta(n)$&$\Theta(n)$\\
    \hline
    Vecteur trié&$\Theta(\log n)$&$\Theta(n)$&$\Theta(n)$&$\Theta(\log n)$&$\Theta(n)$&$\Theta(n)$\\
\hline
ABR&$\Theta(n)$&$\Theta(n)$&$\Theta(n)$&$\Theta(\log n)$&$\Theta(\log n)$&$\Theta(\log n)$\\
\hline
AVL&$\Theta(\log n)$&$\Theta(\log n)$&$\Theta(\log n)$&$\Theta(\log n)$&$\Theta(\log n)$&$\Theta(\log n)$\\
\hline
Table de hachage & $\Theta(n)$&$\Theta(n)$&$\Theta(n)$&$\Theta(1)$&$\Theta(1)$&$\Theta(1)$\\
    \hline\hline
  \end{tabular}
  \end{center}

\begin{itemize}
\item Cas moyen valable uniquement sous l'hypothèse de hachage uniforme
\item Comment obtenir $\Theta(\log n)$ dans le pire cas avec une table de hachage ?
\end{itemize}

\note{Leur demander ici de relever les avantages et inconvénients des arbres et tables de hachage...}
\end{frame}

\begin{frame}{ABR/AVL versus table de hachage}

Tables de hachage:
\begin{itemize}
\item Faciles à implémenter
\item Seule solution pour des clés non ordonnées
\item Accès et insertion très rapides en moyenne (pour des clés simples)
\item Espace gaspillé lorsque $\alpha$ est petit
\item Pas de garantie au pire cas (performances ``instables'')
\end{itemize}

\bigskip

Arbres binaire de recherche (équilibrés):
\begin{itemize}
\item Performance garantie dans tous les cas (stabilité)
\item Taille de structure s'adapte à la taille des données
\item Supportent des opérations supplémentaires lorsque les clés sont ordonnées (parcours en ordre, successeur, prédécesseur, etc.)
\item Accès et insertion plus lente en moyenne
\end{itemize}

\end{frame}


%% \begin{frame}{Applications}

%% Quelles implémentations pour les applications suivantes:
%% \begin{itemize}
%% \item Spotlight
%% \item Itunes song
%% \item IP lookup
%% \item Dictionnaire pour le spell checking
%% \end{itemize}
%% \end{frame}

%% \begin{frame}{Ce qu'on a vu}

%% \begin{itemize}
%% \item D'autres types d'arbres équilibrés: red-black trees, ...
%% \item Fonctions de hachages sophistiquées
%% \item Démonstrations formelles de certaines complexités
%% \end{itemize}

%% \end{frame}

\part{Résolution de problèmes}

% Rajouter les slides d'outline et un découpe des sections par exemple
% Mettre Exemple 1, 2, etc.
% maximum/maximal ?

\begin{frame}{Plan}

\tableofcontents[hideallsubsections]

\end{frame}

\section{Introduction}

\begin{frame}{Méthodes de résolution de problèmes}

Quelques approches génériques pour aborder la résolution d'un problème:
\begin{itemize}
\item \alert{Approche par force brute:} résoudre directement le problème, à partir de sa définition ou par une recherche exhaustive
\semitransp{\item Diviser pour régner: diviser le problème en sous-problèmes, les résoudre, fusionner les solutions pour obtenir une solution au problème original
\item Programmation dynamique: obtenir la solution optimale à un problème en combinant des solutions optimales à des sous-problèmes similaires plus petits et se chevauchant
\item Approche gloutonne: construire la solution incrémentalement, en optimisant de manière aveugle un critère local}
\end{itemize}

\end{frame}

\section{Approche par force brute}

\begin{frame}{Approche par force brute (brute-force)}
\begin{itemize}
\item Consiste à appliquer la solution la plus directe à un problème
\item Généralement obtenue en appliquant à la lettre la définition du problème
\item Exemple simple:
\begin{itemize}
\item Rechercher un élément dans un tableau (trié ou non) en le parcourant linéairement
\item Calculer $a^n$ en multipliant $a$ $n$ fois avec lui-même
\item Implémentation récursive naïve du calcul des nombres de Fibonacci
\item \ldots
\end{itemize}
\item Souvent pas très efficace en terme de temps de calcul mais facile à implémenter et fonctionnel
\end{itemize}

\end{frame}

\begin{frame}{Exemple: tri}

Approches par force brute pour le tri:
\begin{itemize}
\item Un tableau est trié (en ordre croissant) si tout élément est plus petit que l'élément à sa droite
\item $\Rightarrow$ tri à bulle: parcourir le tableau de gauche à droite en échangeant toutes les paires d'éléments consécutifs ne respectant pas cette définition
\item Complexité: $O(n^2)$
\item $\Rightarrow$ tri par sélection: trouver le minimum du tableau, l'échanger avec le premier élément, répéter pour trier le reste du tableau
\item Complexité: $\Theta(n^2)$
\end{itemize}
\end{frame}

\begin{frame}{Recherche exhaustive}
\begin{itemize}
\item Une solution par force brute au problème de la recherche d'un élément possédant une propriété particulière
\item Générer toutes les solutions possibles jusqu'à en obtenir une qui possède la propriété recherchée
\item Exemple pour le tri:
\begin{itemize}
\item Générer toutes les permutations du tableau de départ (une et une seule fois)
\item Vérifier si chaque tableau permuté est trié. S'arrêter si c'est le cas.
\item Complexité: $O(n!\cdot n)$
\end{itemize}
\item Généralement utilisable seulement pour des problèmes de petite taille
\item Dans la plupart des cas, il existe une meilleure solution
\item Dans certains cas, c'est la seule solution possible
\end{itemize}
\end{frame}

\begin{frame}{Problème du voyageur de commerce}
\begin{itemize}
\item Etant donné $n$ villes et les distances entre ces villes
\item Trouver le plus court chemin qui passe par toutes les villes
  exactement une fois avant de revenir à la ville de départ
\end{itemize}
\begin{columns}
\begin{column}{5cm}
\centerline{\includegraphics[width=3cm]{Figures/06-tsp.pdf}}
\end{column}
\begin{column}{5cm}
\footnotesize
\begin{tabular}{ll}
Tour & Coût\\
\hline
A-B-C-D-A & 17\\
A-B-D-C-A & 21\\
A-C-B-D-A & 20\\
A-C-D-B-A & 21\\
A-D-B-C-A & 20\\
A-D-C-B-A & 17\\
\end{tabular}
\end{column}
\end{columns}

\begin{itemize}
\item Recherche exhaustive: $O(n!)$
\item On n'a pas encore pu trouver un algorithme de complexité polynomiale (et il y a peu de chance qu'on y arrive)
\end{itemize}

\end{frame}

\begin{frame}{Force brute/recherche exhaustive}
Avantages:
\begin{itemize}
\item Simple et d'application très large
\item Un bon point de départ pour trouver de meilleurs algorithmes
\item Parfois, faire mieux n'en vaut pas la peine
\end{itemize}

\bigskip

Inconvénients:
\begin{itemize}
\item Produit rarement des solutions efficaces
\item Moins éléguant et créatif que les autres techniques
\end{itemize}

\bigskip

Dans ce qui suit, on commencera la plupart du temps par fournir la solution par force brute des problèmes, qu'on cherchera ensuite à résoudre par d'autres techniques

\end{frame}

% truc interessant:

\section{Diviser pour régner}

\begin{frame}{Méthodes de résolution de problèmes}

Quelques approches génériques pour aborder la résolution d'un problème:
\begin{itemize}
\item \alert{Approche par force brute:} résoudre directement le problème, à partir de sa définition ou par une recherche exhaustive
\item \alert{Diviser pour régner:} diviser le problème en sous-problèmes, les résoudre, fusionner les solutions pour obtenir une solution au problème original
\semitransp{
\item {Programmation dynamique:} obtenir la solution optimale à un problème en combinant des solutions optimales à des sous-problèmes similaires plus petits et se chevauchant
\item {Approche gloutonne:} construire la solution incrémentalement, en optimisant de manière aveugle un critère local}
\end{itemize}

\end{frame}

\begin{frame}{Plan}

\tableofcontents[currentsection,hideothersubsections]

\end{frame}

\begin{frame}{Approche diviser-pour-régner {\it (Divide and conquer)}}

Principe général:
\begin{itemize}
\item Si le problème est trivial, on le résoud directement
\item Sinon:
\begin{enumerate}
\item Diviser le problème en sous-problèmes de taille inférieure (Diviser)
\item Résoudre récursivement ces sous-problèmes (Régner)
\item Fusionner les solutions aux sous-problèmes pour produire une solution au problème original
\end{enumerate}
\end{itemize}

\end{frame}

\begin{frame}{Exemples déjà rencontrés}

\begin{itemize}
\item \alert{Merge sort:}
\begin{enumerate}
\item Diviser: Couper le tableau en deux sous-tableaux de même taille
\item Régner: Trier récursivement les deux sous-tableaux
\item Fusionner: fusionner les deux sous-tableaux
\end{enumerate}
Complexité: $\Theta(n\log n)$ (force brute: $\Theta(n^2)$)
\item \alert{Quicksort:}
\begin{enumerate}
\item Diviser: Partionner le tableau selon le pivot
\item Régner: Trier récursivement les deux sous-tableaux
\item Fusionner: /
\end{enumerate}
Complexité en moyenne: $\Theta(n\log n)$ (force brute: $\Theta(n^2)$)
\item \alert{Recherche binaire} (dichotomique):
\begin{enumerate}
\item Diviser: Contrôler l'élement central du tableau
\item Régner: Chercher récursivement dans un des sous-tableaux
\item Fusionner: trivial
\end{enumerate}
Complexité: $O(\log n)$ (force brute: $O(n)$)
\end{itemize}

\end{frame}

\subsection{Exemple 1: calcul du minimum/maximum d'un tableau}

\begin{frame}{Exemple 1: Calcul du minimum/maximum d'un tableau}

\begin{itemize}
\item Approche par force brute pour trouver le minimum ou le maximum d'un tableau
\end{itemize}

\begin{columns}
\begin{column}{5cm}
\begin{center}
{\small
\fcolorbox{white}{Lightgray}{%
      \begin{codebox}
        \Procname{$\proc{Min}(A)$}
        \li $min\gets A[1]$
        \li \For $i\gets 2$ \To $\attrib{A}{length}$
        \li \Do \If $min>A[i]$
        \li \Then $min\gets A[i]$\End\End
        \li \Return $min$
      \end{codebox}}
}
\end{center}
\end{column}
\begin{column}{5cm}
\begin{center}
{\small
\fcolorbox{white}{Lightgray}{%
      \begin{codebox}
        \Procname{$\proc{Max}(A)$}
        \li $max\gets A[1]$
        \li \For $i\gets 2$ \To $\attrib{A}{length}$
        \li \Do \If $max<A[i]$
        \li \Then $max\gets A[i]$\End\End
        \li \Return $max$
      \end{codebox}}
}
\end{center}
\end{column}
\end{columns}

\bigskip

\begin{itemize}
\item Complexité: $\Theta(n)$ ($n-1$ comparaisons)
\item Peut-on faire mieux ?
\begin{itemize}
\item<2> Non, pas en notation asymptotique (le problème est $\Theta(n)$)
\item<2> Par contre, on peut diminuer le nombre total de comparaisons pour calculer à la fois le minimum et le maximum
\end{itemize}
\end{itemize}

\end{frame}

\begin{frame}{Calcul simultané du minimum et du maximum}

\begin{itemize}
\item Approche diviser-pour-régner pour le calcul simultané du minimum et du maximum
\end{itemize}

\begin{center}
{\small
\fcolorbox{white}{Lightgray}{%
      \begin{codebox}
        \Procname{$\proc{Max-Min}(A,p,r)$}
        \li \If $\id{r}-\id{p}\leq 1$
        \li \Then \If $A[p]<A[r]$
        \li \Then \Return $(A[r],A[p])$
        \li \Else \Return $(A[p],A[r])$\End\End
        \li $q\gets \lfloor\frac{p+r}{2}\rfloor$
        \li $(max1,min1)=\proc{Max-Min}(A,p,q)$
        \li $(max2,min2)=\proc{Max-Min}(A,q+1,r)$
        \li \Return $(\proc{max}(max1,max2),\proc{min}(min1,min2))$
      \end{codebox}}
}
\end{center}

\centerline{Appel initial: $\proc{Max-Min}(A,1,A.length)$}

\bigskip

\begin{itemize}
\item Correct ? Oui (preuve par induction)
\item Complexité ?
\end{itemize}

\end{frame}

\begin{frame}{Analyse de complexité}

\begin{itemize}
\item En supposant que $n$ est une puissance de 2, le nombre de
  comparaisons $T(n)$ est donné par: {\small
\[
T(n) = \left\{
\begin{array}{ll}
 1 & \mbox{si }n=2\\
2 T(n/2)+2 & \mbox{sinon}
\end{array}
\right.
\]}
qui se résoud en:
{\footnotesize
\begin{eqnarray*}
T(n) & = & 2 T(n/2)+2\\
%& = & 2(2 T(n/4)+2)+2\\
& = & 4 T(n/4)+4+2\\
& = & 8 T(n/4)+8+4+2\\
& = & 2^i T(n/2^i)+\sum_{j=1}^i 2^j\\
& = & 2^{\log_2(n)-1} T(2)+\sum_{j=1}^{\log_2(n)-1} 2^j\\
& = & 3/2 n -2
\end{eqnarray*}}
\item C'est-à-dire 25\% de comparaisons en moins que les méthodes séparées
\end{itemize}

\end{frame}

\subsection{Exemple 2: Recherche de pics}

\begin{frame}{Exemple 2: Recherche de pics}

\centerline{\includegraphics[width=6cm]{Figures/06-peakfinding.pdf}}

\bigskip

\begin{itemize}
\item Soit un tableau $A[1\twodots \attrib{A}{length}]$. On supposera
  que $A[0]=A[\attrib{A}{length}+1]=-\infty$.
\item Définition: $A[i]$ est un \alert{pic} s'il n'est pas plus petit que ses voisins:
$$A[i-1]\leq A[i]\geq A[i+1]$$
($A[i]$ est un maximum local)
\item \alert{But:} trouver un pic dans le tableau (n'importe lequel)
\item Note: il en existe toujours un
\end{itemize}

\end{frame}

\begin{frame}{Approche par force brute}
\begin{itemize}
\item Tester toutes les positions séquentiellement:

\bigskip

\begin{center}
{\small
\fcolorbox{white}{Lightgray}{%
      \begin{codebox}
        \Procname{$\proc{Peak1d}(A)$}
        \li \For $i\gets 1$ \To $\attrib{A}{length}$
        \li \Do \If $A[i-1]\leq A[i]\geq A[i+1]$
        \li \Then \Return $i$\End\End
      \end{codebox}}
}
\end{center}

\bigskip

\item Complexité: $O(n)$ dans le pire cas
\end{itemize}

\end{frame}

\begin{frame}{Approche par force brute 2}
\begin{itemize}
\item Le maximum global du tableau est un maximum local et donc un pic

\bigskip

\begin{center}
{\small
\fcolorbox{white}{Lightgray}{%
      \begin{codebox}
        \Procname{$\proc{Peak1d}(A)$}
        \li $m\gets A[0]$
        \li \For $i\gets 1$ \To $\attrib{A}{length}$
        \li \Do \If $A[i]>A[m]$
        \li \Then $m\gets i$\End\End
        \li \Return m
      \end{codebox}}
}
\end{center}

\bigskip

\item Complexité: $\Theta(n)$ dans tous les cas
\end{itemize}

\end{frame}

\begin{frame}{Une meilleure idée}

Approche diviser-pour-régner:
\begin{itemize}
\item Sonder un élément $A[i]$ et ses voisins $A[i-1]$ et $A[i+1]$
\item Si c'est un pic: renvoyer $i$
\item Sinon:
\begin{itemize}
\item les valeurs doivent croître au moins d'un côté
$$A[i-1]>A[i]\mbox{ ou }A[i]<A[i+1]$$
%\item Il doit y avoir un pic de ce côté
\item Si $A[i-1]>A[i]$, on cherche le pic dans $A[1\twodots i-1]$
\item Si $A[i+1]>A[i]$, on cherche le pic dans $A[i+1\twodots \attrib{A}{length}]$
\end{itemize}

\bigskip

\centerline{\includegraphics[width=7cm]{Figures/06-peakfinding-idea.pdf}}

\item A quel position $i$ faut-il sonder ?
\end{itemize}

\note{montrer graphiquement\\
Il faut sonder au milieu pour accélerer les calculs}
\end{frame}

\begin{frame}{Algorithme}

\begin{center}
{\small
\fcolorbox{white}{Lightgray}{%
      \begin{codebox}
        \Procname{$\proc{Peak1d}(A,p,r)$}
        \li $q\gets \lfloor\frac{p+r}{2}\rfloor$
        \li \If $A[q-1]\leq A[q]\geq A[q+1]$
        \li \Then \Return $q$
        \li \ElseIf $A[q-1]>A[q]$
        \li \Then \Return $\proc{Peak1d}(A,p,q-1)$
        \li \ElseIf $A[q]<A[q+1]$
        \li \Then \Return $\proc{Peak1d}(A,q+1,r)$\End
      \end{codebox}}
}
\end{center}

\centerline{Appel initial: $\proc{Peak1d}(A,1,A.length)$}

\end{frame}

\begin{frame}{Analyse}
\begin{itemize}
\item Correction: oui
\begin{itemize}
\item On doit prouver qu'il y aura un pic du côté choisi
\item Preuve par l'absurde:
\begin{itemize}
\item Supposons que $A[q+1]>A[q]$ et qu'il n'y ait pas de pic dans $A[q+1\twodots r]$
\item On doit avoir $A[q+2]>A[q+1]$ (sinon $A[q+1]$ serait un pic)
\item On doit avoir $A[q+3]>A[q+2]$ (sinon $A[q+2]$ serait un pic)
\item \ldots
\item On doit avoir $A[r]>A[r-1]$ (sinon $A[r-1]$ serait un pic)
\item Comme $A[r]>A[r+1]=-\infty$, $A[r]$ est un pic, ce qui contredit l'hypothèse
\end{itemize}
\end{itemize}

\bigskip

\item Complexité:
\begin{itemize}
\item Dans le pire cas, on a $T(n)=T(n/2)+c_1$ et $T(1)=c_2$ (idem recherche binaire)
\item $\Rightarrow T(n)=O(\log n)$
\end{itemize}
\end{itemize}
\end{frame}

\begin{frame}{Extension à un tableau 2D}

\begin{columns}
\begin{column}{5cm}
\begin{itemize}
\item Soit une matrice $n\times n$ de nombres
\item Trouver un élement plus grand ou égal à ses 4 voisins (max)
\end{itemize}

\bigskip

\centerline{\includegraphics[width=2cm]{Figures/06-2dpeak-constraints.pdf}}
\end{column}
\begin{column}{5cm}
\centerline{\includegraphics[width=5cm]{Figures/06-2dpeak.pdf}}
~\hfill{\scriptsize(Demaine \& Leiserson)}
\end{column}
\end{columns}

\bigskip

\begin{itemize}
\item Approche par force brute: $O(n^2)$
\item Recherche du maximum: $\Theta(n^2)$
\end{itemize}

\end{frame}

\begin{frame}{Approche diviser-pour-régner}

\begin{columns}
\begin{column}{7cm}
\begin{itemize}
\item Chercher le maximum global dans la colonne \alert{centrale}
\item Si c'est un pic, le renvoyer
\item Sinon appeler la fonction récursivement sur les colonnes à
  gauche (resp. droite) si le voisin à gauche (resp. droite) est plus grand
\end{itemize}

\end{column}
\begin{column}{4cm}
\centerline{\includegraphics[width=4cm]{Figures/06-2dpeak-dc.pdf}}
~\hfill{\scriptsize(Demaine \& Leiserson)}
\end{column}
\end{columns}

\bigskip

\end{frame}

\begin{frame}{Analyse: correction}

\begin{itemize}
\item On doit prouver qu'il y a bien un pic du côté choisi
\item Preuve par l'absurde:
\begin{itemize}
\item Supposons qu'il n'y ait pas de pic
\item Soient $A[i,j]$ le maximum de la colonne centrale et $A[i,k]$ le voisin le plus grand ($k=j-1$ ou $k=j+1$)
\item $A[i,k]$ doit avoir un voisin $A[p_1,q_1]$ avec une valeur plus élevée (sinon, ce serait un pic)
\item $A[p_1,q_1]$ doit avoir un voisin $A[p_2,q_2]$ avec une valeur plus élevée (sinon, ce serait un pic)
\item \ldots
\item Le voisin doit toujours rester du même côté de la colonne
  centrale (puisque $A[i,k]>A[i,j]$ et $A[i,j]$ est le maximum de la
  colonne $j$)
\item A un certain point, on va manquer de points
\item Il doit donc y avoir un pic
\end{itemize}
\end{itemize}

\end{frame}

\begin{frame}{Analyse: complexité}

\begin{itemize}
\item $O(n)$ pour trouver le maximum d'une colonne
\item $O(\log n)$ itérations
\item $O(n\log n)$ au total
\bigskip

\item Peut-on faire mieux ? Oui, il est possible de proposer un algorithme en $O(n)$ (pas vu dans ce cours)
\end{itemize}

\end{frame}

\subsection{Exemple 3: sous-séquence de somme maximale}

\begin{frame}{Exemple 3: Achat/vente d'actions}

\centerline{\includegraphics[width=8cm]{Figures/06-stockprice.pdf}}

\bigskip

\begin{itemize}
\item Soit le prix d'une action au cours de $n$ jours consécutifs (prix à la fermeture)
\item On aimerait déterminer rétrospectivement:
\begin{itemize}
\item à quel moment, on aurait dû acheter et
\item à quel moment, on aurait dû vendre
\end{itemize}
de manière à maximiser notre gain
\end{itemize}

\end{frame}

\begin{frame}{Exemple 3: Achat/vente d'actions}

Première stratégie:
\begin{itemize}
\item Acheter au prix minimum, vendre au prix maximum
\pause
\item Pas correct: Le prix maximum ne suit pas nécessairement le prix minimum
\end{itemize}

\bigskip
\pause
Deuxième stratégie:
\begin{itemize}
\item Soit acheter au prix minimum et vendre au prix le plus élevé qui suit
\item Soit vendre au prix maximum et acheter au prix le plus bas qui précède
\pause
\item Pas correct:
\end{itemize}
\vspace{-0.5cm}
~\hfill\includegraphics[width=4cm, height=2cm]{Figures/06-stockprice-2.pdf}

\bigskip
\vspace{-0.5cm}
\pause
Troisième stratégie:
\begin{itemize}
\item Tester toutes les paires (force brute)
\item Correct ? Complexité ?
\end{itemize}
\end{frame}

\begin{frame}{Achat/vente d'actions: transformation}

\centerline{\includegraphics[width=11cm]{Figures/06-stockprice-3.pdf}}

\bigskip

\begin{itemize}
\item Transformation du problème:
\begin{itemize}
\item Calculer le tableau $A[i]=\mbox{(prix du jour i)-(prix du jour i-1)}$ (de taille $A.length=n$ en supposant qu'on démarre avec un prix au jour 0)
\item Déterminer la sous-séquence  non vide contiguë de somme maximale dans $A$
\item Soit $A[i\twodots j]$ cette sous-séquence. Il aurait fallu acheter juste avant le jour $i$ (juste après le jour $i-1$) et vendu juste après le jour $j$.
\end{itemize}
\item Exemple dans le tableau ci-dessus: $A[8\twodots 11]$ est la sous-séquence maximale de somme 43 $\Rightarrow$ acheter juste avant le jour 8 et vendre juste après le jour 11.
\item Si on peut trouver la sous-séquence maximale dans un tableau, on aura une solution à notre problème d'achat/vente d'actions
%\item Est-ce que la transformation est utile ?
\end{itemize}
\end{frame}

\begin{frame}{Approche par force brute}
\centerline{\includegraphics[width=8cm]{Figures/06-maxsubarray.pdf}}

\bigskip

\begin{itemize}
\item Implémentation naïve:
\begin{itemize}
\item On génère tous les sous-tableaux 
\item On calcule la somme des éléments de chaque sous-tableau
\item On renvoie les bornes du (d'un) sous-tableau de somme maximale
\end{itemize}
\item Complexité: $\Theta(n^2)$ sous-tableaux et $O(n)$ pour le calcul
  de la somme d'un sous-tableau $\Rightarrow$ $O(n^3)$
\item On peut l'implémenter en $\Theta(n^2)$
\end{itemize}

\end{frame}

\begin{frame}{Approche par force brute}

\begin{center}
{\small
\fcolorbox{white}{Lightgray}{%
      \begin{codebox}
        \Procname{$\proc{Max-subarray-brute-force}(A)$}
        \li $n=A.length$
        \li $\id{max-so-far}\gets -\infty$
        \li \For $i\gets 1$ \To n
        \li \Do $sum=0$
        \li \For $h\gets l$ \To n
        \li \Do $sum\gets sum+A[h]$
        \li \If $sum>\id{max-so-far}$
        \li \Then $\id{max-so-far}=sum$
        \li $low=l$
        \li $high=h$\End\End\End
        \li \Return $(low,high)$
      \end{codebox}}
}
\end{center}

Complexité: $\Theta(n^2)$\\

Peut-on faire mieux ?

\end{frame}

\begin{frame}{Approche diviser-pour-régner}

\begin{itemize}
\item Nouveau problème:
\begin{itemize}
\item trouver un sous-tableau maximal dans $A[low\twodots high]$
\item fonction $\proc{maximum-subarray}(A,low,high)$
\end{itemize}
\item Diviser:
\begin{itemize}
\item diviser le sous-tableau en deux sous-tableaux de tailles aussi proches que possible
\item choisir $mid=\lfloor (low+high)/2 \rfloor$
\end{itemize}
\item Régner:
\begin{itemize}
\item trouver récursivement les sous-tableaux maximaux dans ces deux sous-tableaux
\item appeler $\proc{maximum-subarray}(A,low,mid)$ et $\proc{maximum-subarray}(A,mid+1,high)$
\end{itemize}
\item Fusionner: ?
\end{itemize}

\end{frame}

\begin{frame}{Approche diviser-pour-régner}

\centerline{\includegraphics[width=7cm]{Figures/06-maxsubarray-cross.pdf}}

\bigskip

\begin{itemize}
\item Fusionner:
\begin{itemize}
\item Rechercher un sous-tableau maximum qui traverse la jonction
\item Choisir la meilleure solution parmi les 3
\end{itemize}
\item $\proc{max-crossing-subarray}(A,low,mid,high)$
\begin{itemize}
\item Force brute: $\Theta(n^2)$ (car n/2 choix pour l'extrémité gauche, n/2 choix pour l'extrémité droite)
\item Meilleure solution: on recherche indépendamment les extrémités gauche et droite
\end{itemize}
\end{itemize}

\end{frame}

\begin{frame}{$\proc{max-crossing-subarray}$}

{\footnotesize
\fcolorbox{white}{Lightgray}{%
      \begin{codebox}
        \Procname{$\proc{Max-crossing-subarray}(A,low,mid,high)$}
        \li $\id{left-sum}=-\infty$
        \li $sum=0$
        \li \For $i\gets mid$ \Downto $low$
        \li \Do $sum=sum+A[i]$
        \li \If $sum>\id{left-sum}$
        \li \Then $\id{left-sum}=sum$
        \li $\id{max-left}=i$\End\End
        \li $\id{right-sum}=-\infty$
        \li $sum=0$
        \li \For $j\gets mid+1$ \To $high$
        \li \Do $sum=sum+A[j]$
        \li \If $sum>\id{right-sum}$
        \li \Then $\id{right-sum}=sum$
        \li $\id{max-right}=j$\End\End
        \li \Return $(\id{max-left},\id{max-right},\id{left-sum}+\id{right-sum})$
      \end{codebox}}
}\\
\vspace{-0.5cm}
Complexité: $\Theta(n)$~\hfill\includegraphics[width=6cm]{Figures/06-maxsubarray-cross-algo.pdf}
\end{frame}

\begin{frame}{$\proc{Max-subarray}$}

\begin{center}
{\footnotesize
\fcolorbox{white}{Lightgray}{%
      \begin{codebox}
        \Procname{$\proc{Max-subarray}(A,low,high)$}
        \li \If $high\isequal low$
        \li \Then \Return $(low,high,A[low])$
        \li \Else $mid=\lfloor (low+high)/2 \rfloor$
        \li $(\id{left-low},\id{left-high},\id{left-sum})=\proc{Max-subarray}(A,low,mid)$
        \li $(\id{right-low},\id{right-high},\id{right-sum})=\proc{Max-subarray}(A,mid+1,high)$
        \li $(\id{cross-low},\id{cross-high},\id{cross-sum})=$
        \li \Indent $\proc{Max-crossing-subarray}(A,low,mid,high)$
        \li \If $\id{left-sum}\geq \id{right-sum}$ and $\id{left-sum}\geq \id{cross-sum}$
        \li \Then \Return $(\id{left-low},\id{left-high},\id{left-sum})$
        \li \ElseIf $\id{right-sum}\geq \id{right-sum}$ and $\id{right-sum}\geq \id{cross-sum}$
        \li \Then \Return $(\id{right-low},\id{right-high},\id{right-sum})$
        \li \Else \Return $(\id{cross-low},\id{cross-high},\id{cross-sum})$
      \end{codebox}}
}
\end{center}

\end{frame}

\begin{frame}{Analyse}

\begin{itemize}
\item Si on suppose que $n$ est un multiple de 2, le nombre d'opérations $T(n)$ est donné par:
\[
T(n) = \left\{
\begin{array}{ll}
 c_1 & \mbox{si }n=1\\
2 T(n/2)+c_2 n & \mbox{sinon}
\end{array}
\right.
\]
\item Même complexité que le tri par fusion $\Rightarrow$ $\Theta(n\log n)$
\item Peut-on faire mieux ? On verra plus loin que oui
\end{itemize}

\end{frame}

\begin{frame}{Diviser pour régner: résumé}

\begin{itemize}
\item Mène à des algorithmes très efficaces
\item Pas toujours applicable mais quand même très utile

\bigskip

\item Applications:
\begin{itemize}
\item Tris optimaux
\item Recherche binaire
\item Problème de sélection
\item Trouver la paire de points les plus proches
\item Recherche de l'enveloppe convexe (convex-hull)
\item Multiplication de matrice (méthode de Strassens)
\item \ldots
\end{itemize}
\end{itemize}

\end{frame}

\section{Programmation dynamique}

\begin{frame}{Méthodes de résolution de problèmes}

Quelques approches génériques pour aborder la résolution d'un problème:
\begin{itemize}
\item \alert{Approche par force brute:} résoudre directement le problème, à partir de sa définition ou par une recherche exhaustive
\item \alert{Diviser pour régner:} diviser le problème en sous-problèmes, les résoudre, fusionner les solutions pour obtenir une solution au problème original
\item \alert{Programmation dynamique:} obtenir la solution optimale à un problème en combinant des solutions optimales à des sous-problèmes similaires plus petits et se chevauchant
\semitransp{\item {Approche gloutonne:} construire la solution incrémentalement, en optimisant de manière aveugle un critère local}
\end{itemize}

\end{frame}

\begin{frame}{Plan}

\tableofcontents[currentsection,hideothersubsections]

\end{frame}

\subsection{Exemple 1: découpage de tiges d'acier}

\begin{frame}{Exemple 1: découpage de tiges d'acier}

\bigskip

\centerline{\includegraphics[width=4cm]{Figures/06-steelrod.pdf}}

\bigskip

\begin{itemize}
\item Soit une tige d'acier qu'on découpe pour la vendre morceau par morceau
\item La découpe ne peut se faire que par nombre entier de centimètres
\item Le prix de vente d'une tige dépend (non linéairement) de sa longueur
\item On veut déterminer le revenu maximum qu'on peut attendre de la
  vente d'une tige de $n$ centimètres
\item Problème algorithmique:
\begin{itemize}
\item Entrée: une longueur $n>0$ et une table de prix $p_i$, pour $i=1,2,\ldots,n$
\item Sortie: le revenu maximum qu'on peut obtenir pour des tiges de longueur $n$
\end{itemize}
\end{itemize}
\note{Pourquoi est-ce que je mets non linéairement ? Parce que la dépendance est linéaire, peu importe la manière dont on coupe la tige}
\end{frame}

\begin{frame}{Illustration}
\begin{itemize}
\item Soit la table de prix:
\bigskip

\begin{center}\small
\begin{tabular}{l|llllllllll}
Longueur $i$ & 1 & 2 & 3 & 4 & 5 & 6 & 7 & 8 & 9 & 10\\
\hline
Prix $p_i$ & 1 & 5 & 8 & 9 & 10 & 17 & 17 & 20 & 24 & 30\\
\end{tabular}
\end{center}

\bigskip

\item Découpes possibles d'une tige de longueur $n=4$

\bigskip

\centerline{\includegraphics[width=10cm]{Figures/06-rodcutting.pdf}}

\bigskip

\item Meilleur revenu: découpage en 2 tiges de 2 centimètres, revenu de 10
\end{itemize}
\end{frame}

\begin{frame}{Approche par force brute}

\begin{itemize}
\item Enumérer toutes les découpes, calculer leur revenu, déterminer le revenu maximum
\item Complexité: exponentielle en $n$:
\begin{itemize}
\item Il y a $2^{n-1}$ manières de découper une tige de longueur $n$ (on peut couper ou non après chacun des $n-1$ premiers centimètres)
\item Plusieurs découpes sont équivalentes (1+1+2 et 1+2+1 par
  exemple) mais même en prenant cela en compte, le nombre de découpes
  reste exponentiel
\end{itemize}
\item Infaisable pour $n$ un peu grand
\end{itemize}

\note{dessiner un arbre avec toutes les possibilités. Dire que le nombre de feuille est $2^{n-1}$.}
\end{frame}

\begin{frame}{Idée}
\begin{itemize}
\item Soit $r_i$ le revenu maximum pour une tige de longueur $i$
\item Peut-on formuler $r_n$ de manière récursive ?
\item Déterminons $r_i$ pour notre exemple:
\begin{center}
\footnotesize
\begin{tabular}{l|cl}
i & $r_i$ & solution optimale\\
\hline
1 & 1 & 1 (pas de découpe)\\
2 & 5 & 2 (pas de découpe)\\
3 & 8 & 3 (pas de découpe)\\
4 & 10 & 2+2\\
5 & 13 & 2+3\\
6 & 17 & 6 (pas de découpe)\\
7 & 18 & 1+6 ou 2+2+3\\
8 & 22 & 2+6
\ldots\\
\end{tabular}
\end{center}
\end{itemize}

\note{Dans le cas $n=4$, on doit comparer: 4, 1+3, 2+2, 3+1, 1+1+2, 1+2+1, 2+1+1, 1+1+1+1\\

Le max de 1+3, 1+2+1, 1+1+2, 1+1+1+1 est égal au prix de 1 plus $r_3$}
\end{frame}

\begin{frame}{Formulation récursive de $r_n$: version naïve}
\begin{itemize}
\item $r_n$ peut être calculé comme le maximum de:
\begin{itemize}
\item $p_n$: le prix sans découpe
\item $r_1+r_{n-1}$: le revenu max pour une tige de 1 et une tige de $n-1$
\item $r_2+r_{n-2}$: le revenu max  pour une tige de 2 et une tige de $n-2$
\item \ldots
\item $r_{n-1}+r_1$.
\end{itemize}
\item C'est-à-dire $$r_n=\max(p_n,r_1+r_{n-1},r_2+r_{n-2},\ldots,r_{n-1}+r_1)$$
\end{itemize}
\end{frame}

\begin{frame}{Formulation récursive de $r_n$: version simplifiée}

\begin{itemize}
\item Toute solution optimale a un découpe la plus à gauche
\item On peut calculer $r_n$ en considérant toutes les tailles pour la première découpe et en combinant avec le découpage optimal pour la partie à droite
\item Pour chaque cas, on n'a donc qu'à résoudre un seul sous-problème (au lieu de deux), celui du découpage de la partie droite
\item En supposant $r_0=0$, on obtient ainsi:
$$r_n=\max_{1\leq i \leq n} (p_i+r_{n-i})$$
\end{itemize}

\note{Faire un dessin au tableau avec la partie gauche de toutes les tailles}

\end{frame}

\begin{frame}{Implémentation récursive directe}

\begin{itemize}
\item La formule récursive peut être implémentée directement

\bigskip

\begin{center}
{\footnotesize
\fcolorbox{white}{Lightgray}{%
      \begin{codebox}
        \Procname{$\proc{Cut-rod}(p,n)$}
        \li \If $n\isequal 0$
        \li \Then \Return 0\End
        \li $q=-\infty$
        \li \For $i\gets 1$ \To $n$
        \li \Do $q\gets \max(q,p[i]+\proc{Cut-Rod}(p,n-i))$\End
        \li \Return $q$
      \end{codebox}}
}
\end{center}
(p est un tableau de taille $n$ contenant les prix des tiges de tailles 1 à $n$)
\bigskip
\item Complexité ?
\end{itemize}

\end{frame}

\begin{frame}{Implémentation récursive directe: analyse}

\begin{itemize}
\item L'algorithme est extrêmement inefficace à cause des appels récursifs redondants
\item Exemple: arbre des appels récursifs pour le calcul de $r_4$

\centerline{\includegraphics[width=5cm]{Figures/06-rodcutting-recursiontree.pdf}}

%% \item Le sous-problème de taille 3 est résolu 1 fois, le sous-problème de taille 2 est résolu 2 fois, le sous-problème de taille 1 est résolu 4 fois, le sous-problème de taille 0 est résolu 8 fois
\item En général, le nombre de n\oe uds $T(n)$ de l'arbre est $2^n$.\\
{\small Preuve par induction:
\begin{itemize}
\item Cas de base: $T(0)=1$
\item Cas inductif: $$T(n)=1+\sum_{j=0}^{n-1} T(j)=1+\sum_{j=0}^{n-1} 2^j=1+2^n-1=2^n$$
\end{itemize}}
\item Complexité de l'algorithme est exponentielle en $n$
\end{itemize}
\note{Compter sur le graphe: nombre de 3, nombre de 1, nombre de 2}
\end{frame}

\begin{frame}{Solution par programmation dynamique}

\begin{itemize}
\item Solution: plutôt que de résoudre les mêmes sous-problèmes plusieurs fois,
  s'arranger pour ne les résoudre chacun qu'une seule fois
\item Comment ? En sauvegardant les solutions dans une table et en se
  référant à la table à chaque demande de résolution d'un
  sous-problème déjà rencontré
\item On échange du temps de calcul contre de la mémoire
\item Permet de transformer une solution en temps exponentiel en une solution en temps polynomial
\item Deux implémentations possibles:
\begin{itemize}
\item descendante (top-down) avec \alert{mémoization}
\item ascendante (bottom-up)
\end{itemize}

\end{itemize}

\end{frame}

\begin{frame}{Approche descendante avec mémoization}

\begin{center}
{\footnotesize
\fcolorbox{white}{Lightgray}{%
      \begin{codebox}
        \Procname{$\proc{Memoized-Cut-rod}(p,n)$}
        \li Let $r[0\twodots n]$ be a new array
        \li \For $i\gets 1$ \To $n$
        \li \Do $r[i]\gets -\infty$\End
        \li \Return $\proc{memoized-cut-rod-aux}(p,n,r)$
      \end{codebox}}

\fcolorbox{white}{Lightgray}{%
      \begin{codebox}
        \Procname{$\proc{Memoized-Cut-rod-aux}(p,n,r)$}
        \li \If $r[n]\geq 0$
        \li \Then \Return $r[n]$\End
        \li \If $n\isequal 0$
        \li \Then $q=0$
        \li \Else $q=-\infty$
        \li \For $i\gets 1$ \To $n$
        \li \Do $q=\max(q,p[i]+\proc{memoized-cut-rod-aux}(p,n-i,r))$\End\End
        \li $r[n]=q$
        \li \Return $q$
      \end{codebox}}
}
\end{center}

(Attention: suppose que le tableau est passé par pointeur)

\note{Dire qu'ils doivent bien réfléchir pour être sûr de comprendre\\

\bigskip

{\color{red} Montrer sur l'arbre de récursion comme ça fonctionne !!!}
}
\end{frame}

\begin{frame}{Approche ascendante}

Principe: résoudre les sous-problèmes par taille en commençant d'abord par les plus petits

\bigskip

\begin{center}
{\footnotesize
\fcolorbox{white}{Lightgray}{%
      \begin{codebox}
        \Procname{$\proc{Bottom-up-Cut-rod}(p,n)$}
        \li Let $r[0\twodots n]$ be a new array
        \li $r[0]=0$
        \li \For $j\gets 1$ \To $n$
        \li \Do $q=-\infty$
        \li \For  $i\gets 1$ \To $j$
        \li \Do $q=\max(q,p[i]+r[j-i])$\End
        \li $r[j]=q$\End
        \li \Return $r[n]$
      \end{codebox}}
}
\end{center}

\end{frame}

\begin{frame}{Programmation dynamique: analyse}

\begin{itemize}
\item Solution ascendante est clairement $\Theta(n^2)$ (deux boucles imbriquées)
\item Solution descendante est également $\Theta(n^2)$
\begin{itemize}
\item Chaque sous-problème est résolu une et une seule fois
\item La résolution d'un sous-problème passe par une boucle à $n$ itérations
\end{itemize}
\item Graphes des sous-problèmes:

\centerline{\includegraphics[width=1.5cm]{Figures/06-subproblems-graph.pdf}}

(une flèche de $x$ à $y$ indique que la résolution de $x$ dépend de la résolution de $y$)
\end{itemize}

\note{On voit bien que c'est $n^2$ sur ce graphe\\

On ne doit plus suivre les flèches. Chaque flèche correspond à un lookup dans une table
}

\end{frame}

\begin{frame}{Reconstruction de la solution}
\begin{itemize}
\item Fonction $\proc{Bottom-up-Cut-rod}$ calcule le revenu maximum mais ne donne pas directement la découpe correspondant à ce revenu
\item On peut étendre l'approche ascendante pour enregistrer également la solution dans une autre table
\end{itemize}

\begin{center}
{\footnotesize
\fcolorbox{white}{Lightgray}{%
      \begin{codebox}
        \Procname{$\proc{Extended-Bottom-up-Cut-rod}(p,n)$}
        \li Let $r[0\twodots n]$ and $s[0\twodots n]$ be new arrays
        \li $r[0]=0$
        \li \For $j\gets 1$ \To $n$
        \li \Do $q=-\infty$
        \li \For  $i\gets 1$ \To $j$
        \li \Do \If $q<p[i]+r[j-i]$
        \li \Then $q=p[i]+r[j-i]$
        \li $s[j]=i$\End\End
        \li $r[j]=q$\End
        \li \Return $r$ and $s$
      \end{codebox}}
}
\end{center}

\begin{itemize}
\item $s[j]$ contient la coupure la plus à gauche d'une solution
  optimale au problème de taille $j$
\end{itemize}

\note{Si on met à jour le max dans q, c'est que c'est mieux de couper d'abord en i et puis couper le reste de manière optimale}
\end{frame}

\begin{frame}{Reconstruction de la solution}
\begin{itemize}
\item Pour afficher la solution, on doit ``remonter'' dans $s$
\bigskip

\begin{center}
{\footnotesize
\fcolorbox{white}{Lightgray}{%
      \begin{codebox}
        \Procname{$\proc{Print-cut-rod-solution}(p,n)$}
        \li $(r,s)=\proc{Extended-bottom-up-cut-rod}(p,n)$
        \li \While $n>0$
        \li \Do $\proc{print}$ $s[n]$
        \li $n=n-s[n]$\End
      \end{codebox}}
}
\end{center}

\bigskip

\item Exemple:
\bigskip
\begin{center}\small
\begin{tabular}{c|llllllllll}
$i$ & 0 & 1 & 2 & 3 & 4 & 5 & 6 & 7 & 8 \\
\hline
$p[i]$ & 0 & 1 & 5 & 8 & 9 & 10 & 17 & 17 & 20 \\
$r[i]$ & 0 &  1 & 5 & 8 & 10 & 13 & 17 & 18 & 22\\
$s[i]$ & 0 & 1 & 2 & 3 & 2 & 2 & 6 & 1 & 2\\
\end{tabular}
\end{center}
\bigskip
$$\proc{print-cut-rod-solution}(p,8) \Rightarrow \mbox{"2 6"}$$

\end{itemize}
\end{frame}

\begin{frame}{Programmation dynamique: généralités}
\begin{itemize}
\item La programmation dynamique s'applique aux problèmes d'\alert{optimisation} qui peuvent se décomposer en sous-problèmes de même nature, et qui possèdent les deux propriétés suivantes:
\begin{itemize}
\item \alert{Sous-structure optimale:} on peut calculer la solution d'un problème de taille $n$ à partir de la solution de sous-problèmes de taille inférieure
\item \alert{Chevauchement des sous-problèmes:} Certains sous-problèmes distincts partagent une partie de leurs sous-problèmes
\end{itemize}
\item Implémentation directe récursive donne une solution de complexité exponentielle
\item Sauvegarde des solutions aux sous-problèmes donne une complexité linéaire dans le nombre d'arcs et de sommets du graphe des sous-problèmes
\end{itemize}
\end{frame}

\subsection{Exemple 2: Fibonacci}

\begin{frame}{Exemple 2: Fibonacci}

\begin{itemize}
\item La fonction $\proc{Fibonacci-Iter}$ vue au début du cours est un
  exemple de programmation dynamique (ascendante)
\end{itemize}

\begin{columns}
\begin{column}{5cm}
\begin{center}\footnotesize
\fcolorbox{white}{Lightgray}{
\begin{codebox}
\Procname{$\proc{Fibonacci-Iter}(n)$}
\li \If $n \leq 1$
\li \Then \Return n \End
\li \Else
\li \Then $pprev\gets 0$
\li $prev\gets 1$
\li \For $i\gets 2 \To n$
\li \Do $f\gets prev+pprev$
\li $pprev\gets prev$
\li $prev\gets f$\End
\li \Return f \End
\end{codebox}
}
\end{center}
\end{column}
\begin{column}{5cm}
\centerline{\includegraphics[width=0.8cm]{Figures/06-subproblem-graph-fibonacci.pdf}}
\end{column}
\end{columns}

\begin{itemize}
\item On peut se contenter de ne stocker que les deux dernières valeurs
\item Complexité $\Theta(n)$ (graphe contient $n+1$ n\oe uds et $2n-2$ arcs)
\end{itemize}
{\it (Exercice: écrivez la version descendante avec mémoization)}

\end{frame}

\begin{frame}{Interlude: Fibonacci en $\Theta(\log n)$}
\begin{itemize}
\item Peut-on faire mieux que $\Theta(n)$ pour Fibonacci ? Oui !
\item Propriété: {\small $$\begin{pmatrix} F_{n+1} &F_n\\F_n&F_{n-1}\\\end{pmatrix}=\begin{pmatrix} 1&1\\1&0\\\end{pmatrix}^n$$}
\item Preuve par induction:
\begin{itemize}
\item Cas de base ($n=1$): ok puisque $F_0=0$, $F_1=1$, et $F_2=1$
\item Cas inductif ($n\geq 2$):
{\small
\begin{eqnarray*}
\begin{pmatrix} F_{n+1} &F_n\\F_n&F_{n-1}\\\end{pmatrix} & = & \begin{pmatrix} F_{n} &F_{n-1}\\F_{n-1}&F_{n-2}\\\end{pmatrix} \cdot \begin{pmatrix} 1&1\\1&0\\\end{pmatrix}\\
& =&  \begin{pmatrix} 1&1\\1&0\\\end{pmatrix}^{n-1} \cdot \begin{pmatrix} 1&1\\1&0\\\end{pmatrix}\\
& = & \begin{pmatrix} 1&1\\1&0\\\end{pmatrix}^{n}
\end{eqnarray*}}\qed
\end{itemize}
\end{itemize}

\end{frame}

\begin{frame}{Interlude: Fibonacci en $\Theta(\log n)$}
\begin{itemize}
\item Approche par force brute pour le calcul de $\begin{pmatrix} 1&1\\1&0\\\end{pmatrix}^n$: $\Theta(n)$
\item Idée: utiliser le diviser-pour-régner pour le calcul de $a^n$
\[
a^n = \left\{
\begin{array}{ll}
a^{n/2} \cdot a^{n/2}  & \mbox{si }n\mbox{ est pair}\\
a^{(n-1)/2}\cdot a^{(n-1)/2} \cdot a & \mbox{si }n\mbox{ est impair}
\end{array}
\right.
\]
\item Complexité: $\Theta(\log n)$ (comme la recherche binaire)
\end{itemize}

\bigskip

{\it (Exercice: implémenter l'algorithme)}
\end{frame}

\subsection{Exemple 3: sous-séquence de somme maximale}

\begin{frame}{Exemple 3: sous-séquence maximale}
%(ou la revenge de la programmation dynamique)

\bigskip

%Algorithme de Kadane pour la recherche d'un sous-séquence de somme maximale dans un tableau

\begin{columns}
\begin{column}{7cm}
\begin{center}
{\footnotesize
\fcolorbox{white}{Lightgray}{%
      \begin{codebox}
        \Procname{$\proc{Max-subarray-linear}(A)$}
        \li Let $m[1\twodots n]$ be a new array
        \li $\id{max-so-far}\gets A[1]$
        \li $m[1]=A[1]$
        \li \For $i\gets 2$ \To $A.length$
        \li \Do \If $m[i-1]>0$
        \li \Then $m[i]=m[i-1]+A[i]$
        \li \Else $m[i]=A[i]$\End
        \li \If $m[i]>\id{max-so-far}$
        \li \Then $\id{max-so-far}=m[i]$\End\End 
        \li \Return $\id{max-so-far}$
      \end{codebox}}
}
\end{center}
\end{column}
\begin{column}{3cm}
\centerline{\includegraphics[width=0.7cm]{Figures/06-subproblems-graph-max.pdf}}
\end{column}
\end{columns}

\bigskip

\begin{itemize}
\item Complexité: $\Theta(n)$ (diviser pour régner: $\Theta(n\log n)$)
\item $m[i]$ est la somme de la sous-séquence maximale qui se termine en $i$
\item L'algorithme calcule $m[i]$ à partir de $m[i-1]$
\item Forme de programmation dynamique ascendante (très simple)
\end{itemize}

{\it\small (Exercice: ajouter le calcul des bornes d'un sous-tableau solution, remplacer le tableau $m$ par une seule variable)}

\end{frame}

\subsection{Exemple 4: plus longue sous-séquence commune}

\begin{frame}{Exemple 4: plus longue sous-séquence commune}

\begin{itemize}
\item Définition: Une \alert{sous-séquence} (non contiguë) d'une séquence $\langle x_1,\ldots,x_m\rangle$ est une séquence $\langle x_{i_1}, x_{i_2}, \ldots, x_{i_k}\rangle$, où $1\leq i_1 < i_2 < \ldots <i_k\leq m$.
\item Problème: Etant donné 2 séquences, $X=\langle
  x_1,\ldots,x_m\rangle$ et $Y=\langle y_1,\ldots,y_n\rangle$, trouver une plus grande sous-séquence commune aux deux séquences
\item Exemples:
\centerline{\includegraphics[width=8cm]{Figures/06-exemples-lcs.pdf}}
\end{itemize}
\end{frame}

\begin{frame}{Solution par force brute}

\begin{itemize}
\item On énumère toutes les sous-séquences de la séquence la plus courte
\item Pour chacune d'elles, on vérifie si c'est une sous-séquence de la séquence la plus longue

\bigskip

\item Complexité: $\Theta(n\cdot 2^m)$ (en supposant que $n<m$)
\begin{itemize}
\item $2^m$ sous-séquences possibles dans une séquence de longueur $m$
\item Vérification de l'occurence d'une sous-séquence dans une
  séquence de longueur $n$ en $\Theta(n)$

\bigskip
~\hfill{\it (Exercice:
    implémenter la vérification)}
\end{itemize}
\end{itemize}

\end{frame}

\begin{frame}{Solution par programmation dynamique}

Propriété de sous-structure:
\begin{itemize}
\item Soit $X_i=\langle x_1,\ldots,x_i\rangle$ un préfixe de $X$ et
  $Y_i=\langle y_1,\ldots,y_i\rangle$ un préfixe de $Y$
\item Soit $Z=\langle z_1,\ldots,z_k\rangle$ une plus longue sous-séquence commune de $X$ et $Y$
\item Les propriétés suivantes sont vérifiées:
\begin{itemize}
\item Si $x_m=y_n$, alors $z_k=x_m=y_n$ et $Z_{k-1}$ est une plus
  longue sous-séquence commune de $X_{m-1}$ et $Y_{n-1}$.
\item Si $x_m\neq y_n$, alors $z_k\neq x_m\Rightarrow Z$ est une plus
  longue sous-séquence commune à $X_{m-1}$ et $Y$
\item Si $x_m\neq y_n$, alors $z_k\neq y_n\Rightarrow$ $Z$ est une
  plus longue sous-séquence commune à $X$ et $Y_{n-1}$
\end{itemize}
\end{itemize}
\alert{$\Rightarrow$} Une plus longue sous-séquence commune de deux séquences a pour préfixe
une plus longue sous-séquence des préfixes des deux séquences.

\end{frame}

\begin{frame}{Solution par programmation dynamique}
\begin{itemize}
\item Soit $c[i,j]$ la longueur d'une plus longue sous-séquence de $X_i$ et $Y_j$.
\item Formulation récursive:
\[c[i,j]=\left\{
\begin{array}{ll}
0 & \mbox{ si } i=0\mbox{ ou }j=0,\\
c[i-1,j-1]+1 & \mbox{ si }i,j>0\mbox{ et }x_i=y_j,\\
\max(c[i-1,j],c[i,j-1]) & \mbox{ si }i,j>0\mbox{ et }x_i\neq y_j,\\
\end{array}
\right.
\]
\item Graphe des sous-problèmes:
\end{itemize}

\centerline{\includegraphics[width=4cm]{Figures/06-lcs-subproblemsgraph.pdf}}

\note{Leur faire remplir la table. En mettant un exemple: amputation et spanking}

\end{frame}

\begin{frame}{Implémentation (ascendante)}

{\small
\begin{center}
\fcolorbox{white}{Lightgray}{%
      \begin{codebox}
        \Procname{$\proc{LCS-Length}(X,Y,m,n)$}
        \li Let $c[0\twodots m, 0\twodots n]$ be a new table
        \li \For $i=1$ \To $m$
        \li \Do c[i,0]=0 \End
        \li \For $j=0$ \To $n$
        \li \Do c[0,j]=0 \End
        \li \For $i=1$ \To $m$
        \li \Do \For $j=1$ \To $n$
        \li \Do \If $x_i\isequal y_j$
        \li \Then $c[i,j]=c[i-1,j-1]+1$
        \li \ElseIf  $c[i-1,j]\geq c[i,j-1]$
        \li \Then $c[i,j]=c[i-1,j]$
        \li \Else $c[i,j]=c[i,j-1]$\End\End\End
        \li \Return $c$
      \end{codebox}}
\end{center}
}

\bigskip

Complexité: $\Theta(m\cdot n)$

\end{frame}

\begin{frame}{Illustration}
$amputation$ versus $spanking$

\bigskip

\centerline{\includegraphics[width=8cm]{Figures/06-lcs-algoexemple.pdf}}

\end{frame}

\begin{frame}{Trouver la plus longue sous-séquence}

\begin{columns}
\begin{column}{5cm}
{\scriptsize
\begin{center}
\fcolorbox{white}{Lightgray}{%
      \begin{codebox}
        \Procname{$\proc{LCS-Length}(X,Y,m,n)$}
        \li Let $c[0\twodots m, 0\twodots n]$ be a new table
        \li {\color{red} Let $b[1\twodots m,1\twodots n]$ be a new table}
        \li \For $i=1$ \To $m$
        \li \Do c[i,0]=0 \End
        \li \For $j=0$ \To $n$
        \li \Do c[0,j]=0 \End
        \li \For $i=1$ \To $m$
        \li \Do \For $j=1$ \To $n$
        \li \Do \If $x_i\isequal y_j$
        \li \Then $c[i,j]=c[i-1,j-1]+1$
        \li {\color{red} $b[i,j]="\nwarrow"$}
        \li \ElseIf  $c[i-1,j]\geq c[i,j-1]$
        \li \Then $c[i,j]=c[i-1,j]$
        \li {\color{red} $b[i,j]="\uparrow"$}
        \li \Else $c[i,j]=c[i,j-1]$
        \li {\color{red} $b[i,j]="\leftarrow"$}\End\End\End        
        \li \Return $c$ {\color{red} and $b$}
      \end{codebox}}
\end{center}
}
\end{column}
\begin{column}{5cm}
{\scriptsize
\begin{center}
\fcolorbox{white}{Lightgray}{%
      \begin{codebox}
        \Procname{$\proc{Print-LCS}(b,X,i,j)$}
        \li \If $i\isequal 0$ or $j\isequal 0$
        \li \Then \Return\End
        \li \If $b[i,j]\isequal"\nwarrow"$
        \li \Then $\proc{Print-LCS}(b,X,i-1,j-1)$
        \li print $x_i$\End
        \li \ElseIf $b[i,j]\isequal "\uparrow"$
        \li \Then $\proc{Print-LCS}(b,X,i-1,j)$
        \li \Else $\proc{Print-LCS}(b,X,i,j-1)$
      \end{codebox}}
\end{center}
}
\end{column}
\end{columns}

\end{frame}

\subsection{Exemple 5: le problème 0-1 du sac à dos}

\begin{frame}{Exemple 5: le problème du sac à dos (knapsack)}

Problème:
\begin{itemize}
\item Un voleur se rend dans un musée pour commettre un méfait avec un
  sac à dos pouvant contenir $W$ kg.
\item Le musée comprend $n$ \oe uvres d'art, chacune de poids $p_i$ et
  de prix $v_i$ ($i=1,\ldots,n$)
\item Le problème pour le voleur est de déterminer une sélection
  d'objets de valeur totale maximale et n'excédant pas le poids total
  admissible dans le sac à dos.
\end{itemize}

\bigskip

Formellement:
\begin{itemize}
\item Soit un ensemble $S$ de $n$ objets de poids $p_i>0$ et de valeurs $v_i>0$
\item Trouver $x_1, x_2, \ldots, x_n \in \{0,1\}$ tels que:
\begin{itemize}
\item $\sum_{i=1}^n x_i\cdot p_i\leq W$, et
\item $\sum_{i=1}^n x_i\cdot v_i$ est maximal.
\end{itemize}
\end{itemize}

\end{frame}

\begin{frame}{Exemple}

\begin{columns}
\begin{column}{5cm}

Capacité du sac à dos:

$$W=11$$
\end{column}
\begin{column}{5cm}
\begin{center}
\begin{tabular}{ccc}
$i$ & $v_i$ & $p_i$\\
\hline
1 & 1 & 1\\
2 & 6 & 2\\
3 & 18 & 5\\
4 & 22 & 6\\
5 & 28 & 7
\end{tabular}
\end{center}
\end{column}
\end{columns}

\bigskip

Exemple:
\begin{itemize}
\item $\{5,2,1\}$ a un poids de 10 et une valeur de 35
\item $\{3,4\}$ a un poids de 11 et une valeur de 40
\end{itemize}

\end{frame}

\begin{frame}{Approche par force brute}

\begin{itemize}
\item Recherche exhaustive: on énumère tous les sous-ensembles de $S$, et on calcule leur poids et leur valeur
\item Complexité en temps: $O(n 2^n)$
\item Améliorations:
\begin{itemize}
\item Ne tester que les sous-ensembles de $W/p_{min}$ objets où
  $p_{min}$ est la taille minimale
\item Tester les objets par ordre croissant et s'arrêter dès que l'un
  d'entre eux n'entre plus
\end{itemize}
\item Diminue la constante mais la complexité reste la même
\end{itemize}

\end{frame}

\begin{frame}{Approche par programmation dynamique}

\begin{itemize}
\item Définition: soit $M(k,w)$, $0\leq k\leq n$ et $0\leq w\leq W$, le bénéfice
  maximum qu'on peut obtenir avec les objets 1,\ldots,$k$ de $S$ et
  un sac à dos de charge maximale $w$\\
{\it (On suppose que les poids $p_i$ et $W$ sont entiers)}
\item Deux cas:
\begin{itemize}
\item On ne sélectionne pas l'objet $k$: $M(k,w)$ est le bénéfice maximum
  en sélectionnant parmi les $k-1$ premiers objets avec
  comme limite $w$ ($M(k-1,w)$)
\item On sélectionne l'objet $k$: $M(k,w)$ est la valeur de l'objet
  $k$ plus le bénéfice maximum en sélectionnant parmi les $k-1$
  premiers objets avec la limite $w-p_k$
\end{itemize}
\end{itemize}

\[M(k,w)=\left\{\begin{array}{ll}
0 & \mbox{ si }k=0\\
M(k-1,w) & \mbox{ si } p_k>w\\
\max\{M(k-1,w),v_k+M(k-1,w-p_k)\} &  \mbox{ sinon}
\end{array}
\right.
\]
\label{part6:knapsack}
\end{frame}

\begin{frame}{Implementation}

{\small
\begin{center}
\fcolorbox{white}{Lightgray}{%
      \begin{codebox}
        \Procname{$\proc{KnapSack}(p,v,n,W)$}
        \li Let $M[0\twodots n, 0\twodots W]$ be a new table
        \li \For $w=0$ \To $W$
        \li \Do M[0,w]=0 \End
        \li \For $k=1$ \To $n$
        \li \Do $M[k,0]=0$ \End
        \li \For $k=1$ \To $n$
        \li \Do \For $w=1$ \To $W$
        \li \Do \If $p[k]>w$
        \li \Then $M[k,w]=M[k-1,w]$
        \li \ElseIf $M[k-1,w]>v[k]+M[k-1,w-p[k]]$
        \li     \Do $M[k,w]=M[k-1,w]$
        \li \Else $M[k,w]=v[k]+M[i-1,w-p[k]]$
        \End\End\End
        \li \Return $M[n,W]$
      \end{codebox}}
\end{center}
}


\end{frame}

\begin{frame}{Exemple}

\begin{tabular}{c|cccccccccccc}
$M$ & 0 & 1 & 2 & 3 & 4 & 5 & 6 & 7 & 8 & 9 & 10 & 11\\
\hline
$\emptyset$ &     {\color{red}0} & 0 & 0 & 0 & 0 & 0 & 0 & 0 & 0 & 0 & 0 & 0\\
$\{1\}$ &         {\color{red}0} & 1 & 1 & 1 & 1 & 1 & 1 & 1 & 1 & 1 & 1 & 1\\
$\{1,2\}$ &       {\color{red}0} & 1 & 6 & 7 & 7 & 7 & 7 & 7 & 7 & 7 & 7 & 7\\
$\{1,2,3\}$ &     0 & 1 & 6 & 7 & 7 & {\color{red}18}& 19& 24& 25& 25& 25& 25\\
$\{1,2,3,4\}$ &   0 & 1 & 6 & 7 & 7 & 18& 22& 24& 28& 29& 29& {\color{red}40}\\
$\{1,2,3,4,5\}$ & 0 & 1 & 6 & 7 & 7 & 18& 22& 28& 29& 34& 35& {\color{red}40}\\
\end{tabular}

\begin{columns}
\begin{column}{5cm}
Solution optimale: $\{4, 3\}$\\
Bénéfice: $22+18=40$
\end{column}
\begin{column}{5cm}
\begin{center}
$W=11$~~~~
\begin{tabular}{ccc}
$i$ & $v_i$ & $p_i$\\
\hline
1 & 1 & 1\\
2 & 6 & 2\\
3 & 18 & 5\\
4 & 22 & 6\\
5 & 28 & 7
\end{tabular}
\end{center}
\end{column}
\end{columns}

\end{frame}


\begin{frame}{Récupération des $x_i$}

En remontant dans le tableau $M$:

{\small
\begin{center}
\fcolorbox{white}{Lightgray}{%
      \begin{codebox}
        \Procname{$\proc{KnapSack}(p,v,n,W)$}
        \li \Comment Compute M
        \li \ldots
        \li \Comment Retrieve solution
        \li Let $x[1\twodots n]$ be a new table
        \li $w=W$
        \li \For $k=n$ \Downto $1$
        \li \Do \If $M[k,w]\isequal M[k-1,w]$
        \li    \Then $x[k]=0$
        \li \Else
        \li $x[k]=1$
        \li $w=w-p[k]$ \End\End
        \li \Return $x$
      \end{codebox}}
\end{center}
}

\end{frame}

\begin{frame}{Complexité}

\begin{itemize}
\item Complexité en temps et en espace: $\Theta(n W)$
\begin{itemize}
\item Remplissage de la matrice $M$: $\Theta(n W)$
\item Recherche de la solution: $\Theta(n)$
\end{itemize}
{\it (Exercice: proposez une version $\Theta(n+W)$ en espace)}

\bigskip

\item Note: L'algorithme n'est en fait pas polynomial en fonction de la taille de l'entrée
\begin{itemize}
\item Si $W$ nécessite $n_w$ bits pour son codage, la complexité est $\Theta(n 2^{n_w})$
\item Comme pour le voyageur de commerce, on n'a pas encore trouvé d'algorithme polynomial pour le problème du sac à dos (et il y a peu de chance qu'on y arrive)
\end{itemize}
\end{itemize}
\end{frame}

\begin{frame}{Programmation dynamique: résumé}

Grandes étapes:
\begin{itemize}
\item Caractériser la structure du problème
\item Définir de manière récursive la \alert{valeur} de la solution optimale
\item Calculer les valeurs de la solution optimale (c'est-à-dire remplir un tableau)
\item Reconstruire la (une) solution optimale à partir de l'information calculée (``bottom-up'')
\end{itemize}
\bigskip

Applications:
\begin{itemize}
\item Command unix diff (comparaison de fichiers)
\item Algorithme de Viterbi (reconnaissance vocale)
\item Alignement de séquences d'ADN (Smith-Waterman)
\item Plus court chemin dans un graphe (Bellman-Ford)
\item Compilateurs (analyse syntaxique et optimisation du code)
\item \ldots
\end{itemize}

\end{frame}

\begin{frame}{Programmation dynamique versus diviser-pour-régner}
\begin{itemize}
\item L'approche Diviser-pour-régner décompose aussi le problème en sous-problèmes
\item Mais ces sous-problèmes sont significativement plus petits que le problème de départ ($n \rightarrow n/2$)
\begin{itemize}
\item Alors que la programmation dynamique réduit généralement un problème de taille $n$ en sous-problèmes de taille $n-1$
\end{itemize}
\item Et ces sous-problèmes sont indépendants
\begin{itemize}
\item Alors qu'en programmation dynamique, ils se recouvrent
\end{itemize}
\item Pour ces deux raisons, la récursivité ne fonctionne pas pour la programmation dynamique
\end{itemize}

\end{frame}

\section{Algorithmes gloutons}

\begin{frame}{Méthodes de résolution de problèmes}

Quelques approches génériques pour aborder la résolution d'un problème:
\begin{itemize}
\item \alert{Approche par force brute:} résoudre directement le problème, à partir de sa définition ou par une recherche exhaustive
\item \alert{Diviser pour régner:} diviser le problème en sous-problèmes, les résoudre, fusionner les solutions pour obtenir une solution au problème original
\item \alert{Programmation dynamique:} obtenir la solution optimale à un problème en combinant des solutions optimales à des sous-problèmes similaires plus petits et se chevauchant
\item \alert{Approche gloutonne:} construire la solution incrémentalement, en optimisant de manière aveugle un critère local
\end{itemize}

\end{frame}

\begin{frame}{Plan}

\tableofcontents[currentsection,hideothersubsections]

\end{frame}

\begin{frame}{Algorithme glouton (greedy)}

\begin{itemize}
\item Utilisé pour résoudre des problèmes d'optimisation (comme la programmation dynamique)
\item Idée principale:
\begin{itemize}
\item Quand on a un choix local à faire, faire le choix (glouton) qui semble le meilleur tout de suite (et ne jamais le remettre en question)
\end{itemize}
\item Pour que l'approche fonctionne, le problème doit satisfaire deux propriétés:
\begin{itemize}
\item \alert{Propriété des choix gloutons optimaux}: On peut toujours arriver
  à une solution optimale en faisant un choix localement optimal
\item \alert{Propriété de sous-structure optimale}: Une solution optimale du problème est composée de solutions optimales à des sous-problèmes
\end{itemize}
\item Même si ces propriétés ne sont pas satisfaites, l'approche
  gloutonne peut parfois fournir une approximation intéressante au problème
\item Parfois, il est possible de caractériser la distance de la
  solution gloutonne à la solution optimale
\end{itemize}

\end{frame}

%Change de pieces de Benard Boigelot\\

%Activity scheduling\\

%Code Huffman\\

%fractionnal Knapsack

\subsection{Exemple 1: rendre la monnaie}

\begin{frame}{Exemple 1: rendre la monnaie}

\begin{itemize}
\item Objectif: Etant donné des pièces de 1, 2, 5, 10, et 20
  cents, trouver une méthode pour rembourser une somme de $x$ cents en
  utilisant le moins de pièces possible.
\item Exemple: 34 cents:
\begin{itemize}
\item 1ère possibilité: $\{1,1,2,5,5,20\} \rightarrow 6$ pièces
\item 2ième possibilité: $\{2,2,10,20\} \rightarrow 4$ pièces
\end{itemize}
\bigskip

\item Algorithme de la caissière: A chaque itération, ajouter une
  pièce de la plus grande valeur qui ne dépasse pas la somme restant à rembourser
\item Exemple: 49 cents $\rightarrow \{20,20,5,2,2\}$ (5 pièces)
\end{itemize}

\note{Demander à l'un d'entre eux comment il ferait pour rendre la monnaie sur un montant donné.}

\end{frame}

\begin{frame}{Implémentation}

\begin{itemize}
\item Algorithme de la caissière: A chaque itération, ajouter une
  pièce de la plus grande valeur qui ne dépasse pas la somme restant à rembourser
\end{itemize}

%% \begin{columns}
%% \begin{column}{5cm}
%% {\small
%% \begin{center}
%% \fcolorbox{white}{Lightgray}{%
%%       \begin{codebox}
%%         \Procname{$\proc{CoinchangingGreedy}(x,c,n)$}
%%         \li \Comment $c[1\twodots n]$ contains the $n$ coin values in increasing order
%%         \li Let $S$ the set of coins to give to the customer
%%         \li $S=\emptyset$
%%         \li \While $x\neq 0$\Do
%%         \li  $k=$ the largest integer such that $c_k\leq x$
%%         \li  \If $k\isequal 0$
%%         \li  \Then \Return ``No solution found''\End
%%         \li  $x=x-c[k]$
%%         \li  $S=S\cup \{k\}$\End
%%         \li \Return $S$
%%       \end{codebox}}
%% \end{center}
%% }
%% \end{column}
%% \begin{column}{5cm}
{\small
\begin{center}
\fcolorbox{white}{Lightgray}{%
      \begin{codebox}
        \Procname{$\proc{CoinchangingGreedy}(x,c,n)$}
        \li \Comment $c[1\twodots n]$ contains the $n$ coin values in decreasing order
        \li Let $s[1\twodots n]$ be a new table
        \li \Comment s[i] is the number of $i$th coin in solution
        \li $CoinCount=0$
        \li \For $i=1$ \To $n$ \Do
        \li  $s[i]=\lfloor x/c[i]\rfloor$
        \li  $x=x - s[i]*c[i]$
        \li  $CoinCount=CoinCount+s[i]$\End
        \li \Return $(s, CoinCount)$
      \end{codebox}}
\end{center}
}
%% \end{column}
%% \end{columns}

\begin{itemize}
\item Complexité: $O(n)$ (sans compter le tri des pièces)
\item Cet algorithme permet-il de trouver une solution optimale?
\end{itemize}

\end{frame}

\begin{frame}{Analyse de $\proc{CoinChangingGreedy}$}

\alert{Théorème:} l'algorithme $\proc{CoinChangingGreedy}$ est optimal pour $c=[20,10,5,2,1]$

\bigskip

\alert{Preuve:}
\begin{itemize}
\item Soit $S^*(x)$ l'ensemble optimal de pièces pour un montant $x$ et soit $c^*$ le plus grand $c[i]\leq x$. On doit montrer que:
\begin{enumerate}
\item $S^*(x)$ contient $c^*$\hfill{\it (propriété des choix gloutons optimaux)}
\item $S^*(x)=\{c^*\}\cup S^*(x-c^*)$\hfill{\it (propriété de sous-structure optimale)}
\end{enumerate}

\bigskip

\item Propriété (2) découle directement de (1)
\begin{itemize}
\item $S^*(x)$ contient $c^*$ par (1)
\item Donc $S^*(x)\setminus\{c^*\}$ représente le change pour un montant de $x-c^*$
\item Ce change doit être optimal sinon $S'=\{c^*\}\cup S^*(x-c^*)$ serait une meilleure solution que $S^*(x)$ pour un montant de $x$
\item On a donc $S^*(x)=\{c^*\}\cup S^*(x-c^*)$
\end{itemize}

\end{itemize}

\end{frame}

\begin{frame}
\begin{itemize}
\item Propriété (1): $S^*(x)$ contient $c^*$
\begin{itemize}
\item Avec $c=[20,10,5,2,1]$, une solution optimale ne contient jamais:
\begin{itemize}
\item plus d'une pièce de 1, 5, ou de 10 (car $2\times 1=2$, $2\times 5=10$, $2\times 10=20$)
\item plus de deux pièces de 2 (car $3\times 2=5+1$)
\end{itemize}
\item Analysons les différents cas pour $x$:
\begin{itemize}
\item[$x=1$:] $c^*=1$, meilleure solution $S^*(x)=\{1\}$ contient $c^*$
\item[$2\leq x<5$:] $c^*=2$, avec un seul 1, on ne peut pas obtenir $x$ $\Rightarrow c^*\in S^*(x)$
\item[$x=5$:] $c^*=5$, meilleure solution $S^*(x)=\{5\}$ contient $c^*$
\item[$5< x<10$:] $c^*=5$, avec un seul 1, et deux 2, on ne peut pas obtenir $x$ $\Rightarrow c^*\in S^*(x)$
\item[$x=10$:] $c^*=10$, meilleure solution $S^*(x)=\{10\}$ contient $c^*$
\item[$10<x<20$:] $c^*=10$, avec un seul 1, deux 2, et 1 seul 5, on ne peut pas obtenir $x$ $\Rightarrow c^*\in S^*(x)$
\item[$x=20$:] $c^*=20$, meilleure solution  $S^*(x)=\{20\}$ contient $c^*$
\item[$x>20$:] $c^*=20$, avec un seul 1, deux 2, 1 seul 5 et 1 seul 10, on ne peut pas obtenir $x$ $\Rightarrow c^*\in S^*(x)$
\end{itemize}
\item $S^*(x)$ contient donc toujours bien $c^*$
\end{itemize}
\end{itemize}\qed

\end{frame}

\begin{frame}{Analyse de $\proc{CoinChangingGreedy}$}
\begin{itemize}
\item L'approche greedy n'est correcte que pour certains choix particuliers de valeurs de pièces
\begin{itemize}
\item ok pour la plupart des monnaies courantes, euros, dollars\ldots
\end{itemize}
\item Contre-exemple: $C=[1,10,21,34,70,100]$ (valeurs de timbres aux USA) et $x=140$
\begin{itemize}
\item Algorithme glouton: $100,34,1,1,1,1,1,1$
\item Solution optimale: $70,70$
\end{itemize}
\bigskip
\item Solution pour résoudre le cas général: programmation dynamique
\item Très proche du problème de découpage de tige et du sac à dos
\end{itemize}

\bigskip

{\it (Exercice: écrivez une fonction $\proc{CoinChangeDP}$)}

\end{frame}

\subsection{Exemple 2: sélection d'activités}

\begin{frame}{Exemple 2: sélection d'activités}
\begin{itemize}
\item Un salle est utilisée pour différentes activités
\begin{itemize}
\item Soit $S=\{a_1, a_2,\ldots,a_n\}$ un ensemble de $n$ activités
\item $a_i$ démarre au temps $s_i$ et se termine au temps $f_i$
\item Deux activités $a_i$ et $a_j$ sont \alert{compatibles} si soit $f_i\leq s_j$, soit $f_j\leq s_i$
\end{itemize}
Problème: trouver le plus grand sous-ensemble de tâches compatibles
\item Exemple:
\centerline{\includegraphics[width=5cm]{Figures/06-activityselection-table.pdf}}

\centerline{\includegraphics[width=9cm]{Figures/06-activityselection-graph.pdf}}

\end{itemize}
\note{Solutions optimales: par exemple: $a_1, a_3, a_6, a_8$}
\end{frame}

\begin{frame}{Sélection d'activités: approche gloutonne}


\begin{itemize}
\item Schéma d'une solution gloutonne:
\begin{itemize}
\item définir un ordre ``naturel'' sur les activités
\item sélectionner les activités dans cet ordre pour autant qu'elles soient compatibles avec celles déjà choisies
\end{itemize}
\item Exemples: trier les activités selon $s_i$ (début), selon $f_i$ (fin), selon $f_i-s_i$ (durée), nombre de conflits avec d'autres activités...
\item<2> Montrer par des contre-exemples que seul le tri selon $f_i$ fonctionne
\end{itemize}

\centerline{\includegraphics<2>[width=7cm]{Figures/04-activity-sorted.pdf}}

\note{\centerline{\includegraphics[width=10cm]{Figures/06-activity-contre-exemples.pdf}}}

\end{frame}

\begin{frame}{Sélection d'activités: approche gloutonne}
\begin{itemize}
\item Considérer les activités par ordre croissant de $f_i$ et sélectionner chaque activité compatible avec celles déjà prises
\item Implémentations: en supposant $s$ et $f$ ordonnés selon $f$
\end{itemize}

\begin{columns}
\begin{column}{6cm}
{\scriptsize
\begin{center}
\fcolorbox{white}{Lightgray}{%
      \begin{codebox}
        \Procname{$\proc{Rec-activity-selector}(s,f,k,n)$}
        \li $m=k+1$
        \li \While $m\leq n$ and $s[m]<f[k]$
        \li \Do $m=m+1$\End
        \li \If $m\leq n$
        \li \Then \Return $\{a_m\}\cup...$\Indentmore 
        \li $...\proc{Rec-activity-selector}(s,f,m,n)$\End
        \li \Else \Return $\emptyset$\End
      \end{codebox}}
\end{center}
Appel initial: $\proc{Rec-activity-selector}(s,f,0,s.length)$
}
\end{column}
\begin{column}{4cm}
{\scriptsize
\begin{center}
\fcolorbox{white}{Lightgray}{%
      \begin{codebox}
        \Procname{$\proc{Iter-activity-selector}(s,f)$}
        \li $n=s.length$
        \li $A=\{a_1\}$
        \li $k=1$
        \li \For $m=2$ \To $n$
        \li \Do \If $s[m]\geq f[k]$
        \li \Then $A=A\cup\{a_m\}$
        \li $k=m$\End\End
        \li \Return $A$
      \end{codebox}}
\end{center}
~\\
~\\
}
\end{column}
\end{columns}

\bigskip

\begin{itemize}
\item Complexité: $\Theta(n)$ ($+\Theta(n\log n)$ pour le tri selon $f_i$)
\end{itemize}

\end{frame}

\begin{frame}{Sélection d'activités: analyse}
\begin{itemize}
\item La solution gloutonne est-elle correcte ?

\bigskip

\item \alert{1. Propriété des choix gloutons optimaux:} Soit $a_x\in S$ tel que $f_x\leq f_i$ pour tout $a_i\in S$. Il existe une solution optimale $OPT^*$ qui contient $a_x$.
\item \alert{Preuve:}
\begin{itemize}
\item Soit une solution optimale $OPT$ telle que $a_x\notin OPT$
\item Soit $a_m$ l'activité qui se termine en premier dans $OPT$
\item Construisons $OPT^*=(OPT\setminus \{a_m\})\cup \{a_x\}$
\item $OPT^*$ est valide:
\begin{itemize}
\item Toute activité $a_i\in OPT\setminus \{a_m\}$ débute en un temps $s_i\geq f_m$
\item Par définition de $a_x$, $f_m\geq f_x$ et donc pour tout activité $a_i$, $s_i\geq f_x$
\item Toute activité $a_i$ est donc compatible avec $a_x$
\end{itemize}
\item $OPT^*$ est donc optimale puisque $|OPT^*|=|OPT|$
\end{itemize}\qed
\end{itemize}

\note{Montrer sur le schéma trié}

\end{frame}

\begin{frame}{Sélection d'activités: analyse}
\begin{itemize}
\item La solution gloutonne est-elle correcte ?

\bigskip

\item \alert{2. Propriété de sous-structure optimale:} Soit $a_x\in S$ le choix glouton et $S'=\{a_i|s_i\geq f_x\}$ les activités de $S$ compatibles avec $a_x$. Soit $OPT^*=\{a_x\}\cup OPT'$. Si $OPT'$ est une solution optimale pour $S'$ alors $OPT^*$ est une solution optimale pour $S$.
\item \alert{Preuve:}
\begin{itemize}
\item Soit $OPT$ une solution optimale pour $S$
\item Si $OPT^*$ n'est pas une solution optimale pour $S$, alors $|OPT^*|<|OPT|$ et donc aussi $|OPT'|<|OPT|-1$
\item Soit $a_m$ l'activité qui se termine en premier dans $OPT$ et $\bar{S}=\{a_i|s_i\geq f_m\}$
\item Par construction, $OPT\setminus\{a_m\}$ est une solution pour $\bar{S}$
\item Par construction, $\bar{S}\subseteq S'$ et $OPT\setminus\{a_m\}$
  est une solution valide pour $S'$ (pas nécessairement optimale)
\item Ce qui veut dire qu'il existe une solution pour $S'$ de taille $|OPT|-1$, ce qui contredit  $|OPT'|<|OPT|-1$ et $OPT'$ optimal pour $S'$ (par hypothèse).
\end{itemize}\qed
\end{itemize}

\end{frame}

\begin{frame}{Problèmes similaires}

D'autres problèmes similaires pour lesquels il existe un algorithme glouton:
\begin{itemize}
\item Allocation de resources:
\begin{itemize}
\item Etant donnée un ensemble d'activités $S$ avec leurs temps de début et de fin, trouver le nombre minimum de salles permettant de les réaliser toutes
%\item Solution gloutonne: trier les activités par ordre croissant du temps de départ
\end{itemize}

\bigskip

\item Planification de tâches:
\begin{itemize}
\item Soit un ensemble de tâches avec leur durée et l'instant auquel elles doivent chacune être terminées (leur deadline)
\item Sachant qu'on ne peut exécuter qu'une seule tâche simultanément,
  trouver l'ordonnancement de ces tâches qui minimise le dépassement maximal des deadlines associées aux tâches (latence).
%\item Solution gloutonne: trier les tâches en fonction de leur deadline
\end{itemize}
\end{itemize}

\note{Revenir à ma figure pour la sélection d'activité}

\end{frame}

\subsection{Exemple 3: problème du sac à dos}

\begin{frame}{Exemple 3: problème du sac à dos}

Rappel: problème (0/1) du sac à dos:
\begin{itemize}
\item Soit un ensemble $S$ de $n$ objets de poids $p_i>0$ et de valeur $v_i>0$
\item Trouver $x_1, x_2, \ldots, x_n \in \{0,1\}$ tels que:
\begin{itemize}
\item $\sum_{i=1}^n x_i\cdot p_i\leq W$, et
\item $\sum_{i=1}^n x_i\cdot v_i$ est maximal.
\end{itemize}
\end{itemize}

\bigskip

Solution par programmation dynamique: $\Theta(n W)$

\bigskip

Peut-on le résoudre par une approche gloutonne ?


\note{Leur demander de venir avec une solution.}

\end{frame}


\begin{frame}{Programmation dynamique versus approche gloutonne}
\begin{itemize}
\item Rappel du transparent \pageref{part6:knapsack}:
\begin{itemize}
\item soit $M(k,w)$, $0\leq k\leq n$ et $0\leq w\leq W$, le bénéfice
  maximum qu'on peut obtenir avec les objets 1,\ldots,$k$ de $S$ et un
  sac à dos de charge maximale $w$. On a:
  {\small \[M(k,w)=\left\{\begin{array}{ll} 0 & \mbox{ si
    }i=0\\ M(k-1,w) & \mbox{ si }
    p_i>w\\ \max\{M(k-1,w),v_k+M(k-1,w-p_k)\} & \mbox{ sinon}
\end{array}
\right.
\]}
\end{itemize}
\item Approche gloutonne: consisterait à remplacer le \alert{max} par le choix qui nous semble le meilleur localement
\item Quels choix possibles ?
\begin{itemize}
\item Le moins lourd, le plus lourd ?
\item Le moins coûteux, le plus coûteux ?
\item Le meilleur rapport valeur/poids ?
\end{itemize}
\end{itemize}

\end{frame}

\begin{frame}{Approche gloutonne}

\begin{itemize}
\item Idée d'algorithme:
\begin{itemize}
\item Ajouter à chaque itération l'objet de rapport $\frac{v_i}{p_i}$ maximal qui rentre dans le sac
\item Implémentation très proche du problème de change: $\Theta(n\log n)$
\end{itemize}
\item Est-ce que ça fonctionne ? Non !
%\begin{columns}
%\begin{column}{5cm}
%Solution optimale: $\{4, 3\}$\\
%Bénéfice: $22+18=40$
%\end{column}
%\begin{column}{5cm}
\begin{center}
\begin{tabular}{cccc}
$i$ & $v_i$ & $p_i$ & $v_i/p_i$\\
\hline
1 & 1 & 1 & 1\\
2 & 6 & 2 & 3\\
3 & 18 & 5 & 3,6\\
4 & 22 & 6 & 3,7\\
5 & 28 & 7 & 4\\
\end{tabular}
\end{center}
%\end{column}
%\end{columns}
W=11:
\begin{itemize}
\item Solution greedy: $\{5,2,1\} \Rightarrow$ valeur=35
\item Solution DP: $\{4,3\} \Rightarrow$ valeur=40
\end{itemize}
%\item Peut-on caractériser de combien on se trompe ?
\end{itemize}
\end{frame}

\begin{frame}{Problème fractionnel du sac à dos (fractional knapsack)}

Par rapport au problème 0/1, il est maintenant permis d'inclure
des fractions d'objets ($\leq 1$):
\begin{itemize}
\item Soit un ensemble $S$ de $n$ objets de poids $p_i>0$ et de valeur $v_i>0$
\item Trouver $x_1, x_2, \ldots, x_n \in {\color{red}[0,1]}$ tels que:
\begin{itemize}
\item $\sum_{i=1}^n x_i\cdot p_i\leq W$, et
\item $\sum_{i=1}^n x_i\cdot v_i$ est maximal.
\end{itemize}
\end{itemize}
\bigskip
{\small
Exemple:
\vspace{-0.5cm}
\begin{center}\small
\begin{tabular}{cccc}
$i$ & $v_i$ & $p_i$ & $v_i/p_i$\\
\hline
1 & 1 & 1 & 1\\
2 & 6 & 2 & 3\\
3 & 18 & 5 & 3,6\\
4 & 22 & 6 & 3,7\\
5 & 28 & 7 & 4\\
\end{tabular}
\end{center}
W=11:
\begin{itemize}
\item Solution optimale 0/1: $x_1=0$, $x_2=0$, $x_3=1$, $x_4=1$, $x_5=0$ $\Rightarrow$ valeur=40
\item Solution optimale fractionnelle: $x_1=0$, $x_2=0$, $x_3=0$, $x_4=2/3$, $x_5=1$ $\Rightarrow$ valeur=42,66
\end{itemize}
}
\end{frame}


\begin{frame}{Algorithme glouton}

\begin{itemize}
\item Pour la version fractionnelle, l'algorithme glouton est optimal
\item Implémentation:

\bigskip

{\small
\begin{center}
\fcolorbox{white}{Lightgray}{%
      \begin{codebox}
        \Procname{$\proc{FracKnapSack}(p,v,n,W)$}
        \li \Comment {\color{red}Assume the objects are sorted according to $v[i]/p[i]$}
        \li Let $x[1\twodots n]$ a new table
        \li $w=0$
        \li \For $i=1$ \To $n$
        
        \li \Do $d=min(p[i],W-w)$
        \li $w\gets w+d$
        \li $x[i]=d/p[i]$\End
        \li \Return $x$
      \end{codebox}}
\end{center}
}
\bigskip

\item Complexité: $\Theta(n)$ (+ $\Theta(n\log n)$ pour le tri)
\end{itemize}

\end{frame}

\begin{frame}{Correction}

\alert{Théorème:} Le problème fractionnel du sac à dos possède la propriété des choix gloutons optimaux

\bigskip

\alert{Preuve:}
\begin{itemize}
\item Soit deux objets $i$ et $j$ tels que
$$\frac{v_i}{p_i}>\frac{v_j}{p_j}$$
\item Etant donné un choix $(x_1,x_2,\ldots,x_n)$, on le transforme en $(x'_1,x'_2,\ldots, x'_n)$ tel que:
\begin{itemize}
\item $\forall k\in [1,n]\setminus\{i,j\}: x'_k=x_k,$
\item $x'_i=x_i+\frac{\Delta}{p_i}$, et
\item $x'_j=x_j-\frac{\Delta}{p_j}$,
\end{itemize}
où $\Delta=\min(p_i(1-x_i),p_jx_j)$.
\item Cette transformation ne modifie pas le poids total, mais améliore le bénéfice.
\item On en déduit qu'il est toujours avantageux de prendre la fraction maximale de l'objet $i$ possédant le plus grand rapport $\frac{v_i}{p_i}$.\qed
\end{itemize}

\end{frame}

\begin{frame}{Algorithme glouton: résumé}

\begin{itemize}
\item Très efficaces quand ils fonctionnent. Simples et faciles à
  implémenter
\item Ne fonctionnent pas toujours. Leur correction peut être assez
  difficile à prouver
\end{itemize}

\bigskip

Applications:
\begin{itemize}
\item Arbre de couverture minimal (voir partie 7)
\item Plus court chemin dans un graphe (algorithme de Dijkstra)
\item Allocation de resources
\item Codage de Huffman
\item \ldots
\end{itemize}

\end{frame}

\begin{frame}{Approche gloutonne versus programmation dynamique}

\begin{itemize}
\item Tous deux nécessitent la propriété de sous-structure optimale

\bigskip

\item Les algorithmes gloutons nécessitent que la propriété de choix gloutons optimaux soit satisfaite
\begin{itemize}
\item On n'a pas besoin de solutionner plus d'un sous-problème
\item Le choix glouton est fait \alert{avant} de résoudre le sous-problème
\item Il n'y a pas besoin de stocker les résultats intermédiares
\end{itemize}

\bigskip

\item La programmation dynamique marche sans la propriété des choix gloutons optimaux
\begin{itemize}
\item On doit solutionner plusieurs sous-problèmes et choisir dynamiquement l'un deux pour obtenir la solution globale
\item La solution doit être assemblée ``bottom-up''
\item Les sous-solutions aux sous-problèmes sont réutilisées et doivent donc être stockées
\end{itemize}
\end{itemize}

\end{frame}

\subsection{Exemple 4: codage de Huffman}

\begin{frame}{Exemple 4: codage de Huffman}
\begin{itemize}
\item Soit une séquence $S$ très longue définie sur base de 6 caractères: a, b, c, d, e et f
\begin{itemize}
\item Par exemple, $n=|S|=10^9$
\end{itemize}

\bigskip

\item Quelle est la manière la plus efficace de stocker cette séquence ?

\bigskip

\item Première approche: encoder chaque symbole par un mot binaire de
  longueur fixe:

\begin{center}
\begin{tabular}{c|cccccc}
Symbole & a & b & c & d & e & f\\
\hline
Codage & 000 & 001 & 010 & 011 & 100 & 101\\
\end{tabular}
\end{center}

\begin{itemize}
\item 6 symboles nécessitent 3 bits par symbole
\item $3\times 10^9/8=3.75\times 10^8$bytes (un peu moins de 400Mb)
\end{itemize}
\item Peut-on faire mieux ?
\end{itemize}

\note{Observer que pour e et f par exemple, il n'y a pas besoin de 3 symboles}
\end{frame}

\begin{frame}{Idée}

\begin{itemize}
\item Codage avec des mots de longueur fixe:
\begin{center}
\begin{tabular}{c|cccccc}
Symbole & a & b & c & d & e & f\\
\hline
Codage & 000 & 001 & 010 & 011 & 100 & 101\\
\end{tabular}
\end{center}

\item Observation: l'encodage de e et f est redondant:
\begin{itemize}
\item Le second bit ne nous aide pas à distinguer e de f
\item En d'autres termes, si le premier bit est 1, le second ne nous donne pas d'information et peut être supprimé
\end{itemize}
\item Suggère de considérer un codage avec des mots binaires de
  longueurs variables
\begin{center}
\begin{tabular}{c|cccccc}
Symbole & a & b & c & d & e & f\\
\hline
Codage & 000 & 001 & 010 & 011 & {\color{red}10}& {\color{red}11}\\
\end{tabular}
\end{center}
\item Encodage et décodage sont bien définis et non ambigüs
\item Permet de gagner $n_e+n_f$ bits, où $n_e$ et $n_f$ sont les nombres de e et de f dans la séquence
\end{itemize}
\end{frame}

\begin{frame}{Définition du problème}
\begin{itemize}
\item Soit un ensemble de symboles $C$ et $f(c)$ la fréquence du symbole $c\in C$.
\item Trouver un code $E: C\rightarrow \{0,1\}^*$ tel que
\begin{itemize}
\item $E$ est un code \alert{sans préfixe}
\begin{itemize}
\item Aucun mot de code $E(c_1)$ n'est le préfixe d'un autre mot de code $E(c_2)$
\end{itemize}
\item La longueur moyenne des mots de code est \alert{minimale}
$$B(S)=\sum_{c\in C} f(c)|E(c)|$$
($n B(S)$ est la longueur de l'encodage de $S$)
\end{itemize}
\item Exemple:
\begin{center}\small
\begin{tabular}{c|cccccc|c}
c & a & b & c & d & e & f & \\
f(c) & 45\% & 13\% & 12\% & 16\% & 9\% & 5\% & $B(S)$\\
\hline
Code 1 & 000 & 001 & 010 & 011 & 100& 101 & 3.00\\
Code 2 & 000 & 001 & 010 & 011 & 10& 11 & 2.86\\
Code 3 & 0 & 101 & 100 & 111 & 1101 & 1100 & 2.24\\
\end{tabular}
\end{center}
\end{itemize}

\end{frame}

\begin{frame}{Code sans préfixe}

\centerline{\includegraphics[width=5cm]{Figures/06-huffman-tree1.pdf}~~~\includegraphics[width=3.7cm]{Figures/06-huffman-tree2.pdf}}

\begin{itemize}
\item Un code sans préfixe peut toujours se représenter sous la forme d'une arbre binaire
\begin{itemize}
\item Chaque feuille est associée à un symbole
\item Le chemin de la racine à une feuille est le code du symbole
\item La fréquence d'un n\oe ud est la fréquence du préfixe
\end{itemize}
\item Un code optimal est toujours représenté par un arbre binaire entier {\it (Pourquoi ?)}
\end{itemize}

\end{frame}

\begin{frame}{Algorithme glouton}

On peut montrer que le codage optimal peut être obtenu par un
  algorithme glouton
\begin{itemize}
\item On construit l'arbre de bas en haut en partant des feuilles
\item A chaque étape, on fait le choix ``glouton'' de fusionner les deux n\oe uds les moins fréquents (symboles ou préfixes)
\end{itemize}

\bigskip

Idée de la preuve (pour information):
\begin{itemize}
\item Choix gloutons optimaux:
\begin{itemize}
\item Il existe un code sans préfixe optimal où les deux symboles les moins fréquents sont frères et à la profondeur maximale
\item Par l'absurde: si un tel code n'existait pas, on pourrait l'obtenir en échangeant la position des deux symboles les moins fréquents avec les feuilles les plus profondes sans augmenter $B(S)$
\end{itemize}
\item Sous-structure optimale:
\begin{itemize}
\item Si l'arbre qui a pour feuille le nouveau n\oe ud issu de la fusion gloutonne est optimal, l'arbre complet est optimal
\item Plus difficile à montrer
\end{itemize}
\end{itemize}

\end{frame}

\begin{frame}{Algorithme glouton: exemple}

\centerline{\includegraphics[width=11cm]{Figures/06-huffman-tree-build.pdf}}

\note{Demander comment ils l'implémenteraient}
\end{frame}

\begin{frame}{Algorithme glouton: implémentation}

\begin{center}
{\small
\fcolorbox{white}{Lightgray}{%
      \begin{codebox}
        \Procname{$\proc{Huffman}(C)$}
        \li $n=|C|$
        \li $Q=$"create a min-priority queue from $C$"
        \li \For $i=1$ \To $n-1$
        \li \Do Allocate a new node $z$
        \li $z.left=\proc{Extract-Min}(Q)$
        \li $z.right=\proc{Extract-Min}(Q)$
        \li $z.freq=z.left.freq+z.right.freq$
        \li $\proc{Insert}(Q,z)$\End
        \li \Return $\proc{Extract-min}(Q)$
      \end{codebox}}
}
\end{center}

\bigskip

\begin{itemize}
\item Implémentation avec une file à priorité
\item Complexité: $O(n\log n)$ si $Q$ est implémentée avec un tas (min)
\begin{itemize}
\item Ligne 2: $O(n)$ si on utilise $\proc{Build-min-heap}$
\item Ligne 8: $O(\log n)$ (répétée $n-1$ fois)
\end{itemize}
\end{itemize}

\end{frame}

%% \begin{frame}{Conclusion}

%% \begin{itemize}
%% \item Trois méthodes de résolutions de problèmes
%% \item Plusieurs exemples de problèmes 
%% \item Des problèmes pouvant être 
%% \end{itemize}

%% \end{frame}

\part{Graphes}

\begin{frame}{Plan}

\tableofcontents

\end{frame}

\section{Définitions}

\begin{frame}{Graphes}

\begin{itemize}
\item Un \alert{graphe (dirigé)}  est un couple $(V,E)$ où:
\begin{itemize}
\item $V$ est un ensemble de n\oe uds ({\it nodes}), ou sommets ({\it vertices}) et
\item $E\subseteq V\times V$ est un ensemble d'arcs, ou arêtes ({\it edges}).
\end{itemize}
\item Un graphe \alert{non dirigé} est caractérisé par une relation symmétrique entre les sommets
\begin{itemize}
\item Une arête est un ensemble $e=\{u,v\}$ de deux sommets
\item On la notera tout de même $(u,v)$ (équivalent à $(v,u)$). 
\end{itemize}

\item Applications: modélisation de:
\begin{itemize}
\item Réseaux sociaux
\item Réseaux informatiques
\item World Wide Web
\item Cartes routières
\item \ldots
\end{itemize}
\end{itemize}

\end{frame}

\begin{frame}{Terminologie: graphe non dirigé}

\centerline{\includegraphics[width=3cm]{Figures/07-exemple-graphenondirige.pdf}}
{\small $$V=\{1,2,3,4,5\}, E=\{(1,2),(1,5),(2,4),(2,5),(2,3),(3,4),(4,5)\}$$}
\begin{itemize}
\item Deux n\oe uds sont \alert{adjacents} s'ils sont liés par une même arête
\item Une arête $(v_1,v_2)$ est dite \alert{incidente} aux n\oe uds $v_1$ et $v_2$
\item Le \alert{degré} d'un n\oe ud est égal au nombre de ses arêtes incidentes
\item Le \alert{degré d'un graphe} est le nombre maximal d'arêtes incidentes à tout sommet.
\item Un graphe est \alert{connexe} s'il existe un chemin de tout sommet à tout autre.
\item Une \alert{composante connexe} d'un graphe non orienté est un sous-graphe connexe
  maximal de ce graphe
\end{itemize}

\end{frame}

\begin{frame}{Terminologie: graphe dirigé}

\centerline{\includegraphics[width=4cm]{Figures/07-exemple-graphedirige.pdf}}
{\small $$V=\{1,2,3,4,5,6\}, E=\{(1,2),(1,4),(2,5),(3,5),(3,6),(4,2),(5,4),(6,6)\}$$}
\begin{itemize}
\item Une arête $(v_1,v_2)$ possède l'\alert{origine} $v_1$ et la \alert{destination}
  $v_2$. Cette arête est \alert{sortante} pour $v_1$ et \alert{entrante} pour $v_2$
\item Le degré \alert{entrant} ({\it in-degree}) et le degré \alert{sortant}
  ({\it out-degree}) d'un n\oe ud $v$ sont respectivement égaux aux nombres d'arêtes entrantes et d'arêtes sortantes de $v$
\item Un graphe est \alert{acyclique} s'il n'y a aucun cycle, c'est-à-dire
  s'il n'est pas possible de suivre les arêtes du graphes à partir
  d'un sommet $x$ et de revenir à ce même sommet $x$
\end{itemize}

\end{frame}

\begin{frame}{Type de graphes}
\begin{itemize}
\item Un graphe est \alert{simple} s'il ne possède pas de boucle composée d'une seule arête, c'est-à-dire tel que:
$$\forall v \in V: (v,v)\notin E$$
\item Un \alert{arbre} est un graphe acyclique connexe
\item Un \alert{multigraphe} est une généralisation des graphes pour laquelle
  il est permis de définir plus d'une arête liant un sommet à un autre

\bigskip

\item Un graphe est \alert{pondéré} si les arêtes sont annotées par des \alert{poids}
\begin{itemize}
\item Exemple: réseau entre villes avec comme poids la distance entre
  les villes, réseau internet avec comme poids la bande passante entre routeurs, etc.
\end{itemize}
\end{itemize}
\end{frame}

\section{Représentation des graphes}

\begin{frame}{Représentation I: listes d'adjacences}

Un objet $G$ de type graphe est composé:
\begin{itemize}
\item d'une liste de n\oe uds $G.V=\{1,2,\ldots,|V|\}$
\item d'un tableau $G.Adj$ de $|V|$ listes tel que:
\begin{itemize}
\item Chaque sommet $u\in G.V$ est représenté par un élément du tableau $G.Adj$
\item $G.Adj[u]$ est la liste d'adjacence de $u$, c'est-à-dire la
  liste des sommets $v$ tels que $(u,v)\in E$
\end{itemize}
\end{itemize}

\bigskip

Permet de représenter des graphes dirigés ou non
\begin{itemize}
\item Si le graphe est dirigé (resp. non dirigé), la somme des longueurs des listes de $G.Adj$ est 
$|E|$ (resp. $2|E|$).
\end{itemize}

\bigskip

Permet de représenter un graphe pondéré en associant un poids à chaque
élément de liste

\end{frame}

\begin{frame}{Exemple}

Graphe non dirigé
\centerline{\includegraphics[width=8cm]{Figures/07-adjgraphundirected.pdf}}

\bigskip

Graphe dirigé
\centerline{\includegraphics[width=8cm]{Figures/07-adjgraphdirected.pdf}}

\end{frame}

\begin{frame}{Complexités}
\begin{itemize}
\item Complexité en espace: \uncover<2->{$O(|V|+|E|)$
\begin{itemize}
\item optimal
\end{itemize}}
\item Accéder à un sommet: \uncover<3->{$O(1)$
\begin{itemize}
\item optimal
\end{itemize}}
\item Parcourir tous les sommets: \uncover<4->{$\Theta(|V|)$
\begin{itemize}
\item optimal
\end{itemize}}
\item Parcourir toutes les arêtes: \uncover<5->{$\Theta(|V|+|E|)$
\begin{itemize}
\item ok (mais pas optimal)
\end{itemize}}
\item Vérifier l'existence d'une arête $(u,v)\in E$: \uncover<6->{$O(|V|)$
\begin{itemize}
\item ou encore $O(min(degree(u),degree(v)))$
\item mauvais
\end{itemize}}
\end{itemize}

{\it (Exercice: insertion, suppression de n\oe uds et d'arêtes ?)}

\note{Discuter des opérations d'insertion et de deletion de n\oe uds et d'arêtes}
\end{frame}

\begin{frame}{Réprésentation II: matrice d'adjacence}
\begin{itemize}
\item Les n\oe uds sont les entiers de 1 à $|V|$, $G.V=\{1,2,\ldots,|V|\}$
\item $G$ est décrit par une matrice $G.A$ de dimension $|V|\times |V|$ 
\item $G.A=(a_{ij})$ tel que
\[
a_{ij}=\left\{\begin{array}{ll}
1 & \mbox{si }(i,j)\in E\\
0 & \mbox{sinon}\\
\end{array}\right.
\]
\bigskip

\item Permet de représenter des graphes dirigés ou non
\begin{itemize}
\item $G.A$ est symmétrique si le graphe est non dirigé
\end{itemize}
\item Graphe pondéré: $a_{ij}$ est le poids de l'arête $(i,j)$ si elle existe, NIL (ou 0, ou $+\infty$) sinon
\end{itemize}
\end{frame}

\begin{frame}{Exemple}

Graphe non dirigé
\centerline{\includegraphics[width=8cm]{Figures/07-matgraphundirected.pdf}}

\bigskip

Graphe dirigé
\centerline{\includegraphics[width=8cm]{Figures/07-matgraphdirected.pdf}}

\end{frame}

\begin{frame}{Complexités}
\begin{itemize}
\item Complexité en espace: \uncover<2->{$O(|V|^2)$
\begin{itemize}
\item potentiellement très mauvais
\end{itemize}}
\item Accéder à un sommet: \uncover<3->{$O(1)$
\begin{itemize}
\item optimal
\end{itemize}}
\item Parcourir tous les sommets: \uncover<4->{$\Theta(|V|)$
\begin{itemize}
\item optimal
\end{itemize}}
\item Parcourir toutes les arêtes: \uncover<5->{$\Theta(|V|^2)$
\begin{itemize}
\item potentiellement très mauvais
\end{itemize}}
\item Vérifier l'existence d'une arête $(u,v)\in E$: \uncover<6->{$O(1)$
\begin{itemize}
\item optimal
\end{itemize}}
\end{itemize}
{\it (Exercice: insertion, suppression de n\oe uds et d'arêtes ?)}
\end{frame}

\begin{frame}{Représentations}
\begin{itemize}
\item Listes d'adjacence:
\begin{itemize}
\item Complexité en espace optimal
\item Pas appropriée pour des graphes \alert{denses\footnote{$|E|\approx |V|^2$}} et des algorithmes qui ont besoin d'accéder aux arêtes
\item Préférable pour des graphes \alert{creux\footnote{$|E|\ll |V|^2$}} ou de degré faible
\end{itemize}

\bigskip

\item Matrice d'adjacence:
\begin{itemize}
\item Complexité en espace très mauvaise pour des graphes creux
\item Appropriée pour des algorithmes qui désirent accéder aléatoirement aux arêtes
\item Préférable pour des graphes \alert{denses}
\end{itemize}
\end{itemize}

\end{frame}

\section{Parcours de graphes}

\begin{frame}{Plan}

\tableofcontents[currentsection]

\end{frame}

\begin{frame}{Parcours de graphes}
\begin{itemize}
\item Objectif: parcourir tous les sommets d'un graphe qui sont
  accessibles à partir d'un sommet $v$ donné
\item Un sommet $v'$ est accessible à partir de $v$ si:
\begin{itemize}
\item soit $v'=v$,
\item soit $v'$ est adjacent à $v$,
\item soit $v'$ est adjacent à un sommet $v''$ qui est accessible à partir de $v$
\end{itemize}

\bigskip

\item Différents types de parcours:
\begin{itemize}
\item En profondeur d'abord ({\it depth-first})
\item En largeur d'abord ({\it breadth-first})
\end{itemize}
\end{itemize}

\end{frame}

\begin{frame}{Parcours en largeur d'abord ({\it breadth-first search})}
\begin{itemize}
\item Un des algorithmes les plus simples pour parcourir un graphe
\item A la base de plusieurs algorithmes de graphe importants%(Dijkstra, Prim...)

\bigskip

\item Entrées: un graphe $G=(V,E)$ et un sommet $s\in V$
\begin{itemize}
\item Parcourt le graphe en visitant tous les sommets qui sont accessibles à partir de $s$
\item Parcourt les sommets par ordre croissant de leur distance (en
  nombre minimum d'arêtes) par rapport à $s$
\begin{itemize}
\item on visite $s$
\item tous les voisins de $s$
\item tous les voisins des voisins de $s$
\item etc.
\end{itemize}
%\item Calcule pour chaque sommet $v\in V$ sa distance $v.d$ à $s$
%\item Produit un arbre {\it en profondeur d'abord} ayant pour racine $s$
\item Fonctionne aussi bien pour des graphes dirigés que non dirigés
\end{itemize}
\end{itemize}

\end{frame}

\begin{frame}{Exemple}

\centerline{\includegraphics[width=8cm]{Figures/07-breadth-first-graph.pdf}}
Un parcours en largeur à partir de $s$: $s$-$a$-$c$-$e$-$g$-$b$-$h$-$i$-$f$

\bigskip

Pour l'implémentation:
\begin{itemize}
\item On doit retenir les sommets déjà visités de manière à éviter de
  boucler infiniment
\item On doit retenir les sommets visités dont on n'a pas encore
  visité les voisins
\end{itemize}

\end{frame}

\begin{frame}{Parcours en largeur d'abord: implémentation}

\begin{columns}
\begin{column}{6cm}
\begin{center}
{\small
\fcolorbox{white}{Lightgray}{%
      \begin{codebox}
        \Procname{$\proc{BFS}(G,s)$}
        \li \For each vertex $u \in G.V\setminus \{s\}$
        \li \Do $u.d=\infty$\End
        \li $s.d=0$
        \li $Q=$"create empty Queue"
        \li $\proc{Enqueue}(Q,s)$
        \li \While not $\proc{Queue-Empty}(Q)$
        \li \Do $u=\proc{Dequeue}(Q)$
        \li \For each $v\in G.Adj[u]$
        \li\Do \If $v.d=\infty$
        \li \Then $v.d=u.d+1$
        \li $\proc{Enqueue}(Q,v)$\End\End\End
      \end{codebox}}
}
\end{center}
\end{column}
\begin{column}{6cm}
\begin{itemize}
\item $v.d$ est la distance de $v$ à $s$
\begin{itemize}
\item si un sommet $v$ a été visité, $v.d$ est fini
\item on peut remplacer $d$ par un drapeau binaire
\end{itemize}

\bigskip

\item $Q$ est une file (LIFO) qui contient les sommets visités mais
  dont les voisins n'ont pas encore été visités
\end{itemize}
\end{column}
\end{columns}

\end{frame}


\begin{frame}{Parcours en largeur d'abord: complexité}

\begin{columns}
\begin{column}{6cm}
\begin{center}
{\small\vspace{-0.3cm}
\fcolorbox{white}{Lightgray}{%
      \begin{codebox}
        \Procname{$\proc{BFS}(G,s)$}
        \li \For each vertex $u \in G.V\setminus \{s\}$
        \li \Do $u.d=\infty$\End
        \li $s.d=0$
        \li $Q=\emptyset$
        \li $\proc{Enqueue}(Q,s)$
        \li \While $Q\neq \emptyset$
        \li \Do $u=\proc{Dequeue}(Q)$
        \li \For each $v\in G.Adj[u]$
        \li\Do \If $v.d=\infty$
        \li \Then $v.d=u.d+1$
        \li $\proc{Enqueue}(Q,v)$\End\End\End
        %% \Procname{$\proc{BFS}(G,s)$}
        %% \li \For each vertex $u \in G.V\setminus \{s\}$
        %% \li \Do $u.color=\const{White}$
        %% \li $u.d=\infty$
        %% \li $u.\pi=\const{NIL}$\End
        %% \li $s.color=\const{Gray}$
        %% \li $s.d=0$
        %% \li $s.\pi=\const{NIL}$
        %% \li $Q=\emptyset$
        %% \li $\proc{Enqueue}(Q,s)$
        %% \li \While $Q\neq \emptyset$
        %% \li \Do $u=\proc{Dequeue}(Q)$
        %% \li \For each $v\in G.Adj[u]$
        %% \li\Do \If $v.color\isequal \const{White}$
        %% \li \Then $v.color=\const{Gray}$
        %% \li $v.d=u.d+1$
        %% \li $v.\pi = u$
        %% \li $\proc{Enqueue}(Q,v)$\End\End
        %% \li $u.color=\const{Black}$\End
      \end{codebox}}
}
\end{center}
\end{column}
\begin{column}{5cm}
\begin{itemize}
\item Chaque sommet est enfilé au plus une fois
  ($v.d$ infini $\rightarrow v.d$ fini)
\item Boucle $\While$ exécutée $O(|V|)$ fois
\item Boucle interne: $O(|E|)$ \alert{au total}
\item Au total: $O(|V|+|E|)$
\end{itemize}
\end{column}
\end{columns}

\end{frame}

\begin{frame}{Parcours en largeur d'abord}

\begin{itemize}
\item Correction:
\begin{itemize}
\item L'algorithme fait bien un parcours du graphe en largeur et $v.d$ contient bien la distance minimale de $s$ à $v$
\item Ok intuitivement mais pas évident à montrer formellement. On le
  fera plus loin pour l'algorithme de Dijkstra (calcul du plus court
  chemin)
\end{itemize}

\bigskip

\item Applications:
\begin{itemize}
\item Calcul des plus courtes distances d'un sommet à tous les autres
\item Recherche du plus court chemin entre deux sommets
\item Calcul du diamètre d'un arbre
\item Tester si un graphe est biparti
\item \ldots
\end{itemize}
\end{itemize}
\end{frame}


\begin{frame}{Parcours en profondeur d'abord}

\begin{itemize}
\item Parcours du graphe en profondeur:
\begin{itemize}
\item On suit immédiatement les arêtes incidentes au dernier sommet visité
\begin{itemize}
\item Au lieu de les mettre en attente dans une file
\end{itemize}
\item On revient en arrière ({\it backtrack}) quand le sommet visité
  n'a plus de sommets adjacents non visités
\end{itemize}

\bigskip

\item Exemple:
\centerline{\includegraphics[width=10cm]{Figures/07-dfs-exemple-onenode.pdf}}

\bigskip

Parcours en profondeur à partir de $A$: $A$-$D$-$F$-$G$-$B$-$E$ ($C$ et $H$ pas accessibles)
\end{itemize}

%% \item Entrée: un graphe $G=(V,E)$ (pas de sommet source !)
%% \item Sortie: 2 ``dates'' associées à chaque sommet $v$:
%% \begin{itemize}
%% \item $v.d$=début du traitement du sommet $v$ (découverte du sommet)
%% \item $v.f$=fin du traitement du sommet $v$
%% \end{itemize}
%% \end{itemize}

\note{On ne voit pas un algo qui parcourt le graphe comme le
  bread-first parce que l'algo ici sera utile pour d'autres
  applications. Notamment le tri topologique}
\end{frame}

%% \begin{frame}{Exemple}

%% \centerline{\includegraphics[width=10cm]{Figures/07-dfs-exemple-onenode.pdf}}

%% \bigskip

%% Parcours en profondeur à partir de $A$: $A$-$D$-$F$-$G$-$B$-$E$ ($C$ et $H$ pas accessibles)

%% \end{frame}

\begin{frame}{Parcours en profondeur: implémentation avec une pile}

\begin{columns}
\begin{column}{5.5cm}
\begin{center}
{\small
\fcolorbox{white}{Lightgray}{%
      \begin{codebox}
        \Procname{$\proc{DFS}(G,s)$}
        \li \For each vertex $u \in G.V$
        \li \Do $u.visited=\const{False}$\End
        \li $S=$"create empty stack"
        \li $\proc{Push}(S,s)$
        \li \While not $\proc{Stack-empty}(S)$
        \li \Do $u=\proc{Pop}(S)$
        \li \If $u.visited\isequal \const{False}$
        \li \Then $u.visited=\const{True}$        
        \li \For each $v\in G.Adj[u]$
        \li\Do \If $v.visited\isequal\const{False}$
        \li \Then $\proc{Push}(S,v)$\End\End\End\End
      \end{codebox}}
}
\end{center}
\end{column}
\begin{column}{4.5cm}
\begin{itemize}
\item On remplace la file $Q$ par une pile $S$
\item L'attribut $visited$ marque les sommets visités

\bigskip

\item Initialisation: $\Theta(|V|)$
\item Boucle $\While$: $O(|E|+|V|)$ car:
\begin{itemize}
\item Ligne 8: $O(|V|)$ fois \alert{au total}
\item Ligne 10-11: $O(|E|)$  fois au total
\item Ligne 6: $O(|E|)$ fois au total
\end{itemize}
\item Complexité totale: $O(|V|+|E|)$
\end{itemize}
\end{column}
\end{columns}

\note{Faire tourner l'algorithme sur l'exemple précédent

\bigskip

Au plus une fois: au premier appel sur un sommet, visited est mis à
true et il n'y a plus d'autre appel sur un sommet dont visited est à true

\bigskip

Au lieu de visisted, on pourrait stocker les sommets dans une table hash}

\end{frame}

\begin{frame}{Parcours en profondeur: implémentation récursive}

\begin{columns}
\begin{column}{5cm}
\begin{center}
{\small
\fcolorbox{white}{Lightgray}{%
      \begin{codebox}
        \Procname{$\proc{DFS}(G,s)$}
        \li \For each vertex $u \in G.V$
        \li \Do $u.visited=\const{False}$\End
        \li $\proc{DFS-rec}(G,s)$
      \end{codebox}}

\bigskip

\fcolorbox{white}{Lightgray}{%
      \begin{codebox}
        \Procname{$\proc{DFS-Rec}(G,s)$}
        \li $s.visited=\const{True}$
        \li \For each $v\in G.Adj[s]$
        \li \Do \If $v.visited\isequal\const{False}$
        \li \Then $\proc{DFS-Rec}(G,v)$
      \end{codebox}}
}
\end{center}
\end{column}
\begin{column}{5.5cm}
\begin{itemize}
\item Remplace la pile par la récursion
\bigskip
\item $\proc{DFS-REC}$ appelée au plus $|V|$ fois
\item Chaque arête est considérée au plus une fois dans la boucle $\For$
\item Complexité: $O(|V|+|E|)$
\end{itemize}
\end{column}
\end{columns}

\note{Faire tourner l'algorithme sur l'exemple précédent}

\end{frame}

\begin{frame}{Parcourir tous les sommets d'un graphe}

\begin{itemize}
\item $\proc{BFS}$ et $\proc{DFS}$ ne visitent que les n\oe uds
  accessibles à partir de la source $s$
\begin{itemize}
\item Graphe non dirigé: seule la composante connexe contenant $s$ est visitée
\item Graphe dirigé: certains sommets peuvent ne pas être accessibles
  de $s$ en suivant le sens des arêtes
\end{itemize}
\item Pour parcourir tous les sommets d'un graphe:
\begin{enumerate}
\item On choisit un sommet arbitraire $v$
\item On visite tous les sommets accessibles depuis $v$ (en profondeur ou en largeur)
\item S'il reste certains sommets non visités, on en choisit un et on retourne en (2)
\end{enumerate}
\end{itemize}

\end{frame}

\begin{frame}{Parcours en profondeur de tous les sommets}

\begin{columns}
\begin{column}{5cm}
\begin{center}
{\small
\fcolorbox{white}{Lightgray}{%
      \begin{codebox}
        \Procname{$\proc{DFS-all}(G)$}
        \li \For each vertex $u \in G.V$
        \li \Do $u.visited=\const{False}$\End
        \li \For each vertex $u \in G.V$
        \li \Do \If $u.visited\isequal \const{False}$
        \li \Then $\proc{DFS-Rec}(G,u)$
      \end{codebox}}
}
\end{center}
\end{column}
\begin{column}{5cm}
\begin{center}
{\small
\fcolorbox{white}{Lightgray}{%
      \begin{codebox}
        \Procname{$\proc{DFS-Rec}(G,s)$}
        \li $s.visited=\const{True}$
        \li \For each $v\in G.Adj[s]$
        \li \Do \If $v.visited\isequal\const{False}$
        \li \Then $\proc{DFS-Rec}(G,v)$
      \end{codebox}}
}
\end{center}
\end{column}
\end{columns}

\bigskip

\begin{itemize}
\item Complexité: $\Theta(|V|+|E|)$
\begin{itemize}
\item $\proc{DFS-Rec}$ est appelé sur chaque sommet une et une seule fois
$$\Theta(|V|)$$
\item La boucle $\For$ de $\proc{DFS-Rec}$ parcourt chaque liste
  d'adjacence une et une seule fois $$\Theta(\sum_{u\in G.V} outdegree(u))=\Theta(|E|)$$
\end{itemize}
\end{itemize}

\note{Au moins une fois, par la boucle dans $\proc{DFS-All}$. Au plus
  une fois car $visited$ est mis à true la première fois}

\end{frame}

\begin{frame}{Sous-graphe de liaison}

Un parcours en profondeur de tous les sommets d'un graphe construit un
ensemble d'arbres (une \alert{forêt}), appelé sous-graphe de liaison, où:
\begin{itemize}
\item les sommets sont les sommets du graphe,
\item un sommet $w$ est le fils d'un sommet $v$ dans la forêt si
  $\proc{DFS-rec}(G,w)$ est appelé depuis $\proc{DFS-rec}(G,v)$
\end{itemize}

\bigskip

Exemple:

\centerline{\includegraphics[width=9cm]{Figures/07-dfs-forest.pdf}}

\bigskip

{\it (Exercice: modifiez $\proc{DFS-All}$ et $\proc{DFS-Rec}$ pour construire la forêt)}

\note{L'arbre n'est pas unique !!

\bigskip

Que se passe-t'il si on applique ça à un graphe non orienté ? Combien y aura-t'il d'arbres ?
}
\end{frame}

%% \begin{frame}{Parcours en profondeur d'abord: complexité}

%% \begin{columns}
%% \begin{column}{5cm}
%% \begin{center}
%% {\small
%% \fcolorbox{white}{Lightgray}{%
%%       \begin{codebox}
%%         \Procname{$\proc{DFS}(G)$}
%%         \li \For each vertex $u \in G.V$
%%         \li \Do $u.color=\const{White}$\End
%%         \li $time=0$ \Comment global variable
%%         \li \For each $u\in G.V$
%%         \li  \Do \If $u.color\isequal \const{White}$
%%         \li   \Then $\proc{DFS-Visit}(G,u)$\End\End
%%       \end{codebox}}
%% }
%% \end{center}
%% \end{column}
%% \begin{column}{5cm}
%% \begin{center}
%% {\small
%% \fcolorbox{white}{Lightgray}{%
%%       \begin{codebox}
%%         \Procname{$\proc{DFS-Visit}(G,u)$}
%%         \li $time=time+1$
%%         \li $u.d=time$
%%         \li $u.color=\const{Gray}$
%%         \li \For each $v\in G.Adj[u]$
%%         \li \Do \If $v.color\isequal \const{White}$
%%         \li \Then $\proc{DFS-Visit}(G,v)$\End\End
%%         \li $u.color=\const{Black}$
%%         \li $time = time + 1$
%%         \li $u.f=time$
%%       \end{codebox}}
%% }
%% \end{center}
%% \end{column}
%% \end{columns}

%% \bigskip

%% \begin{itemize}
%% \item Boucle lignes 4-6 de $\proc{DFS-Visit}(G,u)$: $\Theta(out-degree(u))$
%% \item $\proc{DFS-Visit}(G,u)$ est appelé une seule fois pour chaque sommet
%% \begin{itemize}
%% \item On l'appelle sur un sommet blanc uniquement et on le marque gris directement après l'appel
%% \end{itemize}
%% \item Complexité globale: $\Theta(|V|+|E|)$
%% \end{itemize}

%% \note{Pourquoi $\Theta$ ? Parce que l'algorithme parcourt tout le graphe contrairement au breadth-first}

%% \end{frame}

\begin{frame}{Application: tri topologique}
\begin{itemize}
\item Tri topologique:
\begin{itemize}
\item Etant donné un \alert{graphe acyclique dirigé} (DAG), trouver un
  ordre des sommets tel qu'il n'y ait pas d'arête d'un n\oe ud vers un
  des n\oe uds qui le précèdent dans l'ordre
\item On peut montrer que c'est possible si (et seulement si) le
  graphe est acyclique
\end{itemize}

\bigskip

\item Exemples d'applications:
\begin{itemize}
\item Trouver un ordre pour suivre un ensemble de cours qui tienne compte des prérequis de chaque cours
\begin{itemize}
\item Pour suivre SDA, il faut avoir suivi Introduction à la programmation
\end{itemize}
\item Résoudre les dépendances pour l'installation de logiciels
\begin{itemize}
\item Trouver un ordre d'installation de manière à ce que chaque logiciel soit installé après tous ceux dont il dépend
\end{itemize}
\end{itemize}
\end{itemize}

\end{frame}

\begin{frame}{Illustration}
Graphe

\centerline{\includegraphics[width=8cm]{Figures/07-tritopo-exemple.pdf}}

\bigskip

Un tri topologique

\centerline{\includegraphics[width=10cm]{Figures/07-tritopo-exemple-solution.pdf}}

\end{frame}

\begin{frame}{Marquage des sommets pour le parcours en profondeur}

\begin{itemize}
\item Dans le cadre d'un parcours en profondeur de tous les sommets,
  $\proc{DFS-rec}$ est appelé une et une seule fois sur chaque sommet
\item Lors de l'exécution de $\proc{DFS-All}$, on dira qu'un sommet
  $v$ est \alert{fini} si l'appel $\proc{DFS-rec}(G,v)$ est terminé
\item A un moment donné, les sommets peuvent être dans les trois états suivants:
\begin{itemize}
\item pas encore visité (on dira que $v$ est \alert{blanc})
\item visité mais pas encore fini ($v$ est \alert{gris})
\item fini ($v$ est \alert{noir})
\end{itemize}
\end{itemize}
\end{frame}

\begin{frame}{Marquage des sommets pour le parcours en profondeur}

\begin{columns}
\begin{column}{5cm}
\begin{center}
{\small
\fcolorbox{white}{Lightgray}{%
      \begin{codebox}
        \Procname{$\proc{DFS-all}(G)$}
        \li \For each vertex $u \in G.V$
        \li \Do $u.color=\const{White}$\End
        \li \For each vertex $u \in G.V$
        \li \Do \If $u.color\isequal \const{White}$
        \li \Then $\proc{DFS-Rec}(G,u)$
      \end{codebox}}
}
\end{center}
\end{column}
\begin{column}{5cm}
\begin{center}
{\small
\fcolorbox{white}{Lightgray}{%
      \begin{codebox}
        \Procname{$\proc{DFS-Rec}(G,s)$}
        \li $s.color=\const{Gray}$
        \li \For each $v\in G.Adj[s]$
        \li \Do \If $s.color\isequal\const{White}$
        \li \Then $\proc{DFS-Rec}(G,v)$\End\End
        \li $s.color=\const{Black}$
      \end{codebox}}
}
\end{center}
\end{column}
\end{columns}

\bigskip

\bigskip

\begin{itemize}
\item \alert{Lemme.} Soit $s$ un sommet de $G$. Considérons le moment
  de l'exécution de $\proc{DFS-All}(G)$ où $\proc{DFS-Rec}(G,s)$ est
  appelé. Pour tout sommet $v$, on a:
\begin{enumerate}
\item Si $v$ est blanc et accessible depuis $s$, alors $v$ sera noir avant $s$
\item Si $v$ est gris, alors $s$ est accessible depuis $v$
\end{enumerate}
\end{itemize}

\note{Propriété 2: si $v$ est gris, ça veut dire qu'on est dans la
  partie 2-4 de DFS-Rec et donc qu'on a atteint $s$ en parcourant le
  graphe depuis $s$ en profondeur. Donc, $s$ est accessible depuis $v$.}

\end{frame}

\begin{frame}{Trouver un tri topologique par DFS}
\begin{itemize}
\item Soit un graphe $G=(V,E)$ et l'ordre suivant défini sur $V$:
$$s\prec v\Leftrightarrow v\mbox{ devient noir avant }s$$
\item Si $G$ est un DAG, alors $\prec$ définit un ordre topologique sur $G$

\bigskip

\item \alert{Preuve:}
\begin{itemize}
\item Soit $(s,v)\in E$. On doit montrer que $s\prec v$.
\item Considérons le moment où $\proc{DFS-rec}(G,s)$ est appelé:
\begin{itemize}
\item Si $v$ est déjà noir, alors $s\prec v$ par définition de $\prec$
\item Si $v$ est blanc, alors $v$ sera noir avant $s$ par le lemme
  précédent. Donc $s\prec v$
\item Si $v$ est gris, $s$ est accessible depuis $v$ et donc il y a un
  cycle (puisque $(s,v)\in E$). Ce qui ne peut pas arriver vu que $G$ est un DAG
\end{itemize}\qed
\end{itemize}
\end{itemize}
\end{frame}

\begin{frame}{Tri topologique: implémentation}

\begin{columns}
\begin{column}{5.5cm}
\begin{center}
{\small
\fcolorbox{white}{Lightgray}{%
      \begin{codebox}
        \Procname{$\proc{Top-Sort}(G)$}
        \li \For each vertex $u \in G.V$
        \li \Do $u.color=\const{White}$\End
        \li $L=$"create empty linked list"
        \li \For each vertex $u \in G.V$
        \li \Do \If $u.color\isequal \const{White}$
        \li \Then $\proc{Top-Sort-Rec}(G,u,L)$\End\End
        \li \Return $L$
      \end{codebox}}
}
\end{center}
\end{column}
\begin{column}{5.5cm}
\begin{center}
{\small
\fcolorbox{white}{Lightgray}{%
      \begin{codebox}
        \Procname{$\proc{Top-Sort-Rec}(G,s,L)$}
        \li $s.color=\const{Gray}$
        \li \For each $v\in G.Adj[s]$
        \li \Do \If $s.color\isequal\const{White}$
        \li \Then $\proc{Top-Sort-Rec}(G,v,L)$
        \li \ElseIf $s.color\isequal\const{Grey}$
        \li \Then $\proc{Error}$ "$G$ has a cycle"\End\End        
        \li $s.color=\const{Black}$
        \li $\proc{Insert-First}(L,s)$
      \end{codebox}}
}
\end{center}
\end{column}
\end{columns}

\bigskip

Complexité: $\Theta(|V|+|E|)$

\end{frame}

\begin{frame}{Illustration}

Graphe

\centerline{\includegraphics[width=8cm]{Figures/07-tritopo-exemple.pdf}}

\bigskip

Un tri topologique

\centerline{\includegraphics[width=10cm]{Figures/07-tritopo-exemple-solution.pdf}}

\note{Est-ce que vous avez une autre idée, basée sur une approche gloutonne ?}

\end{frame}

\begin{frame}{Une autre solution}

\begin{itemize}
\item Approche gloutonne:
\begin{itemize}
\item Rechercher un sommet qui n'a pas d'arête entrante
\begin{itemize}
\item C'est toujours possible dans un graphe acyclique
\end{itemize}
\item Ajouter ce sommet à un tri topologique du graphe dont on a retiré ce sommet et toutes ses arêtes
\begin{itemize}
\item Ce graphe reste acyclique
\end{itemize}
\end{itemize}
\item Complexité identique à l'approche DFS: $\Theta(|E|+|V|)$
\end{itemize}

\end{frame}


\section{Plus courts chemins}

\begin{frame}{Plan}

\tableofcontents[currentsection]

\end{frame}

\subsection{Définitions et algorithme général}

% definition du probleme + propriétés (cycles négatifs, etc.)

\begin{frame}{Définitions}

\begin{itemize}
\item Soit un graphe dirigé $G=(V,E)$ et une fonction de poids $w:
  E\rightarrow I\!R$
\item Un chemin (du sommet $v_1$ au sommet $v_k$) est une séquence de
  n\oe uds $v_1, v_2,\ldots, v_k$ telle que $\forall i=1,\ldots,k-1$,
  $(v_i,v_{i+1})\in E$.
\item Le poids (ou coût) d'un chemin $p$ est la somme du poids des arêtes qui le composent:
$$w(p)=w(v_1,v_2)+w(v_2,v_3)+\ldots+w(v_{k-1},v_k)$$
\item Exemple
\centerline{\includegraphics[width=4cm]{Figures/07-poidschemin.pdf}}
$$w(s\rightarrow y \rightarrow t\rightarrow x\rightarrow z)=5+1+6+2=14$$
\end{itemize}

\end{frame}

\begin{frame}{Plus courts chemins: définition}

\begin{itemize}
\item Un \alert{plus court chemin} entre deux sommets $u$ et $v$ est un chemin
  $p$ de $u$ à $v$ de poids $w(p)$ le plus faible possible
\item $\delta(u,v)$ est le poids d'un plus court chemin de $u$ à $v$:
$$\delta(u,v)=\min\{w(p)|p \mbox{ est un chemin de }u\mbox{ à }v\}$$
(S'il n'y a pas de chemin entre $u$ et $v$, $\delta(u,v)=\infty$ par définition)
\item Exemples:

\bigskip

\centerline{\includegraphics[width=9cm]{Figures/07-pcc-exemples.pdf}}

%{\small(Chaque n\oe ud $v$ est marqué de la valeur de $\delta(0,v)$)}

\end{itemize}

\note{Plusieurs plus court chemin}

\end{frame}

\begin{frame}{Plus courts chemins: exemple d'application}

\centerline{\includegraphics[width=9cm]{Figures/07-pcc-montef-1.pdf}}

\end{frame}

\begin{frame}{Propriété de sous-structure optimale}

\begin{itemize}
\item \alert{Lemme:} Tout sous-chemin d'un chemin le plus court est un chemin le plus court entre ses extrémités

\bigskip

\item \alert{Preuve:} Par l'absurde
\begin{itemize}
\item Soit un plus court chemin $p_{uv}$ entre $u$ et $v$ et soit un sous-chemin $p_{xy}$ de ce chemin défini par ses extrémités $x$ et $y$
\item S'il existe un chemin plus court que $p_{xy}$ entre $x$ et $y$, on pourrait remplacer $p_{xy}$ par ce chemin dans le chemin entre $u$ et $v$ et obtenir ainsi un chemin plus court que $p_{uv}$ entre $u$ et $v$
\end{itemize}\qed
\end{itemize}

\bigskip

\centerline{\includegraphics[width=7cm]{Figures/07-pcc-sousstructure.pdf}}

\end{frame}

\begin{frame}{Poids négatifs}

\begin{itemize}
\item Les poids peuvent être négatifs
\item Ok la plupart du temps mais non permis par certains algorithmes (Dijkstra)
\item Problème en cas de cycle de poids négatif (\alert{cycle absorbant}):
\begin{itemize}
\item En restant dans le cycle négatif, on peut diminuer arbitrairement le poids du chemin
\item Par définition, on fixera $\delta(u,v)=-\infty$ s'il y a un chemin de $u$ à $v$ qui passe par un cycle négatif
\end{itemize}
\end{itemize}

\centerline{\includegraphics[width=5.5cm]{Figures/07-pcc-negative.pdf}}

$$\delta(s,e)=\delta(s,f)=\delta(s,g)=-\infty$$

\end{frame}

\begin{frame}{Cycles}

Un chemin le plus court ne peut pas contenir de cycles
\begin{itemize}
\item Cycles de poids négatifs: on les a déjà exclus par définition
\item Cycles de poids positifs: on peut obtenir un chemin plus court en les supprimant
\item Cycles de poids nuls: il n'y a pas de raison de les utiliser et donc, on ne le fera pas
\end{itemize}

\end{frame}

\begin{frame}{Plus courts chemins: variantes de problèmes}

Différentes variantes du problème:
\begin{itemize}
\item \alert{Origine unique}: trouver tous les plus courts chemins d'un sommet à tous les autres
\begin{itemize}
\item Algorithmes de Dijkstra (glouton) et Bellman-Ford (programmation dynamique)
\end{itemize}
\item \alert{Destination unique}: trouver tous les plus courts chemins de tous les sommets vers un sommet donné
\begin{itemize}
\item Essentiellement le même problème que l'origine unique
\end{itemize}
\item \alert{Paire unique}: trouver un plus court chemin entre deux sommets donnés.
\begin{itemize}
\item Pas de meilleure solution que de résoudre le problème ``origine unique''.
\end{itemize}
\item \alert{Toutes les paires}: trouver tous les plus courts chemins de $u$ à
  $v$ pour toutes les paires de sommets $(u,v)\in V\times V$.
\begin{itemize}
\item Algorithme de Floyd-Warshall
\end{itemize}
\end{itemize}

\note{On va donc se focaliser sur origine unique et toutes les paires}
\end{frame}

% Principe des algorithmes


\begin{frame}{Recherche du plus court chemin, origine unique}

\begin{itemize}
\item Entrées: un graphe dirigé pondéré et un sommet $s$ (l'origine)
\item Sorties: deux attributs pour chaque sommet:
\begin{itemize}
\item $v.d=\delta(s,v)$, la plus courte distance de $s$ vers chaque n\oe ud
\item $v.\pi=$ le prédécesseur de chaque sommet $v$ dans un plus court chemin de $s$ à $v$\\
(formant \alert{l'arbre} des plus courts chemins partant de $s$)
\end{itemize}
\end{itemize}

\centerline{\includegraphics[width=5cm]{Figures/07-pcc-sssp.pdf}}

\centerline{\small(Chaque n\oe ud $v$ est marqué de la valeur de $\delta(s,v)$)}

\end{frame}

\begin{frame}{Approche force brute}

\begin{itemize}
\item Calculer le poids de tous les chemins entre deux n\oe uds et renvoyer le plus court
\item Problème:
\begin{itemize}
\item Le nombre de chemins peut être infini (dans le cas de cycles)

\item Le nombre de chemins peut être exponentiel par rapport au nombre de sommets et d'arêtes\\

\bigskip

\centerline{\includegraphics[width=8cm]{Figures/07-exponentielnbnodes.pdf}}

\bigskip

\centerline{($O(n)$ n\oe uds et $2^n$ chemins entre $s$ et $v_n$)}

\end{itemize}
\end{itemize}

\end{frame}

\begin{frame}{Schéma général d'un algorithme}

\begin{itemize}
\item Objectif: calculer $v.d=\delta(s,v)$ pour tout $v\in V$

\bigskip

\item Idée d'algorithme:
\begin{itemize}
\item $v.d$ à une itération donnée contient une \alert{estimation} du poids d'un plus court chemin de $s$ à $v$
\item Invariant: $v.d\geq \delta(s,v)$
\item Initialisation: $v.d=+\infty$ ($\forall v \in V$)
\item A chaque itération, on tente d'améliorer (c'est-à-dire diminuer) $v.d$ en maintenant l'invariant
\item A la convergence, on aura $v.d=\delta(s,v)$
\item L'amélioration est basée sur l'utilisation de l'inégalité triangulaire
\end{itemize}
\end{itemize}

\end{frame}

\begin{frame}{Inégalité triangulaire et relâchement}
\begin{itemize}
\item \alert{Théorème:} Pour tout $u, v, x \in V$, on a
$$\delta(u,v)\leq \delta(u,x)+\delta(x,v)$$
\item \alert{Preuve:} Aller de $u$ à $v$ en empruntant un plus court chemin passant par $x$ ne peut pas être plus court qu'un plus court chemin de $u$ à $v$.

\bigskip

\item \alert{Corollaire:} Pour tout $(u,v)\in E$, on a $$\delta(s,v)\leq \delta(s,u)+w(u,v)$$

\bigskip

\item Amélioration d'une arête (\alert{Relâchement}):
\end{itemize}

\begin{columns}
\begin{column}{5cm}
\begin{center}
{\small
\fcolorbox{white}{Lightgray}{%
      \begin{codebox}
        \Procname{$\proc{Relax}(u,v,w)$}
        \li \If $v.d>u.d+w(u,v)$
        \li \Then $v.d=u.d+w(u,v)$
      \end{codebox}
}}
\end{center}
\end{column}
\begin{column}{5cm}
\centerline{\includegraphics[width=5cm]{Figures/07-pcc-relaxation.pdf}}
\end{column}
\end{columns}

\end{frame}

\begin{frame}{Schéma général d'un algorithme}

\begin{columns}
\begin{column}{5cm}
{\small
\fcolorbox{white}{Lightgray}{%
      \begin{codebox}
        \Procname{$\proc{Single-source-SP}(G,w,s)$}
        \li $\proc{Init-single-source}(G,s)$
        \li \While $\exists (u,v): v.d> u.d+w(u,v)$
        \li \Do Pick one edge (u,v)
        \li $\proc{Relax}(u,v,w)$\End
      \end{codebox}
}}
\end{column}
\begin{column}{5cm}

\begin{center}
{\small
\fcolorbox{white}{Lightgray}{%
      \begin{codebox}
        \Procname{$\proc{Init-single-source}(G,s)$}
        \li \For each $v \in G.V$
        \li \Do $v.d=\infty$\End
        \li $s.d=0$
      \end{codebox}
}}
\end{center}

\begin{center}
{\small
\fcolorbox{white}{Lightgray}{%
      \begin{codebox}
        \Procname{$\proc{Relax}(u,v,w)$}
        \li \If $v.d>u.d+w(u,v)$
        \li \Then $v.d=u.d+w(u,v)$
      \end{codebox}
}}
\end{center}
\end{column}
\end{columns}

\begin{itemize}
\item On obtient différents algorithmes en modifiant la manière dont on sélectionne les arêtes
\end{itemize}

\end{frame}

\begin{frame}{Schéma général d'un algorithme}


\begin{columns}
\begin{column}{5cm}
\begin{center}
{\small
\fcolorbox{white}{Lightgray}{%
      \begin{codebox}
        \Procname{$\proc{Single-source-SP}(G,w,s)$}
        \li $\proc{Init-single-source}(G,s)$
        \li \While $\exists (u,v): v.d\geq u.d+w(u,v)$
        \li \Do Pick one edge (u,v)
        \li $\proc{Relax}(u,v,w)$\End
      \end{codebox}
}}
\end{center}
\end{column}
\begin{column}{5cm}

\begin{center}
{\small
\fcolorbox{white}{Lightgray}{%
      \begin{codebox}
        \Procname{$\proc{Init-single-source}(G,s)$}
        \li \For each $v \in G.V$
        \li \Do $v.d=\infty$
        \li  {\color{red}$v.\pi=\const{NIL}$} \End
        \li $s.d=0$
      \end{codebox}
}}
\end{center}

\begin{center}
{\small
\fcolorbox{white}{Lightgray}{%
      \begin{codebox}
        \Procname{$\proc{Relax}(u,v,w)$}
        \li \If $v.d>u.d+w(u,v)$
        \li \Then $v.d=u.d+w(u,v)$
        \li {\color{red} $v.\pi=u$}\End
      \end{codebox}
}}
\end{center}
\end{column}
\end{columns}

\begin{itemize}
\item En ajoutant la construction de l'arbre des plus courts chemins
\end{itemize}

\end{frame}

\begin{frame}{Propriétés de l'algorithme général}

\begin{itemize}
\item \alert{Propriété 1:} L'algorithme général maintient toujours l'invariant

\bigskip

\item \alert{Preuve:} Par induction sur le nombre d'itérations
\begin{itemize}
\item Cas de base: l'invariant est vérifié après l'initialisation
\item Cas inductif:
\begin{itemize}
\item Soit un appel à $relax(u,v,w)$
\item Avant l'appel, on suppose que l'invariant est vérifié et donc $u.d\geq \delta(s,u)$
\item Par l'inégalité triangulaire:
\begin{eqnarray*}
\delta(s,v) &\leq& \delta(s,u)+\delta(u,v)\\
&\leq& u.d+w(u,v)
\end{eqnarray*}
\item Suite à l'assignation $v.d=u.d+w(u,v)$, on a bien $$v.d\geq \delta(s,v)$$
\end{itemize}
\end{itemize}

\end{itemize}
\end{frame}

\bigskip

\begin{frame}{Propriétés de l'algorithme général}

\begin{itemize}
\item \alert{Propriété 2:} Une fois que $v.d=\delta(s,v)$, il n'est plus modifié
\item \alert{Preuve:} On a toujours $v.d\geq \delta(s,v)$ et un relâchement ne peut que diminuer $v.d$

\bigskip

\bigskip

\item Vu les propriétés 1 et 2, pour montrer qu'un algorithme du plus court chemin est
  correct, on devra montrer que le \alert{choix} des arêtes à relâcher
  mènera bien à $v.d=\delta(s,v)$ pour tout $v$.

\end{itemize}

\end{frame}



% Algorithme de Bellman-Ford

\subsection{Bellman-Ford}

\begin{frame}{Algorithme de Bellman-Ford}

\begin{center}
{\small
\fcolorbox{white}{Lightgray}{%
      \begin{codebox}
        \Procname{$\proc{Single-source-SP}(G,w,s)$}
        \li $\proc{Init-single-source}(G,s)$
        \li \While $\exists (u,v): v.d\geq u.d+w(u,v)$
        \li \Do Pick one edge (u,v)
        \li $\proc{Relax}(u,v,w)$\End
      \end{codebox}
}}
\end{center}

\bigskip

\begin{itemize}
\item Algorithme basé sur le relâchement
\item Soit les arêtes $e_1,\ldots,e_m$, dans un ordre quelconque.
\item Le relâchement se fait dans cet ordre:
$$\underbrace{\underbrace{e_1,e_2,\ldots,e_m};\underbrace{e_1,e_2,\ldots,e_m};\ldots;\underbrace{e_1,e_2,\ldots,e_m}}_{|V|-1 fois}$$
\end{itemize}

\end{frame}

\begin{frame}{Algorithme de Bellman-Ford}

\begin{center}
{\small
\fcolorbox{white}{Lightgray}{%
      \begin{codebox}
        \Procname{$\proc{Bellman-Ford}(G,w,s)$}
        \li $\proc{Init-single-source}(G,s)$
        \li \For i=1 \To $|G.V|-1$
        \li \Do \For each edge $(u,v)\in G.E$
        \li \Do $\proc{Relax}(u,v,w)$\End\End
      \end{codebox}
}}
\end{center}

\bigskip

Illustration sur un exemple:

\centerline{\includegraphics[width=5cm]{Figures/07-bellman-ford.pdf}}

\end{frame}


\begin{frame}{Analyse: complexité}

\begin{center}
{\small
\fcolorbox{white}{Lightgray}{%
      \begin{codebox}
        \Procname{$\proc{Bellman-Ford}(G,w,s)$}
        \li $\proc{Init-single-source}(G,s)$
        \li \For i=1 \To $|G.V|-1$
        \li \Do \For each edge $(u,v)\in G.E$
        \li \Do $\proc{Relax}(u,v,w)$\End\End
      \end{codebox}
}}
\end{center}

\begin{itemize}
\item La boucle principale relâche toutes les arêtes $|V|-1$ fois
\item Complexité: $\Theta(|V|\cdot |E|)$
\begin{itemize}
\item En supposant qu'on puisse parcourir les arêtes en $O(|E|)$
\end{itemize}
\end{itemize}

\end{frame}

\begin{frame}{Analyse: correction}

\begin{itemize}
\item On supposera qu'il n'y a pas de cycle de poids négatif
\item Propriété 3: Après $i$ itérations de l'algorithme, $v.d$ est le
  poids d'un plus court chemin de $s$ à $v$ utilisant au plus $i$ arêtes:
$$v.d\leq \min\{w(p): |p|\leq i\}$$
\item Preuve: Par induction:
\begin{itemize}
\item Cas de base: $v.d=+\infty$ si $i=0$ et $s.d=0$
\item Cas inductif:
\begin{itemize}
\item Avant l'itération $i$, on a $v.d\leq \min\{w(p):|p|\leq i-1\}$
\item Cette propriété reste vraie à tout moment de l'itération puisque
  $\proc{Relax}$ ne peut que diminuer les $v.d$
\item L'itération $i$ considère tous les chemins avec $i$ arêtes ou
  moins en relâchant toutes les arêtes entrantes en $v$
\end{itemize}
\end{itemize}
\end{itemize}

\centerline{\includegraphics[width=4cm]{Figures/07-pcc-bellmanford-proof.pdf}}

\note{Pas un égal mais un plus petit ou égal car l'ordre des n\oe uds n'étant pas fixé, on va mixer les updates de path}

\end{frame}

\begin{frame}{Analyse: correction}

\begin{itemize}
\item Si le graphe ne contient pas de cycles de poids négatif, alors,
  à la fin de l'exécution de l'algorithme de Bellman-Ford, on a
  $v.d=\delta(s,v)$ pour tout $v\in V$.

\bigskip

\item Preuve:
\begin{itemize}
\item Sans cycle négatif, tout plus court chemin est \alert{simple}, c'est-à-dire sans cycle
\item Tout chemin simple a au plus $|V|$ sommets et donc $|V|-1$ arêtes
\item Par la propriété 3, on a $v.d\leq \delta(s,v)$ après 
$|V|-1$ itérations
\item Par l'invariant, on a $v.d\geq \delta(s,v)$ $\Rightarrow v.d=\delta(s,v)$
\end{itemize}\qed
\end{itemize}
\end{frame}


\begin{frame}{Programmation dynamique}

\begin{itemize}
\item L'algorithme de Bellman-Ford implémente en fait une approche par
  programmation dynamique\footnote{Bellman est en fait l'inventeur de la programmation dynamique}
\item Soit $v.d[i]$, la longueur du plus court chemin de $s$ à $v$ utilisant au plus $i$ arêtes
\item On a
{\footnotesize
\[v.d[i]=\left\{\begin{array}{l}
0 \mbox{   si }v=s\mbox{ et }i=0\\
+\infty \mbox{   si }v\neq s\mbox{ et }i=0\\
\min\{v.d[i-1],\min_{(u,v)\in E}\{u.d[i-1]+w(u,v)\}\} \mbox{   sinon}\\
  \end{array}\right.\]}
\end{itemize}

\bigskip

{\it (Exercice: implémenter l'algorithme de Bellman-Ford à partir de
  la récurrence et le comparer avec la version précédente)}

\note{Différences:
\begin{itemize}
\item Récurrence suggère de parcourir les noeuds et puis pour chaque noeud les arêtes entrantes
\item Bellman-Ford parcourt les arêtes dans n'importe quel ordre (notre preuve montre que l'ordre n'a pas d'importance)
\item Pour la récurrence, on doit stocker un tableau de la taille du nombre de noeud. L'autre implémentation ne nécessite pas cette table !
\end{itemize}
}
\end{frame}

\begin{frame}{Détection des cycles négatifs}

\begin{columns}
\begin{column}{4.5cm}
\begin{center}
{\small
\fcolorbox{white}{Lightgray}{%
      \begin{codebox}
        \Procname{$\proc{Bellman-Ford}(G,w,s)$}
        \li $\proc{Init-single-source}(G,s)$
        \li \For i=1 \To $|G.V|-1$
        \li \Do \For each edge $(u,v)\in G.E$
        \li \Do $\proc{Relax}(u,v,w)$\End\End
        \li {\color{red}\For each edge $(u,v)\in G.E$}
        \li \Do {\color{red}\If $v.d>u.d+w(u,v)$}
        \li \Then {\color{red}\Return $\const{True}$}\End\End
        \li {\color{red}\Return $\const{False}$}
      \end{codebox}
}}
\end{center}
\end{column}
\begin{column}{5.5cm}
\begin{itemize}
\item Renvoie $\const{True}$ si un cycle négatif (accessible depuis
  $s$) existe, $\const{False}$ sinon
\item En cas de cycle négatif, il existe toujours (et donc aussi en sortie de boucle) au moins un $v.d$ qu'on
  peut améliorer par relâchement d'un arc $(u,v)$
\end{itemize}
\end{column}
\end{columns}

\end{frame}

\begin{frame}{Graphe dirigé acyclique (DAG)}

Version plus efficace dans le cas d'un graphe dirigé acyclique:
\begin{center}
{\small
\fcolorbox{white}{Lightgray}{%
      \begin{codebox}
        \Procname{$\proc{DAG-Shortest-Path}(G,w,s)$}
        \li $L=\proc{Top-Sort}(G)$
        \li $\proc{Init-single-source}(G,s)$
        \li \For each vertex $u$, taken in their order in $L$
        \li \Do \For each vertex $v\in G.Adj[u]$
        \li \Do $\proc{Relax}(u,v,w)$\End\End
      \end{codebox}
}}
\end{center}

\bigskip

Exemple:
\centerline{\includegraphics[width=7cm]{Figures/07-bellmanford-dag.pdf}}

\bigskip

Complexité: $\Theta(|V|+|E|)$

\end{frame}

\subsection{Dijkstra}

\begin{frame}{Poids unitaires: parcours en largeur d'abord}

\begin{itemize}
\item On peut obtenir une solution plus rapide en imposant certaines
  contraintes sur la nature des poids
\item Si les poids sont tous égaux à 1, le parcours en largeur permet
  de calculer les $v.d$ en $O(|V|+|E|)$
\item Rappel:
\end{itemize}

\begin{center}
{\small
\fcolorbox{white}{Lightgray}{%
      \begin{codebox}
        \Procname{$\proc{BFS}(G,s)$}
        \li \For each vertex $u \in G.V\setminus \{s\}$
        \li \Do $u.d=\infty$\End
        \li $s.d=0$
        \li $Q=$"create empty Queue"
        \li $\proc{Enqueue}(Q,s)$
        \li \While not $\proc{Queue-Empty}(Q)$
        \li \Do $u=\proc{Dequeue}(Q)$
        \li \For each $v\in G.Adj[u]$
        \li\Do \If $v.d=\infty$
        \li \Then $v.d=u.d+1$
        \li $\proc{Enqueue}(Q,v)$\End\End\End
      \end{codebox}}
}
\end{center}

\end{frame}

\begin{frame}{Poids unitaires: parcours en largeur d'abord}

\centerline{\includegraphics[width=8cm]{Figures/07-pcc-bfs.pdf}}

\end{frame}

\begin{frame}{Poids entiers, positifs et bornés}

\begin{itemize}
\item Si les poids sont des entiers compris entre $1\ldots W$:
\begin{itemize}
\item On définit un nouveau graphe en éclatant chaque arête de poids $w$ en $w$ arêtes de poids 1
\item On applique le parcours en largeur sur ce nouveau graphe
\end{itemize}
\item Complexité: $O(|V|+W |E|)$
\end{itemize}

\centerline{\includegraphics[width=7cm]{Figures/07-integerweights.pdf}}

\end{frame}

\begin{frame}{Poids positifs: approche gloutonne}

\begin{itemize}
\item Algorithme de Dijkstra: généralisation du parcours en largeur à
  des poids positifs réels

\bigskip

\item Idée:
\begin{itemize}
\item On maintient un ensemble $S$ de sommets dont le poids d'un plus
  court chemin à partir de $s$ est connu
\item A chaque étape, on ajoute à $S$ un sommet $v\in V\setminus S$ dont la distance à $s$ est minimale
\item On met à jour, par relâchement, les distances estimées des sommets adjacents à $v$
\end{itemize}
\end{itemize}

\end{frame}

\begin{frame}{Algorithme de Dijkstra}

\begin{center}
{\small
\fcolorbox{white}{Lightgray}{%
      \begin{codebox}
        \Procname{$\proc{Dijkstra}(G,w,s)$}
        \li $\proc{Init-Single-Source}(G,s)$
        \li $S=\emptyset$
        \li $Q=$"create an empty min priority queue from $G.V$"
        \li \While not $\proc{Empty}(Q)$
        \li \Do $u=\proc{Extract-Min}(Q)$
        \li $S=S\cup \{u\}$
        \li \For each $v\in G.Adj[u]$
        \li\Do $\proc{Relax}(u,v,w)$ {\color{red}\Comment ! $\proc{Relax}$ doit modifier la clé de $v$ dans $Q$}\End\End
      \end{codebox}}
}
\end{center}

\bigskip

Illustration sur un exemple:\vspace{-0.5cm}
\centerline{\includegraphics[width=5cm]{Figures/07-pcc-dijkstra-exemple.pdf}}

\end{frame}


\begin{frame}{Analyse: complexité}

\begin{itemize}
\item Si la file à priorité est implémentée par un tas (min), l'extraction et l'ajustement de la clé sont $O(\log |V|)$
\item Chaque sommet est extrait de la file à priorité une et une seule fois
\begin{itemize}
\item[$\Rightarrow$] $O(|V|\log |V|)$
\end{itemize}
\item Chaque arête est parcourue une et une seule fois et entraîne au plus un ajustement\footnote{Similaire à un $\proc{Heap-Decrease-Key}$} de clé
\begin{itemize}
\item[$\Rightarrow$] $O(|E|\log |V|)$
\end{itemize}
\item Total: $O(|V|\log |V| + |E|\log |V|)=O(|E| \log |V|)$
\begin{itemize}
\item $|E| \log |V|$ domine $|V|\log |V|$ si le graphe est connexe
\end{itemize}
\end{itemize}
\note{Au moins autant d'arête que de noeuds dans un graphe connexe}
\end{frame}

% Calculer tous les chemins: algorithme de Floyd...

\begin{frame}{Analyse: correction}
\begin{itemize}
\item \alert{Théorème}: l'algorithme de Dijkstra se termine avec $v.d=\delta(s,v)$ pour tout $v\in V$
\item \alert{Preuve}:
\begin{itemize}
\item Lorsqu'un n\oe ud $v$ est extrait de la file, son $v.d$ n'est plus modifié. Il suffit donc de montrer que $v.d=\delta(s,v)$ lorsque $v$ est extrait de $Q$
\item Par l'invariant (propriété 1), on a $v.d\geq \delta(s,v)$ à tout moment
\item Par l'absurde, supposons qu'il existe un n\oe ud $u$ tel que $u.d>\delta(s,u)$ lors de son extraction et soit $u$ le premier n\oe ud satisfaisant cette propriété.
\item Soit $y$ le premier n\oe ud d'un plus court chemin de $s$ à $u$ qui se trouve dans $Q$ avant l'extraction de $u$ et soit $x$ son prédécesseur
\end{itemize}
\end{itemize}

\centerline{\includegraphics[width=4.5cm]{Figures/07-dijkstra-proof.pdf}}


\note{S'il n'y a pas de y, alors, u est ce y et ce qui suit montre que $y.d=\delta(s,y)$}

\end{frame}

\begin{frame}{Analyse: correction}

\centerline{\includegraphics[width=4.5cm]{Figures/07-dijkstra-proof.pdf}}

\begin{itemize}
\item[]
\begin{itemize}
\item Puisque $u$ est le premier n\oe ud violant l'invariant, on a $x.d=\delta(s,x)$
\item Par la propriété de sous-structure optimale, le sous-chemin de $s$ à $y$ est un plus court chemin et $y.d$ a été assigné à $x.d+w(x,y)=\delta(s,x)+w(x,y)=\delta(s,y)$ lors de l'extraction de $x$
\item On a donc $y.d=\delta(s,y)\leq \delta(s,u)\leq u.d$
\item Or, $y.d\geq u.d$ puisqu'on s'apprête à extraire $u$ de la file
\item D'où $y.d=\delta(s,y)={\color{red}\delta(s,u)=u.d}$, ce qui contredit notre hypothèse\qed
\end{itemize}
\end{itemize}

\end{frame}

\subsection{Floyd-Warshall}

\begin{frame}{Plus court chemin pour toutes les paires de sommets}

Déterminer les plus courts chemins pour toutes les paires de sommets:
\begin{itemize}
\item Entrées: un graphe dirigé $G=(V,E)$, une fonction de poids $w$. Les sommets sont numérotés de 1 à $n$
\item Sortie: une matrice $D=(d_{ij})$ de taille $n\times n$ où $d_{ij}=\delta(i,j)$ pour tous sommets $i$ et $j$
\end{itemize}

\bigskip

\centerline{\includegraphics[width=10cm]{Figures/07-allpairs-exemple.pdf}}

\end{frame}

\begin{frame}{Plus court chemin pour toutes les paires de sommets}

\begin{itemize}
\item Dans le cas général, on peut appliquer Bellman-Ford sur chaque sommet
\begin{itemize}
\item $O(|V|^2 |E|)$, ou $O(V^4)$ si le graphe est dense ($E=\Theta(V^2)$)
\end{itemize}

\bigskip

\item S'il n'y a pas de poids négatifs, on peut appliquer Dijkstra sur chaque sommet
\begin{itemize}
\item $O(|V| |E| \log |V|)$, ou $O(V^3 \log |V|)$ si le graphe est dense
\end{itemize}

\bigskip

\item Il est possible d'obtenir $O(V^3)$ par programmation dynamique
\end{itemize}

\end{frame}

\begin{frame}{Une solution par programmation dynamique}
\begin{itemize}
%\item Soit $V=\{1,\ldots,n\}$ les sommets du graphe
\item Pour un chemin $p=\langle v_1,v_2,\ldots,v_l\rangle$, un sommet
\alert{intermédiaire} est un sommet de $p$ autre que $v_1$ ou $v_l$
\item Soit $d_{ij}^{(k)}$ le poids d'un plus court chemin entre $i$ et $j$ tel que tous les sommets intermédiaires sont dans le sous-ensemble de sommets $\{1,2,\ldots,k\}$
\item Soit un plus court chemin $p$ entre $i$ et $j$ avec tous les sommets dans $\{1,2,\ldots,k\}$:
\begin{itemize}
\item Si $k$ n'est pas un sommet intermédiaire de $p$, alors tous les sommets intermédiaires de $p$ sont dans $\{1,2,\ldots,k-1\}$
\item Si $k$ est un sommet intermédiaire, tous les sommets intermédiaires des sous-chemins entre $i$ et $k$ et entre $k$ et $j$ appartiennent à $\{1,2,\ldots,k-1\}$
\end{itemize}
\end{itemize}

\centerline{\includegraphics[width=6cm]{Figures/07-floydwarshall-recursion.pdf}}

\note{Pourquoi la deuxième: parce qu'un plus court-chemin ne contient pas de cycle. Au plus une fois $k$ dans le chemine entre $i$ et $j$}

\end{frame}



\begin{frame}{Algorithme de Floyd-Warshall}

\begin{itemize}
\item Formulation récursive:

\[d_{ij}^{(k)}=\left\{ \begin{array}{ll}
w(i,j)&\mbox{si }k=0,\\
\min(d_{ij}^{(k-1)},d_{ik}^{(k-1)}+d_{kj}^{(k-1)}) & \mbox{si } k\geq 1.
\end{array}\right.\]

\item Implémentation ascendante: $\Theta(|V|^3)$
\begin{itemize}
\item $W=(w_{ij})$ est la matrice d'adjacence pondérée
\item $w_{ij}=w(i,j)$ si $(i,j)\in E$, $+\infty$ sinon
\end{itemize}

\begin{center}
{\small
\fcolorbox{white}{Lightgray}{%
      \begin{codebox}
        \Procname{$\proc{Floyd-Warshall}(W,n)$}
         \li $D^{(0)}=W$
         \li \For $k=1$ \To $n$
         \li \Do let $D^{(k)}=(d_{ij}^{(k)})$ be a new $n\times n$ matrix
         \li \For $i=1$ \To $n$
         \li \Do \For $j=1$ \To $n$
         \li \Do $d_{ij}^{(k)}=\proc{min}(d_{ij}^{(k-1)},d_{ik}^{(k-1)}+d_{kj}^{(k-1)})$ \End\End\End
         \li \Return $D^{(n)}$
      \end{codebox}}
}
\end{center}


\end{itemize}

\note{\centerline{\includegraphics[width=11cm]{Figures/07-floydwarshall-exemple.pdf}}}

\end{frame}


\begin{frame}{Fermeture transitive d'un graphe}
\begin{itemize}
\item Soit un graphe dirigé $G=(V,E)$. La \alert{fermeture transitive} de $G$ est le graphe $G^*=(V,E^*)$ tel que:
$$E^*=\{(i,j): \exists\mbox{ un chemin de }i\mbox{ à }j\mbox{ dans }G\}$$

\item Exemple:

\bigskip

\centerline{\includegraphics[width=9cm]{Figures/07-closure.pdf}}

\bigskip

\item Solution directe:
\begin{itemize}
\item Assigner un poids $w_{ij}=1$ à toute arête $(i,j)\in E$
\item Appliquer l'algorithme de Floyd-Warshall
\item Si $d_{ij}<\infty$, il y a un chemin entre $i$ et $j$ dans $G$
\item Sinon, $d_{ij}=\infty$ et il n'y a pas de chemin
\end{itemize}
\end{itemize}

\end{frame}

\begin{frame}{Fermeture transitive d'un graphe}

Une solution plus simple en modifiant l'algorithme de Floyd-Warshall:
\begin{itemize}
\item Soit $t_{ij}^{(k)}=1$ s'il y a un chemin de $i$ à $j$ avec tous les n\oe uds intermédiaires dans $\{1,2,\ldots,k\}$, 0 sinon
\item On a:
{\small
\[t_{ij}^{(k)}=\left\{ \begin{array}{ll}
0&\mbox{si }k=0, i\neq j\mbox{ et }(i,j)\notin E\\
1& \mbox{si }k=0\mbox{ et }i=j\mbox{ ou }(i,j)\in E\\
t_{ij}^{(k-1)}\vee (t_{ik}^{(k-1)}\wedge t_{kj}^{(k-1)}) & \mbox{si } k\geq 1.
\end{array}\right.\]
}
\item Même implémentation que Floyd-Warshall
\begin{itemize}
\item on remplace $\min$ par $\vee$ et $+$ par $\wedge$
\end{itemize}
\end{itemize}

\end{frame}

\begin{frame}{Fermeture transitive d'un graphe: algorithme}

\begin{center}
{\small
\fcolorbox{white}{Lightgray}{%
      \begin{codebox}
        \Procname{$\proc{Transitive-closure}(G,n)$}
         \li Let $T^{(0)}=(t_{ij}^{(0)})$ be a new $n\times n$ matrix
         \li \For $i=1$ \To $n$
         \li \Do \For $j=1$ \To $n$
         \li \Do \If $i\isequal j$ or $(i,j)\in G.E$
         \li \Then $t_{ij}^{(0)}=1$
         \li \Else $t_{ij}^{(0)}=0$\End\End\End
         \li \For $k=1$ \To $n$
         \li \Do let $T^{k}=(t_{ij}^{(k)})$ be a new $n\times n$ matrix
         \li \For $i=1$ \To $n$
         \li \Do \For $j=1$ \To $n$
         \li \Do $t_{ij}^{(k)}= t_{ij}^{(k-1)}\vee (t_{ik}^{(k-1)}\wedge t_{kj}^{(k-1)})$ \End\End\End
         \li \Return $D^{(n)}$
      \end{codebox}}
}
\end{center}

\begin{itemize}
\item Complexité: $\Theta(|V|^3)$
\begin{itemize}
\item Idem Floyd-Warshall mais opérations plus simples
\end{itemize}
\end{itemize}

\end{frame}

\begin{frame}{Plus court chemin: synthèse}

\begin{itemize}
\item Origine unique, graphe dirigé acyclique:
\begin{itemize}
\item Relâchement en suivant un ordre topologique
\item $\Theta(|V|+|E|)$
\end{itemize}
\item Origine unique, graphe dirigé, poids positifs:
\begin{itemize}
\item Algorithme de Dijkstra
\item $\Theta(|E|\log |V|)$
\end{itemize}
\item Origine unique, graphe dirigé, poids quelconques:
\begin{itemize}
\item Algorithme de Bellman-Ford
\item $\Theta(|V|\cdot |E|)$
\end{itemize}
\item Toutes les paires, graphe dirigé ou non:
\begin{itemize}
\item Algorithme de Floyd-Warshall
\item $\Theta(|V|^3)$
\end{itemize}
\end{itemize}

\note{On peut transformer un graphe non dirigé en un graphe dirigé en introduisant deux arêtes par arête non dirigée}

\end{frame}

\section{Arbre couvrant}

\begin{frame}{Plan}

\tableofcontents[currentsection]

\end{frame}

\begin{frame}{Arbre couvrant}

\begin{itemize}
\item Définition: un \alert{arbre couvrant} {\it (spanning tree)}
  pour un graphe connexe $(V,E)$ non dirigé est un arbre (i.e. un
  graphe acyclique) $T$ tel que:
\begin{itemize}
\item l'ensemble des n\oe uds de $T$ est égal à $V$, et
\item l'ensemble des arcs de $T$ est un sous-ensemble de $E$
\end{itemize}
\item Construction: par un parcours en largeur ou en profondeur (graphe de liaison)
\item Exemple

\bigskip

\centerline{\includegraphics[width=3.5cm]{Figures/07-spanningtree.pdf}}

\end{itemize}

\end{frame}

\begin{frame}{Arbre couvrant de poids minimum}
\begin{itemize}
\item Définition: un \alert{arbre couvrant de poids minimum}, ACM, {\it (minimum
  spanning tree, MST)} pour un graphe pondéré connexe $(V,E)$ est un arbre $(V,E')$ tel que:
\begin{itemize}
\item $(V,E')$ est un arbre couvrant de $(V,E)$, et
\item la valeur de $\sum_{e\in E'} w(e)$ est minimale parmi tous les
  arbres couvrants de $(V,E)$, où $w(e)$ dénote le poids de l'arc $e$
\end{itemize}

\bigskip

\item Exemple:
\centerline{\includegraphics[width=6cm]{Figures/07-mstexemple.pdf}}
\end{itemize}

\end{frame}

\begin{frame}{Applications}

\centerline{\includegraphics[width=6cm]{Figures/07-applimst.pdf}}

\bigskip

\begin{itemize}
\item Conception de réseaux: connecter des entités en minimisant le coût de la connection
\begin{itemize}
\item Raccorder des maisons à un central téléphonique en minimisant les longueurs de cables
\item Elaborer un système routier pour connecter des maisons
\item ...
\end{itemize}
\item Dissémination de contenu/routage sur internet
\item Design de circuits imprimés
\item ...
\end{itemize}

\end{frame}

\begin{frame}{Approche générique}


\begin{itemize}
\item Idée:
\begin{itemize}
\item Un ACM est un sous-ensemble d'arêtes du graphe initial
\item On démarre avec un ensemble d'arêtes $A=\emptyset$ vide
\item On ajoute dans $A$ des arêtes en respectant l'invariant suivant:
\begin{itemize}
\item Il existe un ACM qui contient les arêtes de $A$
\end{itemize}
\item S'il existe un ACM contenant les arêtes de $A$, une arête $(u,v)$
  est \alert{sûre} pour $A$ ssi il existe un ACM contenant les arêtes de $A\cup \{(u,v)\}$
\end{itemize}

\bigskip

\item Algorithme générique:

\begin{center}
{\small
\fcolorbox{white}{Lightgray}{%
      \begin{codebox}
        \Procname{$\proc{Generic-MST}(G,w)$}
        \li $A=\emptyset$
        \li \While $A$ is not a spanning tree
        \li \Do find an edge $(u,v)$ that is safe for $A$
        \li $A=A\cup \{(u,v)\}$\End
        \li \Return $A$
      \end{codebox}}
}
\end{center}

\item Comment trouver des arêtes sûres ?
\end{itemize}

\note{Discuter de la correction: assez évident}

\end{frame}

\begin{frame}{Arêtes sûres}
\begin{itemize}
\item Soit $S\subset V$ et $A\subseteq E$:
\begin{itemize}
\item Une \alert{coupure} ({\it cut}) $(S,V\setminus S)$ est une partition des sommets en deux ensembles disjoints $S$ et $V\setminus S$
\item Une arête \alert{traverse} ({\it crosses}) une coupure $(S,V\setminus
  S)$ si une extrémité est dans $S$ et l'autre dans $V\setminus S$
\item Une coupure \alert{respecte} $A$ ssi il n'y a pas d'arête dans $A$ qui traverse la coupure
\end{itemize}

\bigskip

\item \alert{Théorème:} Soit $A$ un sous-ensemble d'un ACM,
  $(S,V\setminus S)$ une coupure qui respecte $A$ et $(u,v)$ une arête
  de \alert{poids minimal} qui traverse la coupure $(S,V\setminus
  S)$. $(u,v)$ est sûre pour $A$.\\
\medskip
{\it (Propriété des choix gloutons optimaux)}
\end{itemize}

\end{frame}

\begin{frame}{Arêtes sûres}

\centerline{~~~~~~~~~~~~~~~~~~~~~~~~~~~~~~\includegraphics[width=4cm]{Figures/07-mstproof.pdf}}
\vspace{-0.8cm}
\alert{Preuve:}
\begin{itemize}
\item Soit $T$ un ACM qui inclut $A$ % existe puisque $A$ est un sous-ensemble d'un MST
\item Supposons que $T$ ne contienne pas $(u,v)$ et montrons qu'il est possible de construire un arbre $T'$ qui inclut $A\cup\{(u,v)\}$
\item Puisque $T$ est un arbre, il n'y a qu'un unique chemin $p$ entre
  $u$ et $v$ et ce chemin traverse la coupure $(S,V\setminus S)$.
\item Soit $(x,y)$ une arête de $p$ qui traverse la coupure $(S,V\setminus S)$
\item Puisque $(u,v)$ est l'arête de poids minimum qui traverse la coupure, on a:
$$w(u,v)\leq w(x,y)$$
\end{itemize}

\end{frame}

\begin{frame}{Arêtes sûres}

\centerline{\includegraphics[width=4cm]{Figures/07-mstproof.pdf}}

\begin{itemize}
\item Puisque la coupure respecte $A$, l'arête $(x,y)$ n'est pas dans $A$
\item Soit $T'=(T\setminus\{(x,y)\})\cup \{(u,v)\}$:
\begin{itemize}
\item $T'$ est un spanning tree
\item $w(T')=w(T)-w(x,y)+w(u,v)\leq w(T)$ puisque $w(u,v)\leq w(x,y)$
\end{itemize}
\item $T'$ est donc bien un ACM tel que $A\cup\{(u,v)\}\subseteq T'$
\item $\Rightarrow (u,v)$ est sûre pour $A$
\end{itemize}\qed

\end{frame}

\begin{frame}{Algorithme de Kruskal}
\begin{itemize}
\item Approche gloutonne:
\begin{itemize}
\item On construit incrémentalement une forêt (c'est-à-dire, un
  ensemble d'arbres), en ajoutant progressivement des arêtes à un
  graphe initialement dépourvu d'arcs
\item On maintient en permanence une partition du graphe en cours de construction en ses composantes connexes
\item Pour relier des composantes connexes, on choisit à chaque fois l'arête de poids minimal qui les connecte
\item On s'arrête dès qu'il ne reste plus qu'une composante connexe
\end{itemize}

\bigskip

\item Correction:
\begin{itemize}
\item Puisqu'on connecte à chaque fois deux composantes connexes
  disjointes, le graphe reste acyclique et à la terminaison, on obtient un arbre couvrant
\item Puisqu'on sélectionne une arête de poids minimal à chaque étape,
  le théorème précédent garantit qu'on arrivera à un ACM
\end{itemize}
\end{itemize}

\end{frame}

\begin{frame}{Algorithme de Kruskal}
\begin{center}
{\small
\fcolorbox{white}{Lightgray}{%
      \begin{codebox}
        \Procname{$\proc{Kruskal}(G,w)$}
        \li $A=\emptyset$
        \li $P=\emptyset$
        \li \For each vertex $v \in G.V$
        \li \Do $P=P\cup\{\{v\}\}$\End
        \li \For each $(u,v)\in G.E$ taken into nondecreasing order of weight $w$
        \li \Do $P_1=$ subset in $P$ containing $u$
        \li $P_2=$ subset in $P$ containing $v$
        \li \If $P_1\neq P_2$
        \li \Then $A=A\cup\{(u,v)\}$
        \li Merge $P_1$ and $P_2$ in $P$\End\End
        \li \Return $A$
      \end{codebox}}
}
\end{center}

\begin{itemize}
\item Les choix des composantes connexes à combiner est arbitraire
\item On fixe cet ordre en parcourant les arêtes par ordre croissant
\end{itemize}

\end{frame}

\begin{frame}{Illustration}

%\begin{columns}
%\begin{column}{5cm}
\centerline{\includegraphics[width=6cm]{Figures/07-mstexemple.pdf}}
%\end{column}
%\begin{column}{5cm}
\begin{center}
\footnotesize
\begin{tabular}{lll}
& & $P$\\
\hline
 & & $\{\{a\},\{b\},\{c\},\{d\},\{e\},\{f\},\{g\},\{h\},\{i\}\}$\\
\alert{$(c,f)$} & fusion & $\{\{a\},\{b\},\{c,f\},\{d\},\{e\},\{g\},\{h\},\{i\}\}$\\
\alert{$(g,i)$} & fusion & $\{\{a\},\{b\},\{c,f\},\{d\},\{e\},\{g,i\},\{h\}\}$\\
\alert{$(e,f)$} & fusion & $\{\{a\},\{b\},\{c,f,e\},\{d\},\{g,i\},\{h\}\}$\\
$(c,e)$ & rejet & \\
\alert{$(d,h)$} & fusion & $\{\{a\},\{b\},\{c,f,e\},\{d,h\},\{g,i\}\}$\\
\alert{$(f,h)$} & fusion & $\{\{a\},\{b\},\{c,f,e,d,h\},\{g,i\}\}$\\
$(e,d)$ & rejet & \\
\alert{$(b,d)$} & fusion & $\{\{a\},\{b,c,f,e,d,h\},\{g,i\}\}$\\
\alert{$(d,g)$} & fusion & $\{\{a\},\{b,c,f,e,d,h,g,i\}\}$\\
$(b,c)$ & rejet & \\
$(g,h)$ & rejet & \\
\alert{$(a,b)$} & fusion & $\{\{a,b,c,f,e,d,h,g,i\}\}$\\
\end{tabular}
\end{center}
%\end{column}
%\end{columns}

\end{frame}

\begin{frame}{Implémentation}
\begin{itemize}
\item Problème: comment représenter les partitions de l'ensemble des sommets du graphe ?
\item Une solution possible:
\begin{itemize}
\item On numérote les parties $P_1$, $P_2$,\ldots, $P_k$ d'une partition $\{P_1,P_2,\ldots, P_k\}$ à l'aide des nombres de 1 à $k$
\item Pour chaque sommet $v$, on retient le numéro de la partie à
  laquelle il appartient (attribut $v.p$)
\item Pour chaque numéro de partie, on retient une liste des sommets contenus dans cette partie
\item Lors de la fusion de deux parties, on insère la plus petite
  partie à fusionner dans l'autre et on met à jour les numéros de partie
\end{itemize}
\item Complexité:
\begin{itemize}
\item Trouver la partie associée à un sommet: $O(1)$
\item Fusionner deux parties de tailles $n_1$ et $n_2$, avec $n_1<n_2$: $\Theta(n_1)$
\end{itemize}
\end{itemize}

\end{frame}

\begin{frame}{Complexité}
\begin{itemize}
\item Initialisation: $O(|V|)$
\item Tri des arêtes: $O(|E|\log |V|)$
\begin{itemize}
\item Tri: $O(|E|\log |E|)$
\item Or, $|E|<|V|^2 \Rightarrow \log |E|=O(2\log|V|)=O(\log |V|)$
\end{itemize}
\item Coût total des fusions: $O(|V|\log |V|)$
\begin{itemize}
\item Chaque fusion est linéaire par rapport à la taille de la plus
  petite partie
\item Chaque fusion produit un nouvelle partie au moins deux fois plus
  grande que la plus petite
\item Chaque sommet n'est ajouté à une partie qu'au plus $O(\log|V|)$ fois
\end{itemize}
\item Temps d'exécution total: $O(|E|\log |V|+|V|\log|V|) = O(|E|\log |V|)$
\begin{itemize}
\item Car $|E|$ domine $|V|$ dans le cas d'un graphe connexe
\end{itemize}
\end{itemize}

\end{frame}

\begin{frame}{Algorithme de Prim}

\begin{itemize}
\item Principe:
\begin{itemize}
\item $A$ est toujours un arbre (plus une forêt)
\item Initialisé comme une seule racine $r$ choisie de manière arbitraire
\item A chaque étape, choisir une arête de poids minimal traversant la
  coupure $(V_A,V\setminus V_A)$, où $V_A$ est l'ensemble des sommets
  connectés par des arêtes de $A$, et l'ajouter à $A$.
\end{itemize}

\bigskip

\begin{center}
{\small
\fcolorbox{white}{Lightgray}{%
      \begin{codebox}
        \Procname{$\proc{Prim}(G,w,r)$}
        \li $A=\emptyset$
        \li $V_A=\{r\}$
        \li \While $|V_A|<|G.V|$
        \li \Do $(u,v)=$"an edge of minimal weight from $V_A$ to $V\setminus V_A$"
        \li $V_A=V_A\cup \{u,v\}$
        \li $A=A\cup \{(u,v)\}$\End
        \li \Return $A$
      \end{codebox}}
}
\end{center}

\bigskip

\item Correction: toujours en application du théorème
\end{itemize}

\end{frame}

\begin{frame}{Implémentation}

\begin{itemize}
\item Comment extraire efficacement l'arête de poids minimal?
\item Utiliser une file à priorité:
\begin{itemize}
\item Chaque élément de la file est un sommet de $V\setminus V_A$ (pas
  encore couvert par l'arbre courant)
\item La clé de $v$ est le poids minimum de tout arête $(u,v)$ où $u\in V_A$
\item Cette clé est mise à jour à chaque ajout d'un sommet dans $V_A$
\end{itemize}

\bigskip

\item L'arbre est ``stocké'' par le biais d'un pointeur $v.\pi$
\begin{itemize}
\item $v.\pi$ est le parent de $v$ dans l'arbre couvrant minimal
\item $v.\pi=\const{NIL}$ si $v=r$ ou $v$ n'a pas de parents
\end{itemize}
\end{itemize}

\end{frame}

\begin{frame}{Implémentation}

\begin{center}
{\small
\fcolorbox{white}{Lightgray}{%
      \begin{codebox}
        \Procname{$\proc{Prim}(G,w,r)$}
        \li $Q=\emptyset$
        \li \For each $u\in G.V$
        \li \Do $u.key=\infty$
        \li $u.\pi=\const{NIL}$
        \li $\proc{Insert}(Q,u)$\End
        \li $\proc{Decrease-key}(Q,r,0)$ \Comment $r.key=0$
        \li \While $Q\neq \emptyset$
        \li \Do $u=\proc{Extract-min}(Q)$
        \li \For each $v\in G.Adj[u]$
        \li \Do \If $v\in Q$ and $w(u,v)< v.key$
        \li \Then $v.\pi=u$
        \li $\proc{Decrease-Key}(Q,v,w(u,v))$\End\End\End
      \end{codebox}}
}
\end{center}

\end{frame}

\begin{frame}{Illustration}

\centerline{\includegraphics[width=6cm]{Figures/07-mstexemple.pdf}}

A partir du n\oe ud $d$
\begin{center}\small
\begin{tabular}{cccccccccc}
N\oe ud  & $Q^0$ & $Q^1$ & $Q^2$ & $Q^3$ & $Q^4$ & $Q^5$ & $Q^6$ & $Q^7$ & $Q^8$\\
\hline
$a$ & $\infty$ & $\infty$& $\infty$& $\infty$ & 12 & 12 & 10 & 10 & 10\\
$b$ & $\infty$ & 8& 8 & 8 & 8 & 8\\%%
$c$ & $\infty$ & $\infty$& $\infty$ & 1\\%%
$d$ & 0 \\%%
$e$ & $\infty$ & 7& 7 & 3 & 3\\%%
$f$ & $\infty$ & $\infty$ & 6\\%%
$g$ & $\infty$ & 8 & 8 & 8 & 8 & 8\\%%
$h$ & $\infty$ & 5\\%%
$i$ & $\infty$ & $\infty$ & 11 & 11 & 11 & 11 & 2\\%%
\end{tabular}\\

\medskip

{\footnotesize (valeurs de clé des sommets dans $Q$ au fur et à mesure des itérations)}
\end{center}


\end{frame}

\begin{frame}{Complexité}

\begin{itemize}
\item En supposant que $Q$ est implémentée à l'aide d'un tas (min)
\item Initialisation et première boucle $\For$: $O(|V|\log|V|)$
\item Diminuer la clé de $r$: $O(\log |V|)$
\item Boucle $\While$: $O(|V|\log|V|+|E|\log |V|)$
\begin{itemize}
\item $|V|$ appels à $\proc{Extract-Min} \Rightarrow O(|V|\log |V|)$
\item $|E|$ appels à $\proc{Decrease-Key} \Rightarrow O(|E|\log|V|)$
\end{itemize}
\item Temps d'exécution total: $O(|E|\log |V|+|V|\log|V|) = O(|E|\log |V|)$
\begin{itemize}
\item Car $|E|$ domine $|V|$ dans le cas d'un graphe connexe
\end{itemize}
\end{itemize}

\end{frame}

\begin{frame}

\centerline{Fin}

\bigskip

Pour aller plus loin:\\
\centerline{\includegraphics[width=5cm]{Figures/clrscover.pdf}}

\end{frame}


\end{document}
