% A FAIRE VERIFIER ETC !!!

% - regarder à ce que les définitions d'arbres soient cohérente entre le chapitre 3 et le chapitre 5 (notamment, arbre ordonné, arbre complet, etc.)
% - ajouter les slides ``ce qu'on a vu'', ``ce qu'on n'a pas vu'' (ce qu'on aurait pu voir)

% - ajouter des exemples d'application: table hash -> recherche genome, arbre binaire: scheduling des avions (voir le cours du MIT de Kellis et autres)

% - ajouter le tas binomial dans le cours 4 ? Pour la concaténation de tas ?

%\documentclass[12pt,handout]{beamer}
\documentclass[notes]{beamer}
\usepackage[utf8]{inputenc}
\usepackage[french]{babel}

% PACKAGES
% --------

\usetheme{default}

\usepackage{pst-node}
\usepackage{clrscode3epg}
\usepackage{multido}
\usepackage{fancybox}
\usepackage{graphics}
\usepackage{verbatim}

%\usepackage{pgfpages}
%\pgfpagesuselayout{4 on 1}[a4paper,border shrink=5mm,landscape]
%\pgfpagesuselayout{2 on 1}[a4paper,border shrink=5mm,portrait]
%\pgfpagesuselayout{resize to}[a4paper,border shrink=5mm,landscape]

\usepackage{multirow}
\usepackage{comment}

% BEAMER OPTION
% -------------
% Dérivé du style de Julien Brusten et de Antonio Carzaniga

\setbeamertemplate{navigation symbols}{}

\usepackage[footheight=1em]{beamerthemeboxes}

\addfootboxtemplate{\color{white}}{~~~~~\color{lightgray}\insertpart
     \hfill\insertframenumber~~~}

\defbeamertemplate*{part page}{mypartpage}[1][]
{
\begin{centering}
{\usebeamerfont{part name}\usebeamercolor[fg]{part name}
Partie~\insertpartnumber}
\vskip1em\par
\begin{beamercolorbox}[rounded=true,shadow=true,sep=8pt,center,#1]{part title}
\usebeamerfont{part title}\insertpart\par
\end{beamercolorbox}
\end{centering}
}

\setbeamertemplate{part page}[mypartpage][]

\setbeamertemplate{itemize item}{\vbox{\hrule width1ex height1ex depth0pt}}

\newhsbcolor{Lightgray}{0 0 .95}

\newcommand{\semitransp}[2][35]{\color{fg!#1}#2}

\setbeamersize{text margin left=.5cm,text margin right=.5cm,sidebar width left=0pt,sidebar width right=0pt}

\makeatletter

\numberwithin{section}{part}

\AtBeginPart{%
\beamer@tocsectionnumber=0\relax\addtocontents{toc}{\protect\beamer@partintoc{\the\c@part}{\beamer@partnameshort}{\the\c@page}}
\frame{\partpage}
}%


%% number, shortname, page.
\providecommand\beamer@partintoc[3]{%
  \ifnum\c@tocdepth=-1\relax
    % requesting onlyparts.
    \makebox[6em]{Partie #1:} #2
    \par
  \fi
}

\define@key{beamertoc}{onlyparts}[]{%
  \c@tocdepth=-1\relax

}

\makeatother%


%\AtBeginPart{\frame{\partpage}}

% debut du document

\definecolor{darkgreen}{rgb}{0,0.5,0}
\definecolor{darkred}{rgb}{0.6,0,0}
\definecolor{gold}{rgb}{1,0.84,0}

\setbeamercolor{alerted text}{fg=beamer@blendedblue}

\newcommand{\pgcd}{\mbox{pgcd}}
\newcommand{\arete}[2]{#1\mbox{---}#2}

\title{Structures de données et algorithmes}
\author{Pierre Geurts}
\date{
2011-2012

\bigskip

\bigskip

{\small
\begin{tabular}{lcl}
E-mail & : & {\tt p.geurts@ulg.ac.be}\\
URL    & : & {\tt http://www.montefiore.ulg.ac.be/}\\
       &   & {\tt ~~~~\~{}geurts/sda.html}\\
Bureau & : & R 141 (Montefiore)\\
Téléphone & : & 04.366.48.15 | 04.366.99.64
\end{tabular}
}
}

\newcommand{\nats}{\mathbb{N}}
\newcommand{\ints}{\mathbb{Z}}
\newcommand{\reals}{\mathbb{R}}
\newcommand{\rationals}{\mathbb{Q}}

\setbeamertemplate{section in toc}[sections numbered]
\setbeamertemplate{subsection in toc}[sections numbered]

\begin{document}

\begin{frame}
\titlepage
\end{frame}


\begin{frame}{Contact}
\begin{itemize}
\item Chargé de cours: 
\begin{itemize}
\item Pierre Geurts, \url{p.geurts@ulg.ac.be}, I141 Montefiore, 04/3664815
\end{itemize}
\item Assistants:
\begin{itemize}
\item Gilles Louppe, \url{g.loupp@ulg.ac.be}, GIGA-R (B34,+1), CHU, 04/3662766
\item Julien Becker, \url{j.becker@ulg.ac.be}, GIGA-R (B34,+1), CHU, 04/3669805
\end{itemize}
\item Site web du cours:
\begin{itemize}
\item \url{http://www.montefiore.ulg.ac.be/~geurts/sda.html}
\end{itemize}
\end{itemize}
\end{frame}

\begin{frame}{Objectif du cours}

\begin{itemize}
\item Introduction à l'étude systématique des algorithmes et des
  structures de données
\item Vous fournir une boîte à outils contenant:
\begin{itemize}
\item Des structures de données permettant d'organiser et d'accéder efficacement
  aux données
\item Les algorithmes les plus populaires
\item Des méthodes génériques pour la modélisation, l'analyse et la résolution de problèmes algorithmiques
\end{itemize}
\item On insistera sur la généralité des algorithmes et structures de données et on les étudiera de manière formelle
\item Les projets vise à vous familiariser à la résolution de problème
\end{itemize}

\note{
Maintenant qu'on est débarassé de l'apprentissage, on va pouvoir se focaliser sur l'algorithmique. Vous avez appris à écrire des programmes, on va apprendre à écrire des algorithmes: procédure de résolution de problèmes.

Suite du cours d'introduction à l'informatique et précède le cours de techniques de programmation que vous aurez pour certains l'an prochain.
}
\end{frame}

\begin{frame}{Organisation du cours}
\begin{itemize}
\item Cours théoriques:
\begin{itemize}
\item Les vendredis de 14h à 16h, S94, Bâtiment B4 (Europe).
\item 10-12 cours
%\item Transparents disponibles sur la page web du cours avant chaque cours
\end{itemize}
\item Répétitions:
\begin{itemize}
\item Certains vendredis de 16h à 18h, S94, Bâtiment B4 (Europe)
\item 5 répétitions (+ debriefing des projets)
\end{itemize}
\item Projets:
\begin{itemize}
\item Trois projets tout au long de l'année, de difficultés croissantes
\item Les deux premiers individuels, le troisième en binôme
\item En C
\end{itemize}

\bigskip
%\item Assistance au cours théorique n'est pas requise mais fortement conseillée.
\item Evaluation sur base des projets (30\%) et d'un examen écrit (70\%).
\end{itemize}
\note{

Le cours théorique sera totalement indépendant du langage de programmation. On utilisera ce qu'on appelle le pseudo-code

\bigskip

Demander s'ils connaissant tous bien le C.

\bigskip

Si certains points vus au cours théorique ne sont pas clairs, on peut organiser une répétition mais il est préférable que vous posiez vos questions directement au cours. Avant chaque cours, séance de questions-réponses sur la séance précédente.

\bigkskip

En plus de ça, il est recommandé d'essayer d'implémenter et de tester les algorithmes et structures de données vues au cours.
}
\end{frame}


\begin{frame}{Notes de cours}
\begin{itemize}
\item Transparents disponibles sur la page web du cours avant chaque cours
\item Pas de livre de référence obligatoire mais les ouvrages suivants ont été utilisé pour préparer le cours:
{\small
\begin{itemize}
\item Introduction to algorithms, Cormen, Leiserson, Rivest, Stein, MIT press, Third edition, 2009.
\begin{itemize}
\item \url{http://mitpress.mit.edu/algorithms/}
\end{itemize}
\item Algorithms, Sedgewick and Wayne, Addison Wesley, Fourth edition, 2011.
\begin{itemize}
\item \url{http://algs4.cs.princeton.edu/home/}
\end{itemize}
\item Data structures and algorithms in Java, Goodrich and Tamassia, Fifth edition, 2010.
\begin{itemize}
\item \url{http://ww0.java4.datastructures.net/}
\end{itemize}
\item Algorithms, Dasgupta, Papadimitriou, and Vazirani, McGraw-Hill, 2006.
\begin{itemize}
\item \url{http://cseweb.ucsd.edu/users/dasgupta/book/index.html}
\item \url{http://www.cs.berkeley.edu/~vazirani/algorithms/all.pdf}
\end{itemize}
\end{itemize}
}
\end{itemize}
\note{Bouquin de référence: CLRS (pas obligatoire, mais vous avez peu de chance de regretter votre achat). Plus un ouvrage de réference qu'un ouvrage pédagogique. On ne verra qu'une toute petite partie. Une version française existe mais je ne l'ai pas et je me baserai sur la version anglaise
}
\end{frame}

\begin{frame}{Cours sur le web}
Ce cours s'inspire également de plusieurs cours disponibles sur le web:
\begin{itemize}
\item Antonio Carzaniga, Faculty of Informatics, University of Lugano
\begin{itemize}
\item \url{http://www.inf.usi.ch/carzaniga/edu/algo/index.html}
\end{itemize}
\item Marc Gaetano, Polytechnique, Nice-Sophia Antipolis
\begin{itemize}
\item \url{http://users.polytech.unice.fr/~gaetano/asd/}
\end{itemize}
\item Robert Sedgewick, Princeton University
\begin{itemize}
\item \url{http://www.cs.princeton.edu/courses/archive/spr10/cos226/lectures.html}
\end{itemize}
\item Charles Leiserson and Erik Demaine, MIT.
\begin{itemize}
\item \url{http://ocw.mit.edu/courses/electrical-engineering-and-computer-science/6-046j-introduction-to-algorithms-sma-5503-fall-2005/index.htm}
\end{itemize}
\item Le cours de 2009-2011 de Bernard Boigelot
\end{itemize}
\end{frame}


\begin{frame}{Contenu du cours}

\tableofcontents[onlyparts]

\end{frame}


\part{Introduction}

\section{Algorithms $+$ Data structures $=$ Programs (\footnotesize{Niklaus Wirth})}

\begin{frame}{Plan}

\tableofcontents

\note{la citation vient du titre d'un bouquin de 1976 dont le message principal était que les structures de données et les algorithmes étaient intimmement liés}
\end{frame}

\begin{frame}{Algorithmes}

\begin{itemize}
\item Un {\bf algorithme} est une suite {\em finie} et {\em non-ambiguë} d'opérations ou d'instructions permettant de résoudre un {\em problème}
\item Provient du nom du mathématicien persan {\em Al-Khawarizmi}
  ($\pm 820$), le père de l'algèbre
\item Un problème algorithmique est souvent formulé comme la
  transformation d'un ensemble de valeurs, {\bf d'entrée}, en un
  nouvel ensemble de valeurs, {\bf de sortie}.
\item Exemples d'algorithmes:
\begin{itemize}
\item Une recette de cuisine (ingrédients $\rightarrow$ plat préparé)
\item La recherche dans un dictionnaire (mot $\rightarrow$ définition)
\item La division entière (deux entiers $\rightarrow$ leur quotient)
\item Le tri d'une séquence (séquence $\rightarrow$ séquence ordonnée)

\end{itemize}
\end{itemize}

\note{dictionnaire: structure de donnée: le dictionnaire: mots écrits
  en français et surtout, classés par ordre alphabétique}
\end{frame}

\begin{frame}{Algorithmes}
\begin{itemize}
\item On étudiera essentiellement les algorithmes {\bf corrects}.
\begin{itemize}
\item Un algorithme est (totalement) {\em correct} lorsque pour chaque instance, il se termine en produisant la bonne sortie.
\item Il existe également des algorithmes {\em partiellement corrects} dont la terminaison n'est pas assurée mais qui fournissent la bonne sortie lorsqu'ils se terminent.
\item Il existe également des algorithmes {\em approximatifs} qui fournissent une sortie inexacte mais néanmoins proche de l'optimum.
\end{itemize}
\bigskip
\item Les algorithmes seront évalués en termes d'{\em utilisation de resources}, essentiellement par rapport aux {\bf temps de calcul} mais aussi à l'utilisation de la {\bf mémoire}.
\end{itemize}
\end{frame}

\begin{frame}{Algorithmes}

Un algorithme peut être spécifié de différentes manières:
\begin{itemize}
\item en langage naturel,
\item graphiquement,
\item en pseudo-code,
\item par un programme écrit dans un langage informatique
\item ...
\end{itemize}
La seule condition est que la description soit précise.

\end{frame}

% objectif: présenter le pseudo-code

\begin{frame}{Exemple: le tri}
\begin{itemize}
\item Le problème de tri:
\begin{itemize}
\item Entrée: une séquence de $n$ nombres $\langle a_1,a_2,\ldots,a_n\rangle$
\item Sortie: une permutation de la séquence de départ $\langle a_1',a_2',\ldots,a_n'\rangle$ telle que $a_1'\leq a_2'\leq\ldots\leq a_n'$
\end{itemize}
\bigskip
\item Exemple:
\begin{itemize}
\item Entrée: $\langle 31,41,59,26,41,58\rangle$
\item Sortie: $\langle 26,31,41,41,58,59\rangle$
\end{itemize}
\end{itemize}

\end{frame}

\begin{frame}{Tri par insertion}

\hfill\includegraphics[height=3cm]{Figures/CLRS/Fig-2-1.pdf}

Description en langage naturel:

\bigskip

On parcourt la séquence de gauche à droite

\bigskip

Pour chaque élément $a_j$:
\begin{itemize}
\item On l'{\alert{insère}} à sa position dans une nouvelle séquence ordonnée contenant les éléments le précédant dans la séquence.
\end{itemize}
On s'arrête dès que le dernier élément a été inséré à sa place dans la séquence.

\note{Un peu l'inverse du tri par sélection: on recherche le minimum de la liste et on le place au sommet}
\end{frame}

\begin{frame}
\frametitle{Tri par insertion}
\centerline{\includegraphics[width=8cm]{Figures/01-insertionsort.pdf}}
\end{frame}

\begin{frame}[fragile]
\frametitle{Tri par insertion}

Description en C (sur des tableaux d'entiers):

%\fcolorbox{white}{Lightgray}{
{\small
\begin{verbatim}
void InsertionSort (int *a, int length) {
  int key, i;
  for(int j = 1; j < length; j++) {
    key = a[j];
    /* Insert a[j] into the sorted sequence a[0...j-1] */
    i = j-1;
    while (i>=0 && a[i]>key) {
      a[i+1] = a[i];
      i = i-1;
    }
    a[i+1] = key;
  }
}
\end{verbatim}
}

\end{frame}

\begin{frame}{Insertion sort}
Description en {\bf pseudo-code} (sur des tableaux d'entiers):

\bigskip

\fcolorbox{white}{Lightgray}{
\begin{codebox}
\Procname{$\proc{Insertion-Sort}(A)$}
\li \For $j \gets 2$ \To $\attrib{A}{length}$
\li     \Do
$\id{key} \gets A[j]$
\li \Comment Insert $A[j]$ into the sorted sequence
    $A[1 \twodots j-1]$.
\li $i \gets j-1$
\li \While $i > 0$ and $A[i] > \id{key}$
\li   \Do
        $A[i+1] \gets A[i]$
\li        $i \gets i-1$
    \End
\li $A[i+1] \gets \id{key}$
\End
\end{codebox}
}

\end{frame}

\begin{frame}{Pseudo-code}

Objectifs:
\begin{itemize}
\item Décrire les algorithmes de manière à ce qu'ils soient compris
par des humains.

~\\

\item Rendre la description indépendante de l'implémentation

~\\

\item S'affranchir de détails tels que la gestion d'erreurs, les déclarations de type, etc.

\end{itemize}

~\\

Très proche du C (langage procédural plutôt qu'orienté objet)

~\\

Peut contenir certaines instructions en langage naturel si nécessaire

% donner quelques règles de pseudo-code
\end{frame}


\begin{frame}{Pseudo-code}
Quelques règles
\begin{itemize}
\item Structures de blocs indiquées par l'indentation
\item Boucles ($\For$, $\While$, $\Repeat$) et conditions ($\If$, $\textbf{else}$, $\textbf{elseif}$) comme en C.
\item Le compteur de boucle garde sa valeur à la sortie de la boucle
\item En sortie d'un $\For$, le compteur a la valeur de la borne max+1.
\begin{columns}
\begin{column}{3cm}
\fcolorbox{white}{Lightgray}{\begin{codebox}
\zi \For $i \gets 1$ \To $Max$
\zi     \Do $Code$
\End
\end{codebox}}
\end{column}
$\Leftrightarrow$
\begin{column}{5cm}
\fcolorbox{white}{Lightgray}{\begin{codebox}
\zi $i\gets 1$
\zi \While $i \leq Max$
\zi     \Do $Code$
\zi     $i\gets i+1$
\End
\end{codebox}}
\end{column}
\end{columns}
\item Commentaires indiqués par $\Comment$
\item Affectation ($\gets$) et test d'égalité ($\isequal$) comme en C.
\end{itemize}
\end{frame}

\begin{frame}{Pseudo-code}

\begin{itemize}
\item Les variables ($i$, $j$ et $key$ par exemple) sont locales à la fonction.
\item $A[i]$ désigne l'élément $i$ du tableau $A$. $A[i..j]$ désigne
  un intervalle de valeurs dans un tableau. $A.length$ est la taille du tableau.
\item L'indexation des tableaux commence à 1.
\item Les types de données composés sont organisés en {\it objets}, qui sont composés d'attributs. On accède à la valeur de l'attribut $attr$ pour un objet $x$ par $\attrib{x}{attr}$.
\item Un variable représentant un tableau ou un objet est considérée comme un pointeur vers ce tableau ou cet objet.
\item Paramètres passés par valeur comme en C (mais tableaux et objets sont passés par pointeur).
\item ...
\end{itemize}
\note{y=x, x.f=3 implique que y.f=3 également}
\end{frame}

\begin{frame}{Trois questions récurrentes face à un algorithme}

\begin{enumerate}
\item Mon algorithme est-il correct, se termine-t-il ? %\textcolor{darkred}{oui}

\bigskip

\item Quelle est sa vitesse d'exécution ? %\textcolor{darkred}{$o(n^2)$}

\bigskip

\item Y-a-t'il moyen de faire mieux ? %\textcolor{darkred}{oui}

\end{enumerate}

\bigskip

Exemple du \textcolor{darkred}{tri par insertion}
\begin{enumerate}
\item Oui $\rightarrow$ technique des invariants (partie 2)
\item $O(n^2)$ $\rightarrow$ analyse de complexité (partie 2)
\item Oui $\rightarrow$ il existe un algorithme $O(n\log n)$ (partie 1)
\end{enumerate}

% commenter sur la facilité de répondre à ces questions
\note{On va passer en revue rapidement ces trois points pour l'insertion-sort. On reviendra sur ça la semaine prochaine}
\end{frame}

\begin{frame}{Correction de $\proc{Insertion-Sort}$}

\begin{center}
\fcolorbox{white}{Lightgray}{
\begin{codebox}
\Procname{$\proc{Insertion-Sort}(A)$}
\li \For $j \gets 2$ \To $\attrib{A}{length}$
\li     \Do
$\id{key} \gets A[j]$
\li $i \gets j-1$
\li \While $i > 0$ and $A[i] > \id{key}$
\li   \Do
        $A[i+1] \gets A[i]$
\li        $i \gets i-1$
    \End
\li $A[i+1] \gets \id{key}$
\End
\end{codebox}
}
\end{center}

\bigskip
\begin{itemize}
\item \alert{Invariant:} (pour la boucle externe) le sous-tableau $A[1\twodots j-1]$ contient les éléments du tableau original $A[1\twodots j-1]$ ordonnés.
\item On doit montrer que
\begin{itemize}
\item l'invariant est vrai avant la première itération %({\em initialisation})
\item l'invariant est vrai avant chaque itération suivante %({\em maintenance})
\item En sortie de boucle, l'invariant implique que l'algorithme est correct %({\em terminaison})
\end{itemize}
\end{itemize}
\note{Si c'est vrai à la terminaison, alors on aura montrer que le tableau est trié à la fin}
\end{frame}

\begin{frame}{Correction de $\proc{Insertion-Sort}$}
\begin{itemize}
\item Avant la première itération:
\begin{itemize}
\item $j=2 \Rightarrow A[1]$ est trivialement ordonné.
\end{itemize}
\bigskip
\item Avant la $j$ème itération:
\begin{itemize}
\item Informellement, la boucle interne déplace $A[j-1]$, $A[j-2]$, $A[j-3]\ldots$ d'une position vers la droite jusqu'à la bonne position pour $key$ ($A[j]$).
\end{itemize}
\bigskip
\item En sortie de boucle:
\begin{itemize}
\item A la sortie de boucle, $j=A.length+1$. L'invariant implique que $A[1\twodots A.length]$ est ordonné.
\end{itemize}
\end{itemize}
\end{frame}

\begin{frame}{Complexité de $\proc{Insertion-Sort}$}
\begin{center}
\fcolorbox{white}{Lightgray}{
\begin{codebox}
\Procname{$\proc{Insertion-Sort}(A)$}
\li \For $j \gets 2$ \To $\attrib{A}{length}$
\li     \Do
$\id{key} \gets A[j]$
\li $i \gets j-1$
\li \While $i > 0$ and $A[i] > \id{key}$
\li   \Do
        $A[i+1] \gets A[i]$
\li        $i \gets i-1$
    \End
\li $A[i+1] \gets \id{key}$
\End
\end{codebox}
}
\end{center}

\bigskip

\begin{itemize}
\item Nombre de comparaisons $T(n)$ pour trier un tableau de taille $n$?
\item Dans le pire des cas:
\begin{itemize}
\item La boucle $\For$ est exécutée $n-1$ fois ($n=A.length$).
\item La boucle $\While$ est exécutée $j-1$ fois
\end{itemize}
\end{itemize}

\note{Pire cas est lorsqu'on doit mettre l'élément à insérer en
  première position: on le compare au $j-1$, au $j-2$, ..., au 1er. Ce
  qui donnej- comparaisons.}
\end{frame}

\begin{frame}{Complexité de $\proc{Insertion-Sort}$}

\begin{itemize}
\item Le nombre de comparaisons est borné par:
$$T(n)\leq \sum_{j=2}^n (j-1)$$
\item Puisque $\sum_{i=1}^n i=n(n+1)/2$, on a:
$$T(n)\leq\frac{n(n-1)}{2}$$
\item Finalement, $T(n)=O(n^2)$
\end{itemize}

\bigskip

(borne inférieure ?)

\note{Lequel est le plus rapide: tri par sélection ou tri par insertion ? Le second, car le premier est toujours $O(n^2)$}
\end{frame}

\begin{frame}{Structures de données}

\begin{itemize}
\item Méthode pour stocker et organiser les données pour en faciliter
  l'accès et la modification
\item Une structure de données regroupe:
\begin{itemize}
\item un certain nombre de données à gérer, et
\item un ensemble d'opérations pouvant être appliquées à ces données
\end{itemize}
\item Dans la plupart des cas, il existe
\begin{itemize}
\item plusieurs manières de représenter les données et
\item différents algorithmes de manipulation.
\end{itemize}
\item On distingue généralement l'\alert{interface} des structures de
  leur \alert{implémentation}.
\end{itemize}

\end{frame}

\begin{frame}{Types de données abstraits}

\begin{itemize}
\item Un type de données abstrait (TDA) représente l'interface d'une structure de données.
\item Un TDA spécifie précisément:
\begin{itemize}
\item la nature et les propriétés des données
\item les modalités d'utilisation des opérations pouvant être effectuées
\end{itemize}
\item En général, un TDA admet différentes implémentations (plusieurs représentations possibles des données, plusieurs algorithmes pour les opérations).
\end{itemize}

\end{frame}

\begin{frame}{Exemple: file à priorités}

\begin{itemize}
\item Données gérées: des objets avec comme attributs:
\begin{itemize}
\item une clé, munie d'un opérateur de comparaison selon un ordre total
\item une valeur quelconque
\end{itemize}
\medskip
\item Opérations:
\begin{itemize}
\item Création d'une file vide
\item $\proc{Insert}(S,x)$: insère l'élément $x$ dans la file $S$.
%\item $\proc{Maximum}(S)$: renvoie l'élément de $S$ avec la clé la plus grande.
\item $\proc{Extract-Max}(S)$: retire et renvoie l'élément de $S$ avec
  la clé la plus grande.
\end{itemize}
\medskip
\item Il existe de nombreuses façons d'implémenter ce TDA:
\begin{itemize}
\item Tableau non trié;
\item Liste triée;
\item Structure de tas;
\item $\ldots$
\end{itemize}
Chacune mène à des complexités différentes des opérations $\proc{Insert}$ et $\proc{Extract-Max}$
\end{itemize}

\end{frame}

\begin{frame}{Structures de données et algorithmes en pratique}
% voir slides de sedgewick, intro-pas-mal.ppt

\begin{itemize}
\item La résolution de problème algorithmiques requiert presque
  toujours la combinaison de structures de données et d'algorithmes
  sophistiqués pour la gestion et la recherche dans ces structures.
\item D'autant plus vrai qu'on a à traiter des volumes de données importants.
\item Quelques exemples de problèmes réels:
\begin{itemize}
\item Routage dans les réseaux informatiques
\item Moteurs de recherche
\item Alignement de séquences ADN en bio-informatique
\end{itemize}
\end{itemize}
\centerline{\includegraphics[width=12cm]{Figures/01-exemplesproblemes.pdf}}
\note{
Routeur: Il faut stocker les chemins de manière distribuée. Trouver le plus cours chemin. Traiter les paquets rapidement.

~\\

Moteurs de recherche: Pour que google puisse répondre aussi rapidement, il faut créer des structures de données très sophistiquées (appelé des indexs)

~\\

Alignement de séquence: séquence du génome humain fait 3 milliards de paires de base. Aligner deux séquences ou faire une recherche dans une séquence est un problème assez difficile.
}

\end{frame}

\section{Introduction à la récursivité}

\begin{frame}{Plan}

\tableofcontents
\end{frame}


\begin{frame}{Algorithmes récursifs}

Un algorithme est {\bf récursif} s'il s'invoque lui-même
directement ou indirectement.

\bigskip

Motivation: Simplicité d'expression de certains algorithmes

\bigskip

Exemple: Fonction factorielle:

\[n!=\left\{\begin{array}{ll}
1 &\mbox{si }n=0\\
n \cdot (n-1)! &\mbox{si }n>0$$
\end{array}
\right.
\]

\begin{center}
\fcolorbox{white}{Lightgray}{
\begin{codebox}
\Procname{$\proc{Factorial}(n)$}
\li \If $n\isequal 0$
\li \Then \Return 1 \End
\li \Return $n \cdot \proc{Factorial}(n-1)$
\end{codebox}
}
\end{center}

\end{frame}

\begin{frame}{Algorithmes récursifs}

\begin{center}
\fcolorbox{white}{Lightgray}{
\begin{codebox}
\Procname{$\proc{Factorial}(n)$}
\li \If $n\isequal 0$
\li \Then \Return 1 \End
\li \Return $n \cdot \proc{Factorial}(n-1)$
\end{codebox}
}
\end{center}

\bigskip

Règles pour développer une solution récursive:
\bigskip
\begin{itemize}
\item On doit définir un cas de base ($n==0$)
\item On doit diminuer la ``taille'' du problème à chaque étape ($n\rightarrow n-1$)
\item Quand les appels récursifs se partagent la même structure de données, les sous-problèmes ne doivent pas se superposer (pour éviter les effets de bord)
\end{itemize}
\end{frame}

\begin{frame}{Exemple de récursion multiple}

Calcul du $k$ième nombre de Fibonacci:
\begin{eqnarray*}
F_0&=&0\\
F_1&=&1\\
\forall i\geq 2: F_i& = &F_{i-2}+F_{i-1}
\end{eqnarray*}

Algorithme:
\begin{center}
\fcolorbox{white}{Lightgray}{
\begin{codebox}
\Procname{$\proc{Fibonacci}(n)$}
\li \If $n \leq 1$
\li \Then \Return n \End
\li \Return $\proc{Fibonacci}(n-2)+\proc{Fibonacci}(n-1)$
\end{codebox}
}
\end{center}
% Prendre le début du cours de Carnagazi

\end{frame}

\begin{frame}{Exemple de récursion multiple}

\begin{center}
\fcolorbox{white}{Lightgray}{
\begin{codebox}
\Procname{$\proc{Fibonacci}(n)$}
\li \If $n \leq 1$
\li \Then \Return n \End
\li \Return $\proc{Fibonacci}(n-2)+\proc{Fibonacci}(n-1)$
\end{codebox}
}
\end{center}
% Prendre le début du cours de Carnagazi

\bigskip

\begin{enumerate}
\item L'algorithme est-il correct?
\item Quelle est sa vitesse d'exécution?
\item Y-a-t'il moyen de faire mieux?
\end{enumerate}

\end{frame}

\begin{frame}{Exemple de récursion multiple}

\begin{center}
\fcolorbox{white}{Lightgray}{
\begin{codebox}
\Procname{$\proc{Fibonacci}(n)$}
\li \If $n \leq 1$
\li \Then \Return n \End
\li \Return $\proc{Fibonacci}(n-2)+\proc{Fibonacci}(n-1)$
\end{codebox}
}
\end{center}

\bigskip

\begin{enumerate}
\item L'algorithme est correct?
\begin{itemize}
\item Clairement, l'algorithme est correct.
\item En général, la correction d'un algorithme récursif se démontre par induction.
\end{itemize}
\item Quelle est sa vitesse d'exécution?
\item Y-a-t'il moyen de faire mieux?
\end{enumerate}

\note{faire la démonstration au tableau:
Pour n<=1, l'algo renvoie n, ce qui est correct
Si l'algorithme est correct pour tout n<n', on doit montrer qu'il est correct pour n+1. C'est évident...
}
\end{frame}

\begin{frame}{Vitesse d'exécution}
\begin{itemize}
\item Nombre d'opérations pour calculer $\proc{Fibonacci}(n)$ en fonction de $n$
\item Empiriquement:

\begin{center}
\includegraphics[width=7.5cm]{Figures/cpu-fibonacci.pdf}\\
~\hfill\scriptsize(Carzaniga)
\end{center}

\item Toutes les implémentations atteignent leur limite, plus ou moins loin
\end{itemize}
\end{frame}

\begin{frame}{Trace d'exécution}
\centerline{\includegraphics[width=8cm]{Figures/trace-fibonacci.pdf}}
~\hfill\scriptsize(Boigelot)
\end{frame}

\begin{frame}{Complexité}
\begin{center}
\fcolorbox{white}{Lightgray}{
\begin{codebox}
\Procname{$\proc{Fibonacci}(n)$}
\li \If $n \leq 1$
\li \Then \Return n \End
\li \Return $\proc{Fibonacci}(n-2)+\proc{Fibonacci}(n-1)$
\end{codebox}
}
\end{center}

\bigskip

\begin{itemize}
\item Soit $T(n)$ le nombre d'opérations de base pour calculer $\proc{Fibonacci}(n)$:
\begin{eqnarray*}
T(0) & = & 2, T(1)=2\\
T(n) & = & T(n-1)+T(n-2)+2\\
\end{eqnarray*}
\item On a donc $T(n)\geq F_n$ (= le $n$ème nombre de Fibonacci).
\end{itemize}

\end{frame}

\begin{frame}{Complexité}

\begin{itemize}
\item Comment croît $F_n$ avec $n$ ?
$$T(n)\geq F_n=F_{n-1}+F_{n-2}$$
Puisque $F_n\geq F_{n-1}\geq F_{n-2}\geq\ldots$
$$F_n\geq 2 F_{n-2}\geq 2(2 F_{n-4})\geq 2(2(2 F_{n-6})) \geq 2^{\frac{n}{2}}$$
Et donc
$$T(n)\geq (\sqrt{2})^n \approx (1.4)^n$$
\item $T(n)$ croît \alert{exponentiellement} avec $n$

\bigskip

\item Peut-on faire mieux ?
\end{itemize}
\note{Dire que le problème vient du fait qu'on calcule plusieurs fois les mêmes choses

pprev va stocker la valeur de $F_{n-2}$ et prev va stocker la valeur de $F_{n-1}$.
}
\end{frame}

\begin{frame}{Solution itérative}

\begin{center}
\fcolorbox{white}{Lightgray}{
\begin{codebox}
\Procname{$\proc{Fibonacci-Iter}(n)$}
\li \If $n \leq 1$
\li \Then \Return n \End
\li \Else
\li \Then $pprev\gets 0$
\li $prev\gets 1$
\li \For $i\gets 2 \To n$
\li \Do $f\gets prev+pprev$
\li $pprev\gets prev$
\li $prev\gets f$\End
\li \Return f \End
\end{codebox}
}
\end{center}

%Complexité: $O(n)$

\end{frame}

\begin{frame}{Vitesse d'exécution}
Complexité: $O(n)$

\bigskip

\centerline{\includegraphics[width=8cm]{Figures/cpu-fibonacci-iter.pdf}}
~\hfill\scriptsize(Carzaniga)
\end{frame}

\begin{frame}{Tri par fusion}

Idée d'un tri basé sur la récursion:
\begin{itemize}
\item on sépare le tableau en deux sous-tableaux de la même taille
\item on trie (récursivement) chacun des sous-tableaux
\item on fusionne les deux sous-tableaux triés en maintenant l'ordre
\end{itemize}
Le cas de base correspond à un tableau d'un seul élément.

\bigskip

\begin{center}
\fcolorbox{white}{Lightgray}{%
    \begin{codebox}
      \Procname{$\proc{merge-sort}(A,p,r)$}
      \li \If $\id{p}<\id{r}$
      \li \Then $q \gets \lfloor \frac{p+r}{2} \rfloor$
      \li       $\proc{merge-sort}(A,p,q)$
      \li       $\proc{merge-sort}(A,q+1,r)$
      \li       $\proc{merge}(A,p,q,r)$ \End
    \end{codebox}}
\end{center}

\centerline{Appel initial: $\proc{merge-sort}(A,1,A.length)$}

\bigskip

Exemple d'application du principe général de ``\alert{diviser pour régner}''

\note{

Idée: on suppose qu'on dispose d'un algorithme pour trier un tableau, comment l'utiliser pour trier un tableau: on l'applique sur le tableau. Un autre manière de faire est de diviser le tableau en deux parties, trier ces deux parties et ensuite les fusionner. On applique la même idée sur les sous-tableaux. On s'arrête quand le tri est trivial: 1 seul élément.

Les contraintes mentionnées plus haut sont bien respectées: cas de base, division, pas de recouvrement.

Illustrer l'idée sur l'exemple: $[5,2,4,7,1,3,2,6]$

\includegraphics[width=5cm]{Figures/01-note-mergesort.pdf}

ce principe reviendra souvent dans ce cours


}

\end{frame}

\begin{frame}{Tri par fusion: illustration}

\centerline{\includegraphics[width=6cm]{Figures/CLRS/mergesort-power2.pdf}}

\note{\centerline{\includegraphics[width=6cm]{Figures/CLRS/mergesort-notpower2.pdf}}}

\end{frame}

\begin{frame}{Fonction $\proc{merge}$}

$\proc{Merge}(A,p,q,r)$:
\begin{itemize}
\item {\bf Entrée:} tableau $A$ et indice $p$, $q$, $r$ tels que:
\begin{itemize}
\item $p\leq q<r$ (pas de tableaux vides)
\item Les sous-tableaux $A[p\twodots q]$ et $A[q+1\twodots r]$ sont ordonnés
\end{itemize}
\item {\bf Sortie:} Les deux sous-tableaux sont fusionnés en seul sous-tableau ordonné dans $A[p\twodots r]$
\end{itemize}

\bigskip

Idée:
\begin{itemize}
\item Utiliser un pointeur vers le début de chacune des listes;
\item Déterminer le plus petit des deux éléments pointés;
\item Déplacer cet élément vers le tableau fusionné;
\item Avancer le pointeur correspondant
\end{itemize}

\note{Faire l'analogie avec deux jeux de cartes triés qu'on veut rassembler. prendre un jeu de carte.

\includegraphics[width=6cm]{Figures/01-note-merge.pdf}

}

\end{frame}

\begin{frame}{Fusion: algorithme}

\begin{center}\small
   \fcolorbox{white}{Lightgray}{
    \begin{codebox}
      \Procname{$\proc{Merge}(A,p,q,r)$}
      \li $n_1 = q-p+1$; $n2 = r-q$
      \li Soit $L[1..n_1+1]$ and $R[1..n_2+1]$ deux nouveaux tableaux
      \li \For $i=1$ \To $n_1$
      \li \Do $L[i]=A[p+i-1]$ \End
      \li \For $j=1$ \To $n_2$
      \li \Do $R[j]=A[q+j]$ \End
      \li $L[n_1+1]=\infty$; $R[n_2+1]=\infty$ \RComment \textcolor{red}{Sentinels}
      \li i=1;j=1
      \li \For $k\gets p$ \To $r$
      \li \Do \If $L[i]\leq R[j]$
      \li \Then $A[k]\gets L[i]$
      \li       $i\gets i+1$
      \li \Else
      \li $A[k]=R[j]$
      \li       $j\gets j+1$
          \End
    \end{codebox}
}
\end{center}
\note{Sentinel: pas toujours une bonne idée. Difficile d'avoir une valeur infinie.}
\end{frame}

\begin{frame}{Fusion: illustration}

\centerline{\includegraphics[width=8cm]{Figures/mergeillustration.pdf}}

\bigskip

Complexité: $O(n)$ (où $n=r-p+1$)

\end{frame}

\begin{frame}{Vitesse d'exécution}

Complexité de $\proc{merge-sort}$: $O(n\log n)$ (voir partie 2)

\bigskip

\centerline{\includegraphics[width=8cm]{Figures/compare-sort.pdf}}

\note{Dans un prochain cours, on verra que le tri par fusion est
  optimal en terme de temps de calcul (N log N)}

\end{frame}

\begin{frame}{Remarques}

\begin{itemize}
\item La fonction $\proc{merge}$ nécessite d'allouer deux tableaux $L$
  et $R$ (dont la taille est $O(n)$). Exercice (difficile): écrire une
  fonction $\proc{merge}$ qui ne nécessite pas d'allocation
  supplémentaire.
\item On pourrait réécrire $\proc{merge-sort}$ de manière itérative (au prix de la simplicité)
\item Version récursive du tri par insertion:
\begin{center}
\fcolorbox{white}{Lightgray}{
\begin{codebox}
\Procname{$\proc{Insertion-Sort-rec}(A,n)$}
\li \If $n>1$
\li \Then $\proc{Insertion-Sort-rec}(A,n-1)$
\li       $\proc{Merge}(A,1,n-1,n)$ \End
\end{codebox}
}
\end{center}
\end{itemize}
\note{Attirer l'attention sur le fait qu'il faut aussi tenir compte de
  la taille mémoire requise par un algorithme.

\bigskip

Dire que l'insertion sort est une version particulière du merge sort où au lieu de diviser en deux, on divise en 1-n-1 et n. C'est donc nettement moins efficace. On obtiendrait l'insertion-sort à partir du mergesort en changeant le q=p+r/2 en q=r-1.
}
\end{frame}

\begin{frame}{Note sur l'implémentation de la récursivité}
\begin{itemize}
\item Trace d'exécution de la factorielle
\centerline{\includegraphics[width=2cm]{Figures/trace-factorielle.pdf}}
\item Chaque appel récursif nécessite de mémoriser le \alert{contexte d'invocation}
\item L'espace mémoire utilisé est donc $O(n)$ ($n$ appels récursifs)
\end{itemize}
\note{!! Il est important de garder ça en mémoire: les algos récursifs ont un coût en terme d'espace mémoire}
\end{frame}

\begin{frame}{Récursivité terminale}

\begin{itemize}
\item Définition: Une procédure est \alert{récursive terminale} (tail récursive) si elle n'effectue plus aucune opération après s'être invoquée récursivement.

\bigskip

\item Avantages:
\begin{itemize}
\item Le contexte d'invocation ne doit pas être mémorisé et donc l'espace mémoire nécessaire est réduit
\item Les procédures récursives terminales peuvent facilement être converties en procédures itératives
\end{itemize}
\end{itemize}

\end{frame}

\begin{frame}{Version récursive terminale de la factorielle}

\begin{center}
\fcolorbox{white}{Lightgray}{
\begin{codebox}
\Procname{$\proc{Factorial2}(n)$}
\li \Return $\proc{Factorial2-rec}(n,2,1)$
\end{codebox}}
\end{center}

\begin{center}
\fcolorbox{white}{Lightgray}{
\begin{codebox}
\Procname{$\proc{Factorial2-rec}(n,i,f)$}
\li \If $i>n$
\li \Then \Return $f$\End
\li \Return $\proc{Factorial2-rec}(n,i+1,f\cdot i)$
\end{codebox}}
\end{center}

\bigskip

Espace mémoire utilisé: $O(1)$ (si la récursion terminale est implémentée efficacement)

\note{
le i est le numéro de l'itération et le f est l'accumulateur.
}

\end{frame}

\begin{frame}{Ce qu'on a vu}

\begin{itemize}
\item Définitions générales: algorithmes, structures de données, structures de données abstraites...
\item Analyse d'un algorithme itératif ($\proc{Insertion-Sort}$)
\item Notions de récursivité
\item Analyse d'un algorithme récursif ($\proc{Fibonacci}$)
\item Tri par fusion ($\proc{MergeSort}$)
\end{itemize}

\end{frame}


\part{Outils d'analyse}

\begin{frame}{Plan}

\tableofcontents

\note{Ce qu'on va voir aujourd'hui va être essentiellement des rappels
  de ce que vous avez déjà vu. L'idée est d'aller un peu plus en
  profondeur et de voir aussi comment les idées qu'on a développées
  pour les algorithmes itératifs peut aussi s'appliquer pour les algorithmes récursifs.

On va aussi introduire certaines nouvelles notions.
}

\end{frame}

\section{Correction d'un algorithme}

\subsection{Algorithmes itératifs}

\begin{frame}{Analyse d'algorithmes}

Questions à se poser lors de la définition d'un algorithme:
\begin{itemize}
\item Mon algorithme est-il correct ?
\item Mon algorithme est-il efficace ? %en termes d'utilisation des
%  resources, temps CPU et/ou espace mémoire ?
\end{itemize}

\bigskip

Autre question importante seulement marginalement abordée dans ce cours:
\begin{itemize}
\item Modularité, fonctionnalité, robustesse, facilité d'utilisation, temps
  de programmation, simplicité, extensibilité, fiabilité,
  existence d'une solution algorithmique...
\end{itemize}

\end{frame}

\begin{frame}{Correction d'un algorithme}%, complétude, terminaison}

\begin{itemize}
\item La correction d'un algorithme s'étudie par rapport à un problème donné
\item Un problème est une collection d'instances de ce problème.
\begin{itemize}
\item Exemple de problème: trier un tableau
\item Exemple d'instance de ce problème: trier le tableau $[8,4,15,3]$
\end{itemize}
\item Un algorithme est correct pour une instance d'un problème s'il
  produit une solution correcte pour cette intance
\item Un algorithme est correct pour un problème s'il est correct pour
  toutes ses instances
\item On s'intéressera ici à la correction d'un algorithme pour un
  problème (et pas pour seulement certaines de ses instances)
\end{itemize}

\note{}

\end{frame}

\begin{frame}{Comment vérifier la correction?}
\begin{itemize}
\item Première solution: en testant concrètement l'algorithme:
\begin{itemize}
\item Suppose d'implémenter l'algorithme dans un langage (programme)
  et de le faire tourner
\item Supppose qu'on peut déterminer les instances du problème à vérifier
\item Il est très difficile de prouver empiriquement qu'on n'a pas de bug %On peut prouver qu'il y a un bug, pas qu'il n'y en a pas.
\end{itemize}
\item Deuxième solution: en dérivant une preuve mathématique formelle:
\begin{itemize}
\item Pas besoin d'implémenter et de tester toutes les instances du problème
\item Sujet à des ``bugs'' également
\end{itemize}
\item En pratique, on combinera les deux

\bigskip

\item Preuves de correction:
\begin{itemize}
\item Pré-condition, post-condition
\item Algorithmes itératifs: assertions, invariants de boucle et induction
\item Algorithmes récursifs: preuves par induction directement
\end{itemize}
\end{itemize}
\end{frame}

\begin{frame}{Correction: cas itératif}

Pour prouver qu'un algorithme itératif est correct:
\begin{itemize}
\item On analyse chaque boucle de l'algorithme séparément, en
  démarrant avec la boucle la plus interne s'il y a plusieurs
  boucles imbriquées.
\item Pour chaque boucle, on met en évidence un invariant de boucle
\begin{itemize}
\item Ensemble de propriétés qui relient les variables du programme
\item Ces propriétés doivent être vraies avant, pendant et après la boucle
\end{itemize}
\item On prouve que l'invariant est vérifié.
\item On utilise l'invariant pour prouver que l'algorithme se termine.
\item On utilise l'invariant pour prouver que l'algorithme calcule le résultat correct.
\end{itemize}

\end{frame}

\begin{frame}{Assertion}

\begin{itemize}
\item Relation entre les variables qui est vraie à un moment donné dans l'exécution
\item Assertions particulières:
\begin{itemize}
\item Pre-condition $P$: conditions que doivent remplir les entrées de l'algorithme
\item Post-condition $Q$: conditions qui expriment que le résultat de l'algorithme est correcte
%\item Invariant: condition que doivent remplir
\end{itemize}
\item $P$ et $Q$ définissent les instances et solutions valides du problème
\item Un algorithme est correct si $P$ \{code\} $Q$ est vrai.
\item Exemple: 
%\item Pour vérifier la correction d'une boucle, on introduit la notion d'invariant.%On suppose $P$ vérifié, on montre que $I$ est un invariant valide et on en déduit que $R$ est vrai également.
\end{itemize}

\end{frame}

\begin{frame}{Invariant}
\begin{itemize}
\item \alert{Invariant:} Une assertion qui définit ce qui est vrai
  avant et après chaque itération de la boucle
\item \alert{Initialisation:} Prouver que l'invariant est vrai avant la
  première itération (sous l'hypothèse que la pré-condition $P$ est
  vérifiée)
\item \alert{Maintenance:} Prouver que si l'invariant est vrai avant la
  $i$-ième itération, il l'est également après celle-ci, et par
  conséquent aussi avant la $i+1$-ième itération
\item \alert{Terminaison:} Prouver que si l'invariant est vrai après la
  dernière itération, la post-condition $Q$ est vérifiée
\end{itemize}

\centerline{\includegraphics[width=8cm]{Figures/02-invariant.pdf}}

\end{frame}

\begin{frame}{Mise en \oe uvre}
\begin{itemize}
\item Trouver un invariant de boucle une fois l'algorithme mis au point peut être assez complexe
\item Idéalement, l'invariant devrait être la propriété clé qui définit l'algorithme
\item Dans la suite du cours, on ne fournira un invariant que dans quelques cas
\end{itemize}
\end{frame}

\begin{frame}{Exemples: exponentielle}
\end{frame}

\begin{frame}{Exemples: fibonacci itératif}
\begin{itemize}
\item Invariant de boucle: 
\end{itemize}
\end{frame}

\begin{frame}{Exemples: insertion sort}
\end{frame}

\begin{frame}{Preuve de terminaison d'un algorithme itératif}

Pour chaque boucle: on définit une fonction de terminaison $t$ qui est telle que $t$ décroit strictement à chaque itération de la boucle et que sa valeur est bornée vers le bas par le gardien de la boucle.

\end{frame}

\begin{frame}{Correction: cas récursif}
\begin{itemize}
\item Preuve par induction: donner le schéma général
\end{itemize}
\end{frame}

\begin{frame}{Exemple: Fibonacci}
\end{frame}

\begin{frame}{Exemple: merge sort}
\end{frame}

\begin{frame}{Preuve de terminaison}
\begin{itemize}
\item itératif: Définir une fonction $t$ montrer que $t$ décroit lors des itérations et 
\item récursif: montrer que le problème est réduit à chaque appel récursif
\end{itemize}
\end{frame}

% Pre-condition=assertion qui est vrai avant le programme
% Post-condition=assertion qui est vrai après l'exécution du programme
% Boucle: invariante:
% Tel que invariant & non G (gardien de boucle) => Post-condition

% Terminaison: on définit une fonction naturel, on montre qu'elle
% décroit à chaque itération et que le gardien


\subsection{Algorithmes récursifs}

\begin{frame}{Preuve par induction}

\end{frame}

\begin{frame}{Conclusion sur la correction}
Dans la suite, on ne présentera des invariants ou des preuves par induction que lorsque ce sera nécessaire (cas non triviaux)
\end{frame}

\section{Complexité algorithmique}

% A VOIR
%  -  notations grand omega et grand theta
%     (borne min et borne tight)
%  -  somme et produit de fonction 
%  -  complexité en temps versus complexité en espace
%  -  complexité d'un algorithme et complexité d'un problème
%  -  algorithmes récursifs

% Plan
% - motiver les temps de calcul versus le reste. On regardera aussi la memoire
% - problem=instances -> asymptotique + worst-case ou average case
% - comment on mesure:
%     - experience: pourri, machine dependen, programmeur, language, demande de coder, etc.
%     - abstraction: modèle de calcul -> notation grand O
% - grand-O: borne supérieure: on aimerait avoir une borne inférieure aussi

% - Autre notation: 

% - Complexité d'un algorithme versus complexité d'un problème

% - comment mesurer la complexité:
%      - algo itératif: somme, produit, boucle, etc.
%      - algo récursif: par une équation récurrente. Pas aussi évident.
%        Fibonacci: dire qu'on a montré que T(n)=OMEGA(1.4^n). On pourrait montrer que T(n)=...
%        Merge-sort: dire que la complexité
%        En général: on peut montrer que T(n)=T(n/b)+O(n) -> T(n)=n log(n)

% - Remarques: attention à la constante qui en pratique peut jouer un rôle important. Dépend de l'utilisation
%   Raffinement du merge-sort qui fait une différence: utiliser le tri par insertion lorsque la taille diminue

% - analyse amortie ?

% Prendre un exemple: recherche du maximum dans un vecteur non trié
% (ou trié). Donner les deux algorithmes.

\begin{frame}{Plan}

\tableofcontents

\end{frame}

\subsection{Introduction}

\begin{frame}{Performance d'un algorithme}

\begin{itemize}
\item Plusieurs métriques possibles:
\begin{itemize}
\item Longueur du programme (nombre de lignes)
\item Simplicité du code
\item \alert{Espace mémoire consommé}
\item \alert{Temps de calcul}
\item ...
\end{itemize}

\bigskip

\item Les temps de calcul sont la plupart du temps utilisés
\begin{itemize}
\item Ils peuvent être quantifiés et sont faciles à comparer
\item Souvent ce qui compte réellement
\end{itemize}
\item Nous étudierons aussi l'espace mémoire consommé par nos algorithmes
\end{itemize}

\end{frame}

\begin{frame}{Comment mesurer les temps d'exécution ?}

Expérimentalement:
\begin{itemize}
\item On écrit un programme qui implémente l'algorithme et on l'exécute sur des données
\item Problèmes:
\begin{itemize}
\item Les temps de calcul vont dépendre de l'implémentation: CPU, OS, langage, compilateur, charge de la machine, OS, etc.
\item Sur quels données tester l'algorithme ?
\end{itemize}
\end{itemize}

\begin{center}
\includegraphics[width=6cm]{Figures/cpu-fibonacci.pdf}\\
~\hfill\scriptsize(Carzaniga)
\end{center}

\end{frame}

\begin{frame}{Comment mesurer les temps d'exécution ?}

En se basant sur un modèle de calcul abstrait:
\begin{itemize}
\item Random-access machine (RAM):
\begin{itemize}
\item Types de données de base:
\begin{itemize}
\item Entiers et nombres flottants
\item Chaque mot est de taille limitée (par ex, 64 bits)
\end{itemize}
\item Opérations de base de la machine:
\begin{itemize}
\item addition, soustraction, multiplication...
\item affectation, accès à un élément d'un tableau...
\item branchement conditionnel, saut
\item appel de sous-routines, renvoi d'une valeur
\end{itemize}
\item Les opérations sont exécutées les unes après les autres (pas de parallélisme)
\item \alert{Les opérations de base prennent un temps constant}
\end{itemize}
\end{itemize}
\end{frame}

\begin{frame}{Comment mesurer les temps d'exécution ?}
\begin{itemize}
\item Calculer les temps de calcul = sommer le temps d'exécution associé à chaque instruction du pseudo-code
\item Modèle RAM:
\begin{itemize}
\item Opérations de base: temps constant
\item Appel de sous-routines: temps de l'appel (constant) + temps de l'exécution de la sous-routine (calculé récursivement)
\end{itemize}

\bigskip

\item Le temps dépend de l'entrée (l'instance particulière du problème)
\item On étudie généralement les temps de calcul en fonction de la \alert{``taille''} de l'entrée
\begin{itemize}
\item Généralement, le nombre de valeurs pour la décrire
\item Mais ça peut être autre chose (Ex: $n$ pour $\proc{Fibonacci}$)
\end{itemize}
\end{itemize}
\end{frame}

\begin{frame}{Analyse du tri par insertion}
\centerline{\includegraphics[width=10cm]{Figures/02-analysisinsertionsort.pdf}}
\begin{itemize}
\item $t_j=$ nombre de fois que la boucle $\While$ est exécutée.
\item Temps exécution $T(n)$ (pour un tableau de taille $n$) donné par:
{\footnotesize
\begin{eqnarray*}
T(n)&=&c_1 n +c_2(n-1)+c_4(n-1)+c_5\sum_{j=2}^n t_j + c_6 \sum_{j=2}^n (t_j-1)\\
&&+c_7\sum_{j=2}^n (t_j-1)+c_8(n-1)
\end{eqnarray*}}
\end{itemize}
\end{frame}

\begin{frame}{Différents types de complexité}
\begin{itemize}
\item Même pour une taille fixée, la complexité peut dépendre de l'instance particulière
%\begin{itemize}
%\item Les valeurs de $t_j$ dépendent du tableau
%\end{itemize}
\item Soit $D_n$ l'ensemble des instances de taille $n$ d'un problème et $T(i_n)$ le temps de calcul pour une instance $i_n\in D_n$.
\item Sur quelles instances les performances d'un algorithme devraient être jugées:
\begin{itemize}
\item Cas le plus favorable (best case): $T(n)=\min\{T(i_n)| i_n\in D_n\}$
\item Cas le plus défavorable (worst case): $T(n)=\max\{T(i_n)| i_n\in D_n\}$
\item Cas moyen (average case): $T(n)=\sum_{i_n\in D_n} Pr(i_n) T(i_n)$ où $Pr(i_n)$ est la probabilité de rencontrer $i_n$
\end{itemize}
\item On se focalise généralement sur le cas \alert{le plus défavorable}
\begin{itemize}
\item Donne une borne supérieure sur le temps d'exécution.
\item Le meilleur cas n'est pas représentatif et le cas moyen est difficile à calculer.
\end{itemize}
\end{itemize}
\note{On va voir que le cas le plus favorable est tout de même intéressant dans certains contextes}
\end{frame}

\begin{frame}{Analyse du tri par insertion}
Meilleur cas:
\begin{itemize}
\item le tableau est trié $\Rightarrow t_j=1$.
\item Le temps de calcul devient:
\begin{eqnarray*}
T(n)&=&c_1 n+ c_2 (n-1)+c_4 (n-1)+c_5 (n-1)+c_8(n-1)\\
&=&(c_1+c_2+c_4+c_5+c_8)n- (c_2+c_4+c_5+c_8)
\end{eqnarray*}
\item $T(n)=an+b$ $\Rightarrow$ $T(n)$ est une fonction \alert{linéaire} de $n$\\

\end{itemize}

\end{frame}

\begin{frame}{Analyse du tri par insertion}
Pire cas:
\begin{itemize}
\item le tableau est trié par ordre décroissant $\Rightarrow t_j=j$.
\item Le temps de calcul devient:
{\footnotesize
\begin{eqnarray*}
T(n)&=&c_1 n+ c_2 (n-1)+c_4 (n-1)+c_5 \left( \frac{n(n+1)}{2}-1\right)\\
& & +c_6\left(\frac{n(n-1)}{2}\right)+c_7 \left(\frac{n(n-1)}{2}\right)+c_8 (n-1)\\
&=&(\frac{c_5}{2}+\frac{c_6}{2}+\frac{c_7}{2}) n^2 + (c_1+c_2+c_4+\frac{c_5}{2}-\frac{c_6}{2}-\frac{c_7}{2}+c_8)n\\
& & - (c_2+c_4+c_5+c_8)
\end{eqnarray*}
}
\item $T(n)=an^2+bn+c$ $\Rightarrow$ $T(n)$ est une fonction \alert{quadratique} de $n$
\end{itemize}

\end{frame}


\begin{frame}{Analyse asymptotique}
% Slide 30 intro-pas-mal (voir aussi CLRS)
\begin{itemize}
\item On s'intéresse à la vitesse de croissance (``order of growth'') de $T(n)$ lorsque $n$ croît.
\begin{itemize}
\item Tous les algorithmes sont rapides pour des petites valeurs de $n$
\end{itemize}
\item On simplifie généralement $T(n)$:
\begin{itemize}
\item en ne gardant que le terme dominant
\begin{itemize}
\item Exemple: $T(n)=10 n^3+n^2+40n+800$
\item T(1000)=100001040800, $10\cdot 1000^3=100000000000$
\end{itemize}
\item en ignorant le coefficient du terme dominant
\begin{itemize}
\item Asymptotiquement, ça n'affecte pas l'ordre relatif
\end{itemize}
\end{itemize}
\centerline{\includegraphics[width=5cm]{Figures/02-dropconstant.pdf}}
\item Exemple: Tri par insertion: $T(n)=an^2+bn+c \rightarrow n^2$.
%\item On dira que $T(n)$ croît en $n^2$ et on le notera $T(n)\in\Theta(n^2)$ ou $T(n)=\Theta(n^2)$.
\end{itemize}
\end{frame}

\begin{frame}{Pourquoi est-ce important?}

\begin{itemize}
\item Supposons qu'on puisse traiter une opération de base en $1\mu s$.
\item Temps d'exécution pour différentes valeurs de $n$

\bigskip

\begin{center}
\footnotesize
\begin{tabular}{ccccc}
\hline
f(n) & $n=10$ & $n=100$ & $n=1000$ & $n=10000$\\
\hline
$n$ & $10\mu s$ & $0.1ms$ & $1ms$ & $10ms$\\
$400n$ & $4ms$ & $40ms$ & $0.4s$ & 4$s$\\
$2n^2$ &$200\mu s$ & $20ms$ & $2s$ & 3.3$m$\\
$n^4$ &$10ms$& $100s$ & $\sim 11.5$ jours & 317 années\\
$2^n$ & $1ms$ & $4\times 10^{16}$ années & $3.4\times 10^{287}$ années & $\ldots$\\
\hline
\end{tabular}
\end{center}

\end{itemize}

~\hfill{\scriptsize(Dupont)}

\end{frame}

\begin{frame}{Pourquoi est-ce important?}

\begin{itemize}
\item Taille maximale du problème qu'on peut traiter en un temps donné:
\begin{center}
\footnotesize
\begin{tabular}{cccc}
\hline
f(n) & en 1 seconde & en 1 minute & en 1 heure\\
\hline
$n$ & $1\times 10^6$ & $6\times 10^7$ & $3.6\times 10^9$\\
$400n$ & 2500 & 150000 & $9\times 10^6$\\
$2n^2$ & 707 & 5477 & 42426\\
$n^4$ & 31 & 88 & 244\\
$2^n$ & 19 & 25 & 31\\
\hline
\end{tabular}
\end{center}

\bigskip

\item Si $m$ est la taille maximale que l'on peut traiter en un temps
  donnée, que devient cette valeur si on reçoit une machine 256 fois plus puissante?
\begin{center}
\footnotesize
\begin{tabular}{cc}
\hline
f(n) &Temps\\
\hline
$n$ & $256m$\\
$400n$ & $256m$\\
$2n^2$ & $16m$\\
$n^4$ & $4m$\\
$2^n$ & $m+8$\\
\hline
\end{tabular}
\end{center}
\end{itemize}

~\hfill{\scriptsize(Dupont)}

\end{frame}

\subsection{Notations asymptotiques}

\begin{frame}{Notations asymptotiques}
\begin{itemize}
\item Permettent de caractériser le taux de croissance de fonctions $f:\nats\rightarrow \reals^+$

\bigskip

\item Trois notations:
\begin{itemize}
\item Grand-O: $f(n)\in O(g(n)) \approx f(n)\leq g(n)$
\item Grand-Omega: $f(n)\in \Omega(g(n)) \approx f(n)\geq g(n)$
\item Grand-Theta: $f(n) \in \Theta(g(n)) \approx f(n) = g(n)$
\end{itemize}
\end{itemize}
\end{frame}

\begin{frame}{Notation grand-O}

{\small
$$O(g(n))=\{f(n)| \exists c>0, \exists n_0\geq 1\mbox{ tels que } 0\leq f(n)\leq c g(n), \forall n\geq n_0\}$$
}

\medskip

\centerline{\includegraphics[width=4cm]{Figures/02-grando.pdf}}

\medskip

\begin{itemize}
\item $f(n)\in O(g(n))\Rightarrow g(n)$ est une borne \alert{supérieure} asymptotique pour $f(n)$.
\item Par abus de notation, on écrira aussi: $f(n)=O(g(n))$.
\end{itemize}

\end{frame}

\begin{frame}{Notation grand-Omega}

{\small
$$\Omega(g(n))=\{f(n)| \exists c>0, \exists n_0\geq 1\mbox{ tels que } 0\leq c g(n)\leq f(n), \forall n\geq n_0\}$$
}

\medskip

\centerline{\includegraphics[width=4cm]{Figures/02-grandomega.pdf}}

\medskip

\begin{itemize}
\item $f(n)\in \Omega(g(n))\Rightarrow g(n)$ est une borne \alert{inférieure} asymptotique pour $f(n)$.
\item Par abus de notation, on écrira aussi: $f(n)=\Omega(g(n))$.
\end{itemize}

\end{frame}

\begin{frame}{Notation grand-Theta}

{\small
\begin{eqnarray*}
\Theta(g(n))&=&\{f(n)| \exists c_1,c_2>0, \exists n_0\geq 1\\
&&\mbox{ tels que } 0\leq c_1 g(n)\leq f(n)\leq c_2 g(n), \forall n\geq n_0\}
\end{eqnarray*}
}

\medskip

\centerline{\includegraphics[width=4cm]{Figures/02-grandtheta.pdf}}

\medskip

\begin{itemize}
\item $f(n)\in \Theta(g(n))\Rightarrow g(n)$ est une borne \alert{serrée} (``tight'') asymptotique pour $f(n)$.
\item Par abus de notation, on écrira aussi: $f(n)=\Theta(g(n))$.
\end{itemize}

\end{frame}

\begin{frame}{Exemples}
\begin{itemize}
\item $3n^2-16n+2\in O(n^5)$ ? $\in O(n)$ ? $\in O(n^{17})$ ?
\item $3n^2-16n+2\in \Omega(n^5)$ ? $\in \Omega(n)$ ? $\in \Omega(n^{17})$ ?
\item $3n^2-16n+2\in \Theta(n^5)$ ? $\in \Theta(n)$ ? $\in \Theta(n^{17})$ ?

\bigskip

\item Classes de complexité: 
$$O(1)\subset O(\log n)\subset O(n) \subset O(n\log n) \subset O(n^{a>1})\subset (2^n)$$
\centerline{\includegraphics[width=6cm]{Figures/02-complexclass.pdf}}
\end{itemize}
\end{frame}

\begin{frame}{Quelques propriétés}
\small
\begin{itemize}
\item $f(n)\in\Omega(g(n)) \Leftrightarrow g(n)\in O(f(n))$
\item $f(n)\in\Theta(g(n))\Leftrightarrow f(n)\in O(g(n))$ et $f(n)\in \Omega(g(n))$
\item $f(n)\in\Theta(g(n))\Leftrightarrow g(n)\in\Theta(f(n))$
%\item $f(n)\in\Theta(g(n))\Rightarrow g(n)\in O(f(n))$
\bigskip
\item Si $f(n)\in O(g(n))$, alors pour tout $k\in \nats$, on a $k\cdot f(n)\in O(g(n))$
\begin{itemize}
\item Example: $\log_a(n)\in O(\log_b(n))$, $a^{n+b}\in O(a^n)$
\end{itemize}
\item Si $f_1(n)\in O(g_1(n))$ et $f_2(n)\in O(g_2(n))$, alors $f_1(n)\cdot f_2(n)\in O(g_1(n)\cdot g_2(n))$
\begin{itemize}
\item Example: $\sum_{i=1}^m a_i n^i\in O(n^m)$
\end{itemize}
\item Si $f_1(n)\in O(g_1(n))$ et $f_2(n)\in O(g_2(n))$, alors $f_1(n)+f_2(n)\in O(g_1(n)+g_2(n))$ et $f_1(n)+f_2(n)\in O(\max\{g_1(n),g_2(n)\})$
\end{itemize}
\end{frame}

\subsection{Complexité d'algorithmes et de problèmes}

\begin{frame}{Complexité d'un algorithme}
\begin{itemize}
\item On utilise les notations asymptotiques pour caractériser la
  \alert{complexité} d'un algorithme.
\item Il faut préciser de quelle complexité on parle: générale, au
  pire cas, au meilleur cas, en moyenne...
\item La notation grand-O est de loin la plus utilisée
\begin{itemize}
\item $f(n)\in O(g(n))$ sous-entend généralement que $O(g(n))$ est le
  plus petit sous-ensemble qui contient $f(n)$ et que $g(n)$ est la plus concise possible
\item Exemple: $n^3+100n^2-n \in O(n^3)=O(n^3+n^2)\subset O(n^4) \subset O(2^n)$
\end{itemize}
\item Idéalement, les notations $O$ et $\Omega$ devraient être
  limitées au cas où on n'a pas de borne serrée.
\end{itemize}
\end{frame}

\begin{frame}{Complexité d'un algorithme}
Exemples:
\begin{itemize}
\item On dira:\\
``la complexité au pire cas du tri par insertion est $\Theta(n^2)$''\\
plutôt que\\
``La complexité au pire cas du tri par insertion est $O(n^2)$'' \\
ou ``La complexité du tri par insertion est $O(n^2)$''
\item On dira\\
``La complexité au meilleur cas du tri par insertion est $\Theta(n)$''\\
plutôt que\\
``La complexité au meilleur cas du tri par insertion est $\Omega(n)$''\\
ou ``La complexité du tri par insertion est $\Omega(n)$''
\item Par contre, on dira ``La complexité de $\proc{Fibonacci}$ est $\Omega(1.4^n)$'', car on n'a pas de borne plus précise à ce stade.
\end{itemize}
\note{Fibonacci est le bon exemple de l'utilisation de $\Omega$. On ne peut rien dire de plus à ce stade}
\end{frame}

\begin{frame}{Complexité d'un problème}
\begin{itemize}
\item Les notations asymptotiques servent aussi à caractériser la complexité
  d'un problème
\begin{itemize}
\item Un problème est $O(g(n))$  s'il existe un algorithme $O(g(n))$ pour le résoudre
\item Un problème est $\Omega(g(n))$ si tout algorithme qui le résoud est forcément $\Omega(g(n))$
\item Un problème est $\Theta(g(n))$ s'il est $O(g(n))$ et $\Omega(g(n))$
\end{itemize}
\item Exemple du problème de tri:
\begin{itemize}
\item Le problème du tri est $O(n \log n)$ (voir plus loin)
\item On peut montrer facilement que le problème du tri est $\Omega(n)$ (voir le transparent suivant)
\item On montrera plus tard que le problème de tri est en fait $\Omega(n \log n)$ et donc qu'il est $\Theta(n\log n)$.
\end{itemize}
\item Exercice: montrez que la recherche du maximum dans un tableau est $\Theta(g(n))$
\end{itemize}
\end{frame}


\begin{frame}{Le problème du tri est $\Omega(n)$}
Preuve par l'absurde (ou par contraposition):
\begin{itemize}
\item Supposons qu'il existe un algorithme moins que $O(n)$ pour résoudre le problème du tri
\item Cet algorithme ne peut pas parcourir tous les éléments du tableau, sinon il serait au moins $O(n)$
\item Il y a donc au moins un élément du tableau qui n'est pas vu par cet algorithme
\item Il existe donc des instances de tableau qui ne seront pas triées correctement par cet algorithme
\item Il n'y a donc pas d'algorithme plus rapide que $O(n)$ pour le tri.
\end{itemize}
\end{frame}

\subsection{Complexité d'algorithmes itératifs}

\begin{frame}{Comment calculer la complexité en pratique ?}
%\begin{itemize}
Quelques règles pour les algorithmes itératifs:
\begin{itemize}
\item Affectation, accès à un tableau, opérations arithmétiques, appel de fonction: $O(1)$
\item Instruction If-Then-Else: $O(\mbox{complexité max des deux branches})$
\item Séquence d'opérations: l'opération la plus couteuse domine (règle de la somme)
\item Boucle simple: $O(n f(n))$ si le corps de la boucle est $O(f(n))$
\end{itemize}
\end{frame}

\begin{frame}{Comment calculer la complexité en pratique ?}
%\begin{itemize}
%Quelques règles pour les algorithmes itératifs:
\begin{itemize}
\item Double boucle: $O(n^2 f(n))$ où $f(n)$ est la complexité du corps de la boucle
\item Boucles incrémentales: $O(n^2)$ (si corps $O(1)$)
\begin{center}\footnotesize
   \fcolorbox{white}{Lightgray}{
    \begin{codebox}
      \zi \For $i\gets 1$ \To $n$
      \zi \Do \For $j\gets 1$ \To $i$
      \zi \Do $\ldots$
      \End\End
    \end{codebox}}
\end{center}
\item Boucles avec un incrément exponentiel: $O(\log n)$ (si corps $O(1)$)
\begin{center}\footnotesize
   \fcolorbox{white}{Lightgray}{
    \begin{codebox}
      \zi $i\gets 1$
      \zi \While  $i\leq n$
      \zi \Do $\ldots$
      \zi $i\gets 2i$
      \End\End
    \end{codebox}}
\end{center}
\end{itemize}
\end{frame}

\begin{frame}{Exemple: }

$\proc{prefixAverages}(X)$:
\begin{itemize}
\item {\bf Entrée:} tableau $X$ de taille $n$
\item {\bf Sortie:} tableau $A$ de taille $n$ tel que $A[i]=\frac{\sum_{j=1}^i X[j]}{i}$
\end{itemize}

\begin{columns}
\begin{column}{5cm}
\begin{center}\small
   \fcolorbox{white}{Lightgray}{
    \begin{codebox}
      \Procname{$\proc{prefixAverages}(X)$}
      \li \For $i\gets 1$ \To $\attrib{X}{length}$
      \li \Do $a\gets 0$
      \li \For $j\gets 1$ \To $i$
      \li \Do $a\gets a+X[j]$ \End
      \li $A[i]\gets a/i$
      \End
      \li \Return A 
    \end{codebox}}
\end{center}
Complexité: $\Theta(n^2)$
\end{column}
\begin{column}{5cm}
\begin{center}\small
   \fcolorbox{white}{Lightgray}{
    \begin{codebox}
      \Procname{$\proc{prefixAverages2}(X)$}
      \li $s\gets 0$
      \li \For $i\gets 1$ \To $\attrib{X}{length}$
      \li \Do $s\gets s+X[i]$
      \li $A[i]\gets s/i$
      \End
      \li \Return A 
    \end{codebox}}
\end{center}
Complexité: $\Theta(n)$
\end{column}
\end{columns}
\note{interieur du for: O(1) puis O(i) puis O(1) => O(i) => on peut ne garder que le for j=1 to i. L'instruction au center est exécutée exactement...}
\end{frame}



%%%%%%%%%% a part
%% algo récursif -> complexité sous une forme récursive. 
%% Exemple du merge-sort.

\subsection{Complexité d'algorithmes récursifs}

\begin{frame}{Complexité d'algorithmes récursifs}

\begin{itemize}
\item La complexité d'algorithme récursif mène généralement à une équation de récurrence
\item La résolution de cette équation n'est généralement pas triviale
\item On se contentera d'étudier quelques cas particuliers importants dans ce cours
\end{itemize}

\end{frame}

\begin{frame}{$\proc{Factorial}$ et $\proc{Fibonacci}$}
% mettre les algos et la récurrence

\begin{columns}
\begin{column}{5.5cm}
\begin{center}\footnotesize
   \fcolorbox{white}{Lightgray}{
    \begin{codebox}
\Procname{$\proc{Factorial}(n)$}
\li \If $n\isequal 0$
\li \Then \Return 1 \End
\li \Return $n \cdot \proc{Factorial}(n-1)$
    \end{codebox}}
\end{center}
{\footnotesize
\begin{eqnarray*}
T(0) & = & c_0\\
T(n) & = & T(n-1)+c_1\\
\end{eqnarray*}
$$\Rightarrow T(n)\in \Theta(n)$$
}
\end{column}
\begin{column}{5.5cm}
\begin{center}\footnotesize
   \fcolorbox{white}{Lightgray}{
    \begin{codebox}
      \Procname{$\proc{Fib}(n)$}
      \li \If $n \leq 1$
      \li \Then \Return n \End
      \li \Return $\proc{Fib}(n-2)+\proc{Fib}(n-1)$
    \end{codebox}}
\end{center}
{\footnotesize
\begin{eqnarray*}
T(0) & = & c_0, T(1)= c_0\\
T(n) & = & T(n-1)+T(n-2)+c_1\\
\end{eqnarray*}
}
$$\Rightarrow T(n)\in \Omega(1.4^n)$$
\end{column}
\end{columns}
\end{frame}


\begin{frame}{Analyse du tri par fusion}

\begin{center}
\fcolorbox{white}{Lightgray}{%
    \begin{codebox}
      \Procname{$\proc{merge-sort}(A,p,r)$}
      \li \If $\id{p}<\id{r}$
      \li \Then $q \gets \lfloor \frac{p+r}{2} \rfloor$
      \li       $\proc{merge-sort}(A,p,q)$
      \li       $\proc{merge-sort}(A,q+1,r)$
      \li       $\proc{merge}(A,p,q,r)$ \End
    \end{codebox}}
\end{center}

\begin{itemize}
\item Récurrence:
\begin{columns}
\begin{column}{5cm}
\begin{eqnarray*}
T(1) & = & c_1\\
T(n) & = & 2 T(n/2)+c_2 n+c_3\\
\end{eqnarray*}
\end{column}
\begin{column}{5cm}
\begin{eqnarray*}
T(1) & = & \Theta(1)\\
T(n) & = & 2 T(n/2)+\Theta(n)\\
\end{eqnarray*}
\end{column}
\end{columns}
\end{itemize}
\note{Dire que celle-ci est importante}
\end{frame}

\begin{frame}{Analyse du tri par fusion}
\begin{columns}
\begin{column}{6cm}
\begin{itemize}
\item Simplifions la récurrence en:
\begin{eqnarray*}
T(1) & = & c\\
T(n) & = & 2 T(n/2)+c n
\end{eqnarray*}
\item On peut représenter la récurrence par un arbre de récursion
\item La complexité est la somme du coût de chaque noeud
\end{itemize}
\end{column}
\begin{column}{5.5cm}
\centerline{\includegraphics[width=5.5cm]{Figures/02-recurtree-merge.pdf}}
\end{column}
\end{columns}
\end{frame}

\begin{frame}{Analyse du tri par fusion}
\begin{columns}
\begin{column}{5cm}
\begin{itemize}
\item Chaque niveau a un coût $cn$
\item En supposant que $n$ est une puissance de 2, il y a $\log_2 n+1$ niveaux
\item Le coût total est $cn\log_2 n+cn \in \Theta(n\log n)$
\end{itemize}
\end{column}
\begin{column}{7cm}
\centerline{\includegraphics[width=6.5cm]{Figures/02-recurtree-merge-annote.pdf}}
\end{column}
\end{columns}
\end{frame}

\begin{frame}{Remarques}
Limitations de l'analyse asymptotique
\begin{itemize}
\item Les facteurs constants ont de l'importance pour des problèmes de petite taille
\begin{itemize}
\item Le tri par insertion est plus rapide que le tri par fusion pour $n$ petit
\end{itemize}
\item Deux algorithmes de même complexité (grand-O) peuvent avoir des propriétés très différentes
\begin{itemize}
\item Le tri par insertion est en pratique beaucoup plus efficace que
  le tri par sélection sur des tableaux presque triés
\end{itemize}
\end{itemize}

\bigskip

Complexité en espace
\begin{itemize}
\item S'étudie de la même manière, avec les mêmes notations
\item Elle est bornée par la complexité en temps (pourquoi ?)
\end{itemize}

\end{frame}

\begin{frame}{Ce qu'on a vu}

\begin{itemize}
\item Correction d'algorithmes itératifs (par invariant) et récursifs (par induction)
\item Notions de complexité algorithmique
\item Notations asymptotiques
\item Calcul de complexité d'algorithmes itératifs et récursifs
\end{itemize}

\end{frame}

%A la fin de chaque cours, je vais dire toutes les simplifications
%qu'on a prise et encourager les étudiants à se poser certaines
%questions. Exemple:



\part{Algorithmes de tri}

%% % Superbe site web avec les differents algorithmes

%% % http://www.sorting-algorithms.com/

%% % Voir les slides de Carzaniga !!!


\begin{frame}{Plan}

\tableofcontents

\end{frame}

\section{Algorithmes de tri}

\begin{frame}{Tri}

\begin{itemize}
\item Un des probl�mes algorithmiques les plus fondamentaux.
\item En g�n�ral, on veut trier des enregistrements avec une cl� et des
  donn�es attach�es.
\medskip
\begin{center}
\begin{tabular}{ccccccc}
Record$_1$ & & Record$_2$ & & Record$_3$ & & Record$_n$\\
\cline{1-1} \cline{3-3} \cline{5-5} \cline{7-7}
\multicolumn{1}{|c|}{Key$_1$} & & \multicolumn{1}{|c|}{Key$_2$} & & \multicolumn{1}{|c|}{Key$_3$} & \ldots & \multicolumn{1}{|c|}{Key$_n$}\\
 \cline{1-1} \cline{3-3} \cline{5-5} \cline{7-7}
\multicolumn{1}{|c|}{Data$_1$} & & \multicolumn{1}{|c|}{Data$_2$} & & \multicolumn{1}{|c|}{Data$_3$} & & \multicolumn{1}{|c|}{Data$_n$}\\
 \cline{1-1} \cline{3-3} \cline{5-5} \cline{7-7}
\end{tabular}
\end{center}
\medskip
\item Ici, on va ignorer ces donn�es satellites et se focaliser sur
  les algorithmes de tri

\bigskip

\item Le probl�me de tri:
\begin{itemize}
\item Entr�e: une s�quence de $n$ nombres $\langle a_1,a_2,\ldots,a_n\rangle$
\item Sortie: une permutation de la s�quence de d�part $\langle a_1',a_2',\ldots,a_n'\rangle$ telle que $a_1'\leq a_2'\leq\ldots\leq a_n'$
\end{itemize}
\end{itemize}

\end{frame}

\begin{frame}{Applications}
\begin{itemize}
\item Tri des enregistrements dans une base de donn�es
\item Visualisation 3D: trier les objets du plus pr�s au plus �loign� pour l'affichage
\item ...
\end{itemize}
\end{frame}

\begin{frame}{Diff�rents types de tri}

\begin{itemize}
\item {\bf Tri interne:} tri en m�moire centrale. Tris externes: donn�es sur un disque externe.
\item {\bf Tri de tableau:} tri qui trie un tableau. Extensible � toutes structure de donn�es offrant un acc�s en temps (quasi) constant � ses �l�ments.
\item {\bf Tri g�n�rique:} peut trier n'imporque quel type d'objets pour autant qu'on puisse comparer ces objets.
\item {\bf Tri comparatif:} bas� sur la comparaison entre les �l�ments (cl�s)

\end{itemize}

\end{frame}

\begin{frame}{Diff�rents types de tri}
\begin{itemize}
\item {\bf Tri it�ratif:} bas� sur un ou plusieurs parcours it�ratif du tableau
\item {\bf Tri r�cursif:} bas� sur une proc�dure r�cursive
\item {\bf Tri en place:} ne n�cessite qu'une quantit� constante de m�moire suppl�mentaire.
\item {\bf Tri  stable:} conserve l'ordre relatif des �l�ments �gaux (au sens de la m�thode de comparaison).
\end{itemize}

\note{Stable: expliquer ce que ça veut dire en illustrant avec les colonnes dans excel: tri sur deux colonnes: un tri stable va maintenir l'ordre sur la deuxi�me colonne}
\end{frame}

\begin{frame}{Jusqu'ici}

  \begin{center}
    \def\arraystretch{1.5}
  \begin{tabular}{@{}lccc@{}c@{}}
    \emph{Algorithme}&\multicolumn{3}{c}{\emph{Complexit�}}&\emph{En place?}\\
    & \emph{\small Pire} & \emph{\small Moyenne} & \emph{Meilleure} & \\
    \hline\hline
    \proc{Insertion-Sort}&$\Theta(n^2)$&$\Theta(n^2)$&$\Theta(n)$&oui\\
    \hline
    \proc{Selection-Sort}&$\Theta(n^2)$&$\Theta(n^2)$&$\Theta(n^2)$&oui\\
    \hline
    \proc{Bubble-Sort}&$\Theta(n^2)$&$\Theta(n^2)$&$\Theta(n^2)$&oui\\
    \hline
    \proc{Merge-Sort}&$\Theta(n\log{n})$&$\Theta(n\log{n})$&$\Theta(n\log{n})$&non\\
    \hline\hline
    \hspace{1em}\alert{??}& &\alert{$\Theta(n\log{n})$}& &\alert{oui}\\
    \hline
    \hspace{1em}\alert{??}&\alert{$\Theta(n\log{n})$}& & &\alert{oui}\\
    \hline\hline
  \end{tabular}
  \end{center}

\end{frame}

\section{Tri rapide}

\begin{frame}{Tri rapide}
\begin{itemize}
\item {\em Quicksort} en anglais
\item Invent� par Hoare en 1960
\item Dans le top 10 des algorithmes du 20-i�me si�cle (SIAM)
\item L'exemple le plus c�l�bre de la technique du ``diviser pour r�gner''
\item Tri en place, comme tri par insertion, et contrairement au tri par fusion
\item Complexit�: $O(n^2)$ dans le pire des cas, $O(n\log n)$ en moyenne
\end{itemize}
\end{frame}

\begin{frame}{\proc{QuickSort}: principe}
Pour trier un sous-tableau $A[p\twodots r]$:
\begin{itemize}
\item Partitionner $A[p\twodots r]$ en deux sous-tableaux: $A[p..q-1]$ et $A[q+1\twodots r]$ tels que tout �l�ment de $A[p\twodots q-1]$ est $\leq A[q]$ et $A[q]\leq$ � tout �l�ment de $A[q+1\twodots r]$.\\~\hfill\alert{\emph{(diviser)}}

\item Appeler r�cursivement l'algorithme pour trier $A[p\twodots q-1]$ et $A[q+1\twodots r]$\\~\hfill\alert{\emph{(r�gner)}}
\end{itemize}

\bigskip

Remarques:
\begin{itemize}
\item $A[q]$ est appel� le ``\alert{pivot}''
\item Par rapport au tri par fusion, il n'y a pas d'op�ration de combinaison
\end{itemize}

\end{frame}

\begin{frame}{\proc{QuickSort}: principe}

\centerline{\includegraphics[width=8cm]{Figures/03-quicksort-illustration.pdf}}

% Gaetano -> � modifier pour mettre les q,r, etc.

\end{frame}

\begin{frame}\frametitle{\proc{QuickSort} Algorithm}

  \begin{center}
%    \begin{small}

    %% \fcolorbox{white}{Lightgray}{%
    %% \begin{codebox}
    %%   \Procname{$\proc{Partition}(A,\id{begin},\id{end})$}
    %%   \li $q\gets \id{begin}$
    %%   \li $v\gets A[\id{end}]$
    %%   \li \For $i\gets \id{begin}+1$ \To $\id{end}-1$
    %%   \li \Do \If $A[i]\le v$
    %%   \li     \Then $\id{swap}(A[i], A[q])$
    %%   \li           $q\gets q+1$
    %%           \End
    %%       \End
    %%   \li $\id{swap}(A[\id{end}], A[q])$
    %%   \li \Return $q$
    %% \end{codebox}}

    %% \bigskip
    \fcolorbox{white}{Lightgray}{
    \begin{codebox}
      \Procname{$\proc{QuickSort}(A,\id{begin},\id{end})$}
      \li \If $\id{begin}<\id{end}$
      \li \Then $q\gets\proc{Partition}(A,\id{begin},\id{end})$
      \li $\proc{QuickSort}(A,\id{begin},q-1)$
      \li $\proc{QuickSort}(A,q+1,\id{end})$
      \End
    \end{codebox}}
%    \end{small}
  \end{center}
\end{frame}

\begin{frame}{Partition}

\begin{itemize}
\item On d�marre avec $q=1$
\item On parcourt le tableau de gauche � droite, en commen�ant � la position 2
\item Si l'�l�ment $A[i]$ est plus petit ou �gal au pivot, on l'�change avec la position $q$ courante et on d�place $q$ vers la droite
\end{itemize}

\begin{center}
    \fcolorbox{white}{Lightgray}{%
    \begin{codebox}
      \Procname{$\proc{Partition}(A,\id{begin},\id{end})$}
      \li $q\gets \id{begin}$
      \li $v\gets A[\id{end}]$
      \li \For $i\gets \id{begin}+1$ \To $\id{end}-1$
      \li \Do \If $A[i]\le v$
      \li     \Then $\id{swap}(A[i], A[q])$
      \li           $q\gets q+1$
              \End
          \End
      \li $\id{swap}(A[\id{end}], A[q])$
      \li \Return $q$
    \end{codebox}}
\end{center}

\end{frame}

\begin{frame}{Partition: illustration}

\centerline{\includegraphics[width=6cm]{Figures/03-partition.pdf}}

\end{frame}

\begin{frame}{Correction}

Invariant pour la boucle:
\begin{enumerate}
\item Toutes les valeurs dans $A[p\twodots i]$ sont $\leq$ au pivot
\item Toutes les valeurs dans $A[i+1\twodots j-1]$ sont $>$ que le pivot
\item $A[r]=pivot$
\end{enumerate}

\bigskip


{\bf Initialisation:} $A[p\twodots i]$ et $A[i+1\twodots q]$ sont vides

{\bf Maintenance:} 

{\bf Terminaison:} $j=r$.


\end{frame}

\begin{frame}{Algorithme complet}
  \begin{center}
   \begin{small}

    \fcolorbox{white}{Lightgray}{%
    \begin{codebox}
      \Procname{$\proc{Partition}(A,\id{begin},\id{end})$}
      \li $q\gets \id{begin}$
      \li $v\gets A[\id{end}]$
      \li \For $i\gets \id{begin}+1$ \To $\id{end}-1$
      \li \Do \If $A[i]\le v$
      \li     \Then $\id{swap}(A[i], A[q])$
      \li           $q\gets q+1$
              \End
          \End
      \li $\id{swap}(A[\id{end}], A[q])$
      \li \Return $q$
    \end{codebox}}

    \bigskip
    \fcolorbox{white}{Lightgray}{
    \begin{codebox}
      \Procname{$\proc{QuickSort}(A,\id{begin},\id{end})$}
      \li \If $\id{begin}<\id{end}$
      \li \Then $q\gets\proc{Partition}(A,\id{begin},\id{end})$
      \li $\proc{QuickSort}(A,\id{begin},q-1)$
      \li $\proc{QuickSort}(A,q+1,\id{end})$
      \End
    \end{codebox}}
    \end{small}
  \end{center}

\end{frame}

\begin{frame}{Illustration}

\centerline{\includegraphics[width=8cm]{Figures/03-quicksort-illustration2.pdf}}

% Gaetano: attention � modifier ou changer

\end{frame}

\begin{frame}{Complexit� de $\proc{Partition}$}

\begin{center}
    \fcolorbox{white}{Lightgray}{%
    \begin{codebox}
      \Procname{$\proc{Partition}(A,\id{begin},\id{end})$}
      \li $q\gets \id{begin}$
      \li $v\gets A[\id{end}]$
      \li \For $i\gets \id{begin}+1$ \To $\id{end}-1$
      \li \Do \If $A[i]\le v$
      \li     \Then $\id{swap}(A[i], A[q])$
      \li           $q\gets q+1$
              \End
          \End
      \li $\id{swap}(A[\id{end}], A[q])$
      \li \Return $q$
    \end{codebox}}
\end{center}

$$T(n)=\Theta(n)$$

\end{frame}

\begin{frame}{Complexit� de $\proc{QuickSort}$}

\begin{center}
    \fcolorbox{white}{Lightgray}{
    \begin{codebox}
      \Procname{$\proc{QuickSort}(A,\id{begin},\id{end})$}
      \li \If $\id{begin}<\id{end}$
      \li \Then $q\gets\proc{Partition}(A,\id{begin},\id{end})$
      \li $\proc{QuickSort}(A,\id{begin},q-1)$
      \li $\proc{QuickSort}(A,q+1,\id{end})$
      \End
    \end{codebox}}
\end{center}

\begin{itemize}
\item Pire cas:
\begin{itemize}
\item $q=begin$ ou $q=end$
\item Le partitionnement transforme un probl�me de taille $n$ en un probl�me de taille $n-1$
$$T(n)=T(n-1)+\Theta(n)$$
\medskip
\item M�me complexit� que le tri par insertion:
$$T(n)=\Theta(n^2)$$
\end{itemize}
\end{itemize}
\note{complexit�$=\sum_{i=1}^n i$

\bigskip

Ca se produit quand ? (tableau d�j� tri�)
}
\end{frame}

\begin{frame}{Complexit� de \proc{QuickSort}}

\begin{center}
    \fcolorbox{white}{Lightgray}{
    \begin{codebox}
      \Procname{$\proc{QuickSort}(A,\id{begin},\id{end})$}
      \li \If $\id{begin}<\id{end}$
      \li \Then $q\gets\proc{Partition}(A,\id{begin},\id{end})$
      \li $\proc{QuickSort}(A,\id{begin},q-1)$
      \li $\proc{QuickSort}(A,q+1,\id{end})$
      \End
    \end{codebox}}
\end{center}

\begin{itemize}
\item Meilleur cas:
\begin{itemize}
\item $q=\lfloor n/2 \rfloor$
\item Le partitionnement transforme un probl�me de taille $n$ en deux
  probl�mes de taille $\lceil n/2\rceil$ et $\lfloor n/2 \rfloor-1$ respectivement
$$T(n)=2 T(n/2)+\Theta(n)$$
\medskip
\item M�me complexit� que le tri par fusion:
$$T(n)=\Theta(n\log n)$$
\end{itemize}
\end{itemize}
\note{complexit�=$\sum_{i=1}^n i$}
\end{frame}

\begin{frame}{Complexit� moyenne de \proc{QuickSort}}
\begin{itemize}
\item Complexit� moyenne identique � la complexit� du meilleur cas
$$T(n)=\Theta(n\log n)$$
\item Intuitivement:
\begin{itemize}
\item En moyenne, on s'attend � une alternance de ``bons'' et de
  ``mauvais'' partitionnements
\item La complexit� d'un mauvais partitionnement suivi d'un bon est
  identique � la complexit� d'une bon partitionnement directement
  (seule la constante est modifi�e).
\end{itemize}
\end{itemize}
\centerline{\includegraphics[width=10cm]{Figures/03-quicksort-casmoyen.pdf}}

\note{Peut-�tre dire aussi qu'un partitionnement tr�s d�s�quilibr� (1/10 - 9/10) garde une complexit� de $n\log n$}

\end{frame}

\begin{frame}{Variantes de \proc{Quicksort}}

\begin{itemize}
\item Choix du pivot:
\begin{itemize}
\item Prendre un �l�ment au hasard plut�t que le dernier.
\item Prendre la m�diane de 3 �l�ments
\item Diminue drastiquement les chances d'�tre dans le pire cas
\end{itemize}
\end{itemize}

\begin{small}
\begin{center}
    \fcolorbox{white}{Lightgray}{%
     \begin{codebox}
       \Procname{$\proc{Randomized-Partition}(A,\id{begin},\id{end})$}
       \li i=\proc{Random}(begin,end)
       \li \id{swap}(A[end],A[i])
       \li \Return $\proc{Partition}(A,begin,end)$
     \end{codebox}}

\bigskip

    \fcolorbox{white}{Lightgray}{%
     \begin{codebox}
       \Procname{$\proc{Median-Of-3-Partition}(A,\id{begin},\id{end})$}
       \li i=\proc{median}(A,begin,(begin+end)/2,end)
       \li \id{swap}(A[end],A[i])
       \li \Return $\proc{Partition}(A,begin,end)$
     \end{codebox}}
\end{center}
\end{small}

\end{frame}

\begin{frame}{Variantes de \proc{Quicksort}}

\begin{itemize}
\item Petits sous-tableaux
\begin{itemize}
\item \proc{Quicksort} est trop lourd pour des petits tableaux
\item Utiliser un tri na�f (par ex., par insertion) sur les sous-tableaux de longueur inf�rieure � $k$ ($k\approx 20$).
\end{itemize}
\end{itemize}

\begin{small}
\begin{center}
    \fcolorbox{white}{Lightgray}{
    \begin{codebox}
      \Procname{$\proc{QuickSort}(A,\id{begin},\id{end})$}
      \li \If $\id{begin}\leq end+\const{CUTOFF}-1$
      \li \Then $\proc{InsertionSort}(A,\id{begin},\id{end})$
      \li \Return \End
      %\li \If $\id{begin}<\id{end}$
      \li $q\gets\proc{Partition}(A,\id{begin},\id{end})$
      \li $\proc{QuickSort}(A,\id{begin},q-1)$
      \li $\proc{QuickSort}(A,q+1,\id{end})$
      \End
    \end{codebox}}

\end{center}
\end{small}

\end{frame}

\begin{frame}{Conclusion sur $\proc{QuickSort}$}

\begin{itemize}
\item Rapide en moyenne $\Theta(n\log n)$
\item Pire cas en $\Theta(n^2)$ mais tr�s improbable avec choix du pivot bien fait
\item Bonne performance au niveau du cache
\item Tri \alert{en place} (mais utilise de la m�moire pour la trace r�cursive)
\item \alert{Pas stable}
\item En pratique souvent un peu plus rapide que \proc{Merge-Sort}
\end{itemize}

\end{frame}

\begin{frame}{Jusqu'ici}

  \begin{center}
    \def\arraystretch{1.5}
  \begin{tabular}{@{}lccc@{}c@{}}
    \emph{Algorithme}&\multicolumn{3}{c}{\emph{Complexit�}}&\emph{En place?}\\
    & \emph{\small Pire} & \emph{\small Moyenne} & \emph{Meilleure} & \\
    \hline\hline
    \proc{Insertion-Sort}&$\Theta(n^2)$&$\Theta(n^2)$&$\Theta(n)$&oui\\
    \hline
    \proc{Selection-Sort}&$\Theta(n^2)$&$\Theta(n^2)$&$\Theta(n^2)$&oui\\
    \hline
    \proc{Bubble-Sort}&$\Theta(n^2)$&$\Theta(n^2)$&$\Theta(n^2)$&oui\\
    \hline
    \proc{Merge-Sort}&$\Theta(n\log{n})$&$\Theta(n\log{n})$&$\Theta(n\log{n})$&non\\
    \hline
    \proc{QuickSort} & $\Theta(n^2)$ & $\Theta(n\log{n})$ & $\Theta(n\log{n})$ & oui\\
    \hline\hline
    \hspace{1em}\alert{??}&\alert{$\Theta(n\log{n})$}& & &\alert{oui}\\
    \hline\hline
  \end{tabular}
  \end{center}

\end{frame}

\section{Tri par tas}

\begin{frame}{Tri par tas: introduction}

\begin{itemize}
\item \emph{Heapsort} en anglais
\item invent� par Williams en 1964
\item bas� sur une structure de donn�e tr�s utile, le \emph{tas}
\item complexit� born�e par $\Theta(n\log n)$ (dans tous les cas)
\item tri en place
\item mise en oeuvre tr�s simple

\bigskip

\item Suite du cours:
\begin{itemize}
\item Introduction aux arbres
\item Tas
\item Tri par tas
\end{itemize}
\end{itemize}

% Voir slides fran�ais.

\note{On va d'abord voir le tas, puis le heapsort, On verra la semaine
  prochaine comment utiliser le tas pour faire une file � priorit�s}
\end{frame}

\subsection{Introduction aux arbres}

\begin{frame}{Arbres: d�finition}
\begin{itemize}
\item D�finition: Un arbre (\emph{tree}) $T$ est un graphe dirig� $(N,E)$, o�:
\begin{itemize}
\item $N$ est un ensemble de n\oe uds, et
\item $E\subset N\times N$ est un ensemble d'arcs,
\end{itemize}
poss�dant les propri�t�s suivantes:
\begin{itemize}
\item T est connexe et acyclique
\item Si $T$ n'est pas vide, alors il poss�de un n\oe ud distingu� appel� racine (\emph{root node}). Cette racine est unique.
\item Pour tout arc $(n_1,n_2)\in E$, le n\oe ud $n_1$ est le \alert{parent} de $n_2$.
\begin{itemize}
\item La racine de $T$ ne poss�de pas de parent.
\item Les autres n\oe uds de $T$ poss�dent un et un seul parent.
\end{itemize}
\end{itemize}
\end{itemize}

\centerline{\includegraphics[width=5cm]{Figures/03-exemple-arbre2.pdf}}

\end{frame}

\begin{frame}{Arbres: terminologie}
\begin{itemize}
\item Si $n_2$ est le parent de $n_1$, alors $n_1$ est le \alert{fils} (\emph{child}) de $n_2$.
\item Deux n\oe uds $n_1$ et $n_2$ qui poss�dent le m�me parent sont
  des \alert{fr�res} (\emph{siblings}).
\item Un n\oe ud qui poss�de au moins un fils est un n\oe ud \alert{interne}.
\item Un n\oe ud externe (c'est-�-dire, non interne) est une \alert{feuille}
  (\emph{leaf}) de l'arbre.
\item Un n\oe ud $n_2$ est un \alert{anc�tre} (\emph{ancestor}) d'un n\oe ud
  $n_1$ si $n_2$ est le parent de $n_1$ ou un anc�tre du parent de $n_1$.
\item Un n\oe ud $n_2$ est un \alert{descendant} d'un n\oe ud $n_1$ si $n_1$ est un anc�tre de $n_2$.
\end{itemize}

\centerline{\includegraphics[width=5cm]{Figures/03-exemple-arbre2.pdf}}

\end{frame}

\begin{frame}{Arbres: terminologie}

\begin{itemize}
\item Un \alert{chemin} est une s�quence de n\oe uds $n_1$, $n_2$, \ldots, $n_m$ telle que pour tout $i\in [1,m-1]$, $(n_i,n_{i+1})$ est un arc de l'arbre.\\
Remarque: Il n'existe jamais de chemin reliant deux feuilles distinctes.
\item La \alert{hauteur} (\emph{height}) d'un n\oe ud $n$ est le nombre d'arcs d'un plus long chemin de ce n\oe ud vers une feuille. La \emph{hauteur de l'arbre} est la hauteur de sa racine.
\item La \alert{profondeur} (\emph{depth}) d'un no\oe ud $n$ est le nombre d'arcs sur le chemin qui le relie � la racine.
\end{itemize}

\centerline{\includegraphics[width=5cm]{Figures/03-exemple-arbre2.pdf}}

\end{frame}

\begin{frame}{Arbre binaire}

\begin{itemize}
\item Un arbre \alert{ordonn�} est un arbre dans lequel les ensembles de fils de chacun de ses n\oe uds sont ordonn�s.
\item Un arbre \alert{binaire} est un arbre ordonn� poss�dant les propri�t�s suivantes:
\begin{itemize}
\item Chacun de ses n\oe uds poss�de au plus deux fils.
\item Chaque n\oe ud fils est soit un fils gauche, soit un fils droit.
\item Le fils gauche pr�c�de le fils droit dans l'ordre des fils d'un n\oe ud.
\end{itemize}
\item Un arbre \alert{binaire propre} (\emph{full}) est un arbre binaire dans lequel tous les n\oe uds internes poss�dent exactement deux fils.
\end{itemize}

\end{frame}

\begin{frame}{Propri�t�s des arbres binaires propres}

% fig: Dupont
\centerline{\includegraphics[width=5cm]{Figures/03-exemple-arbre-binaire.pdf}}

% text: Dupont
\begin{itemize}
\item Le nombre de n\oe uds externes est �gal au nombre de n\oe uds internes plus 1.
\item Le nombre de n\oe uds interne est �gal � $\frac{n-1}{2}$, o�
  $n$ d�signe le nombre de n\oe uds.
\item Le nombre de n\oe uds � la profondeur (ou niveau) $i$ est $\leq 2^i$.
\item La hauteur $h$ de l'arbre est $\leq$ au nombre de n\oe uds internes.
\item Le lien entre hauteur et nombre de n\oe uds peut �tre r�sum� comme suit:
$$n=\Omega(h)\mbox{ et }n=O(2^h)$$
\end{itemize}
\end{frame}

\subsection{Tas}

\begin{frame}{Tas: d�finition}\label{sec:03tas}

Un tas binaire (\emph{binary heap}) est un arbre binaire ordonn� tel que:
\begin{itemize}
\item Si $h$ d�note la hauteur de l'arbre:
\begin{itemize}
\item Pour tout $i\in [0,h-1]$, il y a exactement $2^i$ n\oe uds � la profondeur $i$.
\item Une feuille a une profondeur $h$ ou $h-1$.
\item Les feuilles de profondeur maximale ($h$) sont ``tass�es'' sur la gauche.
\end{itemize}
\medskip

\item Chacun de ses n\oe uds est associ� � une cl�.
\item La cl� de chaque n\oe ud est sup�rieure ou �gale � celle de ses
  fils (\alert{propri�t� d'ordre du tas}).
\end{itemize}

% fig: CLRS
\centerline{\includegraphics[width=5cm]{Figures/03-tas-binaire.pdf}}

\end{frame}

\begin{frame}{Propri�t� d'un tas}
\label{03:hauteurtas}
\begin{itemize}
\item Soit $T$ un arbre binaire complet contenant $n$ entr�es et de hauteur $h$:
\begin{itemize}
\item $n$ est sup�rieur ou �gal � la taille de l'arbre complet de hauteur $h-1$ plus un, soit $2^{h-1+1}-1+1=2^h$
\item $n$ est inf�rieur ou �gal � la taille de l'arbre complet de hauteur $h$, soit $2^{h+1}-1$
\begin{eqnarray*}
2^h\leq n\leq 2^{h+1} & \Leftrightarrow & 2^h\leq n < 2^{h+1}\\
& \Leftrightarrow & h \leq \log_2 n < h+1 \\
& \Leftrightarrow & h=\lceil \log_2 n\rceil
\end{eqnarray*}
\end{itemize}
\end{itemize}

% fig: CLRS
\centerline{\includegraphics[width=5cm]{Figures/03-tas-binaire.pdf}}

\end{frame}

\begin{frame}{Tas: interface}

Op�rations d�finies sur un tas $H$:
\begin{itemize}
\item $\attrib{H}{heap-size}$: le nombre de cl�s dans $H$.
\item $\proc{Build-Max-Heap}(A)$ construit un tas � partir du tableau $A$.
\item $\proc{Heap-Insert}(H,key)$ ins�re $key$ dans le tas.
\item $\proc{Heap-Extract-Max}(H)$ extrait la cl� maximale.
\end{itemize}

\bigskip

\centerline{\includegraphics[width=5cm]{Figures/03-tas-binaire.pdf}}

\end{frame}

\begin{frame}{Impl�mentation par un tableau}

\centerline{\includegraphics[width=10cm]{Figures/03-tas-binaire-tableau.pdf}}

\bigskip

Un tas peut �tre repr�sent� de mani�re compacte � l'aide d'un tableau $A$.

\begin{itemize}
\item La racine de l'abre est le premier �l�ment du tableau.
\item $\proc{Parent}(i)=\lfloor i/2\rfloor$
\item $\proc{Left}(i)=2i$
\item $\proc{Right}(i)=2i+1$
\end{itemize}

Propri�t� d'ordre du tas: $\forall i, A[\proc{Parent}(i)]\geq A[i]$

\end{frame}

\begin{frame}{$\proc{Heap-Extract-Max}$}

\begin{itemize}
\item Proc�dure $\proc{Heap-Extract-Max}$:
\begin{itemize}
\item Extrait la cl� maximale. Elle est toujours � la racine \emph{(Pourquoi ?)}
\item R�arrange le tas pour maintenir la propri�t� d'ordre du tas
\end{itemize}

% FIG: Gaetano -> a refaire
\centerline{\includegraphics[width=5cm]{Figures/03-heap-extract-max.pdf}}

\item On remplace la racine par la feuille la plus � droite.
\item Les sous-arbres de droite et de gauche sont des tas.
\end{itemize}
\end{frame}

\begin{frame}{$\proc{Max-Heapify}$}

\begin{itemize}
\item Proc�dure $\proc{Max-Heapify}(A,i)$:
\begin{itemize}
\item Suppose que le sous-arbre de gauche du n\oe ud $i$ est un tas
\item Suppose que le sous-arbre de droite du n\oe ud $i$ est un tas
\item But: r�arranger le tas pour maintenir la propri�t� d'ordre du tas
\end{itemize}
\item Ex: $\proc{Max-Heapify}(A,2)$
\centerline{\includegraphics[width=8cm]{Figures/03-max-heapify.pdf}}
\end{itemize}

\end{frame}

\begin{frame}{$\proc{Max-Heapify}$}

\begin{center}
\begin{small}
\fcolorbox{white}{Lightgray}{
      \begin{codebox}
        \Procname{$\proc{Max-Heapify}(A,i)$}
        \li $l\gets\proc{Left}(i)$
        \li $r\gets\proc{Right}(i)$
        \li \If $l\le \attrib{A}{heap-size} \wedge A[l]>A[i]$
        \li \Then $\id{largest}\gets l$
        \li \Else $\id{largest}\gets i$
        \End
        \li \If $r\le \attrib{A}{heap-size} \wedge A[r]>A[\id{largest}]$
        \li \Then $\id{largest}\gets r$
        \End
        \li \If $\id{largest}\ne i$
        \li \Then $\id{swap}(A[i],A[\id{largest}])$
        \li       $\proc{Max-Heapify}(A,\id{largest})$
        \End
      \end{codebox}}
\end{small}
\end{center}

\bigskip

\begin{itemize}
\item Complexit� ? La hauteur du n\oe ud: $T(n)=\Theta(\log n)$ (Transp. \pageref{03:hauteurtas})
\end{itemize}
\end{frame}

\begin{frame}{Construction d'un tas}

\begin{center}
\fcolorbox{white}{Lightgray}{
      \begin{codebox}
        \Procname{$\proc{Build-Max-Heap}(A)$}
        \li $\attrib{A}{heap-size}\gets\attrib{A}{length}$
        \li \For $i\gets\lfloor\id{length}(A)/2\rfloor$ \Downto $1$
        \li \Do $\proc{Max-Heapify}(A,i)$
        \End
      \end{codebox}}
\end{center}

\bigskip

\centerline{\includegraphics[width=10cm]{Figures/03-build-max-heap.pdf}}

\bigskip

\begin{itemize}
\item La tableau initial est interpr�t� comme un arbre binaire complet
\item On tasse les n\oe uds internes de bas en haut et de droite � gauche
\end{itemize}

\note{Montrer sur le dessin comment �a marche}

\end{frame}

\begin{frame}{Complexit� de $\proc{Build-Max-Heap}$}

\begin{itemize}
\item Borne simple:
\begin{itemize}
\item $O(n)$ appels � $\proc{Max-Heapify}$, chacun �tant $O(\log n)$ $\Rightarrow O(n\log n)$.
\end{itemize}

\bigskip


\item Analyse plus fine:
\begin{itemize}
\item Pour simplifier l'analyse, on suppose que l'arbre binaire est complet.
\item On a donc $n=2^{h+1}-1$ pour un $h\geq 0$, qui est aussi la hauteur de l'arbre r�sultant
\end{itemize}
\end{itemize}

\end{frame}

\begin{frame}{Complexit� de $\proc{Build-Max-Heap}$}
\centerline{\includegraphics[width=9cm]{Figures/03-max-heapify-complex.pdf}}

\begin{itemize}
\item Il y a $2^i$ n\oe uds � la profondeur $i$ (= hauteur $h-i$).
\item On doit appeler $\proc{Max-Heapify}$ sur chacun d'eux $\Rightarrow O(h-i)$.
\item Nombre d'op�rations en fonction de $h$:
$$T(h)=\sum_{i=0}^{h-1} 2^i O(h-i)=O(\sum_{i=0}^{h-1} 2^i (h-i))$$
\end{itemize}
\end{frame}

\begin{frame}{Complexit� de $\proc{Build-Max-Heap}$}
\begin{itemize}
\item $\sum_{i=0}^{h-1} 2^i (h-i)=1\cdot 2^{h-1}+2\cdot 2^{h-2}+\ldots+(h-1)\cdot 2^1+h\cdot 2^0$

\bigskip

\begin{small}
\begin{tabular}{ccccccccccl}
$2^{h-1}$ & + & $2^{h-2}$ & + & $2^h-3$ & + & \ldots & +& $2^0$ & = & $2^{h}-1$\\
 & + & $2^{h-2}$ & + & $2^h-3$ & + & \ldots & +& $2^0$ & = & $2^{h-1}-1$\\
 & + & $2^{h-2}$ & + & $2^h-3$ & + & \ldots & +& $2^0$ & = & $2^{h-2}-1$\\
 &   &        &   &      &   & \ldots &  &    & \ldots & \\
 &   &        &   &     &    &        & & $2^0$ & = & $2^1-1$\\
\hline
 &   &        &   &     &    &        & &     & = & $\left(\sum_{i=1}^h 2^i\right)-h$
\end{tabular}
\end{small}

\medskip

(En utilisant $\sum_{i=0}^n x^i=\frac{x^{n+1}-1}{x-1}$)

\bigskip

\item On obtient donc $T(h)=O(2^{h+1}-h-2).$
\item Puisque $h=log_2(n+1)-1=O(\log n)$, on a $$T(n)=O(n).$$
\end{itemize}
\end{frame}

%\subsection{File � priorit�s}

\subsection{Tri par tas}

\begin{frame}{Tri par tas: algorithme}

\begin{center}
\fcolorbox{white}{Lightgray}{
        \begin{codebox}
          \Procname{$\proc{Heap-Sort}(A)$}
          \li $\proc{Build-Max-Heap}(A)$
          \li \For $i\gets\id{length}(A)$ \Downto $1$
          \li \Do $\id{swap}(A[i],A[1])$
          \li     $\id{heap-size}(A)\gets\id{heap-size}(A)-1$
          \li     $\proc{Max-Heapify}(A,1)$
          \End
        \end{codebox}}
\end{center}

\end{frame}

\begin{frame}{Tri par tas: illustration}
Tableau initial: $A=[7,4,3,1,2]$

\bigskip

%Fig:CLRS
\centerline{\includegraphics[width=7cm]{Figures/03-heapsort-exemple.pdf}}

\end{frame}

\begin{frame}{Complexit� de \proc{Heap-Sort}}

\begin{itemize}
\item $\proc{Build-Max-Heap}$: $O(n)$
\item Boucle $\For$: $n-1$ fois
\item Echange d'�l�ments: $O(1)$
\item $\proc{Max-Heapify}$: $O(\log n)$
\end{itemize}
Total: $O(n\log n)$ \alert{(pour le pire cas et le cas moyen)}

\bigskip

Le tri par tas est cependant g�n�ralement battu par le tri rapide
\end{frame}

\section{Synth�se}

\begin{frame}{R�sum�}

 \begin{center}
    \def\arraystretch{1.5}
  \begin{tabular}{@{}lccc@{}c@{}}
    \emph{Algorithme}&\multicolumn{3}{c}{\emph{Complexit�}}&\emph{En place?}\\
    & \emph{\small Pire} & \emph{\small Moyenne} & \emph{Meilleure} & \\
    \hline\hline
    \proc{Insertion-Sort}&$\Theta(n^2)$&$\Theta(n^2)$&$\Theta(n)$&oui\\
    \hline
    \proc{Selection-Sort}&$\Theta(n^2)$&$\Theta(n^2)$&$\Theta(n^2)$&oui\\
    \hline
    \proc{Bubble-Sort}&$\Theta(n^2)$&$\Theta(n^2)$&$\Theta(n^2)$&oui\\
    \hline
    \proc{Merge-Sort}&$\Theta(n\log{n})$&$\Theta(n\log{n})$&$\Theta(n\log{n})$&non\\
    \hline
    \proc{Quick-Sort} & $\Theta(n^2)$ & $\Theta(n\log{n})$ & $\Theta(n\log{n})$ & oui\\
    \hline
    \proc{Heap-Sort} & $\Theta(n\log{n})$ & $\Theta(n\log{n})$ & $\Theta(n\log{n})$ & oui\\
    \hline\hline
  \end{tabular}
  \end{center}

\end{frame}

\begin{frame}{Peut-on faire mieux que $O(n\log n)$?}

% Gaetano

\begin{itemize}
\item Non, si on se restreint aux tri \emph{comparatifs}, c'est-�-dire:
\begin{itemize}
\item Aucune hypoth�se sur les �l�ments � trier
\item N�cessit� de les comparer entre eux
\end{itemize}
\item Complexit� d'un probl�me algorithmique versus complexit� d'un algorithme
\item Dans ce cas, un algorithme de tri est:
\begin{itemize}
\item Une suite de comparaisons d'�l�ments suivant une certaine m�thode
\item Un processus qui transforme un tableau $[e_0,e_1,\ldots,e_{n-1}]$ en un autre tableau $[e_{\sigma_0},e_{\sigma_1},\ldots,e_{\sigma_{n-1}}]$ o� $(\sigma_0,\sigma_1,\ldots,\sigma_{n-1})$ est une permutation de $(0,1,\ldots,n-1)$.
\end{itemize}
\end{itemize}

\end{frame}

\begin{frame}{Arbre de d�cision: exemple}

Un algorithme de tri = un arbre binaire de d�cision

\bigskip

% FIGS: Gaetano
\centerline{\includegraphics[width=10cm]{Figures/03-arbre-decision.pdf}}

\bigskip

(arbre de d�cision pour le tri par insertion du tableau $[e_0,e_1,e_2]$)

\bigskip

\emph{Exercice: construire l'arbre pour le tri par fusion}

\end{frame}

\begin{frame}{Arbre de d�cision: d�finition}

Un algorithme de tri = un arbre binaire de d�cision

\begin{itemize}
\item feuille de l'arbre: une permutation des �l�ments du tableau initial
\item tri: le chemin de la racine � la feuille correspondant au tableau tri�
\item hauteur de l'arbre: le pire cas pour le tri
\item branche la plus courte: le meilleur cas pour le tri
\item hauteur moyenne de l'arbre: la complexit� en moyenne du tri
\end{itemize}

\end{frame}

\begin{frame}{Arbre de d�cision: propri�t�}
\begin{itemize}
\item Un arbre binaire de hauteur $h$ a au plus $2^h$ feuilles (r�currence)
\item Le nombre de feuilles de l'arbre de d�cision est  $n!$ o�  $n$ est la taille du tableau � trier
% 
\item On a donc:
$$n!\leq 2^h \Rightarrow \log(n!)\leq h$$
\item Formule de Stirling:
$$n!=\sqrt{2\pi n} (\frac{n}{e})^n (1+\Theta(\frac{1}{n}))\Rightarrow n!\geq (\frac{n}{e})^n$$

$$h\geq \log(n!)>\log((\frac{n}{e})^n)=n\log n - n\log e\Rightarrow h=\Omega(n\log n)$$

\end{itemize}
\bigskip

\centerline{\bf Le probl�me du tri comparatif est $\Omega(n\log n)$}

\note{Dire qu'il ne peut pas y en avoir moins que $n!$ feuilles sinon, un cas ne serait pas traiter}

\end{frame}

\begin{frame}{Ce qu'on a vu}

\begin{itemize}
\item Cat�gorisation des algorithmes de tri
\item \proc{QuickSort} (tri en place en $O(n\log n))$
\item Analyse du cas moyen d'un algorithme
\item Notre premi�re structure de donn�es: le tas
\item \proc{HeapSort}
\item Borne inf�rieure sur les tris comparatifs

\bigskip

\item Liens:
\begin{itemize}
\item \url{http://www.sorting-algorithms.com/}
\end{itemize}

\end{itemize}

\end{frame}

\begin{frame}{Ce qu'on n'a pas vu}

\begin{itemize}
\item Analyse formelle de la complexit� moyenne du $\proc{QuickSort}$.
\item Invariant pour le tri par tas
\item M�thodes de tri lin�aire
\item M�thode de s�lection: trouver l'�l�ment de rang $i$
\end{itemize}

\note{Demander comment il ferait �a de mani�re naive}
\end{frame}


\part{Structures de données élémentaires}

% Structures de pile, file, liste liée (notion de liste liée comme une implémentation), arbre (?), files à priorité
% complexité amortie ? ajout d'élément à une structure de ce type.

\begin{frame}{Plan}

\tableofcontents

\end{frame}

\section{Introduction}

\begin{frame}{Concept}
\begin{itemize}
\item Une \alert{structure de données} est une manière d'organiser et de stocker l'information
\begin{itemize}
\item Pour en faciliter l'accès ou dans d'autres buts
\end{itemize}
\item Une structure de données a une \alert{interface} qui consiste en un ensemble de procédures pour ajouter, effacer, accéder, réorganiser, etc. les données.
\item Une structure de données conserve des \alert{données} et éventuellement des \alert{méta-données}
\begin{itemize}
\item Par exemple: un tas utilise un tableau pour stocker les clés et une variable $\attrib{A}{heap-size}$ pour retenir le nombre d'éléments qui sont dans le tas.
\end{itemize}
\item Un type de données abstrait (TDA) = définition des propriétés
  de la structure et de son interface (``cahier des charges'')
\end{itemize}
\end{frame}

\begin{frame}{Structures de données}

Dans ce cours:
\begin{itemize}
\item Principalement des \alert{ensembles dynamiques} (dynamic sets), amenés à croître, se
  rétrécir et à changer au cours du temps.
\item Les objets de ces ensembles comportent des attributs.
\item Un de ces attributs est une \alert{clé} qui permet d'identifier
  l'objet, les autres attributs sont la plupart du temps non
  pertinents pour l'implémentation de la structure.
\item Certains ensembles supposent qu'il existe un \alert{ordre total}
  entre les clés.
\end{itemize}

\end{frame}

\begin{frame}{Opérations standards sur les structures}

\begin{itemize}
\item Deux types: opérations de recherche/accès aux données et opérations de modifications
\item Recherche: exemples:
\begin{itemize}
\item $\proc{Search}(S,k)$: retourne un pointeur $x$ vers un élément dans $S$ tel que $\attrib{x}{key}=k$, ou $\const{NIL}$ si un tel élément n'appartient pas à $S$.
\item $\proc{Minimum}(S)$, $\proc{Maximum}(S)$: retourne un pointeur
  vers l'élément avec la plus petite (resp. grande) clé.
\item $\proc{Successor}(S,x)$,$\proc{Predecessor}(S,x)$ retourne un pointeur vers l'élément tout juste plus grand (resp. petit) que $x$ dans $S$, $\const{NIL}$ si $x$ est le maximum (resp. minimum).
\end{itemize}
\item Modification: exemples:
\begin{itemize}
\item $\proc{Insert}(S,x)$: insère l'élément $x$ dans $S$.
\item $\proc{Delete}(S,x)$: retire l'élément $x$ de $S$.
\end{itemize}
\end{itemize}

\end{frame}

\begin{frame}{Implémentation d'une structure de données}
\begin{itemize}
\item Etant donné un TDA (interface), plusieurs implémentations sont généralement possibles
\item La complexité des opérations dépend de l'implémentation, \alert{pas du TDA}.

\bigskip

\item Les briques de base pour implémenter une structure de données
  dépendent du langage d'implémentation
\begin{itemize}
\item Dans ce cours, les principaux outils du C: tableaux, structures
  à la C (objets avec attributs), liste liées (simples, doubles,
  circulaires), etc.
\end{itemize}
\item Une structure de données peut être implémentée à l'aide d'une
  autre structure de données (de base ou non)
\end{itemize}
\end{frame}

\begin{frame}{Quelques structures de données standards}

\begin{itemize}
\item Pile: collection d'objets accessible selon une politique LIFO
\item File: collection d'objets accessible selon une politique FIFO
\item File double: combine accès LIFO et FIFO
\item Liste: collection d'objets ordonnés accessible à partir de leur position
\item Vecteur: collection d'objets ordonnés accessible à partir de leur rang
\item File à priorité: accès uniquement à l'élément de clé (priorité) maximale

\bigskip

\item Dictionnaire: structure qui implémente les 3 opérations
  recherche, insertion, suppression (cf. partie 5)
\end{itemize}

\end{frame}

\section{Pile}

\begin{frame}{Pile}
\begin{itemize}
\item Ensemble dynamique d'objets accessibles selon une discipline
  \alert{LIFO} (``Last-in first-out'').
\item Interface
\begin{itemize}
\item $\proc{Stack-Empty}(S)$ renvoie vrai si et seulement si la pile est vide
\item $\proc{Push}(S,x)$ pousse la valeur $x$ sur la pile $S$
\item $\proc{Pop}(S)$ extrait et renvoie la valeur sur le sommet de la pile $S$
\end{itemize}
\item Applications:
\begin{itemize}
\item Option 'undo' dans un traitement de texte
\item Langage postscript
\item Appel de fonctions dans un compilateur
\item \ldots
\end{itemize}
\item Implémentations:
\begin{itemize}
\item avec un tableau (taille fixée a priori)
\item au moyen d'une liste liée (allouée de manière dynamique)
\item \ldots
\end{itemize}
\end{itemize}
\end{frame}

\begin{frame}{Implémentation par un tableau}

\begin{itemize}
\item $S$ est un tableau qui contient les éléments de la pile
\item $\attrib{S}{top}$ est la position courante de l'élément au sommet de $S$

\medskip

\begin{columns}
\begin{column}{5cm}
\centerline{\includegraphics[width=3cm]{Figures/04-piletableau.pdf}}

\bigskip

\begin{center}
\begin{small}
\fcolorbox{white}{Lightgray}{%
        \begin{codebox}
          \Procname{$\proc{Push}(S,x)$}
          \li \If $\attrib{S}{top}\isequal \attrib{S}{length}$
          \li \Then \Error ``overflow''\End
          \li $\attrib{S}{top}\gets \attrib{S}{top}+1$
          \li $S[\attrib{S}{top}]\gets x$
        \end{codebox}}
\end{small}
\end{center}

\end{column}
\begin{column}{5cm}
    \begin{center}
      \begin{small}
      \fcolorbox{white}{Lightgray}{%
        \begin{codebox}
          \Procname{$\proc{Stack-Empty}(S)$}
          \li \Return $\attrib{S}{top}\isequal 0$
          \End
        \end{codebox}}

\bigskip

      \fcolorbox{white}{Lightgray}{%
        \begin{codebox}
          \Procname{$\proc{Pop(S)}$}
          \li \If $\proc{Stack-Empty}(S)$
          \li \Then \Error ``underflow''
          \li \Else $\attrib{S}{top}\gets \attrib{S}{top}-1$
          \li       \Return $S[top(S)+1]$
          \End
        \end{codebox}}

      \end{small}
    \end{center}
\end{column}
\end{columns}

\item Complexité en temps \alert{et en espace}: $O(1)$\\
(Inconvénient: L'espace occupé ne dépend pas du nombre d'objets)
\end{itemize}

\end{frame}

\begin{frame}{Rappel: liste simplement et doublement liée}

\centerline{\includegraphics[width=8cm]{Figures/04-listeliee.pdf}}

\bigskip

\begin{itemize}
\item Structure de données composée d'une séquence d'éléments de liste.
\item Chaque élément $x$ de la liste est composé:
\begin{itemize}
\item d'un contenu utile $\attrib{x}{data}$ de type arbitraire (par exemple une clé),
\item d'un pointeur $\attrib{x}{next}$ vers l'élément suivant dans la séquence
\item \emph{Doublement liée: }d'une pointeur $\attrib{x}{prev}$ vers l'élément précédent dans la séquence
\end{itemize}
\item Soit $L$ une liste liée
\begin{itemize}
\item $\attrib{L}{head}$ pointe vers le premier élément de la liste
\item \emph{Doublement liée:} $\attrib{L}{tail}$ pointe vers le dernier élément de la liste
\end{itemize}
\item Le dernier élément possède un pointeur $\attrib{x}{next}$ vide (noté $\const{NIL}$)
\item \emph{Doublement liée:} Le premier élément possède un pointeur $\attrib{x}{prev}$ vide
%% \item Avantage
%% \begin{itemize}
%% \item L'insertion et la suppression d'éléments est réalisable en temps $O(1)$ en tête de liste, ainsi qu'à la suite d'un élément donné (à n'importe quel endroit pour une liste doublement liée).
%% \item Les éléments de la liste peuvent être alloués dynamiquement
%% \end{itemize}
\end{itemize}
\note{Mais l'espace mémoire est deux fois plus important qu'avec un tableau classique}
\end{frame}

%% \begin{frame}{Opérations sur une liste doublement liée}

%% %Interface:
%% {\small
%%     \begin{itemize}
%% %      \medskip
%%     \item $\proc{List-Insert}(L,x)$ ajoute l'élément $x$ au début de la liste $L$
%% %      \medskip
%%     \item $\proc{List-Delete}(L,x)$ supprime l'élément $x$ de la liste $L$
%% %      \medskip
%%     \item $\proc{List-Search}(L,k)$ trouve une élément dont la clé $k$
%%       est dans la liste $L$
%%     \end{itemize}
%% }

%% %% \fcolorbox{white}{Lightgray}{%
%% %%       \begin{codebox}
%% %%         \Procname{$\proc{List-Init}(L)$}
%% %%         \li $\id{prev}(\id{nil}(L))\gets\id{nil}(L)$
%% %%         \li $\id{next}(\id{nil}(L))\gets\id{nil}(L)$
%% %%       \end{codebox}}

%% \begin{center}
%% \begin{footnotesize}
%% \fcolorbox{white}{Lightgray}{%
%%       \begin{codebox}
%%         \Procname{$\proc{List-Insert}(L,x)$}
%%         \li $\attrib{x}{next}\gets \attrib{L}{head}$
%%         \li \If $\attrib{L}{head}\ne \const{NIL}$
%%         \li \Then $\attrib{L}{head}.\id{prev} \gets x$ \End
%%         \li $\attrib{L}{head}\gets x$
%%         \li $\attrib{x}{prev}\gets \const{NIL}$
%%       \end{codebox}}
%% ~~~~~
%% \fcolorbox{white}{Lightgray}{%
%%       \begin{codebox}
%%         \Procname{$\proc{List-Delete}(L,x)$}
%%         \li \If $\attrib{x}{prev}\ne \const{NIL}$
%%         \li \Then $\attrib{x}{prev}.\id{next}\gets \attrib{x}{next}$
%%         \li \Else $\attrib{L}{head}\gets \attrib{x}{next}$\End
%%         \li \If $\attrib{x}{next}\ne \const{NIL}$
%%         \li \Then $\attrib{x}{next}.\id{prev}\gets \attrib{x}{prev}$ \End
%%       \end{codebox}}

%% \bigskip

%% \fcolorbox{white}{Lightgray}{%
%%       \begin{codebox}
%%         \Procname{$\proc{List-Search}(L,k)$}
%%         \li $x\gets \attrib{L}{head}$
%%         \li \While $x\ne \const{NIL}\wedge\attrib{x}{key}\ne k$
%%         \li \Do $x\gets \attrib{x}{next}$
%%             \End
%%         \li \Return $x$
%%       \end{codebox}}
%% \end{footnotesize}
%% \end{center}

%% Complexité: $O(1)$ pour l'insertion et la suppression, $O(n)$ pour la recherche.
%% \end{frame}


%% \begin{frame}{Sentinelle}
%% \begin{itemize}\small
%% \item On peut simplifier le code en ajoutant une \alert{sentinelle} $L.nil$ qui représente le $\const{NIL}$
%% \item $L.nil.next$ pointe vers le premier élément et $L.nil.prev$ pointe vers le dernier élément (, $L.nil.next=L.nil$ et $L.nil.prev=L.nil$)
%% \end{itemize}

%% \centerline{\includegraphics[width=8cm]{Figures/04-sentinel.pdf}}

%% \begin{center}
%% \begin{footnotesize}
%% \fcolorbox{white}{Lightgray}{%
%%       \begin{codebox}
%%         \Procname{$\proc{List-Insert'}(L,x)$}
%%         \li $\attrib{x}{next}\gets \attrib{L}{nil}.\id{next}$
%%         \li $\attrib{L}{nil}.\id{next}.\id{prev} \gets x$
%%         \li $\attrib{L}{nil}.\id{next} \gets x$
%%         \li $\attrib{x}{prev}\gets \attrib{L}{nil}$
%%       \end{codebox}}
%% ~~~~~
%% \fcolorbox{white}{Lightgray}{%
%%       \begin{codebox}
%%         \Procname{$\proc{List-Delete'}(L,x)$}
%%         \li $\attrib{x}{prev}.\id{next}\gets \attrib{x}{next}$
%%         \li $\attrib{x}{next}.\id{prev}\gets \attrib{x}{prev}$
%%       \end{codebox}}

%% \bigskip

%% \fcolorbox{white}{Lightgray}{%
%%       \begin{codebox}
%%         \Procname{$\proc{List-Search'}(L,k)$}
%%         \li $x\gets \attrib{L}{nil}.\id{next}$
%%         \li \While $x\ne \attrib{L}{nil}\wedge\attrib{x}{key}\ne k$
%%         \li \Do $x\gets \attrib{x}{next}$
%%             \End
%%         \li \Return $x$
%%       \end{codebox}}
%% \end{footnotesize}
%% \end{center}

%% \note{
%% Toutes les pointeurs vers NIL sont remplacés par un pointeur vers $L.nil$

%% Ne permet de gagner qu'un facteur constant

%% \bigskip

%% Fait perdre de la place si on a beaucoup de petites listes


%% \bigskip

%% Principal intérêt est le clareté du code
%% }
%% \end{frame}


\begin{frame}{Implémentation d'une pile à l'aide d'une liste liée}

\begin{itemize}
\item $S$ est une liste simplement liée ($S.head$ pointe vers le premier élément de la liste)

\medskip

\begin{columns}
\begin{column}{5cm}
\begin{center}
\includegraphics[width=5cm]{Figures/04-pilelisteliee.pdf}
\end{center}

\bigskip

\fcolorbox{white}{Lightgray}{%
        \begin{codebox}
          \Procname{$\proc{Push}(S,x)$}
          \li $\attrib{x}{next}\gets \attrib{S}{head}$
          \li $\attrib{S}{head}\gets x$
        \end{codebox}}
\end{column}
\begin{column}{5cm}
\begin{center}
  \footnotesize
  \fcolorbox{white}{Lightgray}{%
    \begin{codebox}
      \Procname{$\proc{Stack-Empty}(S)$}
      \li \If $\attrib{S}{head}\isequal \const{NIL}$
      \li \Then \Return \const{true}
      \li \Else \Return \const{false}
        \End
  \end{codebox}}\hfill

\bigskip

  \fcolorbox{white}{Lightgray}{%
    \begin{codebox}
      \Procname{$\proc{Pop(S)}$}
      \li \If $\proc{Stack-Empty}(S)$
          \li \Then \Error ``underflow''
          \li \Else $x = S.head$
          \li       $\attrib{S}{head}\gets \attrib{S}{head}.\id{next}$
          \li       \Return $x$
          \End
        \end{codebox}}
    \end{center}
\end{column}
\end{columns}

\item Complexité en temps $O(1)$, complexité en espace $O(n)$ pour $n$ opérations
\end{itemize}

\end{frame}

\begin{frame}{Application}
\begin{itemize}
\item Vérifier l'appariement de parenthèses ($[]$,$()$ ou $\{\}$) dans une chaîne de caractères
\begin{itemize}
\item Exemples: $((x)+(y)]/2\rightarrow$ non, $[- (b) + \mbox{sqrt}(4*(a)*c)]/ (2*a) \rightarrow$ oui
\end{itemize}
\item Solution basée sur une pile:
\begin{center}
\begin{small}
\fcolorbox{white}{Lightgray}{%
        \begin{codebox}
          \Procname{$\proc{ParenthesesMatch}(A)$}
          \li $S\gets$ pile vide
          \li \For $i\gets 1 \To \attrib{A}{length}$
          \li \Do \If $A[i]$ est une parenthèse gauche
          \li \Then $\proc{Push}(S,A[i])$
          \li \ElseIf $A[i]$ est une parenthèse droite
          \li \Then \If $\proc{Stack-Empty}(S)$
          \li \Then \Return \const{False}

          \li \ElseIf  $\proc{Pop}(S)$ n'est pas du même type que $A[i]$
          \li \Then \Return \const{False}
          \End\End\End
          \li \End \Return $\proc{Stack-Empty}(S)$
          \End
        \end{codebox}}
\end{small}
\end{center}
\end{itemize}
\end{frame}

\section{Files simple et double}

\begin{frame}{File}
\begin{itemize}
\item Ensemble dynamique d'objets accessibles selon une discipline \alert{FIFO} (``First-in first-out'').
\item Interface
\begin{itemize}
\item $\proc{Enqueue}(Q,s)$ ajoute l'élément $x$ à la fin de la file $Q$
\item $\proc{Dequeue}(Q)$  retire l'élément à la tête de la file $Q$
\end{itemize}

\bigskip

\item Implémentation à l'aide d'un tableau circulaire
\begin{itemize}
\item $Q$ est un tableau de taille fixe $\attrib{Q}{length}$
\begin{itemize}
\item Mettre plus de $\attrib{Q}{length}$ éléments dans la file provoque une erreur de dépassement
\end{itemize}
\item $\attrib{Q}{head}$  est la position à la tête de la file
\item $\attrib{Q}{tail}$ est la première position vide à la fin de la file
\item Initialement: $\attrib{Q}{head}=\attrib{Q}{tail}=1$
\end{itemize}
\end{itemize}
\end{frame}

\begin{frame}{\proc{Enqueue} et \proc{Dequeue}}
\begin{small}
Etat initial:
\bigskip
\centerline{\includegraphics[width=6cm]{Figures/04-file1.pdf}}
$\proc{Enqueue}(Q,17)$, $\proc{Enqueue}(Q,3)$, $\proc{Enqueue}(Q,5)$
\bigskip
\centerline{\includegraphics[width=6cm]{Figures/04-file2.pdf}}
$\proc{Dequeue}(Q) \rightarrow 15$
\bigskip
\centerline{\includegraphics[width=6cm]{Figures/04-file3.pdf}}
\end{small}
\end{frame}

\begin{frame}{\proc{Enqueue} et \proc{Dequeue}}

\begin{center}
\begin{small}
\fcolorbox{white}{Lightgray}{%
  \begin{codebox}
    \Procname{$\proc{Enqueue(Q,x)}$}
    \li $Q[\attrib{Q}{tail}]\gets x$
    \li \If $\attrib{Q}{tail}\isequal \attrib{Q}{length}$
    \li       \Then $\attrib{Q}{tail}\gets 1$
    \li       \Else $\attrib{Q}{tail}\gets\attrib{Q}{tail}+1$
  \end{codebox}}
~~~~~
\fcolorbox{white}{Lightgray}{%
  \begin{codebox}
    \Procname{$\proc{Dequeue(Q)}$}
    \li $x\gets Q[\attrib{Q}{head}]$
    \li \If $\attrib{Q}{head}\isequal \attrib{Q}{length}$
    \li       \Then $\attrib{Q}{head}\gets 1$
    \li       \Else $\attrib{Q}{head}\gets\attrib{Q}{head}+1$\End
    \li \Return $x$
  \end{codebox}}
\end{small}
\end{center}

\bigskip

\begin{itemize}
\item Complexité en temps $O(1)$, complexité en espace $O(1)$.
\item {\it Exercice: ajouter la gestion d'erreur}
\end{itemize}
\end{frame}

\begin{frame}{Implémentation à l'aide d'une liste liée}
\centerline{\includegraphics[width=6cm]{Figures/04-filelisteliee.pdf}}
\begin{itemize}
\item $Q$ est une liste simplement liée
\item $\attrib{Q}{head}$ (resp. $\attrib{Q}{tail}$) pointe vers la tête (resp. la queue) de la liste

\begin{center}
\begin{small}
\fcolorbox{white}{Lightgray}{%
  \begin{codebox}
    \Procname{$\proc{Enqueue(Q,x)}$}
    \li $\attrib{x}{next}\gets \const{NIL}$
    \li \If $\attrib{Q}{head}\isequal \const{NIL}$
    \li \Then $\attrib{Q}{head}\gets x$
    \li \Else $\attrib{Q}{tail}.\id{next} \gets x$
    \End
    \li $\attrib{Q}{tail}\gets x$
  \end{codebox}}
~~~~~
\fcolorbox{white}{Lightgray}{%
  \begin{codebox}
    \Procname{$\proc{Dequeue(Q)}$}
    \li \If $\attrib{Q}{head}\isequal \const{NIL}$
    \li \Then \Error ``underflow''
    \End
    \li $x\gets \attrib{Q}{head}$
    \li $\attrib{Q}{head}\gets\attrib{Q}{head}.\id{next}$
    \li \If $\attrib{Q}{head}\isequal \const{NIL}$
    \li \Then $\attrib{Q}{tail}\gets \const{NIL}$ \End
    \li \Return $x$
  \end{codebox}}
\end{small}
\end{center}

\item Complexité en temps $O(1)$, complexité en espace $O(n)$ pour $n$ opérations
\end{itemize}

\end{frame}

\begin{frame}{File double}
Double ended-queue (deque)

\bigskip

\begin{itemize}
\item Généralisation de la pile et de la file
\item Collection ordonnée d'objets offrant la possibilité
\begin{itemize}
\item d'insérer un nouvel objet \alert{avant le premier} ou \alert{après le dernier}
\item d'extraire le \alert{premier} ou le \alert{dernier} objet
\end{itemize}
\item Interface:
\begin{itemize}
\item $\proc{insert-first}(Q,x)$: ajoute $x$ au début de la file double
\item $\proc{insert-last}(Q,x)$: ajoute $x$ à la fin de la file double
\item $\proc{remove-first}(Q)$: extrait l'objet situé en première position
\item $\proc{remove-last}(Q)$: extrait l'objet situé en dernière position
\item \ldots
\end{itemize}
\item Application: équilibrage de la charge d'un serveur
\end{itemize}

\end{frame}

\begin{frame}{Implémentation à l'aide d'une liste doublement liée}
\begin{itemize}
\item A l'aide d'une liste double liée
\item Soit la file double $Q$:
\begin{itemize}
\item $Q.head$ pointe vers un élément \alert{sentinelle} en début de liste
\item $Q.tail$ pointe vers un élément \alert{sentinelle} en fin de liste
\item $Q.size$ est la taille courante de la liste
\end{itemize}

\centerline{\includegraphics[width=10cm]{Figures/04-dequeimplem.pdf}}

\item Les sentinelles ne contiennent pas de données. Elles permettent
  de simplifier le code (pour un coût en espace constant).
\end{itemize}

\bigskip

{\it Exercice: implémentation de la file double sans
  sentinelles, implémentation de la file simple avec sentinelle}

\end{frame}

\begin{frame}{Implémentation à l'aide d'une liste doublement liée}
\begin{center}
      \begin{small}\hfill
      \fcolorbox{white}{Lightgray}{%
        \begin{codebox}
          \Procname{$\proc{insert-first}(Q,x)$}
          \li $\attrib{x}{prev}\gets \attrib{Q}{head}$
          \li $\attrib{x}{next}\gets \attrib{Q}{head}.\id{next}$
          \li $\attrib{Q}{head}.\id{next}.\id{prev} \gets \id{x}$
          \li $\attrib{Q}{head}.\id{next}\gets \id{x}$
          \li $\attrib{Q}{size}\gets \attrib{Q}{size}+1$
        \end{codebox}}\hfill
\fcolorbox{white}{Lightgray}{%
        \begin{codebox}
          \Procname{$\proc{insert-last}(Q,x)$}
          \li $\attrib{x}{prev}\gets \attrib{Q}{tail}.\id{prev}$
          \li $\attrib{x}{next}\gets \attrib{Q}{tail}$
          \li $\attrib{Q}{tail}.\id{prev}.\id{next} \gets \id{x}$
          \li $\attrib{Q}{head}.\id{prev}\gets \id{x}$
          \li $\attrib{Q}{size}\gets \attrib{Q}{size}+1$
        \end{codebox}}\hfill~

      \bigskip
      \fcolorbox{white}{Lightgray}{%
        \begin{codebox}
          \Procname{$\proc{remove-first}(Q)$}
          \li \If $(\attrib{Q}{size}\isequal 0)$
          \li \Then \Error\End
          \li $x\gets \attrib{Q}{head}.\id{next}$
          \li $\attrib{Q}{head}.\id{next}\gets \attrib{Q}{head}.\id{next}.\id{next}$
          \li $\attrib{Q}{head}.\id{next}.\id{prev}\gets \attrib{Q}{head}$
          \li $\attrib{Q}{size}\gets \attrib{Q}{size}-1$
          \li \Return $x$
        \end{codebox}}
      \hfill
      \fcolorbox{white}{Lightgray}{%
        \begin{codebox}
          \Procname{$\proc{remove-last}(Q)$}
          \li \If $(\attrib{Q}{size}\isequal 0)$
          \li \Then \Error\End
          \li $x\gets \attrib{Q}{tail}.\id{prev}$
          \li $\attrib{Q}{tail}.\id{prev}\gets \attrib{Q}{tail}.\id{prev}.\id{prev}$
          \li $\attrib{Q}{tail}.\id{prev}.\id{next}\gets \attrib{Q}{head}$
          \li $\attrib{Q}{size}\gets \attrib{Q}{size}-1$
          \li \Return $x$
        \end{codebox}}
\end{small}
    \end{center}

Complexité $O(1)$ en temps et $O(n)$ en espace pour $n$ opérations.

\end{frame}

\section{Liste}

\begin{frame}{Liste}

\begin{itemize}
\item Ensemble dynamique d'objets ordonnés accessibles \alert{relativement}
  les uns aux autres, sur base de leur position
\item Généralise toutes les structures vues précédemment
\item Interface:
\begin{itemize}
\item Les fonctions d'une liste double (insertion et retrait en début et fin de liste)
\item $\proc{Insert-Before}(L,p,x)$: insére $x$ avant $p$ dans la liste
\item $\proc{Insert-After}(L,p,x)$: insère $x$ après $p$ dans la liste
\item $\proc{Remove}(L,p)$: retire l'élement à la position $p$
\item $\proc{replace}(L,p,x)$: remplace par l'objet $x$ l'objet situé à la position $p$
\item $\proc{first}(L)$, $\proc{last}(L)$: renvoie la première, resp. dernière position dans la liste
\item $\proc{prev}(L,p)$, $\proc{next}(L,p)$: renvoie la position précédant (resp. suivant) $p$ dans la liste
\end{itemize}

\bigskip

\item Implémentation similaire à la file double, à l'aide d'une liste doublement liée (avec sentinelles)
\end{itemize}

\end{frame}

\begin{frame}{Implémentation à l'aide d'une liste doublement liée}


\begin{columns}
\begin{column}{5cm}
\centerline{\includegraphics[width=6cm]{Figures/04-listelisteliee.pdf}}

\bigskip

\begin{center}
  \begin{small}\hfill
      \fcolorbox{white}{Lightgray}{%
        \begin{codebox}
          \Procname{$\proc{insert-before}(L,p,x)$}
          \li $\attrib{x}{prev}\gets \attrib{p}{prev}$
          \li $\attrib{x}{next}\gets \id{p}$
          \li $\attrib{p}{prev}.\id{next} \gets \id{x}$
          \li $\attrib{p}{prev} \gets \id{x}$
          \li $\attrib{L}{size}\gets \attrib{L}{size}+1$
        \end{codebox}}
  \end{small}
\end{center}
\end{column}
\begin{column}{5cm}
\begin{center}

\fcolorbox{white}{Lightgray}{%
        \begin{codebox}
          \Procname{$\proc{remove}(L,p)$}
          \li $\attrib{p}{prev}.\id{next} \gets \attrib{p}{next}$
          \li $\attrib{p}{next}.\id{prev} \gets \attrib{p}{prev}$
          \li $\attrib{L}{size}\gets \attrib{L}{size}-1$
          \li \Return $p$
        \end{codebox}}

\bigskip

\fcolorbox{white}{Lightgray}{%
        \begin{codebox}
          \Procname{$\proc{insert-after}(L,p,x)$}
          \li $\attrib{x}{prev}\gets \id{p}$
          \li $\attrib{x}{next}\gets \attrib{p}{next}$
          \li $\attrib{p}{next}.\id{prev} \gets \id{x}$
          \li $\attrib{p}{next} \gets \id{x}$
          \li $\attrib{L}{size}\gets \attrib{L}{size}+1$
        \end{codebox}}\hfill~
\end{center}
\end{column}
\end{columns}

Complexité $O(1)$ en temps et $O(n)$ en espace pour $n$ opérations.
%\note{\centerline{\includegraphics[width=8cm]{Figures/04-insertionliste.pdf}}}
\end{frame}

\section{Vecteur}

\begin{frame}{Vecteur}

\begin{itemize}
\item Ensemble dynamique d'objets occupant des rangs entiers
  successifs, permettant la consultation, le remplacement, l'insertion
  et la suppression d'éléments à des rangs arbitraires
\item Interface
\begin{itemize}
\item $\proc{Elem-At-Rank}(V,r)$ retourne l'élément au rang $r$ dans $V$.
\item $\proc{Replace-At-Rank}(V,r,x)$ remplace l'élément situé au rang $r$ par $x$ et retourne cet objet.
\item $\proc{Insert-At-Rank}(V,r,x)$ insère l'élément $x$ au rang $r$, en augmentant le rang des objets suivants.
\item $\proc{Remove-At-Rank}(V,r)$ extrait l'élément situé au rang $r$ et le retire de $r$, en diminuant le rang des objets suivants.
\item $\proc{Vector-Size}(V)$ renvoie la taille du vecteur.
\end{itemize}
\item Applications: tableau dynamique, gestion des éléments d'un menu,\ldots
\item Implémentation: liste liée, tableau extensible\ldots
\end{itemize}

\note{Inconvénient des structures précédentes: il faut parcourir la liste pour retrouver le kième élément ($O(1)$ dans un tableau\\

\bigskip

Avantage: espace mémoire $O(n)$.

\bigskip

Comment combiner les deux ?
}
\end{frame}

\begin{frame}{Implémentation par un tableau extensible}

\begin{itemize}
\item Les éléments sont stockés dans un tableau extensible $\attrib{V}{A}$ de taille initiale $\attrib{V}{c}$.
\item En cas de dépassement, la capacité du tableau est \alert{doublée}.
\item $\attrib{V}{n}$ retient le nombre de composantes.
\item Insertion et suppression:

\begin{center}
\begin{footnotesize}
\fcolorbox{white}{Lightgray}{%
  \begin{codebox}
    \Procname{$\proc{Insert-At-Rank}(V,r,x)$}
    \li \If $\attrib{V}{n}\isequal \attrib{V}{c}$
    \li \Then $\attrib{V}{c}\gets 2\cdot\attrib{V}{c}$
    \li  $W\gets $ ``Tableau de taille $\attrib{V}{c}$''
    \li  \For $i\gets 1 \To n$
    \li \Do $W[i]\gets V.A[i]$ \End
    \li $\attrib{V}{A}\gets W$\End
    \li \For $i\gets \attrib{V}{n} \Downto r$
    \li \Do $\attrib{V}{A}[i+1]\gets \attrib{V}{A}[i]$ \End
    \li $\attrib{V}{A}[r]\gets x$
    \li $\attrib{V}{n}\gets \attrib{V}{n}+1$
  \end{codebox}}
~~~~~
\fcolorbox{white}{Lightgray}{%
  \begin{codebox}
    \Procname{$\proc{Remove-At-Rank}(V,r)$}
    \li $tmp\gets \attrib{V}{A}[r]$
    \li \For $i\gets r \To \attrib{V}{n}-1$ \Do
    \li $\attrib{V}{A}[i]\gets \attrib{V}{A}[i+1]$
    \End
    \li $\attrib{V}{n}\gets \attrib{V}{n}-1$
    \li \Return $tmp$
  \end{codebox}}
\end{footnotesize}
\end{center}


\end{itemize}

\end{frame}

\begin{frame}{Complexité en temps}\label{sec04:amortie}
\begin{itemize}
%\item Accès à un élément, remplacement de sa valeur: $O(1)$
\item $\proc{Insert-At-Rank}$:
\begin{itemize}
\item $O(n)$ pour une opération individuelle, où $n$ est le nombre de composantes du vecteur
\item $O(n^2)$ pour $n$ opérations d'insertion en \alert{début} de vecteur
\item $O(n)$ pour $n$ opérations d'insertion en \alert{fin} de vecteur
\end{itemize}
\item Justification:
\begin{itemize}
\item Si la capacité du tableau passe de $c_0$ à $2^k c_0$ au cours des $n$ opérations, alors le coût des transferts entre tableaux s'élève à
$$c_0+2 c_0+\ldots+2^{k-1} c_0=(2^k-1) c_0.$$
Puisque $2^{k-1} c_0 < n \leq 2^k c_0$, ce coût est \alert{$O(n)$}.
\item On dit que le \alert{coût amorti} par opération est $O(1)$
\item Si on avait élargi le tableau avec un incrément constant $m$, le coût aurait été
$$c_0+(c_0+m)+(c_0+2m)+\ldots+(c_0+(k-1)m)=kc_0+\frac{k(k-1)}{2}m.$$
Puisque $c_0+(k-1)m<n\leq c_0+km$, ce coût aurait donc été \alert{$O(n^2)$}.
\end{itemize}
\end{itemize}
\end{frame}

\begin{frame}{Complexité en temps}
\begin{itemize}
%\item Accès à un élément, remplacement de sa valeur: $O(1)$
\item $\proc{Remove-At-Rank}$:
\begin{itemize}
\item $O(n)$ pour une opération individuelle, où $n$ est le nombre de composantes du vecteur
\item $O(n^2)$ pour $n$ opérations de retrait en \alert{début} de vecteur
\item $O(n)$ pour $n$ opérations de retrait en \alert{fin} de vecteur
\end{itemize}
\item Remarque: Un tableau circulaire permettrait d'améliorer l'efficacité des opérations d'ajout et de retrait en début de vecteur.
\end{itemize}

\end{frame}

\section{File à priorité}

% Implémentation par un tableau, implémentation par un tas.
% Tas de Fibonacci ou binomial pour la fusion de deux tas ?

\begin{frame}{File à priorité}
\begin{itemize}
\item Ensemble dynamique d'objets classés par ordre de \alert{priorité}
\begin{itemize}
\item Permet d'extraire un objet possédant la plus grande priorité
\item En pratique, on représente les priorités par les clés
\item Suppose un ordre total défini sur les clés
\end{itemize}
\item Interface:
\begin{itemize}
\item $\proc{Insert}(S,x)$: insère l'élément $x$ dans $S$.
\item $\proc{Maximum}(S)$: renvoie l'élément de $S$ avec la plus grande clé.
\item $\proc{Extract-Max}(S)$: supprime et renvoie l'élément de $S$ avec la plus grande clé.

\end{itemize}
\item Remarques:
\begin{itemize}
\item Extraire l'élément de clé maximale ou minimale sont des problèmes équivalents
\item La file FIFO est une file à priorité où la clé correspond à l'ordre d'arrivée des élements.
\end{itemize}
\item Application: gestion des jobs sur un ordinateur partagé
\end{itemize}
\note{Ordre total=relation binaire avec les propriétés suivantes:
\begin{itemize}
\item totalité: $\forall p_1, p_2: p_1\leq p_2 \mbox{ ou }p_2\leq p_1$
\item antisymétrie: $\forall p_1,p_2: p_1\leq p_2 \mbox{ et } p_2\leq p_1 \Rightarrow p_1=p_2$
\item Transitivité...
\end{itemize}

\bigskip

File FIFO est un cas particulier ou la priorité est représenté par l'ordre d'arrivé des éléments
}
\end{frame}

\begin{frame}{Implémentations}
\begin{itemize}
\item Implémentation à l'aide d'un tableau statique
\begin{itemize}
\item $Q$ est un tableau statique de taille fixée $\attrib{Q}{length}$.
\item Les éléments de $Q$ sont triés par ordre \alert{croissant} de clés. $\attrib{Q}{top}$ pointe vers le dernier élément.
\item Complexité en temps: extraction en $O(1)$ et insertion en $O(n)$ où $n$ est la taille de la file
\item Complexité en espace: $O(1)$
\end{itemize}
\item Implémentation à l'aide d'une liste liée
\begin{itemize}
\item $Q$ est une liste liée où les éléments sont triés par ordre \alert{décroissant} de clés
\item Complexité en temps: extraction en $O(1)$ et insertion en $O(n)$ où $n$ est la taille de la file
\item Complexité en espace: $O(n)$
\end{itemize}
\item Peut-on faire mieux?
\end{itemize}

{\it Exercice: comment obtenir $O(1)$ pour l'insertion et $O(n)$ pour l'extraction?}
 \note{Leur demander pourquoi on trie le tableau par ordre croissant de clé et pas par ordre décroissant: parce que retirer le premier élément veut dire qu'on doit décaler tous les autres et donc l'opération serait $O(n)$}
\end{frame}

\begin{frame}{Implémentation à l'aide d'un tas}
\begin{itemize}
\item La file est implémentée à l'aide d'un tas(-max) (voir slide \pageref{sec:03tas})
\item Accès et extraction du maximum:

\bigskip

\begin{center}
\begin{small}
\fcolorbox{white}{Lightgray}{
  \begin{codebox}
    \Procname{$\proc{Heap-Maximum}(A)$}
    \li \Return $A[1]$
\end{codebox}}~~~~~\fcolorbox{white}{Lightgray}{
  \begin{codebox}
    \Procname{$\proc{Heap-Extract-Max}(A)$}
    \li \If $\attrib{A}{heap-size}<1$
    \li \Then \Error ``heap underflow''
    \End
    \li $max \gets A[1]$
    \li $A[1]\gets A[\attrib{A}{heap-size}]$
    \li $\attrib{A}{heap-size}\gets \attrib{A}{heap-size}-1$
    \li $\proc{Max-heapify}(A,1)$ \Comment reconstruit le tas
    \li \Return $max$
\end{codebox}}
\end{small}
\end{center}

\bigskip

\item Complexité: $O(1)$ et $O(\log n)$ respectivement (voir chapitre 3)
\end{itemize}

\end{frame}

\begin{frame}{Implémentation à l'aide d'un tas}

\begin{center}
\includegraphics[width=8cm]{Figures/04-fileprio.pdf}
\end{center}

\end{frame}

\begin{frame}{Implémentation à l'aide d'un tas: insertion}
\begin{itemize}
\item $\proc{Increase-Key}(S,x,k)$ augmente la valeur de la clé de $x$
  à $k$ (on suppose que $k\geq$ à la valeur courante de la clé de $x$).

\bigskip

\begin{center}
\begin{small}
\fcolorbox{white}{Lightgray}{
  \begin{codebox}
    \Procname{$\proc{Heap-Increase-Key}(A,i,key)$}
    \li \If $key<A[i]$
    \li \Then \Error ``new key is smaller than current key'' \End
    \li $A[i]\gets key$
    \li \While $i>1$ and $A[\proc{Parent}(i)]<A[i]$
    \li \Do $\id{swap}(A[i],A[\proc{Parent}(i)])$
    \li     $i\gets \proc{Parent}(i)$
    \End
\end{codebox}}
\end{small}
\end{center}

\bigskip

\item Complexité: $O(\log n)$ (la longueur de la branche de la racine à $i$ étant $O(\log n)$ pour un tas de taille $n$).
\end{itemize}
\note{Montrer un exemple sur le slide précédent: augmenter le poids de $A[9]$ de 4 à 15: on échange les noeuds 4 et 9, puis les noeuds 4 et 2.}
\end{frame}

\begin{frame}{Implémentation à l'aide d'un tas: insertion}
\begin{itemize}
\item Pour insérer un élément avec une clé $key$:
\begin{itemize}
\item l'insérer à la dernière position sur le tas avec une clé $-\infty$,
\item augmenter sa clé de $-\infty$ à $key$ en utilisant la procédure précédente
\end{itemize}

\bigskip

\begin{center}
\begin{small}
\fcolorbox{white}{Lightgray}{
  \begin{codebox}
    \Procname{$\proc{Heap-Insert}(A,key)$}
    \li $\attrib{A}{heap-size}\gets \attrib{A}{heap-size}+1$
    \li $A[\attrib{A}{heap-size}]\gets -\infty$
    \li $\proc{Heap-Increase-Key}(A,\attrib{A}{heap-size},key)$
\end{codebox}}
\end{small}
\end{center}

\bigskip

\item Complexité: $O(\log n)$.

\bigskip

\item[$\Rightarrow$] Implémentation d'une file à priorité par un tas: $O(\log n)$ pour l'extraction et l'insertion.
\end{itemize}
\note{}
\end{frame}

\begin{frame}{Ce qu'on a vu}

\begin{itemize}
\item Quelques structures de données classiques et différentes implémentations pour chacune d'elles
\begin{itemize}
\item Pile
\item Liste
\item Files simples, doubles et à priorité
\item Vecteurs
\end{itemize}
\item Structures de type liste liée
\end{itemize}

\end{frame}

\begin{frame}{Ce qu'on n'a pas vu}

\begin{itemize}
\item Structure de type séquence: hybride entre le vecteur et la
  liste
\item Notion d'itérateur
\item Tas binomial: alternative au tas binaire qui permet la fusion
  rapide de deux tas
\item Evolution dynamique de la taille d'un tas (analyse amortie)
\item \ldots
\end{itemize}

\end{frame}


\part{Dictionnaires}

% Superbe site web avec les differents algorithmes

% http://www.sorting-algorithms.com/


% Plan

% recherche séquentielle
% recherche binaire (dichotomique) -> problème insertion reste lente
% binary search trees -> insertion et recherche rapide mais seulement en moyenne
% RB trees -> insertion et recherche rapide dans le pire cas
% table à accès direct: tout en $O(1)$ mais prend beaucoup de mémoire
% table hash: tout en ordre $O(1)$ mais structure non dynamique

\begin{frame}{Plan}

\tableofcontents[hideallsubsections]

\end{frame}

\section{Introduction}

\begin{frame}{Dictionnaires}
\begin{itemize}
\item Définition: un \alert{dictionnaire} est un ensemble dynamique
  d'objets avec des clés comparables qui supportent les opérations
  suivantes:
\begin{itemize}
\item $\proc{Search}(S,k)$ retourne un pointeur $x$ vers un élément dans $S$ tel que $\attrib{x}{key}=k$, ou $\const{NIL}$ si un tel élément n'appartient pas à $S$.
\medskip
\item $\proc{Insert}(S,x)$ insère l'élément $x$ dans le dictionnaire $S$. Si un élément de même clé se trouve déjà dans le dictionnaire, on met à jour sa valeur
\medskip
\item $\proc{Delete}(S,x)$ retire l'élément $x$ de $S$. Ne fait rien si l'élément n'est pas dans le dictionnaire.
\end{itemize}
\medskip
\item Pour faciliter la recherche, on peut supposer qu'il existe un ordre total sur les clés.
\end{itemize}
\end{frame}

\begin{frame}{Dictionnaires}
\begin{itemize}
\item Deux objectifs en général:
\begin{itemize}
\item minimiser le coût pour l'insertion et l'accès au données
\item minimiser l'espace mémoire pour le stockage des données
\end{itemize}
\item Exemples d'applications:
\begin{itemize}
\item Table de symboles dans un compilateur
\item Table de routage d'un DNS
\item \ldots
\end{itemize}
\item Beaucoup d'implémentations possibles
\end{itemize}
\note{Ici:
\begin{itemize}
\item Pas d'ordre sur les clés (du moins, on ne l'exploite pas)
\begin{itemize}
\item tableau à accès direct
\item table de hachage
\end{itemize}
\item Ordre sur les clés
\begin{itemize}
\item tableau trié
\item arbre binaire de recherche
\end{itemize}
\end{itemize}
}
\end{frame}

\begin{frame}{Liste liée}
Première solution:
\begin{itemize}
\item On stocke les paires clé-valeur dans une liste liée
\item Recherche:
\begin{center}
{\footnotesize
\fcolorbox{white}{Lightgray}{%
      \begin{codebox}
        \Procname{$\proc{List-Search}(L,k)$}
        \li $x\gets \attrib{L}{head}$
        \li \While $x\ne \const{NIL}\wedge\attrib{x}{key}\ne k$
        \li \Do $x\gets \attrib{x}{next}$
            \End
        \li \Return $x$
      \end{codebox}}
%% \fcolorbox{white}{Lightgray}{%
%%       \begin{codebox}
%%         \Procname{$\proc{List-Delete}(L,x)$}
%%         \li \If $\attrib{x}{prev}\ne \const{NIL}$
%%         \li \Then $\attrib{x}{prev}.\id{next}\gets \attrib{x}{next}$
%%         \li \Else $\attrib{L}{head}\gets \attrib{x}{next}$\End
%%         \li \If $\attrib{x}{next}\ne \const{NIL}$
%%         \li \Then $\attrib{x}{next}.\id{prev}\gets \attrib{x}{prev}$ \End
%%       \end{codebox}}
}
\end{center}
\item Insertion (resp. Suppression)
\begin{itemize}
\item On recherche la clé dans la liste
\item Si elle existe, on remplace la valeur (resp. on la supprime)
\item Si elle n'existe pas, on la place en début de liste (resp. on ne fait rien)
\end{itemize}
\item Complexité au pire cas\hfill{\it (meilleur cas ?)}
% remplacer ça par une table
\begin{itemize}
\item Insertion: $O(N)$
\item Recherche: $O(N)$
\item Suppression: $O(N)$
\end{itemize}
%\item Peut-on améliorer la recherche ?
\end{itemize}
\end{frame}

\begin{frame}{Vecteur trié}

Deuxième solution:
\begin{itemize}
\item On suppose qu'il existe un ordre total sur les clés
\item On stocke les éléments dans un \alert{vecteur} qu'on maintient trié
\item Recherche dichotomique (approche ``diviser-pour-régner'')

\begin{center}
\begin{small}
\fcolorbox{white}{Lightgray}{%
  \begin{codebox}
    \Procname{$\proc{Binary-Search}(V,k,low,high)$}
    \li \If $low>high$
    \li \Then \Return \const{NIL} \End
    \li $mid\gets \lfloor (low+high)/2\rfloor$
    \li $x\gets \proc{Elem-At-Rank}(V,mid)$
    \li \If $k\isequal x.key$
    \li \Then \Return $x$
    \li \ElseIf $k>x.key$
    \li \Then \Return $\proc{Binary-Search}(V,k,mid+1,high)$
    \li \Else \Return $\proc{Binary-Search}(V,k,low,mid-1)$
    \End
\end{codebox}}
\end{small}
\end{center}
Complexité: $O(\log n)$
\end{itemize}
\note{Comment est-ce que vous feriez l'insertion? comme insertion sort ?}
\end{frame}

\begin{frame}{Vecteur trié}
\begin{itemize}
\item Insertion: recherche de la position par $\proc{Binary-Search}$ puis insertion dans le vecteur par $\proc{Insert-At-Rank}$ (=décalage des éléments vers la droite).
\item Suppression: recherche puis suppression par $\proc{Remove-At-Rank}$ (=décalage des éléments vers la gauche).
\item Complexité au pire cas\hfill{\it (meilleur cas ?)}
\begin{itemize}
\item Insertion: $O(N)$ (on doit décaler les éléments à droite de la clé)
\item Recherche: $O(\log N)$ (recherche dichotomique)
\item Suppression: $O(N)$ (on doit décaler les éléments à gauche de la clé)
\end{itemize}
(Si le vecteur est implémenté par un tableau extensible !)
\end{itemize}

\note{Est-ce que ça ne vous donne pas une autre idée d'algorithme de tri ? Binary insertion sort: complexité ? Intéressant si le coût d'une comparaison est plus grand que le coût d'un swap: example si la comparaison nécessite de faire un calcul couteux (log(n) comparaison dans un cas, n dans l'aute)}

\end{frame}

\begin{frame}{Dictionnaires: jusqu'ici}

  \begin{center}\small
    \def\arraystretch{1.5}\renewcommand{\tabcolsep}{1mm}
    \begin{tabular}{@{}lcccccc@{}}
    &\multicolumn{3}{c}{\emph{Pire cas}} & \multicolumn{3}{c}{\emph{En moyenne}}\\
    \emph{Implémentation}& \proc{Search} & \proc{Insert} & \proc{Delete} & \proc{Search} & \proc{Insert} & \proc{Delete}\\
    \hline\hline
    Liste &$\Theta(n)$&$\Theta(n)$&$\Theta(n)$&$\Theta(n)$&$\Theta(n)$&$\Theta(n)$\\
    \hline
    Vecteur trié&$\Theta(\log n)$&$\Theta(n)$&$\Theta(n)$&$\Theta(\log n)$&$\Theta(n)$&$\Theta(n)$\\
    \hline\hline
  \end{tabular}
  \end{center}

\bigskip

Peut-on obtenir à la fois une insertion et une recherche ``efficaces''?

\end{frame}


\section{Arbres binaires de recherche}

\begin{frame}{Plan}

\tableofcontents[currentsection]

\end{frame}

\subsection{Type de données abstrait pour un arbre}

\begin{frame}{Type de données abstrait pour un arbre}
\begin{itemize}
\item Principe:
\begin{itemize}
\item Des données sont associées aux n\oe uds d'un arbre
\item Les n\oe uds sont accessibles les uns par rapport aux autres selon leur position dans l'arbre
\end{itemize}
\item Interface: Pour un arbre $T$ et un n\oe ud $n$
\begin{itemize}
\item $\proc{Parent}(T,n)$: renvoie le parent d'un n\oe ud $n$ (signale une erreur si $n$ est la racine)
\item $\proc{Children}(T,n)$: renvoie une structure de données (ordonnée
  ou non) contenant les fils du n\oe ud $n$ (exemple: une liste)
\item $\proc{isRoot}(T,n)$: renvoie vrai si $n$ est la racine de l'arbre
\item $\proc{isInternal}(T,n)$: renvoie vrai si $n$ est un n\oe ud interne
\item $\proc{isExternal}(T,n)$: renvoie vrai si $n$ est un n\oe ud externe
\item $\proc{GetData}(T,n)$: renvoie les données associées au n\oe ud $n$
\item $\proc{Left}(T,n)$, $\proc{Right}(T,n)$: renvoie les fils gauche et droit de $n$ (pour un arbre binaire)
\item $\proc{Root}(T)$: renvoie le n\oe ud racine de l'arbre
\item $\proc{Size}(T)$: renvoie le nombre de n\oe uds de l'arbre
\item \ldots
\end{itemize}
\end{itemize}
\end{frame}

\begin{frame}{Exemples d'opération sur un arbre}
\begin{itemize}
\item Calcul de la profondeur d'un n\oe ud

\begin{center}\small
\fcolorbox{white}{Lightgray}{%
  \begin{codebox}
    \Procname{$\proc{Depth-rec}(T,n)$}
    \li \If $\proc{isRoot}(T,n)$
    \li \Then \Return 0\End
    \li \Return $1+\proc{Depth-rec}(T,\proc{Parent}(T,n))$
\end{codebox}}
\end{center}
\item Version itérative

\begin{center}\small
\fcolorbox{white}{Lightgray}{%
  \begin{codebox}
    \Procname{$\proc{Depth-iter}(T,n)$}
    \li $d\gets 0$
    \li \While \textbf{not} $\proc{isRoot}(T,n)$ \Do
    \li $d\gets d+1$
    \li $n\gets \proc{Parent}(T,n)$ \End
    \li \Return $d$
\end{codebox}}
\end{center}
\item Complexité en temps: $O(n)$, où $n$ est la taille de l'arbre (si
  les opérations de l'interface sont $O(1)$)
\end{itemize}
\note{Complexité en espace: $O(n)$ pour la version récursive}
\end{frame}

\begin{frame}{Exemples d'opération sur un arbre}
\begin{itemize}
\item Calcul de la hauteur de l'arbre

\begin{center}\small
\fcolorbox{white}{Lightgray}{%
  \begin{codebox}
    \Procname{$\proc{Height}(T,n)$}
    \li \If $\proc{isExternal}(T,n)$
    \li \Then \Return 0 \End
    \li $h\gets 0$
    \li \For \textbf{each} $n2$ \textbf{in} $\proc{Children}(T,n)$
    \li \Do $h\gets \max(h,\proc{Height}(T,n2))$\End
    \li \Return $h+1$
\end{codebox}}
\end{center}

\item Complexité en temps: $O(n)$, où $n$ est la taille de l'arbre (si
  les opérations de l'interface sont $O(1)$)
\end{itemize}

\end{frame}

\begin{frame}{Implémentation d'un arbre binaire}

Première solution: \alert{numérotation de niveaux}
\begin{itemize}
\item L'arbre est représenté par un vecteur (ou un tableau)
\item Chaque position dans l'arbre est associée à un rang particulier:
\begin{itemize}
\item La racine est en position 1
\item Si un n\oe ud est au rang $r$, son successeur gauche est au rang $2r$, son successeur droit au rang $2r+1$
\end{itemize}
\item Si l'arbre binaire n'est pas un arbre binaire complet, le vecteur contiendra des trous (qu'il faudra pouvoir identifier)
\item Complexité en temps des opérations: $O(1)$
\item Complexité en espace: $O(2^n)$ pour un arbre de $n$ n\oe uds ($\Theta(n)$ pour un arbre binaire complet)
\end{itemize}

\note{Au pire, l'arbre aura une profondeur $h=n$, et donc il faudra pouvoir référencer $O(2^h)=O(2^n)$ éléments

\bigskip

On peut représenter un arbre k-aire avec le même truc. kr, kr+1, etc. pour les fils. Suppose qu'on connaisse le kmax}

\end{frame}

\begin{frame}{Implémentation d'un arbre binaire}

Deuxième solution: \alert{structure liée}
\begin{itemize}
\item Principe: on retient pour chaque n\oe ud $n$ de l'arbre:
\begin{itemize}
\item Un champ de données ($\attrib{n}{data}$)
\item Un pointeur vers son n\oe ud parent ($\attrib{n}{parent}$)
\item Un pointeur vers ses fils gauche et droit ($\attrib{n}{left}$ et $\attrib{n}{right}$)
\end{itemize}
\item Complexité en temps des opérations: $O(1)$
\item Complexité en espace: $\Theta(n)$ pour $n$ n\oe uds
\end{itemize}

\bigskip

{\it (généralise la notion de liste liée)}

\note{Représentation par un tableau (left, right, parent, etc.). Avantage: plus compact mais nécessite un tableau de taille fixée.}
\end{frame}

\begin{frame}{Implémentation d'un arbre quelconque}

\begin{itemize}
\item Même structure liée que pour un arbre binaire
\item Mais on remplace $\attrib{n}{left}$ et $\attrib{n}{right}$ par un pointeur $\attrib{n}{children}$ vers un ensemble dynamique
\item Le type d'ensemble dynamique (vecteur, liste, \ldots) dépendra des opérations devant être effectuées
\end{itemize}

\centerline{\includegraphics[width=9cm]{Figures/05-arbrequelconque.pdf}}

{\it (on peut aussi représenter un arbre quelconque par un arbre binaire)}
\note{Représentation fils-gauche, frère-droit}
\end{frame}

\begin{frame}{Implémentation des arbres binaires}

Implémentation des arbres binaires dans le reste de ce cours:
\begin{columns}
\begin{column}{7.5cm}
\begin{itemize}
\item $T$ représente l'arbre, qui consiste en un ensemble de n\oe uds
\item $T.root$ est le n\oe ud racine de l'arbre $T$
\item N\oe ud $x$
\begin{itemize}
\item $\attrib{x}{parent}$ est le parent du n\oe ud $x$
\item $\attrib{x}{key}$ est la clé stockée au n\oe ud $x$
\item $\attrib{x}{left}$ est le fils de  gauche du n\oe ud $x$
\item $\attrib{x}{right}$ est le fils de droite du n\oe ud $x$
\end{itemize}
\end{itemize}
\end{column}
\begin{column}{4cm}
\begin{center}
\includegraphics[width=3cm]{Figures/05-bstnode.pdf}
\end{center}
\end{column}
\end{columns}

\bigskip
Pour simplifier les notations, nos fonctions seront implémentées directement sur base de cette implémentation (et pas de l'interface générale)

\note{Le passage à une autre implémentation devrait être relativement aisé}

\end{frame}

\begin{frame}{Parcours d'arbres (binaire)}

\begin{itemize}
\item Un parcours d'arbre est une façon d'\alert{ordonner} les n\oe uds d'un arbre afin de les parcourir
\item Différents types de parcours:
\begin{itemize}
\item Parcours en profondeur:
\begin{itemize}
\item Infixe (en ordre)
\item Préfixe (en préordre)
\item Suffixe (en postordre)
\end{itemize}
\item Parcours en largeur
\end{itemize}
\end{itemize}
\note{Parcours en largeur d'un tas ? $\Rightarrow$ on passe simplement séquentiellement sur les éléments}
\end{frame}

\begin{frame}{Parcours infixe}

\begin{center}
\includegraphics[width=5cm]{Figures/05-onebst2.pdf}
\bigskip

$\Rightarrow \langle A, B, D, F, H, K\rangle$
\end{center}

\begin{itemize}
\item Parcours infixe (en ordre): Chaque n\oe ud est visité \alert{après} son fils gauche et \alert{avant} son fils droit

\bigskip
\begin{center}
\begin{small}
\fcolorbox{white}{Lightgray}{%
  \begin{codebox}
    \Procname{$\proc{Inorder-Tree-Walk}(x)$}
    \li \If $x\neq \const{NIL}$
    \li \Then $\proc{Inorder-Tree-Walk}(\attrib{x}{left})$
    \li print $\attrib{x}{key}$
    \li $\proc{Inorder-Tree-Walk}(\attrib{x}{right})$
    \End
\end{codebox}}
\end{small}
\end{center}

\end{itemize}

\end{frame}

\begin{frame}{Parcours préfixe}

\begin{center}
\includegraphics[width=6cm]{Figures/05-onebst2.pdf}
\bigskip

$\Rightarrow \langle F, B, A, D, H, K\rangle$
\end{center}

\begin{itemize}
\item Parcours préfixe (en préordre): chaque n\oe ud est visité \alert{avant} ses fils

\bigskip
\begin{center}
\begin{small}
\fcolorbox{white}{Lightgray}{%
  \begin{codebox}
    \Procname{$\proc{Preorder-Tree-Walk}(x)$}
    \li \If $x\neq \const{NIL}$
    \li \Then print $\attrib{x}{key}$
    \li $\proc{Preorder-Tree-Walk}(\attrib{x}{left})$
    \li $\proc{Preorder-Tree-Walk}(\attrib{x}{right})$
    \End
\end{codebox}}
\end{small}
\end{center}

\end{itemize}

\end{frame}

\begin{frame}{Parcours postfixe}

\begin{center}
\includegraphics[width=6cm]{Figures/05-onebst2.pdf}
\bigskip

$\Rightarrow \langle A, D, B, K, H, F\rangle$
\end{center}

\begin{itemize}
\item Parcours postfixe (en postordre): chaque n\oe ud est visité \alert{après} ses fils

\bigskip
\begin{center}
\begin{small}
\fcolorbox{white}{Lightgray}{%
  \begin{codebox}
    \Procname{$\proc{Postorder-Tree-Walk}(x)$}
    \li \If $x\neq \const{NIL}$
    \li \Then $\proc{Postorder-Tree-Walk}(\attrib{x}{left})$
    \li $\proc{Postorder-Tree-Walk}(\attrib{x}{right})$
    \li print $\attrib{x}{key}$
    \End
\end{codebox}}
\end{small}
\end{center}

\end{itemize}

\end{frame}

\begin{frame}{Complexité des parcours}


Tous les parcours en profondeur sont $\Theta(n)$ en temps
\begin{itemize}
\item Soit $T(n)$ le nombre d'opérations pour un arbre avec $n$ n\oe uds
\item On a $T(n)=\Omega(n)$ (on doit au moins parcourir chaque n\oe ud).
\item Etant donné la récurrence, on a:
$$T(n)\leq T(n_L)+T(n-n_L-1) + d$$ où $n_L$ est le nombre de n\oe uds du sous-arbre à gauche et $d$ une constante
\item On peut prouver par induction que $T(n)< (c+d) n +c$ où $c=T(0)$.
\item $T(n)=\Omega(n)$ et $T(n)=O(n)$ $\Rightarrow$ $T(n)=\Theta(n)$
\end{itemize}

\note{Preuve par induction:
\centerline{\includegraphics[width=10cm]{Figures/05-proofinorder.pdf}}
}
\end{frame}

\begin{frame}{Parcours en largeur}

\begin{itemize}
\item Parcours en largeur: on visite le n\oe ud le plus proche de la racine qui n'a pas déjà été visité. Correspond à une visite de n\oe ud de profondeur 1, puis 2, \ldots.
\item Implémentation à l'aide d'une file en $\Theta(n)$
\end{itemize}

%\bigskip

\begin{columns}
\begin{column}{5cm}
\begin{center}
\includegraphics[width=6cm]{Figures/05-onebst2.pdf}
\bigskip

$\Rightarrow \langle F,B,H,A,D,K\rangle$
\end{center}
\end{column}~~~~~~~
\begin{column}{5cm}
\begin{center}
\begin{footnotesize}
\fcolorbox{white}{Lightgray}{%
  \begin{codebox}
    \Procname{$\proc{Breadth-Tree-Walk}(x)$}
    \li $Q\gets$''Empty queue''
    \li $\proc{Enqueue}(Q,x)$
    \li \While \textbf{not} $\proc{Queue-Empty}(Q)$
    \li \Do $y\gets\proc{Dequeue}(Q)$
    \li print $y.key$
    \li \If $y.left\neq \const{NIL}$
    \li \Then  $\proc{Enqueue}(Q,y.left)$ \End
    \li \If $y.right\neq \const{NIL}$
    \li \Then  $\proc{Enqueue}(Q,y.right)$ \End
    \End
\end{codebox}}
\end{footnotesize}
\end{center}
\end{column}
\end{columns}

\medskip

\emph{(Exercice: Implémenter les parcours en profondeur de manière non récursive)}
\note{Solution: il faut utiliser une pile mais une solution simple peut être trouvée avec une comparaison de pointeurs (voir les solutions du bouquin)

\bigskip

Quid de l'espace: taille max de la file ?? ($O(n/2)$)

}
\end{frame}

\subsection{Arbre binaire de recherche}

\begin{frame}{Plan}

\tableofcontents[currentsection]

\end{frame}


\begin{frame}{Arbres binaires de recherche}

\begin{itemize}
\item Une structure d'arbre binaire implémentant un dictionnaire, avec
  des opérations en $O(h)$ où $h$ est la hauteur de l'arbre


\bigskip

\item Chaque n\oe ud de l'arbre binaire est associé à une clé
\item L'arbre satisfait à la propriété d'arbre binaire de recherche
\begin{itemize}
\item Soient deux n\oe uds $x$ et $y$.
\item Si $y$ est dans le sous-arbre de gauche de $x$, alors $y.key<x.key$
\item Si $y$ est dans le sous-arbre de droite de $x$, alors $y.key\geq x.key$
\end{itemize}
\end{itemize}

\centerline{\includegraphics[width=9cm]{Figures/05-arbresbinaires.pdf}}

\note{Structure qui implémente la recherche binaire qu'on a vue la semaine passée}
\end{frame}

\begin{frame}{Parcours d'un arbre binaire de recherche}

\begin{center}
\includegraphics[width=5cm]{Figures/05-onebst.pdf}
\bigskip

$\Rightarrow \langle 2, 5, 5, 6, 7, 8\rangle$
\end{center}

\begin{itemize}
\item Le parcours infixe d'un arbre binaire de recherche permet d'afficher les clés par ordre croissant

\bigskip
\begin{center}
\begin{small}
\fcolorbox{white}{Lightgray}{%
  \begin{codebox}
    \Procname{$\proc{Inorder-Tree-Walk}(x)$}
    \li \If $x\neq \const{NIL}$
    \li \Then $\proc{Inorder-Tree-Walk}(\attrib{x}{left})$
    \li print $\attrib{x}{key}$
    \li $\proc{Inorder-Tree-Walk}(\attrib{x}{right})$
    \End
\end{codebox}}
\end{small}
\end{center}

\end{itemize}

\end{frame}

\begin{frame}{Recherche dans un arbre binaire}
\begin{itemize}
\item Recherche binaire
\begin{center}
\begin{small}
\fcolorbox{white}{Lightgray}{%
  \begin{codebox}
    \Procname{$\proc{Tree-Search}(x,k)$}
    \li \If $x\isequal \const{NIL}$ or $k\isequal \attrib{x}{key}$
    \li \Then \Return $x$\End
    \li \If $k<\attrib{x}{key}$
    \li \Then \Return $\proc{Tree-Search}(\attrib{x}{left},k)$
    \li \Else \Return $\proc{Tree-Search}(\attrib{x}{right},k)$
\end{codebox}}

\medskip

Appel initial (à partir d'un arbre $T$)\\
\fcolorbox{white}{Lightgray}{%
$\proc{Tree-Search}(\attrib{T}{root},k)$
}
\end{small}
\end{center}

\bigskip

\item Complexité ? $T(n)=O(h)$, où $h$ est la hauteur de l'arbre
\item Pire cas: $h=n$
\end{itemize}

\end{frame}

\begin{frame}{Recherche dans un arbre binaire}
\begin{itemize}
\item $\proc{Tree-Search}$ est récursive terminale.
\item Version itérative
\begin{center}
\begin{small}
\fcolorbox{white}{Lightgray}{%
  \begin{codebox}
    \Procname{$\proc{Iterative-Tree-Search}(T,k)$}
    \li $x\gets \attrib{T}{root}$
    \li \While $x\neq \const{NIL}$ and $k\neq \attrib{x}{key}$
    \li \Do \If $k<\attrib{x}{key}$
    \li \Then $x\gets \attrib{x}{left}$
    \li \Else $x\gets \attrib{x}{right}$
    \End\End
    \li \Return x
\end{codebox}}
\end{small}
\end{center}

\bigskip

\end{itemize}
\note{Invariant: Si $k$ est dans l'arbre, elle se trouve dans le sous-arbre dont la racine est $x$

\bigskip

Faire le lien avec la recherche binaire}

\end{frame}

\begin{frame}{Clés maximale et minimale}
\begin{itemize}
\item Etant donné la propriété d'arbre binaire
\begin{itemize}
\item La clé minimale se trouve dans le n\oe ud le plus à gauche
\item La clé maximale se trouve dans le no\oe ud le plus à droite
\end{itemize}

\bigskip

\begin{center}
\fcolorbox{white}{Lightgray}{%
\begin{codebox}
          \Procname{$\proc{Tree-Minimum}(x)$}
          \li \While $\attrib{x}{left}\ne\const{NIL}$
          \li \Do $x\gets\attrib{x}{left}$
              \End
          \li \Return $x$
        \end{codebox}}~~~~~\fcolorbox{white}{Lightgray}{%
\begin{codebox}
          \Procname{$\proc{Tree-Maximum}(x)$}
          \li \While $\attrib{x}{right}\ne\const{NIL}$
          \li \Do $x\gets\attrib{x}{right}$
              \End
          \li \Return $x$
        \end{codebox}}
\end{center}

\bigskip

\item Complexité: $O(h)$, où $h$ est la hauteur de l'arbre.
\end{itemize}
\end{frame}

\begin{frame}{Successeur et prédécesseur}
\begin{itemize}
\item Etant donné un n\oe ud $x$, trouver le n\oe ud contenant la valeur de clé suivante (dans l'ordre)

\begin{center}
\includegraphics[width=7cm]{Figures/05-treesuccessor.pdf}

\bigskip

Ex: successeur de 15 $\rightarrow 17$, successeur de 4 $\rightarrow 6$.
\end{center}


\item Le successeur de $x$ est le minimum du sous-arbre de droite s'il existe
\item Sinon, c'est le premier ancêtre $a$ de $x$ tel que $x$ tombe dans le sous-arbre de gauche de $a$.
\end{itemize}
\end{frame}

\begin{frame}{Successeur et prédécesseur}

\begin{columns}
\begin{column}{5cm}
\begin{center}
\begin{small}
\fcolorbox{white}{Lightgray}{
\begin{codebox}
      \Procname{$\proc{Tree-Successor}(x)$}
      \li \If $\attrib{x}{right}\ne\const{NIL}$\Then
      \li \Return $\proc{Tree-Minimum}(\attrib{x}{right})$\End
      \li $y\gets\attrib{x}{parent}$
      \li \While $y\ne\const{NIL}$ and $x\isequal\attrib{y}{right}$
      \li \Do $x\gets y$
      \li     $y\gets\attrib{y}{parent}$
      \End
      \li \Return $y$
    \end{codebox}}
\end{small}
\end{center}
\end{column}
\begin{column}{5cm}
\begin{center}

  \bigskip

\bigskip

\includegraphics[width=5cm]{Figures/05-treesuccessor.pdf}
\end{center}
\end{column}
\end{columns}

\bigskip

Complexité: $O(h)$, où $h$ est la hauteur de l'arbre

\bigskip

\emph{(Exercice: $\proc{Tree-Predecessor}$)}

\note{Attention, algo assez tordu}

\end{frame}

\begin{frame}{Insertion}

\begin{center}
\includegraphics[width=8cm]{Figures/05-bstinsertion.pdf}
\end{center}

\begin{itemize}
\item Pour insérer $x$, on recherche la clé $\attrib{x}{key}$ dans l'arbre
\item Si on ne la trouve pas, on l'ajoute à l'endroit où la recherche s'est arrêtée.
\end{itemize}
\end{frame}

\begin{frame}{Insertion}

  \begin{columns}
    \begin{column}{5cm}
\begin{center}
\begin{small}
\fcolorbox{white}{Lightgray}{
      \begin{codebox}
        \Procname{$\proc{Tree-Insert}(T, z)$}
      \li $y\gets\const{NIL}$
      \li $x\gets\attrib{T}{root}$
      \li \While $x\ne\const{nil}$
      \li \Do $y\gets x$
      \li     \If $\attrib{z}{key}<\attrib{x}{key}$
      \li     \Then $x\gets\attrib{x}{left}$
      \li     \Else $x\gets\attrib{x}{right}$
      \End
      \End
      \li $\attrib{z}{parent}\gets y$
      \li \If $y\isequal \const{NIL}$
      \li \Then \Comment Tree $T$ was empty
      \li        $\attrib{T}{root}\gets z$
      \li \ElseIf $\attrib{z}{key}<\attrib{y}{key}$
      \li       \Then $\attrib{y}{left}\gets z$
      \li       \Else $\attrib{y}{right}\gets z$
      \End
      \End
      \end{codebox}}
\end{small}
\end{center}
    \end{column}
    \begin{column}{5cm}
      \begin{center}
      \includegraphics[width=5cm]{Figures/05-bstinsertion.pdf}
      \end{center}

\bigskip

\bigskip

Complexité: $O(h)$ où $h$ est la hauteur de l'arbre
    \end{column}
  \end{columns}

\note{On suppose qu'on insère deux fois une clé qui y serait déjà. Sinon, on peut d'abord la chercher. Si elle est là, on update, sinon, on insère. $x$ trace le chemin, $y$ maintient le pointeur vers le parent de $x$.\\

\bigskip


\centerline{\includegraphics[width=8cm]{Figures/05-deletebstexample.pdf}}
}

\end{frame}

\begin{frame}{Suppression}

3 cas à considérer en fonction du n\oe ud $z$ à supprimer:
\begin{itemize}
\item $z$ n'a pas de fils gauche: remplacer $z$ par son fils droit
\centerline{\includegraphics[width=5cm]{Figures/05-bstdelete-1.pdf}}
\item $z$ n'a pas de fils droit: remplacer $z$ par son fils gauche
\centerline{\includegraphics[width=5cm]{Figures/05-bstdelete-2.pdf}}
\end{itemize}

\note{On pourrait aussi ne rien faire et garder les n\oe uds dans la
  structure et les marquer. Problème: beaucoup de place pour rien

\bigskip

Virer le minimum et le maximum est facile.



}

\end{frame}

\begin{frame}
\begin{itemize}
\item $z$ a deux fils: rechercher le successeur $y$ de $z$.\\\emph{NB: $y$ est dans le sous-arbre de droite et n'a pas de fils gauche.}
\begin{itemize}
\item Si $y$ est le fils droit de $z$, remplacer $z$ par $y$ et conserver le fils droit de $y$
\centerline{\includegraphics[width=5.5cm]{Figures/05-bstdelete-3.pdf}}
\item Sinon, $y$ est dans le sous-arbre droit de $z$ mais n'en est pas la racine. On remplace $y$ par son propre fils droit et on remplace $z$ par $y$.
\centerline{\includegraphics[width=8.5cm]{Figures/05-bstdelete-4.pdf}}
\end{itemize}
\end{itemize}
\note{$y$ n'a pas de fils gauche sinon, le successeur de $z$ se trouverait dans le sous-arbre de gauche qui correspond à des valeurs plus petites que $y$ et plus grande que $z$.
~\\
}
\end{frame}

\begin{frame}{Suppression}
\vspace{-1cm}
\begin{columns}
\begin{column}{6cm}
\begin{center}
\begin{small}
\fcolorbox{white}{Lightgray}{%
  \begin{codebox}
    \Procname{$\proc{Tree-Delete}(T,z)$}
    \li \If $\attrib{z}{left}\isequal \const{NIL}$
    \li \Then $\proc{Transplant}(T,z,z.right)$
    \li \ElseIf $\attrib{z}{right}\isequal\const{NIL}$
    \li \Then $\proc{Transplant}(T,z,z.left)$
    \li \Else \Comment $z$ has two children
    \li $y\gets\proc{Tree-Successor}(z)$
    \li \If $\attrib{y}{parent}\ne z$
    \li \Then $\proc{Transplant}(T,y,y.right)$
    \li $\attrib{y}{right}\gets\attrib{z}{right}$
    \li $\attrib{y}{right}.\id{parent} \gets y$
    \End
    \li \Comment Replace $z$ by $y$
    \li $\proc{Transplant}(T,z,y)$
    \li $\attrib{y}{left}\gets \attrib{z}{left}$
    \li $\attrib{y}{left}.\id{parent}\gets y$
    \End
\end{codebox}}
\end{small}
\end{center}
\end{column}
\begin{column}{6cm}

\bigskip

\bigskip

\begin{center}
\begin{small}
\fcolorbox{white}{Lightgray}{%
  \begin{codebox}
    \Procname{$\proc{Transplant}(T,u,v)$}
    \li\If $\attrib{u}{parent}\isequal \const{NIL}$
    \li\Then $\attrib{T}{root}\gets v$
    \li \ElseIf $u\isequal \attrib{u}{parent}.\id{left}$
    \li \Then $\attrib{u}{parent}.\id{left}\gets v$
    \li \Else $\attrib{u}{parent}.\id{right}\gets v$
    \End
    \li \If $v\ne\const{NIL}$
    \li \Then $\attrib{v}{parent}=\attrib{u}{parent}$
    \End
  \end{codebox}}
\end{small}
\end{center}

\end{column}
\end{columns}

Complexité: $O(h)$ pour un arbre de hauteur $h$\\(Tout est $O(1)$ sauf l'appel à $\proc{Tree-Successor}$).

\note{Transplant: remplace $u$ par $v$ dans $T$. On vérifie qu'il se trouve à gauche ou à la droite de son parent.\\
\bigskip

Pourquoi pas avec le prédécesseur ??? (Parce qu'on peut avoir la même clé que $z$ à droite ($\geq$)}
\end{frame}

\begin{frame}{Arbres binaires de recherche}

\begin{itemize}
\item Toutes les opérations sont $O(h)$ où $h$ est la hauteur de l'arbre
\item Si $n$ éléments ont été insérés dans l'arbre:
\begin{itemize}
\item Au pire, $h=n-1=O(n)$
\begin{itemize}
\item Elements insérés en ordre
\end{itemize}
\item Au mieux, $h=\lceil\log_2 n\rceil=O(\log n)$
\begin{itemize}
\item Pour un arbre binaire complet
\end{itemize}
\item En moyenne, on peut montrer que $h=O(\log n)$
\begin{itemize}
\item En supposant que les éléments ont été insérés en ordre aléatoire
\end{itemize}
\end{itemize}
\end{itemize}

\note{
Autre représentation: externe:

\centerline{\includegraphics[width=8cm]{Figures/03-external_bst.pdf}}
}
\end{frame}

\begin{frame}{Tri avec un arbre binaire de recherche}

\begin{center}
\begin{small}
\fcolorbox{white}{Lightgray}{%
  \begin{codebox}
    \Procname{$\proc{Binary-Search-Tree-Sort}(A)$}
    \li $T\gets$ ``Empty binary search tree''
    \li \For $i\gets 1$ \To $n$
    \li \Do $\proc{Tree-Insert}(T,A[i])$\End
    \li $\proc{Inorder-Tree-Walk}(\attrib{T}{root})$
    \End
  \end{codebox}
  }
\end{small}
\end{center}

\begin{itemize}
\item Exemple: $A=[6,5,7,2,5,8]$

~\hfill\includegraphics[width=5cm]{Figures/05-onebst.pdf}
\item Complexité en temps identique au quicksort
\begin{itemize}
\item Insertion: en moyenne, $n\cdot O(\log n)=O(n\log n)$, pire cas: $O(n^2)$
\item Parcours de l'arbre en ordre: $O(n)$
\item Total: $O(n\log n)$ en moyenne, $O(n^2)$ pour le pire cas
\end{itemize}
\item Complexité en espace cependant plus importante, pour le stockage de la structure d'arbres.
\end{itemize}

\note{File à priorité avec un arbre:
pire cas: $O(N)$ pour l'insertion et l'extraction. Cas moyen: $O(log n)$ dans les deux cas.
}

\end{frame}

\begin{frame}{Dictionnaires: jusqu'ici}

  \begin{center}\small
    \def\arraystretch{1.5}\renewcommand{\tabcolsep}{1mm}
    \begin{tabular}{@{}lcccccc@{}}
    &\multicolumn{3}{c}{\emph{Pire cas}} & \multicolumn{3}{c}{\emph{En moyenne}}\\
    \emph{Implémentation}& \proc{Search} & \proc{Insert} & \proc{Delete} & \proc{Search} & \proc{Insert} & \proc{Delete}\\
    \hline\hline
    Liste &$\Theta(n)$&$\Theta(n)$&$\Theta(n)$&$\Theta(n)$&$\Theta(n)$&$\Theta(n)$\\
    \hline
    Vecteur trié&$\Theta(\log n)$&$\Theta(n)$&$\Theta(n)$&$\Theta(\log n)$&$\Theta(n)$&$\Theta(n)$\\
\hline
ABR&$\Theta(n)$&$\Theta(n)$&$\Theta(n)$&$\Theta(\log n)$&$\Theta(\log n)$&$\Theta(\log n)$\\
    \hline\hline
  \end{tabular}
  \end{center}

\bigskip

\begin{itemize}
\item Peut-on obtenir $\Theta(\log n)$ dans le pire cas? Oui !
\item Deux solutions:
\begin{itemize}
\item Utiliser de la randomisation pour que le probabilité de
  rencontrer le pire cas soit négligeable
\item Maintenir les arbres équilibrés
\end{itemize}
\end{itemize}

\note{
randomization: n'assure pas qu'on ne sera jamais dans le pire cas. Suppose qu'on puisse jouer sur l'ordre d'insertion
}
\end{frame}


\subsection{Arbres équilibrés AVL}

\begin{frame}{Plan}

\tableofcontents[currentsection]

\end{frame}


\begin{frame}{Arbres équilibrés}
\begin{itemize}
\item Solution générale pour obtenir une complexité au pire cas en $O(\log n)$:% maintenir en permanence un arbre plus ou moins complet
\begin{itemize}
\item Définir un \alert{invariant} sur la structure d'arbre
\item Prouver que cet invariant garantit une hauteur $\Theta(\log n)$
\item Implémenter les opérations d'insertion et suppression de manière à maintenir l'invariant
\item Si ces opérations ne sont pas trop coûteuses (p.ex., $O(\log
  n)$), on aura gagné
\end{itemize}

\bigskip

\item Plusieurs types d'arbres équilibrés:
\begin{itemize}
\item \alert{Arbres AVL}
\begin{itemize}
\item \alert{Invariant: Arbres $H$-équilibrés}
\end{itemize}
\item Arbres 2-3-4
\item Arbres rouges et noirs
\item Splay trees, Scapegoat trees, treaps\ldots
\end{itemize}
\end{itemize}
\end{frame}

\begin{frame}{Arbres $H$-équilibrés}
\begin{itemize}
\item \alert{Définition:}

$$T\mbox{ est }H-\mbox{équilibré}\Leftrightarrow |h(g(T'))-h(d(T'))|\leq 1,$$
pour tout sous-arbre $T'$ de $T$, et où $g(X)$, $d(X)$ et $h(X)$ sont resp. le sous-arbre gauche, le sous-arbre droit et la hauteur de l'arbre $X$.

\medskip

\emph{(Les hauteurs des deux sous-arbres d'un même n\oe ud diffèrent au plus de un)}

\bigskip

\item \alert{Propriété:}

Pour tout arbre $H$-équilibré de taille $n$ et de hauteur $h$, on a 
$$h=\Theta(\log n)$$

Plus précisément, on peut prouver:
 $$\log(n+1)\leq h+1< 1,44 \log(n+2)$$
\end{itemize}

\end{frame}

\begin{frame}{Arbres $H$-équilibrés}
\alert{Démonstration}

Etant donné un arbre $H$-équilibré de taille $n$ et de hauteur $h\geq 1$, pour $h$ fixé, $n$ est
\begin{itemize}
\item Maximum: quand l'arbre est complet, soit quand
$n=2^{h+1}-1\Rightarrow n+1\leq 2^{h+1} \Rightarrow \log (n+1)\leq h+1 \Rightarrow h\in\Omega(\log n) $
\item Minimum: quand $n=N(h)$ où $N(h)$ est la taille d'un arbre $H$-équilibré de hauteur $h$ qui a le moins d'éléments.
\begin{itemize}
\item $N(h)$ peut être défini par récurrence par $N(h)=1+N(h-1)+N(h-2)$ avec $N(0)=1$ et $N(1)=2$.
\end{itemize}
\end{itemize}
\centerline{\includegraphics[width=5cm]{Figures/05-avlmaximum.pdf}}

\note{Pour que le nombre n\oe uds soit minimal, il faut que l'arbre soit déséquilibré (sinon, on pourrait enlever des n\oe uds et toujours satisfaire la propriété d'arbre $H$-équilibré)}

\end{frame}

\begin{frame}{~}%{Borne plus simple}
\begin{itemize}
\item[]
\begin{itemize}
\item On a donc\\
\begin{tabular}{cl}
& $N(h)=1+N(h-1)+N(h-2)$\\
$\Rightarrow$ & $N(h)>2 N(h-2)$ (car $N(h-1)>N(h-2)$)\\
$\Rightarrow$ & $N(h)>2^{h/2}$\\
$\Rightarrow$ & $h< 2\log N(h)$\\
\end{tabular}
\item dont on peut tirer que
$h\in O(\log n)$
\end{itemize}
\item On en déduit que
$$h=\Theta(\log n)$$\qed
\end{itemize}

\note{\begin{eqnarray*}
N(h)&<&2 N(h-2)\\
&<& 2^2 N(h-4)\\
&<& 2^3 N(h-6)\\
&<&\ldots\\
&<&2^i N(h-2i)\\
&<&2^(h/2) N(0)=2^(h/2)
\end{eqnarray*}

}
\end{frame}

\begin{frame}{Borne supérieure plus précise}
\begin{itemize}
\item[]
\begin{itemize}
\item En notant $F(h)=N(h)+1$, on a $F(h)=F(h-1)+F(h-2)$ avec $F(0)=2$, $F(1)=3$
\item $F$ est un récurrence de Fibonacci qui a pour solution
 $$F(h)=\frac{1}{\sqrt{5}} (\phi^{h+3}-\phi'^{h+3})\mbox{ avec }\phi=\frac{1+\sqrt{5}}{2}\mbox{ et }\phi'=\frac{1-\sqrt{5}}{2}$$
\item On a $$N(h)+1=\frac{1}{\sqrt{5}} (\phi^{h+3}-\phi'^{h+3})$$
\item ce qui donne
$$n+1\geq \frac{1}{\sqrt{5}} (\phi^{h+3}-\phi'^{h+3})> \frac{1}{\sqrt{5}} (\phi^{h+3}-1)$$
(car $|\phi'|<1$)
\item En prenant le $\log_{\phi}$ des deux membres:
$$h+1<1,44\log (n+2)$$
\end{itemize}
%A VERIFIER
\end{itemize}

\end{frame}

\begin{frame}{Arbres AVL}

\begin{itemize}
\item \alert{Définition:} Un arbre AVL est un arbre binaire de
  recherche $H$-équilibré
\item Inventé par Adelson-Velskii et Landis en 1960
\item Recherche:
\begin{itemize}
\item Par la fonction $\proc{Tree-Search}$ puisque c'est un arbre binaire
\item Complexité $\Theta(\log n)$ étant donné la propriété
\end{itemize}
\item Insertion:
\begin{itemize}
\item On insère l'élément comme dans un arbre binaire classique
\item On vérifie que l'invariant est respecté
\item Si ce n'est pas le cas, on modifie l'arbre
\end{itemize}
\end{itemize}
\end{frame}

\begin{frame}{Rotations}

\centerline{\includegraphics[width=10cm]{Figures/05-rotations.pdf}}

\bigskip

\begin{center}
\fcolorbox{white}{Lightgray}{%
    \begin{codebox}
      \Procname{$\proc{Left-Rotate}(x)$}
      \li $r\gets\attrib{x}{right}$
      \li $\attrib{x}{right}\gets\attrib{r}{left}$
      \li $\attrib{r}{left}\gets x$
      \li \Return $r$
    \end{codebox}}
~~~~~~~~~~~~~~~~~~~
\fcolorbox{white}{Lightgray}{%
    \begin{codebox}
      \Procname{$\proc{Right-Rotate}(x)$}
      \li $l\gets\attrib{x}{left}$
      \li $\attrib{x}{left}\gets\attrib{l}{right}$
      \li $\attrib{l}{right}\gets x$
      \li \Return $l$
    \end{codebox}}
\end{center}

Les rotations maintiennent la propriété d'arbre binaire  de recherche

\note{Deux types d'opération pour maintenir l'équilibre: rotations à gauche et à droite. Implémentée comme sur ce slide. Opération d'ordre $O(1)$.

\bigskip

Donner un exemple: 
\centerline{\includegraphics[width=10cm]{Figures/03-rotation.pdf}}
}
\end{frame}

\begin{frame}{Insertion dans un AVL}

\centerline{\includegraphics[width=10cm]{Figures/05-avlinsertion.pdf}}

\bigskip

\begin{itemize}
\item Insérer le nouvel élément comme dans un arbre binaire de recherche ordinaire
\item L'insertion peut créer un déséquilibre (l'arbre n'est plus $H$-équilibré)
\item Remonter depuis le nouveau n\oe ud jusqu'à la racine en
  restaurant l'équilibre des sous-arbres rencontrés si nécessaire
\end{itemize}

\note{Implémentation récursive:
%voir ici:http://www.enseignement.polytechnique.fr/profs/informatique/Luc.Maranget/421/poly/arbre-bin.html
}

\end{frame}

% Insertion:
% - si l'arbre est équilibré -> pas de risque de déséquilibrage
% - si déséquilibre à gauche ou à droite -> possibilité de violation dans le cas d'une insertion à droite ou à gauche
% - symétrique:
%   deux cas: 1) -> 1 rotation corrige le tir
%             2) -> 2 rotations corrigent le tir

% Implémentation: on doit maintenir la hauteur des noeuds (data augmentation)
% Au plus 2 rotations par insertion: Intuitivement, les opérations du slide précédent font que la hauteur du sous-arbre en $x$ n'est finalement pas augmentée suite à l'insertion. Tous les sous-arbres au dessus de $x$ sont maintenus équilibrés (et $x$ est le premier sous-arbre non $H$-équilibré).

% Deletion: idem mais plus de rotations sont possibles

\begin{frame}{Equilibrage}

\begin{itemize}
\item Soit $x$ le n\oe ud le plus bas violant l'invariant après l'insertion
  \begin{itemize}
  \item Tous ses sous-arbres sont $H$-équilibrés
  \item Il y a une différence d'au plus 2 niveaux entre ses
    sous-arbres gauche et droit
  \end{itemize}
\item Comment rétablir l'équilibre ?
\item Deux cas possibles (selon insertion à droite ou à gauche):
\end{itemize}
\begin{center}
Cas 1\hspace{4cm}Cas 2

\medskip

\includegraphics[width=8cm]{Figures/05-avlcas1-2.pdf}

\medskip
(Déséquilibre à droite)\hspace{1.3cm}(Déséquilibre à gauche)
\end{center}

\note{Tous ses sous-arbres sont $H$-équilibrés puisque c'est le plus bas qui viole l'invariant}

\end{frame}

\begin{frame}{Cas 1: déséquilibre à droite}
\begin{itemize}
\item Deux sous-cas possibles
\end{itemize}

\begin{center}
Cas 1.1\hspace{4cm}Cas 1.2

\medskip

\includegraphics[width=8cm]{Figures/05-avlcas1.pdf}

\medskip
Déséquilibre à l'extérieur\hspace{1.3cm}Déséquilibre à l'intérieur\\
(Cas droite-droite)\hspace{2cm}(Cas droite-gauche)
\end{center}

\bigskip

\emph{(Pourquoi le cas B et C de hauteur $h$ n'est pas possible ?)}

\note{Leur demander pourquoi ce sont les seuls deux cas.

Pourquoi pas B et C de hauteur h ? Parce que sinon, ça voudrait dire que la hauteur du sous-arbre en $y$ n'a pas augmenté (et donc il ne pourrait pas être devenu déséquilibré).

}
\end{frame}

\begin{frame}{Cas 1.1: déséquilibre à droite, extérieur (droite-droite)}
\begin{itemize}
\item Equilibre rétabli par une rotation à gauche de $x$
\end{itemize}

\begin{center}
\includegraphics[width=10cm]{Figures/05-avleqcas11.pdf}
\end{center}

\end{frame}

\begin{frame}{Cas 1.2: déséquilibre à droite, intérieur (droite-gauche)}
\begin{itemize}
\item Une rotation à gauche ne permet pas de rétablir l'équilibre
%\end{itemize}

\begin{center}
\includegraphics[width=10cm]{Figures/05-avlcas12-wrong.pdf}
\end{center}
\item Le sous-arbre $B$ contient au moins un élément (l'élément inséré)
\begin{center}
\includegraphics[width=7cm]{Figures/05-avlcas12-decomp.pdf}
\end{center}

\end{itemize}

\note{Pourquoi pas $B_l$ et $B_r$ de hauteur $h-1$ ? Parce qu'alors l'insertion n'aurait pas modifié la hauteur de $B$}
\end{frame}

\begin{frame}{Cas 1.2: déséquilibre à droite, intérieur (droite-gauche)}
\begin{itemize}
\item Equilibre rétabli par deux rotations
%\end{itemize}

\begin{center}
\includegraphics[width=10cm]{Figures/05-avlcas12-double.pdf}
\end{center}

\end{itemize}

\end{frame}

\begin{frame}{Cas 2: déséquilibre à gauche}
\begin{itemize}
\item Symétrique du cas 1
\item Deux sous-cas possibles
\end{itemize}

\begin{center}
Cas 2.1\hspace{4cm}Cas 2.2

\medskip

\includegraphics[width=8cm]{Figures/05-avlcas2.pdf}

\medskip
Déséquilibre à l'extérieur\hspace{1.3cm}Déséquilibre à l'intérieur\\
(Cas gauche-gauche)\hspace{2cm}(Cas gauche-droite)
\end{center}

\begin{itemize}
\item Résolus respectivement par une rotation (à droite) et une double rotation.
\end{itemize}

\end{frame}

\begin{frame}{Implémentation}
\begin{itemize}
\item Algorithme récursif: Pour insérer une clé dans un arbre $T$:
\begin{itemize}
\item On l'insère (récursivement) dans le sous-arbre approprié (gauche ou droit)
\item Si l'arbre résultant $T$ devient déséquilibré, on effectue une rotation simple ou double selon le cas dans lequel on se trouve
\end{itemize}
\item L'arbre après rééquilibrage étant de la même hauteur qu'avant l'insertion, on n'aura à faire qu'au plus une rotation (simple ou double).
\item L'implémentation est facilitée si on maintient en chaque n\oe ud $x$ un attribut $\attrib{x}{h}$ avec la hauteur du sous-arbre en $x$.
\item Complexité:
\begin{itemize}
\item $O(h)$ où $h$ est la hauteur de l'arbre,
\item c'est-à-dire $O(\log n)$ vu que l'arbre est $H$-équilibré.
\end{itemize}
\end{itemize}
\note{Exemple:
\centerline{\includegraphics[width=5cm]{Figures/03-avlexemple.pdf}}
}
\end{frame}

\begin{frame}{Suppression}
\begin{itemize}
\item Comme pour l'insertion, on doit rétablir l'équilibre suite à la suppression
\item La suppression d'un n\oe ud peut déséquilibrer le parent de ce n\oe ud
\item Contrairement à l'insertion, on peut devoir rééquilibrer plusieurs ancêtres du n\oe ud supprimé.
\item Chaque rotation étant d'ordre $O(1)$, la complexité d'une suppression reste cependant $O(h)$ pour un arbre de hauteur $h$ et donc $O(\log n)$ pour un AVL.
\end{itemize}
\end{frame}

\begin{frame}{Tri avec un AVL}

\begin{itemize}
\item Comme avec un arbre de binaire de recherche ordinaire, on peut trier avec un AVL
\begin{itemize}
\item On insère les éléments successivement dans l'arbre
\item On effectue un parcours en ordre de l'arbre
\end{itemize}
\item Complexité en temps: $\Theta(n\log n)$ (comme pour le tri par tas)
\item Complexité en espace: $\Theta(n)$ (pour la structure d'arbre temporaire) (versus $O(1)$ pour le heap-sort)
%\item Tas: optimisé pour retrouver et supprimer le minimum (ou le max)
%\item AVL: optimisé pour retrouver et supprimer un élément arbitraire 
%\item A ELABORER
\end{itemize}

\end{frame}


%%%%

\begin{frame}{Dictionnaires: jusqu'ici}

  \begin{center}\small
    \def\arraystretch{1.5}\renewcommand{\tabcolsep}{1mm}
    \begin{tabular}{@{}lcccccc@{}}
    &\multicolumn{3}{c}{\emph{Pire cas}} & \multicolumn{3}{c}{\emph{En moyenne}}\\
    \emph{Implémentation}& \proc{Search} & \proc{Insert} & \proc{Delete} & \proc{Search} & \proc{Insert} & \proc{Delete}\\
    \hline\hline
    Liste &$\Theta(n)$&$\Theta(n)$&$\Theta(n)$&$\Theta(n)$&$\Theta(n)$&$\Theta(n)$\\
    \hline
    Vecteur trié&$\Theta(\log n)$&$\Theta(n)$&$\Theta(n)$&$\Theta(\log n)$&$\Theta(n)$&$\Theta(n)$\\
\hline
ABR&$\Theta(n)$&$\Theta(n)$&$\Theta(n)$&$\Theta(\log n)$&$\Theta(\log n)$&$\Theta(\log n)$\\
\hline
AVL&$\Theta(\log n)$&$\Theta(\log n)$&$\Theta(\log n)$&$\Theta(\log n)$&$\Theta(\log n)$&$\Theta(\log n)$\\
    \hline\hline
  \end{tabular}
  \end{center}

\bigskip

\begin{itemize}
\item Peut-on faire mieux?
\item Oui, en changeant radicalement de philosophie
\end{itemize}

\note{Dire que les autres algos d'arbres ont la m\^eme complexit\'e.

\bigskip

Appli: système de réservation: on veut retrouver le prochain (min), }

\end{frame}


\begin{frame}{Demo}
Illustrations:
\begin{small}
\begin{itemize}
\item \url{http://people.ksp.sk/~kuko/bak/}
\item \url{http://www.csi.uottawa.ca/~stan/csi2514/applets/avl/BT.html}
\item \url{http://www.cs.jhu.edu/~goodrich/dsa/trees/avltree.html}
\end{itemize}
\end{small}

\note{Montrer les heap min et max, le BST normal, les AVL et puis peut-être les skip-list

~\bigskip

Pour AVL, demander ce qui va se passer lorsqu'on insère un élément dans l'arbre
}
\end{frame}

\section{Tables de hachage}

\begin{frame}{Plan}

\tableofcontents[currentsection]

\end{frame}

\subsection{Principe}

\begin{frame}{Tableau à accès direct}

\begin{itemize}
\item On suppose:
\begin{itemize}
\item que chaque élément a une clé tirée d'un univers
  $U=\{0,1,\ldots,m-1\}$ où $m$ n'est pas trop grand
\item qu'il ne peut pas y avoir deux éléments avec la même clé.
\end{itemize}
\item Le dictionnaire est implémenté par un tableau $T[0\ldots m-1]$:
\begin{itemize}
\item Chaque position dans la table correspond à une clé de $U$.
\item S'il y a un élément $x$ avec la clé $k$, alors $T[k]$ contient un pointeur vers $x$.
\item Sinon, $T[k]$ est vide ($T[k]=\const{NIL}$).
\end{itemize}
\end{itemize}

\end{frame}

\begin{frame}{Tableau à accès direct}

\centerline{\includegraphics[width=9cm]{Figures/05-directtable.pdf}}

\begin{center}
\begin{small}
\fcolorbox{white}{Lightgray}{%
  \begin{codebox}
    \Procname{$\proc{Direct-Address-Search}(T,k)$}
    \li \Return $T[k]$
\end{codebox}}

\fcolorbox{white}{Lightgray}{%
  \begin{codebox}
    \Procname{$\proc{Direct-Address-Insert}(T,x)$}
    \li \Return $T[\attrib{x}{key}]=x$
\end{codebox}}

\fcolorbox{white}{Lightgray}{%
  \begin{codebox}
    \Procname{$\proc{Direct-Address-Delete}(T,x)$}
    \li \Return $T[\attrib{x}{key}]=\const{NIL}$
\end{codebox}}
\end{small}
\end{center}


\end{frame}

\begin{frame}{Tableau à accès direct}

\begin{itemize}
\item Complexité de toutes les opérations: $O(1)$ (dans tous les cas)
\item Problème:
\begin{itemize}
\item Complexité en espace: $\Theta(|U|)$
\item si l'univers de clés $U$ est grand, stocker une table de taille $|U|$ peut être peu pratique, voire impossible
\end{itemize}
\item Souvent l'ensemble des clés réellement stockées, noté $K$, est petit comparé à $U$ et donc l'espace alloué est gaspillé.

\bigskip\bigskip

\item Comment bénéficier de l'accès rapide d'une table à accès direct avec une table de taille raisonnable ?\\

\medskip

$\Rightarrow$ \alert{Table de hachage:}
\begin{itemize}
\item Réduit le stockage à $\Theta(|K|)$
\item Recherche en $O(1)$ (\alert{en moyenne} !)
\end{itemize}
\end{itemize}
\note{Le prix à payer est qu'on a plus une complexité en $O(1)$ dans le pire cas mais en moyenne. Ce qui n'est pas un problème}
\end{frame}

\begin{frame}{Table de hachage}

\begin{itemize}
\item Inventée en 1953 par Luhn
\item Idée:
\begin{itemize}
\item Utiliser une table $T$ de taille $m\ll|U|$
\item stocker $x$ à la position $h(\attrib{x}{key})$, où $h$ est une fonction de \alert{hachage}: $$h:U\rightarrow \{0,\ldots,m-1\}$$
\end{itemize}
\end{itemize}

\begin{center}
\begin{small}
\fcolorbox{white}{Lightgray}{%
  \begin{codebox}
    \Procname{$\proc{Hash-Insert}(T,x)$}
    \li $T[h(\attrib{x}{key})]\gets x$
\end{codebox}}
~~~~~~~~\fcolorbox{white}{Lightgray}{%
  \begin{codebox}
    \Procname{$\proc{Hash-Delete}(T,x)$}
    \li $T[h(\attrib{x}{key})]\gets \const{NIL}$
\end{codebox}}

\bigskip

\fcolorbox{white}{Lightgray}{%
  \begin{codebox}
    \Procname{$\proc{Hash-Search}(T,x)$}
    \li \Return $T[h(\attrib{x}{key})]$
\end{codebox}}
\end{small}
\end{center}

Est-ce que ces algorithmes sont corrects ?

\end{frame}

\begin{frame}{Table de hachage: collisions}

\centerline{\includegraphics[width=7cm]{Figures/05-hashtable1.pdf}}

\bigskip

\begin{itemize}
\item \alert{Collision:} lorsque deux clés distinctes $k_1$ et $k_2$ sont telles que $h(k_1)=h(k_2)$
\item Cela se produit toujours lorsque le nombre de clés observées est plus grand que la taille du tableau $T$ ($|K|>m$)
\item Très probable, même lorsque la fonction de hachage répartit les clés uniformément $\Rightarrow$ \alert{Paradoxe des anniversaire}
\end{itemize}
\note{Supposons qu'on veuille stocker de l'info sur vous dans un calendrier avec 365 cases}
\end{frame}

\begin{frame}{Paradoxe des anniversaires}

\begin{itemize}
\item Hypothèse:
\begin{itemize}
\item On néglige les années bissextiles
\item Les 365 jours présentent la même probabilité d'être un jour d'anniversaire
\end{itemize}
\item Si $p$ est la probabilité d'une collision d'anniversaires:
$$1-p = \frac{364}{365} \cdot \frac{363}{365} \cdot \frac{362}{365} \ldots \frac{365-(n-1)}{365} = \frac{365!}{(365-n)! 365^n}$$
Exemples:
\begin{itemize}
\item $n=23 \Rightarrow p>0,5$
\item $n=57 \Rightarrow p>0,99$
\item $n=70 \Rightarrow p>0,999$
\end{itemize}
\bigskip

\item Pour une table de hachage:
\begin{itemize}
\item $m=365$ et 57 clés $\Rightarrow$ plus de 99\% de chance de collision
\item $m=1000000$ et 2500 clés $\Rightarrow$ plus de 95\% de chance de collision
\end{itemize}
\end{itemize}
\note{Faire l'essai dans la classe: demander qui est ne en avril-mai}
\end{frame}

\begin{frame}{Collision}

\begin{itemize}
\item Pour éviter les collisions:
\begin{itemize}
\item on veille à utiliser une fonction de hachage qui disperse le
  plus possible les clés vers les différents compartiments.
\item on utilise un nombre de compartiments suffisamment grand
\end{itemize}
Cependant, même dans ce cas, la probabilité de collision peut
  être non négligeable.

\bigskip

\item Deux approches pour prendre en compte les collisions:
\begin{itemize}
\item Le chaînage (adressage fermé)
\item Le sondage (adressage ouvert)
\end{itemize}
\end{itemize}

\note{Voir plus loin pour la première propriété

\bigskip

On va d'abord voir la première solution. On verra ensuite la seconde}

\end{frame}


\begin{frame}{Résolution des collisions par chaînage}

Solution: mettre les éléments qui sont ``hachés'' vers la même
position dans une liste liée (simple ou double)

\bigskip

\centerline{\includegraphics[width=10cm]{Figures/05-hashtable2.pdf}}

\note{Une simple ou une double ? Une liste double est meilleure pour supprimer des éléments}
\end{frame}

\begin{frame}{Implémentation des opérations}

~\bigskip

\begin{center}
\begin{small}
\fcolorbox{white}{Lightgray}{%
  \begin{codebox}
    \Procname{$\proc{Chained-Hash-Insert}(T,x)$}
    \li $\proc{List-Insert}(T[h(\attrib{x}{key})],x)$
\end{codebox}}
~~~~~~~~\fcolorbox{white}{Lightgray}{%
  \begin{codebox}
    \Procname{$\proc{Chained-Hash-Delete}(T,x)$}
    \li $\proc{List-Delete}(T[h(\attrib{x}{key})],x)$
\end{codebox}}

\bigskip

\fcolorbox{white}{Lightgray}{%
  \begin{codebox}
    \Procname{$\proc{Chained-Hash-Search}(T,k)$}
    \li \Return $\proc{List-Search}(T[h(k)],k)$
\end{codebox}}
\end{small}
\end{center}

\bigskip
\bigskip

\begin{itemize}
\item Complexité:
\begin{itemize}
\item Insertion: $O(1)$
\item Suppression: $O(1)$ si liste doublement liée, $O(n)$ pour une liste de taille $n$ si liste simplement liée.
\item Recherche: $O(n)$ si liste de taille $n$.
\end{itemize}
\end{itemize}

\end{frame}

\begin{frame}{Analyse du cas moyen}

%\medskip

\begin{itemize}
\item Recherche d'une clé $k$ dans la table:
\begin{itemize}
\item recherche positive: la clé $k$ se trouve dans la table
\item recherche négative: la clé $k$ n'est pas dans la table
\end{itemize}
\medskip
\item Le \alert{facteur de charge} d'une table de hachage est donné par $\alpha=\frac{n}{m}$ où:
\begin{itemize}
\item $n$ est le nombre d'éléments dans la table
\item $m$ est la taille de la table (c'est-à-dire, le nombre de listes liées)
\end{itemize}
\medskip
\item \alert{hachage uniforme simple}: Pour toute clé $k\in U$,
$$Proba\{h(k)=i\}=\frac{1}{m}, \forall i\in\{0,\ldots,m-1\}$$

\bigskip

\centerline{\includegraphics[width=7cm]{Figures/05-hachageuniforme.pdf}}

\end{itemize}

\end{frame}

\begin{frame}{Analyse du cas moyen}
\begin{itemize}
\item Hypothèses:
\begin{itemize}
\item $h$ produit un hachage uniforme simple
\item le calcul de $h(k)$ est $\Theta(1)$
\item Insertion en début de liste
\end{itemize}
\item $\Rightarrow$ complexités moyennes:
\begin{itemize}
\item recherche négative: $\Theta(1+\alpha)$
\item recherche positive: $\Theta(1+\alpha)$
\end{itemize}
\item Si $n=O(m)$, \hfill {\it ($m$ croît au moins linéairement avec $n$)}, $$\alpha=\frac{O(m)}{m}=O(1)$$
\item Toutes les opérations sont donc $O(1)$ en moyenne
\end{itemize}

\note{
Pourquoi est-ce qu'on n'écrit pas directement $\Theta(\alpha)$ ? Parce que $\alpha=n/m=f(n)$ et pour que l'analyse ait un sens, il faut que $m$ grandisse avec $n$. Donc, on ne sait pas a priori comment $\alpha$ évolue avec $n$ et donc si on peut négliger 1 par rapport à $\alpha$

\bigskip


Si $n=O(m)$, on a $m=\Omega(n)$, ce qui veut dire que la taille de la table croît proportionnellement avec la taille des données. Si elle croît moins que linéairement, lorsque $n$ va croître le alpha va augmenter.}
\end{frame}

\begin{frame}{Analyse du cas moyen: recherche négative}
\begin{itemize}
\item La clé $k$ ne se trouve pas dans la table
\item Par la propriété de hachage uniforme simple, elle a la même probabilité d'être envoyée vers chaque position dans la table.
\item Recherche négative requière le parcours de la liste $T[h(k)]$ complète
\item Cette liste a une longueur moyenne $E[n_{h(k)}]=\alpha$.
\item Le nombre d'éléments à examiner lors d'une recherche négative est donc $\alpha$.
\item En ajoutant le temps de calcul de la fonction de hachage, on arrive à une complexité moyenne $\Theta(1+\alpha)$.
\end{itemize}
\end{frame}


\begin{frame}{Analyse du cas moyen: recherche positive}

\begin{itemize}
\item La clé $k$ se trouve dans la table
\item Supposons qu'elle ait été insérée à la $i$-ième étape (parmi $n$).
\begin{itemize}
\item Le nombre d'éléments à examiner pour trouver la clé est le
  nombre d'éléments insérés à la position $h(k)$ après $k$ plus 1 (la clé $k$ elle-même). 
\item En moyenne, sur les $n-i$ insertions après $k$, il y en aura
  $(n-i)/m$ qui correspondront à la position $h(k)$.
\end{itemize}
\item $k$ ayant pu être insérée à n'importe quelle étape parmi $n$ avec une probabilité $1/n$:
\begin{footnotesize}
$$\sum_{i=1}^n \frac{1}{n} (1+\frac{n-i}{m})=1+\frac{1}{n m}(\sum_{i=1}^n n-\sum_{i=1}^n i)=1+\frac{1}{n m} (n^2-\frac{n(n+1)}{2})=1+\frac{\alpha}{2}-\frac{\alpha}{2n}$$
\end{footnotesize}
\item En tenant compte du coût du hachage, la complexité en moyenne
  est donc $\Theta(2+\alpha/2-\alpha/2n)=\Theta(1+\alpha)$.
\end{itemize}

%% Intuitivement:
%% Supposons qu'on ait inséré $x$ à la $i$th étape (proba de chaque étape est $1/n$.

%% Nombre de comparaison est = au nombre d'éléments inséré après $x$ dans
%% la table, c'est-à-dire $(n-i+1)/m$ vu l'hypothèse d'uniform hashing

%% Proba totale=$\sum_{i=1}^N 1/N (n-i+1)/m$...

\note{ $2+\alpha/2-\alpha/2n\in \Theta(1+\alpha)$ car $2*(1+\alpha/4)<2+\alpha/2-\alpha/2n<2*(1+\alpha)$ pour $n$ grand}

\end{frame}

\subsection{Fonctions de hachage}

\begin{frame}{Plan}

\tableofcontents[currentsection,currentsubsection]

\end{frame}


\begin{frame}{Fonctions de hachage}
\begin{itemize}
\item Idéalement, la fonction de hachage
\begin{itemize}
\item devrait être facile à calculer ($O(1)$)
\item devrait satisfaire l'hypothèse de hachage uniforme simple
\end{itemize}
\item La deuxième propriété est très difficile à assurer en pratique:
\begin{itemize}
\item La distribution des clés est généralement inconnue
\item Les clés peuvent ne pas être indépendantes
\end{itemize}
\item En pratique, on utilise des heuristiques basées sur la nature
  attendue des clés
\item Si toutes les clés sont connues, il existe des algorithmes pour
  construire une fonction de hachage parfaite, sans collision
  (Exemple: le logiciel gperf)
\end{itemize}

\end{frame}

\begin{frame}{Fonctions de hachage: codage préalable}

\begin{itemize}
\item Les fonctions de hachage supposent que les clés sont des nombres naturels
\item Si ce n'est pas le cas, il faut préalablement utiliser une \alert{fonction de codage}
\item Exemple: codage des chaînes de caractères:
\begin{itemize}
\item On interprète la chaîne comme un entier dans une certaine base
\item Exemple pour ``SDA'': valeurs ASCII (128 possibles): $$S=83, D=68, A=65$$
\item Interprété comme l'entier:\hfill {\it (Pourquoi pas 83+68+65 ?)}
 $$(83\cdot 128^2)+(68\cdot 128^1)+(65\cdot 128^0)=1368641$$
\item Calculé efficacement par la méthode de Horner:
$$((83\cdot 128+68)\cdot 128 + 65)$$
\end{itemize}
\end{itemize}
\note{
Pourquoi ne pas prendre la somme des entiers (83+68+65) ? reste la même pour une permutation de l'entrée

\bigskip

Horner: $$C_n x^n+c_{n-1} x^{n-1}+\ldots+C_0 x^0$$
$$(((C_n x+c_{n-1}) x+c_{n-2})x+\ldots C_1)x+C_0$$
}
\end{frame}

\begin{frame}{Méthode de division}

La fonction de hachage calcule le reste de la division entière de la clé par la taille de la table
$$h(k)=k\bmod m.$$
Exemple: $m=20$ et $k=91$ $\Rightarrow h(k)=11$.

\bigskip

\alert{Avantages:} simple et rapide (juste une opération de division)

\bigskip

\alert{Inconvénients:} Le choix de $m$ est très sensible et certaines valeurs doivent être évitées

\bigskip

Exemples:
\begin{itemize}
\item Si $m=2^p$ pour un entier $p$, $h(k)$ ne dépend que des $p$ bits
  les moins significatifs de $k$\\
\begin{itemize}
\item Exemple: ``SDA'' $\bmod\ 128$= ``GAGA''$\bmod\ 128$=65
\end{itemize}
\item Si $k$ est une chaîne de caractères codée en base $2^p$ et $m=2^p-1$, permuter la chaîne ne modifie pas le valeur de hachage\\
\begin{itemize}
\item Exemple: ``SDA''=1368641, ``DSA''=1124801\\
$\Rightarrow 1368641\bmod 127=1124801\bmod 127=89$
\end{itemize}
\end{itemize}
\note{Théorie qui motive les clés de hachage est la théorie des nombres. On va donner le minimum ici.

\bigskip

Pourquoi pas: $k m/k_{max}$ ? Parce qu'on ne connaît pas $k_{max}$

\bigskip

%Dire que si $m$ est premier, alors la valeur hachée est une
%permutation de $\{0,1,\ldots,m-1\}$

On cherche souvent des nombres premiers sous la forme $2^p-1$ car il
existe un test de primalité efficace pour ces nombres. Dans le cas
général, tester la primalité d'une nombre est très complexe.

}
\end{frame}

\begin{frame}{Méthode de division}

\begin{itemize}
\item Si la fonction de hachage produit des séquences périodiques, il vaut mieux choisir $m$ premier
\item En effet, si $m$ est premier avec $b$, on a:
$$\{(a+b\cdot i) \bmod m|i=0,1,2,\ldots\}=\{0,1,2,\ldots,m-1\}$$
\item Exemple: hachage de $\{206, 211, 216, 221,\ldots\}$
\begin{itemize}
\item $m=100$: valeurs hachées possibles: 6, 11,\ldots, 96
\item $m=101$: toutes les entrées sont exploitées
\end{itemize}

\bigskip

\item[$\Rightarrow$] Bonne valeur de $m$: un nombre premier pas trop près d'une puissance exacte de 2
\end{itemize}
\end{frame}

\begin{frame}{Méthode de multiplication}
\begin{itemize}
\item Fonction de hachage:
$$h(k)=\lfloor m\cdot(k A \bmod 1)\rfloor$$
où
\begin{itemize}
%\item $m$ est la taille de la table de hachage
%\item $k$ est la clé
\item $A$ est une constante telle que $0<A<1$.
\item $k A \bmod 1=k A - \lfloor k A \rfloor$ est la partie
  fractionnaire de $kA$.
\end{itemize}
\item Inconvénient: plus lente que la méthode de division
\item Avantage: la valeur de $m$ n'est plus critique
\item La méthode marche mieux pour certaines valeurs de $A$. Par
  exemple:$$A=\frac{\sqrt{5}-1}{2}$$
\end{itemize}
\end{frame}

\begin{frame}{Méthode de multiplication: implémentation}
Calcul aisé si:
\begin{itemize}
\item $m=2^p$ pour un entier $p$
\item Les mots sont codés en $w$ bits et les clés $k$ peuvent être
  codées par un seul mot
\item $A$ de la forme $s/2^w$ pour $0<s<2^w$
\end{itemize}

\centerline{\includegraphics[width=8cm]{Figures/05-hashtablemultiplication.pdf}}

\bigskip

{\small\it Exemple: $m=2^3$, $w=5$ ($\Rightarrow 0<s<2^5$), $s=13$, $A=13/32$ $\Rightarrow h(21)=4$}
\end{frame}

\subsection{Adressage ouvert}

\begin{frame}{Plan}

\tableofcontents[currentsection,currentsubsection]

\end{frame}


\begin{frame}{Adressage ouvert: principe}

\begin{itemize}
\item Alternative au chaînage pour gérer les collisions
\item Tous les éléments sont stockés dans le tableau (pas de listes chaînées)
\item Ne fonctionne que si $\alpha\leq 1$
\item Pour insérer une clé $k$, on \alert{sonde} les cases
  systématiquement à partir de $h(k)$ jusqu'à en trouver une vide.
\item Différentes méthodes en fonction de la stratégie de sondage
\end{itemize}

\end{frame}

\begin{frame}{Adressage ouvert: stratégie de sondage}

On définit une nouvelle fonction de hachage qui dépend de la clé
  et du numéro du sondage:
$$h:U\times\{0,1,\ldots,m-1\} \rightarrow \{0,1,\ldots,m-1\}$$
et qui est telle que
$$\langle h(k,0), h(k,1), \ldots, h(k,m-1)\rangle$$
est une permutation de $\langle 0, 1, \ldots, m-1\rangle$.
%\item La table peut donc être totalement remplie et la suppression est difficile.


\centerline{\includegraphics[width=7cm]{Figures/05-hashopenaddressing.pdf}}

\end{frame}

\begin{frame}{Adressage ouvert: recherche et insertion}

\begin{center}
\fcolorbox{white}{Lightgray}{%
  \begin{codebox}
    \Procname{$\proc{Hash-Search}(T,k)$}
    \li $i\gets 0$
    \li \Repeat
    \li $j\gets h(k,i)$
    \li \If $T[j]\isequal k$
    \li \Then \Return j\End
    \li $i\gets i+1$
    \li \Until $T[j]\isequal \const{NIL}$ or $i\isequal m$
    \li \Return $\const{NIL}$
\end{codebox}}~~~\fcolorbox{white}{Lightgray}{%
  \begin{codebox}
    \Procname{$\proc{Hash-Insert}(T,k)$}
    \li $i\gets 0$
    \li \Repeat
    \li $j\gets h(k,i)$
    \li \If $T[j]\isequal \const{NIL}$
    \li \Then  $T[j]=k$
    \li \Return j
    \li \Else $i\gets i+1$ \End
    \li \Until $i\isequal m$
    \li \Error ``hash table overflow''
\end{codebox}}

\end{center}

\centerline{\includegraphics[width=7cm]{Figures/05-hashopenaddressing.pdf}}

\note{Suppression ??}

\end{frame}

\begin{frame}{Adressage ouvert: suppression}
\begin{itemize}
\item La suppression est possible mais pas aisée
\begin{itemize}
\item On évitera l'utilisation de l'adressage ouvert si on prévoit de nombreuses suppressions de clés dans le dictionnaire
\end{itemize}
\item On ne peut pas naïvement mettre $\const{NIL}$ dans la case contenant la clé $k$ qu'on désire effacer%\hfill (\emph{Pourquoi ?})
\item Solution:
\begin{itemize}
\item Utiliser une valeur spéciale $\const{DELETED}$ au lieu
  de $\const{NIL}$ pour signifier qu'on a effacé une valeur dans cette case
\item Lors d'une recherche: considérer un case contenant
  $\const{DELETED}$ comme une case contenant une clé
\item Lors d'une insertion: considérer une case contenant $\const{DELETED}$ comme une case vide.
\end{itemize}
\item Inconvénient: le temps de recherche ne dépend maintenant plus du facteur de charge $\alpha$ de la table%\hfill(\emph{Pourquoi ?})
\end{itemize}
\note{Demander pourquoi le temps de recherche ne dépend plus du facteur de charge (plutôt du facteur de charge maximum sur la durée de vie de la table)}
\end{frame}

\begin{frame}{Stratégies de sondage}
\begin{itemize}
\item Soit $h_k=\langle h(k,0), h(k,1), \ldots, h(k,m-1)\rangle$ la séquence de sondage correspondant à la clé $k$.
\item Hachage uniforme:
\begin{itemize}
\item chacun des $m!$ permutations de $\langle
  0,1,\ldots,m-1\rangle$ a la même probabilité d'être la séquence de
  sondage d'une clé $k$.
\item Difficile à implémenter.
\end{itemize}
\item En pratique, on se contente d'une garantie que la séquence de
  sondage soit une permutation de $\langle
  0,1,\ldots,m-1\rangle$.

\bigskip

\item Trois techniques pseudo-uniformes:
\begin{itemize}
\item sondage linéaire
\item sondage quadratique
\item double hachage
\end{itemize}
\end{itemize}

\end{frame}

\begin{frame}{Sondage linéaire}

\centerline{\includegraphics[width=7cm]{Figures/05-linearprobing.pdf}}

\bigskip

$$h(k,i)=(h'(k)+i) \bmod m,$$
où $h'(k)$ est une fonction de hachage ordinaire à valeurs dans $\{0,1,\ldots,m-1\}$.

%\bigskip

Propriétés:
\begin{itemize}
\item très facile à implémenter
\item effet de grappe fort: création de longues suites de cellules occupées
\begin{itemize}
\item La probabilité de remplir une cellule vide est $\frac{i+1}{m}$ où $i$ est le nombre de cellules pleines précédant la cellule vide
\end{itemize}
\item pas très uniforme
\end{itemize}

\note{On sonde les positions qui suivent directement la case trouvée en cyclant sur le tableau

\bigskip

longues suites augmentent les temps de calcul pour l'insertion et la recherche: on tombe souvent sur une collision}

\end{frame}

\begin{frame}{Sondage quadratique}

$$h(k,i)=(h'(k)+c_1 i+ c_2 i^2) \bmod m,$$
où $h'$ est une fonction de hachage ordinaire à valeurs dans $\{0,1,\ldots,m-1\}$, $c_1$ et $c_2$ sont deux constantes non nulles.

\bigskip

Propriétés:
\begin{itemize}
\item nécessité de bien choisir les constantes $c_1$ et $c_2$ (pour
  avoir une permutation de $\langle 0,1,\ldots,m-1\rangle$)
\item effet de grappe plus faible mais tout de même existant:
\begin{itemize}
\item  Deux clés de même valeur de hachage suivront le même chemin
$$h(k,0)=h(k',0)\Rightarrow h(k,i)=h(k',i)$$
\end{itemize}
\item meilleur que le sondage linéaire
\end{itemize}

\note{probleme: le saut ne depend pas de la clé $\rightarrow$ on crée des grappes malgré tout. Solution: rendre le saut dépendant de la clé (double hachage)}

\end{frame}

\begin{frame}{Double hachage}

\begin{columns}
\begin{column}{9cm}

$$h(k,i)=(h_1(k)+i h_2(k))\bmod m,$$
où $h_1$ et $h_2$ sont des fonctions de hachage ordinaires à valeurs dans $\{0,1,\ldots,m-1\}$.

\bigskip

Propriétés:
\begin{itemize}
\item difficile à implémenter à cause du choix de $h_1$ et $h_2$ ($h_2(k)$ doit être premier avec $m$ pour avoir une permutation de $\langle 0,1,\ldots,m-1\rangle$).
\item très proche du hachage uniforme
\item bien meilleur que les sondages linéaire et quadratique
\end{itemize}

\bigskip

\emph{Exemple: $h_1(k)=k\bmod 13$, $h_2(k)=1+(k\mod 11)$, insertion de la clé 14}

\end{column}
\begin{column}{2.5cm}
\begin{center}
\includegraphics[width=1.5cm]{Figures/05-doublehachage.pdf}
\end{center}
\end{column}
\end{columns}
\note{Pour la remarque $h_2(k)$ doit être premier avec $m$, voir le slide sur la périodicité}
\end{frame}

\begin{frame}{Adressage ouvert: élément d'analyse}
Pour une table de hachage à adressage ouvert de taille $m$ contenant $n$ éléments ($\alpha=n/m<1$) et en supposant le hachage uniforme
\begin{itemize}
\item Le nombre moyen de sondages pour une recherche négative ou un
  ajout est borné par $\frac{1}{1-\alpha}$
\item Le nombre moyen de sondages pour une recherche positive est borné par $\frac{1}{\alpha} \log \frac{1}{1-\alpha}$
\end{itemize}
%(\emph{pas démontré dans ce cours})

\bigskip

$\Rightarrow$ Si $\alpha$ est constant ($n=O(m)$), la recherche est $O(1)$.
\begin{itemize}
\item Si $\alpha=0.5$, une recherche nécessite en moyenne 2 sondages ($1/(1-0.5)$).
\item Si $\alpha=0.9$, une recherche nécessite en moyenne 10 sondages ($1/(1-0.9)$).
\end{itemize}

\note{Sans démonstration}

\end{frame}

\begin{frame}{Adressage ouvert versus chaînage}
\begin{itemize}
\item Chaînage:
\begin{itemize}
\item Peut gérer un nombre illimité d'éléments et de collisions
\item Performances plus stables
\item Surcoût lié à la gestion et le stockage en mémoire des listes liées
\end{itemize}
\item Adressage ouvert:
\begin{itemize}
\item Rapide et peu gourmand en mémoire
\item Choix de la fonction de hachage plus difficile (pour éviter les grappes)
\item On ne peut pas avoir $n>m$
\item Suppression problématique
\end{itemize}

\bigskip

\item D'autres alternatives existent:
\begin{itemize}
\item Two-probe hashing
\item Cuckoo hashing
\item \ldots
\end{itemize}
\end{itemize}

\note{Cuckoo hashing: on utilise plusieurs fonctions de hachage: si
  collision, on déplace l'élément à une nouvelle position en utilisant
  la fonction de hachage. Si re-collision, on prend la suivante, et
  ainsi de suite. Recherche: on utilise les fonctions de hachage en
  séquence.

\bigskip

Two-probe hashing: deux fonctions de hachage: on hache deux fois en cas de collision et on insère la clé dans la chaîne la plus courte. $\log\log n$ pour la longueur moyenne d'une chaîne.}

\end{frame}

\begin{frame}{Le rehachage}
\begin{itemize}
\item Lorsque $\alpha$ se rapproche de 1, les performances s'effondrent
\item Solution: \alert{rehachage}: création d'une table plus grande
\begin{itemize}
\item allocation d'une nouvelle table
\item détermination d'une nouvelle fonction de hachage, tenant compte du nouveau $m$
\item parcours des entrées de la table originale et insertion dans la nouvelle table
\end{itemize}
\item Si la taille est doublée, le coût asymptotique constant des opérations est conservé (voir slide \pageref{sec04:amortie}).
\end{itemize}

\note{Comment pourrait-on attaquer un système ?}

\end{frame}

\begin{frame}{Universal hashing}
\begin{itemize}
\item Les performances d'un table de hachage se dégrade fortement en
  cas de collisions multiples
\item Connaissant la fonction de hachage, un adversaire malintentionné pourrait s'amuser à entrer des clés créant des collisions. Exemples:
\begin{itemize}
\item Création de fichiers avec des noms bien choisis dans le kernel Linux 2.4.20
\item 28/12/2011: {\scriptsize \url{http://www.securityweek.com/hash-table-collision-attacks-could-trigger-ddos-massive-scale}}
\end{itemize}
\item C'est un exemple d'\alert{attaque par déni de service}

\bigskip

\item Parade: \alert{hachage universel}: choisir la fonction de hachage aléatoirement à chaque création d'une nouvelle instance de la table 
\item Exemple:
$$h(k)=((ak+b) \bmod p)\bmod m,$$
où $p$ est un premier très grand et $a$ et $b$ deux entiers choisis aléatoirement
\end{itemize}

\note{Si il utilise tout le temps la même clé, ça va écraser la valeur et ça ne posera pas de problème

\bigskip

fonction aléatoire $\Rightarrow$ l'utilisateur ne peut pas savoir a priori quelles clés vont créer des collisions
}

\end{frame}

\begin{frame}{Demo}

\begin{small}
\begin{itemize}
\item \url{http://groups.engin.umd.umich.edu/CIS/course.des/cis350/hashing/WEB/HashApplet.htm}
\end{itemize}
\end{small}

\end{frame}

\subsection{Comparaisons}

\begin{frame}{Plan}

\tableofcontents[currentsection,currentsubsection]

\end{frame}

\begin{frame}{Dictionnaires: résumé}

  \begin{center}\small
    \def\arraystretch{1.5}\renewcommand{\tabcolsep}{1mm}
    \begin{tabular}{@{}lcccccc@{}}
    &\multicolumn{3}{c}{\emph{Pire cas}} & \multicolumn{3}{c}{\emph{En moyenne}}\\
    \emph{Implémentation}& \proc{Search} & \proc{Insert} & \proc{Delete} & \proc{Search} & \proc{Insert} & \proc{Delete}\\
    \hline\hline
    Liste &$\Theta(n)$&$\Theta(n)$&$\Theta(n)$&$\Theta(n)$&$\Theta(n)$&$\Theta(n)$\\
    \hline
    Vecteur trié&$\Theta(\log n)$&$\Theta(n)$&$\Theta(n)$&$\Theta(\log n)$&$\Theta(n)$&$\Theta(n)$\\
\hline
ABR&$\Theta(n)$&$\Theta(n)$&$\Theta(n)$&$\Theta(\log n)$&$\Theta(\log n)$&$\Theta(\log n)$\\
\hline
AVL&$\Theta(\log n)$&$\Theta(\log n)$&$\Theta(\log n)$&$\Theta(\log n)$&$\Theta(\log n)$&$\Theta(\log n)$\\
\hline
Table de hachage & $\Theta(n)$&$\Theta(n)$&$\Theta(n)$&$\Theta(1)$&$\Theta(1)$&$\Theta(1)$\\
    \hline\hline
  \end{tabular}
  \end{center}

\begin{itemize}
\item Cas moyen valable uniquement sous l'hypothèse de hachage uniforme
\item Comment obtenir $\Theta(\log n)$ dans le pire cas avec une table de hachage ?
\end{itemize}

\note{Leur demander ici de relever les avantages et inconvénients des arbres et tables de hachage...}
\end{frame}

\begin{frame}{ABR/AVL versus table de hachage}

Tables de hachage:
\begin{itemize}
\item Faciles à implémenter
\item Seule solution pour des clés non ordonnées
\item Accès et insertion très rapides en moyenne (pour des clés simples)
\item Espace gaspillé lorsque $\alpha$ est petit
\item Pas de garantie au pire cas (performances ``instables'')
\end{itemize}

\bigskip

Arbres binaire de recherche (équilibrés):
\begin{itemize}
\item Performance garantie dans tous les cas (stabilité)
\item Taille de structure s'adapte à la taille des données
\item Supportent des opérations supplémentaires lorsque les clés sont ordonnées (parcours en ordre, successeur, prédécesseur, etc.)
\item Accès et insertion plus lente en moyenne
\end{itemize}

\end{frame}


%% \begin{frame}{Applications}

%% Quelles implémentations pour les applications suivantes:
%% \begin{itemize}
%% \item Spotlight
%% \item Itunes song
%% \item IP lookup
%% \item Dictionnaire pour le spell checking
%% \end{itemize}
%% \end{frame}

%% \begin{frame}{Ce qu'on a vu}

%% \begin{itemize}
%% \item D'autres types d'arbres équilibrés: red-black trees, ...
%% \item Fonctions de hachages sophistiquées
%% \item Démonstrations formelles de certaines complexités
%% \end{itemize}

%% \end{frame}

\part{Résolution de problèmes}

% Superbe site web avec les differents algorithmes

% http://www.sorting-algorithms.com/

\begin{frame}{Plan}

\tableofcontents

\end{frame}


\section{Introduction}

\begin{frame}{Introduction}

On va voir 4 manières d'aborder des problèmes nouveaux:
\begin{itemize}
\item Approche naïve/exhaustive
\item Approche gloutonne
\item Programmation dynamique
\item Diviser pour régner
\end{itemize}

\end{frame}

\section{Divide and Conquer}

\begin{frame}{Divide and Conquer}
\end{frame}

\section{Algorithme glouton}

\begin{frame}{Algorithme glouton}
\end{frame}

\section{Programmation dynamique}

\begin{frame}{Programmation dynamique}
\end{frame}


\part{Graphes}

\begin{frame}{Plan}

\tableofcontents

\end{frame}

\section{Définitions}

\begin{frame}{Graphes}

\begin{itemize}
\item Un \alert{graphe (dirigé)}  est un couple $(V,E)$ où:
\begin{itemize}
\item $V$ est un ensemble de n\oe uds ({\it nodes}), ou sommets ({\it vertices}) et
\item $E\subseteq V\times V$ est un ensemble d'arcs, ou arêtes ({\it edges}).
\end{itemize}
\item Un graphe \alert{non dirigé} est caractérisé par une relation symmétrique entre les sommets
\begin{itemize}
\item Une arête est un ensemble $e=\{u,v\}$ de deux sommets
\item On la notera tout de même $(u,v)$ (équivalent à $(v,u)$). 
\end{itemize}

\item Applications: modélisation de:
\begin{itemize}
\item Réseaux sociaux
\item Internet
\item World wide web
\item Cartes routières
\item \ldots
\end{itemize}
\end{itemize}

\end{frame}

\begin{frame}{Exemples}
(mettre des graphes dirigés ou non)
\end{frame}

\begin{frame}{Terminologie}

Graphe non dirigé:
\begin{itemize}
\item Deux n\oe uds sont \alert{adjacents} s'ils sont liés par une même arête
\item Une arête $(v_1,v_2)$ est dite \alert{incidente} aux n\oe uds $v_1$ et $v_2$
\item Le \alert{degré} d'un n\oe ud est égal au nombre de ses arêtes incidentes
\item Le \alert{degrée d'un graphe} est le nombre maximal d'arêtes incidentes à tout sommet.
\item Un graphe est \alert{connexe} s'il existe un chemin de tout sommet à tout autre.
\end{itemize}

\end{frame}

\begin{frame}{Terminologie}

Graphe dirigé:
\begin{itemize}
\item Une arête $(v_1,v_2)$ possède l'\alert{origine} $v_1$ et la \alert{destination}
  $v_2$. Cette arête est \alert{sortante} pour $v_1$ et \alert{entrante} pour $v_2$
\item Le degré \alert{entrant} ({\it in-degree}) et le degré \alert{sortant}
  ({\it out-degree}) d'un n\oe ud $v$ sont respectivement égaux aux nombre d'arêtes entrantes et d'arêtes sortantes de $v$
\item Un graphe est \alert{acyclique} s'il n'y a aucun cycle, c'est-à-dire
  s'il n'est pas possible de suivre les arêtes du graphes à partir
  d'un sommet $x$ et de revenir à ce même sommet $x$
\end{itemize}

\end{frame}

\begin{frame}{Type de graphes}
\begin{itemize}
\item Un graphe est \alert{simple} s'il ne possède pas de boule composées d'une seule arête, c'est-à-dire tel que:
$$\forall v \in V: (v,v)\notin E$$
\item Un \alert{arbre} est un graphe acyclique connexe
\item Un \alert{multigraphe} est une généralisation des graphes pour laquelles
  il est permis de définir plus d'une arête liant un sommet à un autre

\bigskip

\item Un graphe est \alert{pondéré} si les arêtes sont annotées par des \alert{poids}
\begin{itemize}
\item Exemple: réseau entre villes avec comme poids la distance entre
  les villes, réseau internet avec comme poids la bande passante entre routeur, etc.
\end{itemize}
\end{itemize}
\end{frame}

\section{Représentation des graphes}

\begin{frame}{Représentation I: listes d'adjacences}

Un objet $G$ de type graphe est composé:
\begin{itemize}
\item d'une liste de n\oe uds $G.V=\{1,2,\ldots,|V|\}$
\item d'un tableau $G.Adj$ de $|V|$ listes tel que:
\begin{itemize}
\item Chaque sommet $u\in G.V$ est représenté par une élément du tableau $G.Adj$
\item $G.Adj[u]$ est la liste d'adjacence de $u$, c'est-à-dire la
  liste des sommets $v$ tels que $(u,v)\in E$
\end{itemize}
\end{itemize}

\bigskip

Permet de représenter des graphes dirigés ou non
\begin{itemize}
\item Si le graphe est dirigé (resp. non dirigé), la somme des longueurs des listes de $G.Adj$ est 
$|E|$ (resp. $2|E|$).
\end{itemize}

\bigskip

Permet de représenter un graphe pondéré en associant un poids à chaque
élément de liste

\end{frame}

\begin{frame}{Exemple}

Graphe non dirigé
\centerline{\includegraphics[width=8cm]{Figures/07-adjgraphundirected.pdf}}

\bigskip

Graphe dirigé
\centerline{\includegraphics[width=8cm]{Figures/07-adjgraphdirected.pdf}}

\end{frame}

\begin{frame}{Complexités}
\begin{itemize}
\item Complexité en espace: $O(|V|+|E|)$
\begin{itemize}
\item optimal
\end{itemize}
\item Accéder à un sommet: $O(1)$
\begin{itemize}
\item optimal
\end{itemize}
\item Parcourir tous les sommets: $\Theta(|V|)$
\begin{itemize}
\item optimal
\end{itemize}
\item Parcourir toutes les arêtes: $\Theta(|V|+|E|)$
\begin{itemize}
\item ok (mais pas optimal)
\end{itemize}
\item Vérifier l'existence d'une arête $(u,v)\in E$: $O(|V|)$
\begin{itemize}
\item ou encore $O(min(degree(u),degree(v)))$
\item mauvais
\end{itemize}
\end{itemize}
\note{Discuter des opérations d'insertion et de deletion de n\oe uds et d'arêtes}
\end{frame}

\begin{frame}{Réprésentation II: matrice d'adjacence}
\begin{itemize}
\item Les n\oe uds sont les entiers de 1 à $|V|$, $G.V=\{1,2,\ldots,|V|\}$
\item $G$ est décrit par une matrice $G.A$ de dimension $|V|\times |V|$ 
\item $G.A=(a_{ij})$ tel que
\[
a_{ij}=\left\{\begin{array}{ll}
1 & \mbox{si }(i,j)\in E\\
0 & \mbox{sinon}\\
\end{array}\right.
\]
\bigskip

\item Permet de représenter des graphes dirigés ou non
\begin{itemize}
\item $G.A$ est symmétrique si le graphe est non dirigé
\end{itemize}
\item Graphe pondéré: $a_{ij}$ est le poids de l'arête $(i,j)$ si elle existe, NIL (ou 0, ou $+\infty$) sinon
\end{itemize}
\end{frame}

\begin{frame}{Exemple}

Graphe non dirigé
\centerline{\includegraphics[width=8cm]{Figures/07-matgraphundirected.pdf}}

\bigskip

Graphe dirigé
\centerline{\includegraphics[width=8cm]{Figures/07-matgraphdirected.pdf}}

\end{frame}

\begin{frame}{Complexités}
\begin{itemize}
\item Complexité en espace: $O(|V|^2)$
\begin{itemize}
\item potentiellement très mauvais
\end{itemize}
\item Accéder à un sommet: $O(1)$
\begin{itemize}
\item optimal
\end{itemize}
\item Parcourir tous les sommets: $\Theta(|V|)$
\begin{itemize}
\item optimal
\end{itemize}
\item Parcourir toutes les arêtes: $\Theta(|V|^2)$
\begin{itemize}
\item potentiellement très mauvais
\end{itemize}
\item Vérifier l'existence d'une arête $(u,v)\in E$: $O(1)$
\begin{itemize}
\item optimal
\end{itemize}
\end{itemize}
\end{frame}

\begin{frame}{Représentations}
\begin{itemize}
\item Listes d'adjacence:
\begin{itemize}
\item Complexité en espace optimal
\item Mauvais pour des graphes \alert{dense} et des algorithmes qui ont besoin d'accéder aux arêtes
\item Préférable pour des graphes \alert{creux} ou de degré faible
\end{itemize}

\bigskip

\item Matrice d'adjacence:
\begin{itemize}
\item Complexité en espace très mauvaise
\item Bonne pour des algorithmes qui désirent accéder aléatoirement aux arêtes
\item Préférable pour des graphes \alert{denses}
\end{itemize}
\end{itemize}

\end{frame}

\section{Parcours de graphes}

\begin{frame}{Plan}

\tableofcontents[currentsection]

\end{frame}

\begin{frame}{Parcours de graphes}
\begin{itemize}
\item Objectif: parcourir tous les n\oe uds d'un graphe qui sont accessibles à partir d'un n\oe uds $v$ donné
\item Un n\oe ud $v'$ est accessible à partir de $v$ si:
\begin{itemize}
\item soit $v'=v$,
\item soit $v'$ est adjacent à $v$,
\item soit $v'$ est adjacent à un n\oe ud $v''$ qui est accessible à partir de $v$
\end{itemize}

\bigskip

\item Différents types de parcours:
\begin{itemize}
\item En profondeur d'abord ({\it depth-first})
\item En largeur d'abord ({\it breadth-first})
\end{itemize}
\end{itemize}

\end{frame}

\begin{frame}{Parcours en largeur d'abord ({\it breadth-first search})}
\begin{itemize}
\item Un des algorithmes les plus simples pour parcourir un graphe
\item A la base de plusieurs algorithmes de graphe importants%(Dijkstra, Prim...)

\bigskip

\item Entrées: un graphe $G=(V,E)$ et un sommet $s\in V$
\begin{itemize}
\item Parcourt le graphe en ``touchant'' tous les sommets qui sont accessibles à partir de $s$
\item Parcourt les sommets par ordre croissant de leur distance (en
  nombre minimum d'arêtes) par rapport à $s$
\item Calcule pour chaque sommet $v\in V$ sa distance $v.d$ à $s$
%\item Produit un arbre {\it en profondeur d'abord} ayant pour racine $s$
\item Fonctionne aussi bien pour des graphes dirigés que non dirigés
\end{itemize}
\end{itemize}

\end{frame}

\begin{frame}{Exemple}

\centerline{\includegraphics[width=9cm]{Figures/07-breadth-first-graph.pdf}}

\end{frame}

\begin{frame}{Parcours en largeur d'abord: Algorithme}

\begin{columns}
\begin{column}{6cm}
\begin{center}
{\small
\fcolorbox{white}{Lightgray}{%
      \begin{codebox}
        \Procname{$\proc{BFS}(G,s)$}
        \li \For each vertex $u \in G.V\setminus \{s\}$
        \li \Do $u.d=\infty$\End
        \li $s.d=0$
        \li $Q=\emptyset$
        \li $\proc{Enqueue}(Q,s)$
        \li \While $Q\neq \emptyset$
        \li \Do $u=\proc{Dequeue}(Q)$
        \li \For each $v\in G.Adj[u]$
        \li\Do \If $v.d=\infty$
        \li \Then $v.d=u.d+1$
        \li $\proc{Enqueue}(Q,v)$\End\End\End
        %% \Procname{$\proc{BFS}(G,s)$}
        %% \li \For each vertex $u \in G.V\setminus \{s\}$
        %% \li \Do $u.color=\const{White}$
        %% \li $u.d=\infty$
        %% \li $u.\pi=\const{NIL}$\End
        %% \li $s.color=\const{Gray}$
        %% \li $s.d=0$
        %% \li $s.\pi=\const{NIL}$
        %% \li $Q=\emptyset$
        %% \li $\proc{Enqueue}(Q,s)$
        %% \li \While $Q\neq \emptyset$
        %% \li \Do $u=\proc{Dequeue}(Q)$
        %% \li \For each $v\in G.Adj[u]$
        %% \li\Do \If $v.color\isequal \const{White}$
        %% \li \Then $v.color=\const{Gray}$
        %% \li $v.d=u.d+1$
        %% \li $v.\pi = u$
        %% \li $\proc{Enqueue}(Q,v)$\End\End
        %% \li $u.color=\const{Black}$\End
      \end{codebox}}
}
\end{center}
\end{column}
\begin{column}{5cm}
\begin{itemize}
\item $Q$ est une file (LIFO) qui contient les n\oe uds ``touchés'' mais pas encore visités
\item Le calcul de l'attribut $d$ n'est pas nécessaire (on peut le
  remplacer par un drapeau binaire)
\end{itemize}
\end{column}
\end{columns}

\end{frame}


\begin{frame}{Parcours en largeur d'abord: Complexité}

\begin{columns}
\begin{column}{6cm}
\begin{center}
{\small\vspace{-0.3cm}
\fcolorbox{white}{Lightgray}{%
      \begin{codebox}
        \Procname{$\proc{BFS}(G,s)$}
        \li \For each vertex $u \in G.V\setminus \{s\}$
        \li \Do $u.d=\infty$\End
        \li $s.d=0$
        \li $Q=\emptyset$
        \li $\proc{Enqueue}(Q,s)$
        \li \While $Q\neq \emptyset$
        \li \Do $u=\proc{Dequeue}(Q)$
        \li \For each $v\in G.Adj[u]$
        \li\Do \If $v.d=\infty$
        \li \Then $v.d=u.d+1$
        \li $\proc{Enqueue}(Q,v)$\End\End\End
        %% \Procname{$\proc{BFS}(G,s)$}
        %% \li \For each vertex $u \in G.V\setminus \{s\}$
        %% \li \Do $u.color=\const{White}$
        %% \li $u.d=\infty$
        %% \li $u.\pi=\const{NIL}$\End
        %% \li $s.color=\const{Gray}$
        %% \li $s.d=0$
        %% \li $s.\pi=\const{NIL}$
        %% \li $Q=\emptyset$
        %% \li $\proc{Enqueue}(Q,s)$
        %% \li \While $Q\neq \emptyset$
        %% \li \Do $u=\proc{Dequeue}(Q)$
        %% \li \For each $v\in G.Adj[u]$
        %% \li\Do \If $v.color\isequal \const{White}$
        %% \li \Then $v.color=\const{Gray}$
        %% \li $v.d=u.d+1$
        %% \li $v.\pi = u$
        %% \li $\proc{Enqueue}(Q,v)$\End\End
        %% \li $u.color=\const{Black}$\End
      \end{codebox}}
}
\end{center}
\end{column}
\begin{column}{5cm}
\begin{itemize}
\item Chaque sommet est enfilé au plus une fois
  ($v.d infini \rightarrow v.d$ fini)
\item Boucle $\While$ exécutée $O(|V|)$ fois
\item Boucle interne: $O(|E|)$ \alert{au total}
\item Au total: $O(|V|+|E|)$
\end{itemize}
\end{column}
\end{columns}

\end{frame}

\begin{frame}{Parcours en largeur d'abord}

\begin{itemize}
\item Correction:
\begin{itemize}
\item L'algorithme fait bien un parcours du graphe en largeur et $v.d$ contient bien la distance minimale de $s$ à $v$
\item Pas évident à montrer. On le fera plus loin pour l'algorithme de Dijkstra (calcul du plus court chemin)
\end{itemize}

\bigskip

\item Applications:
\begin{itemize}
\item Calcul des plus courtes distances d'un sommet à tous les autres
\item Recherche du plus court chemin entre deux sommets
\item Calcul du diamètre d'un arbre
\item Tester si un graphe est biparti
\item \ldots
\end{itemize}
\end{itemize}
\end{frame}


\begin{frame}{Parcours en profondeur d'abord}

\begin{itemize}
\item Parcours du graphe en profondeur:
\begin{itemize}
\item On suit immédiatement les arêtes incidentes au dernier sommet visité
\begin{itemize}
\item Au lieu de les placer dans une file comme dans le parcours en largeur
\end{itemize}
\item On revient en arrière quand le sommet visité n'a plus de sommets adjacents non visités
\end{itemize}

\bigskip

\item Entrée: un graphe $G=(V,E)$ (pas de sommet source !)
\item Sortie: 2 ``dates'' associées à chaque sommet $v$:
\begin{itemize}
\item $v.d$=début du traitement du sommet $v$ (découverte du sommet)
\item $v.f$=fin du traitement du sommet $v$
\end{itemize}
\end{itemize}

\note{On ne voit pas un algo qui parcourt le graphe comme le
  bread-first parce que l'algo ici sera utile pour d'autres
  applications. Notamment le tri topologique}
\end{frame}

\begin{frame}{Exemple}

\centerline{\includegraphics[width=10cm]{Figures/07-dfs-exemple-onenode.pdf}}

\bigskip

Parcours en profondeur à partir de $A$: $A$-$D$-$F$-$G$-$B$-$E$ ($C$ et $H$ pas accessibles)

\end{frame}

\begin{frame}{Parcours en profondeur d'abord: implémentation}

\begin{columns}
\begin{column}{5cm}
\begin{center}
{\small\vspace{-0.3cm}
\fcolorbox{white}{Lightgray}{%
      \begin{codebox}
        \Procname{$\proc{DFS}(G,s)$}
        \li \For each vertex $u \in G.V\setminus \{s\}$
        \li \Do $u.color=\const{White}$\End
        \li $s.color=\const{Black}$        
        \li $S=\emptyset$
        \li $\proc{Push}(S,s)$
        \li \While $S\neq \emptyset$
        \li \Do $u=\proc{Pop}(S)$
        \li \For each $v\in G.Adj[u]$
        \li\Do \If $v.color=\const{White}$
        \li \Then $v.color=\const{Black}$
        \li $\proc{Push}(S,v)$\End\End\End
      \end{codebox}}
}
\end{center}
\end{column}
\begin{column}{5cm}
\begin{itemize}
\item On remplace la file $Q$ par une pile $S$
\item Chaque sommet est visité au plus une fois ($v.color$ blanc
  $\rightarrow$ noir)
\item Boucle $\While$ executées $O(|V|)$ fois
\item Boucle interne: $O(|E|)$ \alert{au total}
\item Complexité: $O(|V|+|E|)$
\end{itemize}
\end{column}
\end{columns}

\end{frame}

\begin{frame}{Parcours en profondeur d'abord: implémentation}

\begin{columns}
\begin{column}{5cm}
\begin{center}
{\small
\fcolorbox{white}{Lightgray}{%
      \begin{codebox}
        \Procname{$\proc{DFS}(G)$}
        \li \For each vertex $u \in G.V$
        \li \Do $u.color=\const{White}$\End
        \li $time=0$ \Comment global variable
        \li \For each $u\in G.V$
        \li  \Do \If $u.color\isequal \const{White}$
        \li   \Then $\proc{DFS-Visit}(G,u)$\End\End
      \end{codebox}}
}
\end{center}
\end{column}
\begin{column}{5cm}
\begin{center}
{\small
\fcolorbox{white}{Lightgray}{%
      \begin{codebox}
        \Procname{$\proc{DFS-Visit}(G,u)$}
        \li $time=time+1$
        \li $u.d=time$
        \li $u.color=\const{Gray}$
        \li \For each $v\in G.Adj[u]$
        \li \Do \If $v.color\isequal \const{White}$
        \li \Then $\proc{DFS-Visit}(G,v)$\End\End
        \li $u.color=\const{Black}$
        \li $time = time + 1$
        \li $u.f=time$
      \end{codebox}}
}
\end{center}
\end{column}
\end{columns}

\bigskip

L'attribut $color$ permet de marquer les sommets déjà découverts:
\begin{itemize}
\item $\const{White}$: sommets non découverts
\item $\const{Gray}$: sommets découverts mais pas encore traités complètement
\item $\const{Black}$: sommets traités complètement
\end{itemize}
(la couleur $\const{Gray}$ n'est pas nécessaire)

\end{frame}

\begin{frame}{Exemple}

\centerline{\includegraphics[width=10cm]{Figures/07-dfs-exemple.pdf}}

\end{frame}

\begin{frame}{Parcours en profondeur d'abord: complexité}

\begin{columns}
\begin{column}{5cm}
\begin{center}
{\small
\fcolorbox{white}{Lightgray}{%
      \begin{codebox}
        \Procname{$\proc{DFS}(G)$}
        \li \For each vertex $u \in G.V$
        \li \Do $u.color=\const{White}$\End
        \li $time=0$ \Comment global variable
        \li \For each $u\in G.V$
        \li  \Do \If $u.color\isequal \const{White}$
        \li   \Then $\proc{DFS-Visit}(G,u)$\End\End
      \end{codebox}}
}
\end{center}
\end{column}
\begin{column}{5cm}
\begin{center}
{\small
\fcolorbox{white}{Lightgray}{%
      \begin{codebox}
        \Procname{$\proc{DFS-Visit}(G,u)$}
        \li $time=time+1$
        \li $u.d=time$
        \li $u.color=\const{Gray}$
        \li \For each $v\in G.Adj[u]$
        \li \Do \If $v.color\isequal \const{White}$
        \li \Then $\proc{DFS-Visit}(G,v)$\End\End
        \li $u.color=\const{Black}$
        \li $time = time + 1$
        \li $u.f=time$
      \end{codebox}}
}
\end{center}
\end{column}
\end{columns}

\bigskip

\begin{itemize}
\item Boucle lignes 4-6 de $\proc{DFS-Visit}(G,u)$: $\Theta(out-degree(u))$
\item $\proc{DFS-Visit}(G,u)$ est appelé une seule fois pour chaque sommet
\begin{itemize}
\item On l'appelle sur un sommet blanc uniquement et on le marque gris directement après l'appel
\end{itemize}
\item Complexité globale: $\Theta(|V|+|E|)$
\end{itemize}

\note{Pourquoi $\Theta$ ? Parce que l'algorithme parcourt tout le graphe contrairement au breadth-first}

\end{frame}

\begin{frame}{Application: tri topologique}
\begin{itemize}
\item Tri topologique:
\begin{itemize}
\item Etant donné un \alert{graphe acyclique dirigé} (DAG), trouver un
  ordre des sommets tel qu'il n'y ait pas d'arête d'un n\oe ud vers un
  des n\oe uds qui le précède dans l'ordre
\item C'est toujours possible si le graphe est acyclique
\end{itemize}

\bigskip

\item Exemples d'applications:
\begin{itemize}
\item Trouver un ordre pour suivre un ensemble de cours qui tienne compte des prérequis de chaque cours
\begin{itemize}
\item Pour suivre SDA, il faut avoir suivi Introduction à la programmation
\end{itemize}
\item Résoudre les dépendances pour l'installation de logiciels
\begin{itemize}
\item Trouver un ordre d'installation de manière à ce que chaque logiciel soit installé après tous ceux dont il dépend
\end{itemize}
\end{itemize}
\end{itemize}

\end{frame}

\begin{frame}{Illustration}
Graphe

\centerline{\includegraphics[width=8cm]{Figures/07-tritopo-exemple.pdf}}

\bigskip

Une tri topologique

\centerline{\includegraphics[width=10cm]{Figures/07-tritopo-exemple-solution.pdf}}

\end{frame}

\begin{frame}{Algorithme}

Deux solutions:
\begin{itemize}
\item En utilisant le parcours en profondeur
\begin{center}
{\small
\fcolorbox{white}{Lightgray}{%
      \begin{codebox}
        \Procname{$\proc{Topological-sort}(G)$}
        \li $\proc{DFS}(G)$
        \li Return nodes in $G.V$ in reverse order of $v.f$
        \end{codebox}}
}
\end{center}
\bigskip

\item Une solution gloutonne:
\begin{itemize}
\item Rechercher un sommet qui n'a pas d'arête entrante 
\item Ajouter ce sommet à un tri topologique du graphe dont on a retiré ce sommet et toute ses arêtes
\end{itemize}

\bigskip

\item Complexité dans les deux cas: $\Theta(|E|+|V|)$
\end{itemize}

\end{frame}

\begin{frame}{Illustration}

Calcul de $v.d$ et $v.f$ sur l'exemple illustratif

\centerline{\includegraphics[width=8cm]{Figures/07-tritopo-annotated.pdf}}

Tri selon $v.f$

\centerline{\includegraphics[width=10cm]{Figures/07-tritopo-annotated-2.pdf}}

\end{frame}

\section{Plus courts chemins}

\begin{frame}{Plan}

\tableofcontents[currentsection]

\end{frame}

\begin{frame}{Plus courts chemins}

\end{frame}

\section{Arbre de couverture}

\begin{frame}{Plan}

\tableofcontents[currentsection]

\end{frame}

\begin{frame}{Arbre de courverture}
\end{frame}



\end{document}
