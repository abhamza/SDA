
\part{Outils d'analyse}

\begin{frame}{Plan}

\begin{itemize}
\item Correction d'un algorithme
\item Complexité d'un algorithme
\end{itemize}

\end{frame}

\section{Correction d'un algorithme}

\subsection{Algorithme itératif}

\begin{frame}{Analyse d'algorithmes}

Questions à se poser lors de la définition d'un algorithme:
\begin{itemize}
\item Mon algorithme est-il correct ?
\item Mon algorithme est-il efficace ? %en termes d'utilisation des
%  resources, temps CPU et/ou espace mémoire ?
\end{itemize}

\bigskip

Autre question importante seulement marginalement abordée dans ce cours:
\begin{itemize}
\item Modularité,fonctionnalité, robustesse, facilité d'utilisation, temps
  de programmation, simplicité, extensibilité, fiabilité,
  existence d'une solution algorithmique
\end{itemize}

\end{frame}

\begin{frame}{Correction d'un algorithme}%, complétude, terminaison}

\begin{itemize}
\item La correction d'un algorithme s'étudie par rapport à un problème donné
\item Un problème est une collection d'instances de ce problème.
\begin{itemize}
\item Exemple de problème: trier un tableau
\item Exemple d'instance de ce problème: trier le tableau $[8,4,15,3]$
\end{itemize}
\item Un algorithme est correct pour une instance d'un problème s'il
  produit une solution correcte pour cette intance
\item Un algorithme est correct pour un problème s'il est correct pour
  toutes ses instances
\item On s'intéressera ici à la correction d'un algorithme pour un
  problème, par pour seulement certaines de ses instances
\end{itemize}

\note{}

\end{frame}

\begin{frame}{Comment vérifier la correction?}
\begin{itemize}
\item Première solution: en testant concrètement l'algorithme:
\begin{itemize}
\item Suppose d'implémenter l'algorithme dans un langage (programme)
  et de le faire tourner
\item Supppose qu'on peut déterminer les instances du problème à vérifier
\item Il est très difficile de prouver empiriquement qu'on n'a pas de bug %On peut prouver qu'il y a un bug, pas qu'il n'y en a pas.
\end{itemize}
\item Deuxième solution: en dérivant une preuve mathématique formelle:
\begin{itemize}
\item Pas besoin d'implémenter et de tester toutes les instances du problème
\item Sujet à des ``bugs'' également
\end{itemize}
\item En pratique, on combinera les deux

\bigskip

\item Preuves de correction:
\begin{itemize}
\item Précondition, post-conditions
\item Algorithmes itératifs: assertions, invariants de boucle et induction
\item Algorithmes récursifs: preuve par induction directement
\end{itemize}
\end{itemize}
\end{frame}

\begin{frame}{Correction: cas itératif}

Pour prouver qu'un algorithme itératif est correct:
\begin{itemize}
\item On analyse chaque boucle de l'algorithme séparément, en
  démarrant avec la boucle la plus interne s'il y a plusieurs
  boucles imbriquées.
\item Pour chaque boucle, on met en évidence un invariant de boucle
\begin{itemize}
\item Ensemble de propriétés qui relient les variables du programme
\item Ces propriétés doivent être vraies avant, pendant et après la boucle
\end{itemize}
\item On prouve que l'invariant est vérifié.
\item On utilise l'invariant pour prouver que l'algorithme se termine.
\item On utilise l'invariant pour prouver que l'algorithme calcule le résultat correct.
\end{itemize}

\end{frame}

\begin{frame}{Assertion}

\begin{itemize}
\item Relation entre les variables qui est vraies à un moment donné dans l'exécution
\item Assertions particulières:
\begin{itemize}
\item Pre-condition $P$: conditions que doivent remplir les entrées de l'algorithme
\item Post-condition $Q$: conditions qui expriment que le résultat de l'algorithme est correcte
%\item Invariant: condition que doivent remplir
\end{itemize}
\item $P$ et $Q$ définissent les instances et solutions valides du problème
\item Un algorithme est correct si $P$ \{code\} $Q$ est vrai.
\item Pour vérifier la correction d'une boucle, on introduit la notion d'invariant.%On suppose $P$ vérifié, on montre que $I$ est un invariant valide et on en déduit que $R$ est vrai également.
\end{itemize}

\end{frame}

\begin{frame}{Invariant}
\begin{itemize}
\item \alert{Invariant:} Une assertion qui définit ce qui est vrai avant chaque itération de la boucle
\item \alert{Initialisation:} Prouver que l'invariant est vrai avant la
  première itération (sous l'hypothèse que la pré-condition $P$ est
  vérifiée)
\item \alert{Maintenance:} Prouver que si l'invariant est vrai avant la
  $i$-ième itération, il l'est également après celle-ci, et par
  conséquent aussi avant la $i+1$-ième itération
\item \alert{Terminaison:} Prouver que si l'invariant est vrai après la
  dernière itération, la post-condition $Q$ est vérifiée
\end{itemize}

\centerline{\includegraphics[width=8cm]{Figures/02-invariant.pdf}}

\end{frame}

\begin{frame}{Mise en \oe uvre}
\begin{itemize}
\item Trouver un invariant de boucle une fois l'algorithme mis au point peut être assez complexe
\item Idéalement, l'invariant devrait être la propriété clé qui définit l'algorithme
\item Dans la suite du cours, on ne fournira un invariant que dans quelques cas
\end{itemize}
\end{frame}
\begin{frame}{Exemples: exponentielle}
\end{frame}

\begin{frame}{Exemples: fibonacci itératif}
\end{frame}

\begin{frame}{Exemples: insertion sort}
\end{frame}

\begin{frame}{Preuve de terminaison d'un algorithme itératif}

Pour chaque boucle: on définit une fonction de terminaison $t$ qui est telle que $t$ décroit strictement à chaque itération de la boucle et que sa valeur est bornée vers le bas par le gardien de la boucle.

\end{frame}

\begin{frame}{Correction: cas récursif}
\begin{itemize}
\item Preuve par induction: donner le schéma général
\end{itemize}
\end{frame}

\begin{frame}{Exemple: Fibonacci}
\end{frame}

\begin{frame}{Exemple: merge sort}
\end{frame}

\begin{frame}{Preuve de terminaison}
\begin{itemize}
\item itératif: Définir une fonction $t$ montrer que $t$ décroit lors des itérations et 
\item récursif: montrer que le problème est réduit à chaque appel récursif
\end{itemize}
\end{frame}

% Pre-condition=assertion qui est vrai avant le programme
% Post-condition=assertion qui est vrai après l'exécution du programme
% Boucle: invariante:
% Tel que invariant & non G (gardien de boucle) => Post-condition

% Terminaison: on définit une fonction naturel, on montre qu'elle
% décroit à chaque itération et que le gardien


\subsection{Algorithme récursif}

\begin{frame}{Preuve par induction}

\end{frame}

\begin{frame}{Conclusion sur la correction}
Dans la suite, on ne présentera des invariants ou des preuves par induction que lorsque ce sera nécessaire (cas non triviaux)
\end{frame}

\section{Complexité algorithmique}

\subsection{Itératif}

\begin{frame}{Efficacité}
Pourquoi est-ce important?
\begin{itemize}
\item Un algorithme trop lent (dans certaines applications, un algorithm quadratique est déjà trop lent)
\item Il faut pouvoir s'adapter à la croissance des données actuelles
\end{itemize}

\end{frame}

%% Définition mathématique des trois types de complexité

%% Soit Dn
%% l’ensemble des données de taille n. Soit I un sous ensemble de
%% Dn et soit t(I) le nombre d’opérations élémentaires pour exécuter I.

%% Complexité dans le meilleur des cas
%% meilleur(n) = min {t(I), I Dn}
%% }

%% Complexité dans le pire des cas
%% pire(n) = max {t(I), I  Dn
%% }

%% Complexité en moyenne
%% moyenne(n) = S Pr (I) t(I) où Pr(I) est la probabilité de I

%% Note sur la complexité amortie
%% En moyenne sur plusieurs instance...

\begin{frame}{RAM model}

\end{frame}

\begin{frame}{Application à l'insertion sort}
\end{frame}

\subsection{Combinaison de fonctions}

\begin{frame}{règle de la somme et du produit}
Voir slides ``intro-pas-mal.ppt''
\end{frame}

\subsection{Algorithmes récursifs}

\begin{frame}{Algorithmes récursifs}

\begin{itemize}
\item Fibonacci
\item Merge sort
\end{itemize}

\end{frame}

\begin{frame}
Complexité en espace:
\begin{itemize}
\item ...
\end{itemize}
\end{frame}

\begin{frame}{Pour en savoir plus (ou ce qu'on n'a pas vu)}

\begin{itemize}
\item Méthode systématique pour l'analyse de la complexité d'algorithmes récursifs (l'an prochain)
\item Etude au cas moyen
\end{itemize}

\end{frame}

%A la fin de chaque cours, je vais dire toutes les simplifications
%qu'on a prise et encourager les étudiants à se poser certaines
%questions. Exemple:

