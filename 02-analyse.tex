
\part{Outils d'analyse}

% DeuxiÚme cours: outils d'analyse:
% - correction d'algo: rappel bref aux invariants (exemple du insertion sort) -> 30 minutes
% - complexité -> 1h30
%    - rappel des notations O theta, etc. en se basant sur insertion sort
%    - traitement d'algorithme récursif: merge-sort

% - Trois questions à se poser face à un algorithme -> deuxiÚme cours
%    - est-il correct ?
%    - Quel est sa complexité ?
%    - Est-ce qu'on peut l'améliorer ? (Si 

\begin{frame}{Plan}

\begin{itemize}
\item Correction d'un algorithme
\item Complexité d'un algorithme
\item Illustration (Fibonacci ?)
\end{itemize}

\end{frame}

\section{Correction d'un algorithme}

\subsection{Algorithme itératif}

\begin{frame}
Questions à se poser lors de la définition d'un algorithme:
\begin{itemize}
\item Mon algorithme est-il correct ?
\item Mon algorithme est-il efficace en termes d'utilisation des
  resources, temps CPU et/ou espace mémoire ?
\end{itemize}

Autre question importante seulement marginalement abordée dans ce cours:
\begin{itemize}
\item Modularité,fonctionnalité, robustesse, facilité d'usage, temps
  de programmation, simplcité, extensibilité, fiabilité, terminaison,
  existence d'une solution algorithmique
\end{itemize}

\end{frame}

\begin{frame}{Correctness}

Pourquoi est-ce important?
\begin{itemize}
\item Un algorithme incorrect ne sert souvent à rien (*)
\item Utilisation d'algorithmes dans une application critique:
  médicale, ingénieurie (point, avion) etc.
\item Exemples:
\end{itemize}

(*) Notons qu'un algorithme incorrect mais dont l'erreur peut-être
évaluée peut être très utile (seulement marginalement dans ce cours)
\end{frame}

\begin{frame}{RAM model}

\end{frame}

\begin{frame}{Application à l'insertion sort}
\end{frame}

\subsection{Combinaison de fonctions}

\begin{frame}{règle de la somme et du produit}
Voir slides ``intro-pas-mal.ppt''
\end{frame}

\subsection{Algorithme récursif}

\begin{frame}{Preuve par induction}

\end{frame}

\begin{frame}{Conclusion sur la correction}
Dans la suite, on ne présentera des invariants ou des preuves par induction que lorsque ce sera nécessaire (cas non triviaux)
\end{frame}

\section{Complexité algorithmique}

\subsection{Itératif}

\begin{frame}{Efficacité}
Pourquoi est-ce important?
\begin{itemize}
\item Un algorithme trop lent (dans certaines applications, un algorithm quadratique est déjà trop lent)
\item Il faut pouvoir s'adapter à la croissance des données actuelles
\end{itemize}

\end{frame}

%% Définition mathématique des trois types de complexité

%% Soit Dn
%% l’ensemble des données de taille n. Soit I un sous ensemble de
%% Dn et soit t(I) le nombre d’opérations élémentaires pour exécuter I.

%% Complexité dans le meilleur des cas
%% meilleur(n) = min {t(I), I Dn}
%% }

%% Complexité dans le pire des cas
%% pire(n) = max {t(I), I  Dn
%% }

%% Complexité en moyenne
%% moyenne(n) = S Pr (I) t(I) où Pr(I) est la probabilité de I

%% Note sur la complexité amortie
%% En moyenne sur plusieurs instance...

\subsection{Algorithmes récursifs}

\begin{frame}

\begin{itemize}
\item Fibonacci
\item Merge sort
\end{itemize}

\end{frame}

\begin{frame}
Complexité en espace:
\begin{itemize}
\item ...
\end{itemize}
\end{frame}

\begin{frame}{Pour en savoir plus (ou ce qu'on n'a pas vu)}

\begin{itemize}
\item Méthode systématique pour l'analyse de la complexité d'algorithmes récursifs (l'an prochain)
\item Etude au cas moyen
\end{itemize}

\end{frame}

%A la fin de chaque cours, je vais dire toutes les simplifications
%qu'on a prise et encourager les étudiants à se poser certaines
%questions. Exemple:

