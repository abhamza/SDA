
\part{Structures de données élémentaires}

% Structures de pile, file, liste liée (notion de liste liée comme une implémentation), arbre (?), files à priorité
% complexité amortie ? ajout d'élément à une structure de ce type.

\begin{frame}{Plan}

\tableofcontents

\end{frame}

\section{Introduction}

\begin{frame}{Concept}
\begin{itemize}
\item Une \alert{structure de données} est une manière d'organiser et de stocker l'information
\begin{itemize}
\item Pour en faciliter l'accès ou dans d'autres buts
\end{itemize}
\item Une structure de données a une \alert{interface} qui consiste en un ensemble de procédures pour ajouter, effacer, accéder, réorganiser, etc. les données.
\item Une structure de données conserve des \alert{données} et éventuellement des \alert{méta-données}
\begin{itemize}
\item Par exemple: un tas utilise un tableau pour stocker les clés et une variable $\attrib{A}{heap-size}$ pour retenir le nombre d'éléments qui sont dans le tas.
\end{itemize}
\item Un type de données abstrait (TDA) = définition des propriétés
  de la structure et de son interface (``cahier des charges'')
\end{itemize}
\end{frame}

\begin{frame}{Structures de données}

Dans ce cours:
\begin{itemize}
\item Principalement des \alert{ensembles dynamiques} (dynamic sets), amenés à croître, se
  rétrécir et à changer au cours du temps.
\item Les objets de ces ensembles comportent des attributs.
\item Un de ces attributs est une \alert{clé} qui permet d'identifier
  l'objet, les autres attributs sont la plupart du temps non
  pertinents pour l'implémentation de la structure.
\item Certains ensembles supposent qu'il existe un \alert{ordre total}
  entre les clés.
\end{itemize}

\end{frame}

\begin{frame}{Opérations standards sur les structures}

\begin{itemize}
\item Deux types: opérations de recherche/accès aux données et opérations de modifications
\item Recherche: exemples:
\begin{itemize}
\item $\proc{Search}(S,k)$: retourne un pointeur $x$ vers un élément dans $S$ tel que $\attrib{x}{key}=k$, ou $\const{NIL}$ si un tel élément n'appartient pas à $S$.
\item $\proc{Minimum}(S)$, $\proc{Maximum}(S)$: retourne un pointeur
  vers l'élément avec la plus petite (resp. grande) clé.
\item $\proc{Successor}(S,x)$,$\proc{Predecessor}(S,x)$ retourne un pointeur vers l'élément tout juste plus grand (resp. petit) que $x$ dans $S$, $\const{NIL}$ si $x$ est le maximum (resp. minimum).
\end{itemize}
\item Modification: exemples:
\begin{itemize}
\item $\proc{Insert}(S,x)$: insère l'élément $x$ dans $S$.
\item $\proc{Delete}(S,x)$: retire l'élément $x$ de $S$.
\end{itemize}
\end{itemize}

\end{frame}

\begin{frame}{Implémentation d'une structure de données}
\begin{itemize}
\item Etant donné un TDA (interface), plusieurs implémentations sont généralement possibles
\item La complexité des opérations dépend de l'implémentation, \alert{pas du TDA}.

\bigskip

\item Les briques de base pour implémenter une structure de données
  dépendent du langage d'implémentation
\begin{itemize}
\item Dans ce cours, les principaux outils du C: tableaux, structures
  à la C (objets avec attributs), liste liées (simples, doubles,
  circulaires), etc.
\end{itemize}
\item Une structure de données peut être implémentée à l'aide d'une
  autre structure de données (de base ou non)
\end{itemize}
\end{frame}

\begin{frame}{Quelques structures de données standards}

\begin{itemize}
\item Pile: collection d'objets accessible selon une politique LIFO
\item File: collection d'objets accessible selon une politique FIFO
\item File double: combine accès LIFO et FIFO
\item Liste: collection d'objets ordonnés accessible à partir de leur position
\item Vecteur: collection d'objets ordonnés accessible à partir de leur rang
\item File à priorité: accès uniquement à l'élément de clé (priorité) maximale

\bigskip

\item Dictionnaire: structure qui implémente les 3 opérations
  recherche, insertion, suppression (cf. partie 5)
\end{itemize}

\end{frame}

\section{Pile}

\begin{frame}{Pile}
\begin{itemize}
\item Ensemble dynamique d'objets accessibles selon une discipline
  \alert{LIFO} (``Last-in first-out'').
\item Interface
\begin{itemize}
\item $\proc{Stack-Empty}(S)$ renvoie vrai si et seulement si la pile est vide
\item $\proc{Push}(S,x)$ pousse la valeur $x$ sur la pile $S$
\item $\proc{Pop}(S)$ extrait et renvoie la valeur sur le sommet de la pile $S$
\end{itemize}
\item Applications:
\begin{itemize}
\item Option 'undo' dans un traitement de texte
\item Langage postscript
\item Appel de fonctions dans un compilateur
\item \ldots
\end{itemize}
\item Implémentations:
\begin{itemize}
\item avec un tableau (taille fixée a priori)
\item au moyen d'une liste liée (allouée de manière dynamique)
\item \ldots
\end{itemize}
\end{itemize}
\end{frame}

\begin{frame}{Implémentation par un tableau}

\begin{itemize}
\item $S$ est un tableau qui contient les éléments de la pile
\item $\attrib{S}{top}$ est la position courante de l'élément au sommet de $S$

\medskip

\begin{columns}
\begin{column}{5cm}
\centerline{\includegraphics[width=3cm]{Figures/04-piletableau.pdf}}

\bigskip

\begin{center}
\begin{small}
\fcolorbox{white}{Lightgray}{%
        \begin{codebox}
          \Procname{$\proc{Push}(S,x)$}
          \li \If $\attrib{S}{top}\isequal \attrib{S}{length}$
          \li \Then \Error ``overflow''\End
          \li $\attrib{S}{top}\gets \attrib{S}{top}+1$
          \li $S[\attrib{S}{top}]\gets x$
        \end{codebox}}
\end{small}
\end{center}

\end{column}
\begin{column}{5cm}
    \begin{center}
      \begin{small}
      \fcolorbox{white}{Lightgray}{%
        \begin{codebox}
          \Procname{$\proc{Stack-Empty}(S)$}
          \li \Return $\attrib{S}{top}\isequal 0$
          \End
        \end{codebox}}

\bigskip

      \fcolorbox{white}{Lightgray}{%
        \begin{codebox}
          \Procname{$\proc{Pop(S)}$}
          \li \If $\proc{Stack-Empty}(S)$
          \li \Then \Error ``underflow''
          \li \Else $\attrib{S}{top}\gets \attrib{S}{top}-1$
          \li       \Return $S[top(S)+1]$
          \End
        \end{codebox}}

      \end{small}
    \end{center}
\end{column}
\end{columns}

\item Complexité en temps \alert{et en espace}: $O(1)$\\
(Inconvénient: L'espace occupé ne dépend pas du nombre d'objets)
\end{itemize}

\end{frame}

\begin{frame}{Rappel: liste simplement et doublement liée}

\centerline{\includegraphics[width=8cm]{Figures/04-listeliee.pdf}}

\bigskip

\begin{itemize}
\item Structure de données composée d'une séquence d'éléments de liste.
\item Chaque élément $x$ de la liste est composé:
\begin{itemize}
\item d'un contenu utile $\attrib{x}{data}$ de type arbitraire (par exemple une clé),
\item d'un pointeur $\attrib{x}{next}$ vers l'élément suivant dans la séquence
\item \emph{Doublement liée: }d'une pointeur $\attrib{x}{prev}$ vers l'élément précédent dans la séquence
\end{itemize}
\item Soit $L$ une liste liée
\begin{itemize}
\item $\attrib{L}{head}$ pointe vers le premier élément de la liste
\item \emph{Doublement liée:} $\attrib{L}{tail}$ pointe vers le dernier élément de la liste
\end{itemize}
\item Le dernier élément possède un pointeur $\attrib{x}{next}$ vide (noté $\const{NIL}$)
\item \emph{Doublement liée:} Le premier élément possède un pointeur $\attrib{x}{prev}$ vide
%% \item Avantage
%% \begin{itemize}
%% \item L'insertion et la suppression d'éléments est réalisable en temps $O(1)$ en tête de liste, ainsi qu'à la suite d'un élément donné (à n'importe quel endroit pour une liste doublement liée).
%% \item Les éléments de la liste peuvent être alloués dynamiquement
%% \end{itemize}
\end{itemize}
\note{Mais l'espace mémoire est deux fois plus important qu'avec un tableau classique}
\end{frame}

%% \begin{frame}{Opérations sur une liste doublement liée}

%% %Interface:
%% {\small
%%     \begin{itemize}
%% %      \medskip
%%     \item $\proc{List-Insert}(L,x)$ ajoute l'élément $x$ au début de la liste $L$
%% %      \medskip
%%     \item $\proc{List-Delete}(L,x)$ supprime l'élément $x$ de la liste $L$
%% %      \medskip
%%     \item $\proc{List-Search}(L,k)$ trouve une élément dont la clé $k$
%%       est dans la liste $L$
%%     \end{itemize}
%% }

%% %% \fcolorbox{white}{Lightgray}{%
%% %%       \begin{codebox}
%% %%         \Procname{$\proc{List-Init}(L)$}
%% %%         \li $\id{prev}(\id{nil}(L))\gets\id{nil}(L)$
%% %%         \li $\id{next}(\id{nil}(L))\gets\id{nil}(L)$
%% %%       \end{codebox}}

%% \begin{center}
%% \begin{footnotesize}
%% \fcolorbox{white}{Lightgray}{%
%%       \begin{codebox}
%%         \Procname{$\proc{List-Insert}(L,x)$}
%%         \li $\attrib{x}{next}\gets \attrib{L}{head}$
%%         \li \If $\attrib{L}{head}\ne \const{NIL}$
%%         \li \Then $\attrib{L}{head}.\id{prev} \gets x$ \End
%%         \li $\attrib{L}{head}\gets x$
%%         \li $\attrib{x}{prev}\gets \const{NIL}$
%%       \end{codebox}}
%% ~~~~~
%% \fcolorbox{white}{Lightgray}{%
%%       \begin{codebox}
%%         \Procname{$\proc{List-Delete}(L,x)$}
%%         \li \If $\attrib{x}{prev}\ne \const{NIL}$
%%         \li \Then $\attrib{x}{prev}.\id{next}\gets \attrib{x}{next}$
%%         \li \Else $\attrib{L}{head}\gets \attrib{x}{next}$\End
%%         \li \If $\attrib{x}{next}\ne \const{NIL}$
%%         \li \Then $\attrib{x}{next}.\id{prev}\gets \attrib{x}{prev}$ \End
%%       \end{codebox}}

%% \bigskip

%% \fcolorbox{white}{Lightgray}{%
%%       \begin{codebox}
%%         \Procname{$\proc{List-Search}(L,k)$}
%%         \li $x\gets \attrib{L}{head}$
%%         \li \While $x\ne \const{NIL}\wedge\attrib{x}{key}\ne k$
%%         \li \Do $x\gets \attrib{x}{next}$
%%             \End
%%         \li \Return $x$
%%       \end{codebox}}
%% \end{footnotesize}
%% \end{center}

%% Complexité: $O(1)$ pour l'insertion et la suppression, $O(n)$ pour la recherche.
%% \end{frame}


%% \begin{frame}{Sentinelle}
%% \begin{itemize}\small
%% \item On peut simplifier le code en ajoutant une \alert{sentinelle} $L.nil$ qui représente le $\const{NIL}$
%% \item $L.nil.next$ pointe vers le premier élément et $L.nil.prev$ pointe vers le dernier élément (, $L.nil.next=L.nil$ et $L.nil.prev=L.nil$)
%% \end{itemize}

%% \centerline{\includegraphics[width=8cm]{Figures/04-sentinel.pdf}}

%% \begin{center}
%% \begin{footnotesize}
%% \fcolorbox{white}{Lightgray}{%
%%       \begin{codebox}
%%         \Procname{$\proc{List-Insert'}(L,x)$}
%%         \li $\attrib{x}{next}\gets \attrib{L}{nil}.\id{next}$
%%         \li $\attrib{L}{nil}.\id{next}.\id{prev} \gets x$
%%         \li $\attrib{L}{nil}.\id{next} \gets x$
%%         \li $\attrib{x}{prev}\gets \attrib{L}{nil}$
%%       \end{codebox}}
%% ~~~~~
%% \fcolorbox{white}{Lightgray}{%
%%       \begin{codebox}
%%         \Procname{$\proc{List-Delete'}(L,x)$}
%%         \li $\attrib{x}{prev}.\id{next}\gets \attrib{x}{next}$
%%         \li $\attrib{x}{next}.\id{prev}\gets \attrib{x}{prev}$
%%       \end{codebox}}

%% \bigskip

%% \fcolorbox{white}{Lightgray}{%
%%       \begin{codebox}
%%         \Procname{$\proc{List-Search'}(L,k)$}
%%         \li $x\gets \attrib{L}{nil}.\id{next}$
%%         \li \While $x\ne \attrib{L}{nil}\wedge\attrib{x}{key}\ne k$
%%         \li \Do $x\gets \attrib{x}{next}$
%%             \End
%%         \li \Return $x$
%%       \end{codebox}}
%% \end{footnotesize}
%% \end{center}

%% \note{
%% Toutes les pointeurs vers NIL sont remplacés par un pointeur vers $L.nil$

%% Ne permet de gagner qu'un facteur constant

%% \bigskip

%% Fait perdre de la place si on a beaucoup de petites listes


%% \bigskip

%% Principal intérêt est le clareté du code
%% }
%% \end{frame}


\begin{frame}{Implémentation d'une pile à l'aide d'une liste liée}

\begin{itemize}
\item $S$ est une liste simplement liée ($S.head$ pointe vers le premier élément de la liste)

\medskip

\begin{columns}
\begin{column}{5cm}
\begin{center}
\includegraphics[width=5cm]{Figures/04-pilelisteliee.pdf}
\end{center}

\bigskip

\fcolorbox{white}{Lightgray}{%
        \begin{codebox}
          \Procname{$\proc{Push}(S,x)$}
          \li $\attrib{x}{next}\gets \attrib{S}{head}$
          \li $\attrib{S}{head}\gets x$
        \end{codebox}}
\end{column}
\begin{column}{5cm}
\begin{center}
  \footnotesize
  \fcolorbox{white}{Lightgray}{%
    \begin{codebox}
      \Procname{$\proc{Stack-Empty}(S)$}
      \li \If $\attrib{S}{head}\isequal \const{NIL}$
      \li \Then \Return \const{true}
      \li \Else \Return \const{false}
        \End
  \end{codebox}}\hfill

\bigskip

  \fcolorbox{white}{Lightgray}{%
    \begin{codebox}
      \Procname{$\proc{Pop(S)}$}
      \li \If $\proc{Stack-Empty}(S)$
          \li \Then \Error ``underflow''
          \li \Else $x = S.head$
          \li       $\attrib{S}{head}\gets \attrib{S}{head}.\id{next}$
          \li       \Return $x$
          \End
        \end{codebox}}
    \end{center}
\end{column}
\end{columns}

\item Complexité en temps $O(1)$, complexité en espace $O(n)$ pour $n$ opérations
\end{itemize}

\end{frame}

\begin{frame}{Application}
\begin{itemize}
\item Vérifier l'appariement de parenthèses ($[]$,$()$ ou $\{\}$) dans une chaîne de caractères
\begin{itemize}
\item Exemples: $((x)+(y)]/2\rightarrow$ non, $[- (b) + \mbox{sqrt}(4*(a)*c)]/ (2*a) \rightarrow$ oui
\end{itemize}
\item Solution basée sur une pile:
\begin{center}
\begin{small}
\fcolorbox{white}{Lightgray}{%
        \begin{codebox}
          \Procname{$\proc{ParenthesesMatch}(A)$}
          \li $S\gets$ pile vide
          \li \For $i\gets 1 \To \attrib{A}{length}$
          \li \Do \If $A[i]$ est une parenthèse gauche
          \li \Then $\proc{Push}(S,A[i])$
          \li \ElseIf $A[i]$ est une parenthèse droite
          \li \Then \If $\proc{Stack-Empty}(S)$
          \li \Then \Return \const{False}

          \li \ElseIf  $\proc{Pop}(S)$ n'est pas du même type que $A[i]$
          \li \Then \Return \const{False}
          \End\End\End
          \li \End \Return $\proc{Stack-Empty}(S)$
          \End
        \end{codebox}}
\end{small}
\end{center}
\end{itemize}
\end{frame}

\section{Files simple et double}

\begin{frame}{File}
\begin{itemize}
\item Ensemble dynamique d'objets accessibles selon une discipline \alert{FIFO} (``First-in first-out'').
\item Interface
\begin{itemize}
\item $\proc{Enqueue}(Q,s)$ ajoute l'élément $x$ à la fin de la file $Q$
\item $\proc{Dequeue}(Q)$  retire l'élément à la tête de la file $Q$
\end{itemize}

\bigskip

\item Implémentation à l'aide d'un tableau circulaire
\begin{itemize}
\item $Q$ est un tableau de taille fixe $\attrib{Q}{length}$
\begin{itemize}
\item Mettre plus de $\attrib{Q}{length}$ éléments dans la file provoque une erreur de dépassement
\end{itemize}
\item $\attrib{Q}{head}$  est la position à la tête de la file
\item $\attrib{Q}{tail}$ est la première position vide à la fin de la file
\item Initialement: $\attrib{Q}{head}=\attrib{Q}{tail}=1$
\end{itemize}
\end{itemize}
\end{frame}

\begin{frame}{\proc{Enqueue} et \proc{Dequeue}}
\begin{small}
Etat initial:
\bigskip
\centerline{\includegraphics[width=6cm]{Figures/04-file1.pdf}}
$\proc{Enqueue}(Q,17)$, $\proc{Enqueue}(Q,3)$, $\proc{Enqueue}(Q,5)$
\bigskip
\centerline{\includegraphics[width=6cm]{Figures/04-file2.pdf}}
$\proc{Dequeue}(Q) \rightarrow 15$
\bigskip
\centerline{\includegraphics[width=6cm]{Figures/04-file3.pdf}}
\end{small}
\end{frame}

\begin{frame}{\proc{Enqueue} et \proc{Dequeue}}

\begin{center}
\begin{small}
\fcolorbox{white}{Lightgray}{%
  \begin{codebox}
    \Procname{$\proc{Enqueue(Q,x)}$}
    \li $Q[\attrib{Q}{tail}]\gets x$
    \li \If $\attrib{Q}{tail}\isequal \attrib{Q}{length}$
    \li       \Then $\attrib{Q}{tail}\gets 1$
    \li       \Else $\attrib{Q}{tail}\gets\attrib{Q}{tail}+1$
  \end{codebox}}
~~~~~
\fcolorbox{white}{Lightgray}{%
  \begin{codebox}
    \Procname{$\proc{Dequeue(Q)}$}
    \li $x\gets Q[\attrib{Q}{head}]$
    \li \If $\attrib{Q}{head}\isequal \attrib{Q}{length}$
    \li       \Then $\attrib{Q}{head}\gets 1$
    \li       \Else $\attrib{Q}{head}\gets\attrib{Q}{head}+1$\End
    \li \Return $x$
  \end{codebox}}
\end{small}
\end{center}

\bigskip

\begin{itemize}
\item Complexité en temps $O(1)$, complexité en espace $O(1)$.
\item {\it Exercice: ajouter la gestion d'erreur}
\end{itemize}
\end{frame}

\begin{frame}{Implémentation à l'aide d'une liste liée}
\centerline{\includegraphics[width=6cm]{Figures/04-filelisteliee.pdf}}
\begin{itemize}
\item $Q$ est une liste simplement liée
\item $\attrib{Q}{head}$ (resp. $\attrib{Q}{tail}$) pointe vers la tête (resp. la queue) de la liste

\begin{center}
\begin{small}
\fcolorbox{white}{Lightgray}{%
  \begin{codebox}
    \Procname{$\proc{Enqueue(Q,x)}$}
    \li $\attrib{x}{next}\gets \const{NIL}$
    \li \If $\attrib{Q}{head}\isequal \const{NIL}$
    \li \Then $\attrib{Q}{head}\gets x$
    \li \Else $\attrib{Q}{tail}.\id{next} \gets x$
    \End
    \li $\attrib{Q}{tail}\gets x$
  \end{codebox}}
~~~~~
\fcolorbox{white}{Lightgray}{%
  \begin{codebox}
    \Procname{$\proc{Dequeue(Q)}$}
    \li \If $\attrib{Q}{head}\isequal \const{NIL}$
    \li \Then \Error ``underflow''
    \End
    \li $x\gets \attrib{Q}{head}$
    \li $\attrib{Q}{head}\gets\attrib{Q}{head}.\id{next}$
    \li \If $\attrib{Q}{head}\isequal \const{NIL}$
    \li \Then $\attrib{Q}{tail}\gets \const{NIL}$ \End
    \li \Return $x$
  \end{codebox}}
\end{small}
\end{center}

\item Complexité en temps $O(1)$, complexité en espace $O(n)$ pour $n$ opérations
\end{itemize}

\end{frame}

\begin{frame}{File double}
Double ended-queue (deque)

\bigskip

\begin{itemize}
\item Généralisation de la pile et de la file
\item Collection ordonnée d'objets offrant la possibilité
\begin{itemize}
\item d'insérer un nouvel objet \alert{avant le premier} ou \alert{après le dernier}
\item d'extraire le \alert{premier} ou le \alert{dernier} objet
\end{itemize}
\item Interface:
\begin{itemize}
\item $\proc{insert-first}(Q,x)$: ajoute $x$ au début de la file double
\item $\proc{insert-last}(Q,x)$: ajoute $x$ à la fin de la file double
\item $\proc{remove-first}(Q)$: extrait l'objet situé en première position
\item $\proc{remove-last}(Q)$: extrait l'objet situé en dernière position
\item \ldots
\end{itemize}
\item Application: équilibrage de la charge d'un serveur
\end{itemize}

\end{frame}

\begin{frame}{Implémentation à l'aide d'une liste doublement liée}
\begin{itemize}
\item A l'aide d'une liste double liée
\item Soit la file double $Q$:
\begin{itemize}
\item $Q.head$ pointe vers un élément \alert{sentinelle} en début de liste
\item $Q.tail$ pointe vers un élément \alert{sentinelle} en fin de liste
\item $Q.size$ est la taille courante de la liste
\end{itemize}

\centerline{\includegraphics[width=10cm]{Figures/04-dequeimplem.pdf}}

\item Les sentinelles ne contiennent pas de données. Elles permettent
  de simplifier le code (pour un coût en espace constant).
\end{itemize}

\bigskip

{\it Exercice: implémentation de la file double sans
  sentinelles, implémentation de la file simple avec sentinelle}

\end{frame}

\begin{frame}{Implémentation à l'aide d'une liste doublement liée}
\begin{center}
      \begin{small}\hfill
      \fcolorbox{white}{Lightgray}{%
        \begin{codebox}
          \Procname{$\proc{insert-first}(Q,x)$}
          \li $\attrib{x}{prev}\gets \attrib{Q}{head}$
          \li $\attrib{x}{next}\gets \attrib{Q}{head}.\id{next}$
          \li $\attrib{Q}{head}.\id{next}.\id{prev} \gets \id{x}$
          \li $\attrib{Q}{head}.\id{next}\gets \id{x}$
          \li $\attrib{Q}{size}\gets \attrib{Q}{size}+1$
        \end{codebox}}\hfill
\fcolorbox{white}{Lightgray}{%
        \begin{codebox}
          \Procname{$\proc{insert-last}(Q,x)$}
          \li $\attrib{x}{prev}\gets \attrib{Q}{tail}.\id{prev}$
          \li $\attrib{x}{next}\gets \attrib{Q}{tail}$
          \li $\attrib{Q}{tail}.\id{prev}.\id{next} \gets \id{x}$
          \li $\attrib{Q}{head}.\id{prev}\gets \id{x}$
          \li $\attrib{Q}{size}\gets \attrib{Q}{size}+1$
        \end{codebox}}\hfill~

      \bigskip
      \fcolorbox{white}{Lightgray}{%
        \begin{codebox}
          \Procname{$\proc{remove-first}(Q)$}
          \li \If $(\attrib{Q}{size}\isequal 0)$
          \li \Then \Error\End
          \li $x\gets \attrib{Q}{head}.\id{next}$
          \li $\attrib{Q}{head}.\id{next}\gets \attrib{Q}{head}.\id{next}.\id{next}$
          \li $\attrib{Q}{head}.\id{next}.\id{prev}\gets \attrib{Q}{head}$
          \li $\attrib{Q}{size}\gets \attrib{Q}{size}-1$
          \li \Return $x$
        \end{codebox}}
      \hfill
      \fcolorbox{white}{Lightgray}{%
        \begin{codebox}
          \Procname{$\proc{remove-last}(Q)$}
          \li \If $(\attrib{Q}{size}\isequal 0)$
          \li \Then \Error\End
          \li $x\gets \attrib{Q}{tail}.\id{prev}$
          \li $\attrib{Q}{tail}.\id{prev}\gets \attrib{Q}{tail}.\id{prev}.\id{prev}$
          \li $\attrib{Q}{tail}.\id{prev}.\id{next}\gets \attrib{Q}{head}$
          \li $\attrib{Q}{size}\gets \attrib{Q}{size}-1$
          \li \Return $x$
        \end{codebox}}
\end{small}
    \end{center}

Complexité $O(1)$ en temps et $O(n)$ en espace pour $n$ opérations.

\end{frame}

\section{Liste}

\begin{frame}{Liste}

\begin{itemize}
\item Ensemble dynamique d'objets ordonnés accessibles \alert{relativement}
  les uns aux autres, sur base de leur position
\item Généralise toutes les structures vues précédemment
\item Interface:
\begin{itemize}
\item Les fonctions d'une liste double (insertion et retrait en début et fin de liste)
\item $\proc{Insert-Before}(L,p,x)$: insére $x$ avant $p$ dans la liste
\item $\proc{Insert-After}(L,p,x)$: insère $x$ après $p$ dans la liste
\item $\proc{Remove}(L,p)$: retire l'élement à la position $p$
\item $\proc{replace}(L,p,x)$: remplace par l'objet $x$ l'objet situé à la position $p$
\item $\proc{first}(L)$, $\proc{last}(L)$: renvoie la première, resp. dernière position dans la liste
\item $\proc{prev}(L,p)$, $\proc{next}(L,p)$: renvoie la position précédant (resp. suivant) $p$ dans la liste
\end{itemize}

\bigskip

\item Implémentation similaire à la file double, à l'aide d'une liste doublement liée (avec sentinelles)
\end{itemize}

\end{frame}

\begin{frame}{Implémentation à l'aide d'une liste doublement liée}


\begin{columns}
\begin{column}{5cm}
\centerline{\includegraphics[width=6cm]{Figures/04-listelisteliee.pdf}}

\bigskip

\begin{center}
  \begin{small}\hfill
      \fcolorbox{white}{Lightgray}{%
        \begin{codebox}
          \Procname{$\proc{insert-before}(L,p,x)$}
          \li $\attrib{x}{prev}\gets \attrib{p}{prev}$
          \li $\attrib{x}{next}\gets \id{p}$
          \li $\attrib{p}{prev}.\id{next} \gets \id{x}$
          \li $\attrib{p}{prev} \gets \id{x}$
          \li $\attrib{L}{size}\gets \attrib{L}{size}+1$
        \end{codebox}}
  \end{small}
\end{center}
\end{column}
\begin{column}{5cm}
\begin{center}

\fcolorbox{white}{Lightgray}{%
        \begin{codebox}
          \Procname{$\proc{remove}(L,p)$}
          \li $\attrib{p}{prev}.\id{next} \gets \attrib{p}{next}$
          \li $\attrib{p}{next}.\id{prev} \gets \attrib{p}{prev}$
          \li $\attrib{L}{size}\gets \attrib{L}{size}-1$
          \li \Return $p$
        \end{codebox}}

\bigskip

\fcolorbox{white}{Lightgray}{%
        \begin{codebox}
          \Procname{$\proc{insert-after}(L,p,x)$}
          \li $\attrib{x}{prev}\gets \id{p}$
          \li $\attrib{x}{next}\gets \attrib{p}{next}$
          \li $\attrib{p}{next}.\id{prev} \gets \id{x}$
          \li $\attrib{p}{next} \gets \id{x}$
          \li $\attrib{L}{size}\gets \attrib{L}{size}+1$
        \end{codebox}}\hfill~
\end{center}
\end{column}
\end{columns}

Complexité $O(1)$ en temps et $O(n)$ en espace pour $n$ opérations.
%\note{\centerline{\includegraphics[width=8cm]{Figures/04-insertionliste.pdf}}}
\end{frame}

\section{Vecteur}

\begin{frame}{Vecteur}

\begin{itemize}
\item Ensemble dynamique d'objets occupant des rangs entiers
  successifs, permettant la consultation, le remplacement, l'insertion
  et la suppression d'éléments à des rangs arbitraires
\item Interface
\begin{itemize}
\item $\proc{Elem-At-Rank}(V,r)$ retourne l'élément au rang $r$ dans $V$.
\item $\proc{Replace-At-Rank}(V,r,x)$ remplace l'élément situé au rang $r$ par $x$ et retourne cet objet.
\item $\proc{Insert-At-Rank}(V,r,x)$ insère l'élément $x$ au rang $r$, en augmentant le rang des objets suivants.
\item $\proc{Remove-At-Rank}(V,r)$ extrait l'élément situé au rang $r$ et le retire de $r$, en diminuant le rang des objets suivants.
\item $\proc{Vector-Size}(V)$ renvoie la taille du vecteur.
\end{itemize}
\item Applications: tableau dynamique, gestion des éléments d'un menu,\ldots
\item Implémentation: liste liée, tableau extensible\ldots
\end{itemize}

\note{Inconvénient des structures précédentes: il faut parcourir la liste pour retrouver le kième élément ($O(1)$ dans un tableau\\

\bigskip

Avantage: espace mémoire $O(n)$.

\bigskip

Comment combiner les deux ?
}
\end{frame}

\begin{frame}{Implémentation par un tableau extensible}

\begin{itemize}
\item Les éléments sont stockés dans un tableau extensible $\attrib{V}{A}$ de taille initiale $\attrib{V}{c}$.
\item En cas de dépassement, la capacité du tableau est \alert{doublée}.
\item $\attrib{V}{n}$ retient le nombre de composantes.
\item Insertion et suppression:

\begin{center}
\begin{footnotesize}
\fcolorbox{white}{Lightgray}{%
  \begin{codebox}
    \Procname{$\proc{Insert-At-Rank}(V,r,x)$}
    \li \If $\attrib{V}{n}\isequal \attrib{V}{c}$
    \li \Then $\attrib{V}{c}\gets 2\cdot\attrib{V}{c}$
    \li  $W\gets $ ``Tableau de taille $\attrib{V}{c}$''
    \li  \For $i\gets 1 \To n$
    \li \Do $W[i]\gets V.A[i]$ \End
    \li $\attrib{V}{A}\gets W$\End
    \li \For $i\gets \attrib{V}{n} \Downto r$
    \li \Do $\attrib{V}{A}[i+1]\gets \attrib{V}{A}[i]$ \End
    \li $\attrib{V}{A}[r]\gets x$
    \li $\attrib{V}{n}\gets \attrib{V}{n}+1$
  \end{codebox}}
~~~~~
\fcolorbox{white}{Lightgray}{%
  \begin{codebox}
    \Procname{$\proc{Remove-At-Rank}(V,r)$}
    \li $tmp\gets \attrib{V}{A}[r]$
    \li \For $i\gets r \To \attrib{V}{n}-1$ \Do
    \li $\attrib{V}{A}[i]\gets \attrib{V}{A}[i+1]$
    \End
    \li $\attrib{V}{n}\gets \attrib{V}{n}-1$
    \li \Return $tmp$
  \end{codebox}}
\end{footnotesize}
\end{center}


\end{itemize}

\end{frame}

\begin{frame}{Complexité en temps}\label{sec04:amortie}
\begin{itemize}
%\item Accès à un élément, remplacement de sa valeur: $O(1)$
\item $\proc{Insert-At-Rank}$:
\begin{itemize}
\item $O(n)$ pour une opération individuelle, où $n$ est le nombre de composantes du vecteur
\item $O(n^2)$ pour $n$ opérations d'insertion en \alert{début} de vecteur
\item $O(n)$ pour $n$ opérations d'insertion en \alert{fin} de vecteur
\end{itemize}
\item Justification:
\begin{itemize}
\item Si la capacité du tableau passe de $c_0$ à $2^k c_0$ au cours des $n$ opérations, alors le coût des transferts entre tableaux s'élève à
$$c_0+2 c_0+\ldots+2^{k-1} c_0=(2^k-1) c_0.$$
Puisque $2^{k-1} c_0 < n \leq 2^k c_0$, ce coût est \alert{$O(n)$}.
\item On dit que le \alert{coût amorti} par opération est $O(1)$
\item Si on avait élargi le tableau avec un incrément constant $m$, le coût aurait été
$$c_0+(c_0+m)+(c_0+2m)+\ldots+(c_0+(k-1)m)=kc_0+\frac{k(k-1)}{2}m.$$
Puisque $c_0+(k-1)m<n\leq c_0+km$, ce coût aurait donc été \alert{$O(n^2)$}.
\end{itemize}
\end{itemize}
\end{frame}

\begin{frame}{Complexité en temps}
\begin{itemize}
%\item Accès à un élément, remplacement de sa valeur: $O(1)$
\item $\proc{Remove-At-Rank}$:
\begin{itemize}
\item $O(n)$ pour une opération individuelle, où $n$ est le nombre de composantes du vecteur
\item $O(n^2)$ pour $n$ opérations de retrait en \alert{début} de vecteur
\item $O(n)$ pour $n$ opérations de retrait en \alert{fin} de vecteur
\end{itemize}
\item Remarque: Un tableau circulaire permettrait d'améliorer l'efficacité des opérations d'ajout et de retrait en début de vecteur.
\end{itemize}

\end{frame}

\section{File à priorité}

% Implémentation par un tableau, implémentation par un tas.
% Tas de Fibonacci ou binomial pour la fusion de deux tas ?

\begin{frame}{File à priorité}
\begin{itemize}
\item Ensemble dynamique d'objets classés par ordre de \alert{priorité}
\begin{itemize}
\item Permet d'extraire un objet possédant la plus grande priorité
\item En pratique, on représente les priorités par les clés
\item Suppose un ordre total défini sur les clés
\end{itemize}
\item Interface:
\begin{itemize}
\item $\proc{Insert}(S,x)$: insère l'élément $x$ dans $S$.
\item $\proc{Maximum}(S)$: renvoie l'élément de $S$ avec la plus grande clé.
\item $\proc{Extract-Max}(S)$: supprime et renvoie l'élément de $S$ avec la plus grande clé.

\end{itemize}
\item Remarques:
\begin{itemize}
\item Extraire l'élément de clé maximale ou minimale sont des problèmes équivalents
\item La file FIFO est une file à priorité où la clé correspond à l'ordre d'arrivée des élements.
\end{itemize}
\item Application: gestion des jobs sur un ordinateur partagé
\end{itemize}
\note{Ordre total=relation binaire avec les propriétés suivantes:
\begin{itemize}
\item totalité: $\forall p_1, p_2: p_1\leq p_2 \mbox{ ou }p_2\leq p_1$
\item antisymétrie: $\forall p_1,p_2: p_1\leq p_2 \mbox{ et } p_2\leq p_1 \Rightarrow p_1=p_2$
\item Transitivité...
\end{itemize}

\bigskip

File FIFO est un cas particulier ou la priorité est représenté par l'ordre d'arrivé des éléments
}
\end{frame}

\begin{frame}{Implémentations}
\begin{itemize}
\item Implémentation à l'aide d'un tableau statique
\begin{itemize}
\item $Q$ est un tableau statique de taille fixée $\attrib{Q}{length}$.
\item Les éléments de $Q$ sont triés par ordre \alert{croissant} de clés. $\attrib{Q}{top}$ pointe vers le dernier élément.
\item Complexité en temps: extraction en $O(1)$ et insertion en $O(n)$ où $n$ est la taille de la file
\item Complexité en espace: $O(1)$
\end{itemize}
\item Implémentation à l'aide d'une liste liée
\begin{itemize}
\item $Q$ est une liste liée où les éléments sont triés par ordre \alert{décroissant} de clés
\item Complexité en temps: extraction en $O(1)$ et insertion en $O(n)$ où $n$ est la taille de la file
\item Complexité en espace: $O(n)$
\end{itemize}
\item Peut-on faire mieux?
\end{itemize}

{\it Exercice: comment obtenir $O(1)$ pour l'insertion et $O(n)$ pour l'extraction?}
 \note{Leur demander pourquoi on trie le tableau par ordre croissant de clé et pas par ordre décroissant: parce que retirer le premier élément veut dire qu'on doit décaler tous les autres et donc l'opération serait $O(n)$}
\end{frame}

\begin{frame}{Implémentation à l'aide d'un tas}
\begin{itemize}
\item La file est implémentée à l'aide d'un tas(-max) (voir slide \pageref{sec:03tas})
\item Accès et extraction du maximum:

\bigskip

\begin{center}
\begin{small}
\fcolorbox{white}{Lightgray}{
  \begin{codebox}
    \Procname{$\proc{Heap-Maximum}(A)$}
    \li \Return $A[1]$
\end{codebox}}~~~~~\fcolorbox{white}{Lightgray}{
  \begin{codebox}
    \Procname{$\proc{Heap-Extract-Max}(A)$}
    \li \If $\attrib{A}{heap-size}<1$
    \li \Then \Error ``heap underflow''
    \End
    \li $max \gets A[1]$
    \li $A[1]\gets A[\attrib{A}{heap-size}]$
    \li $\attrib{A}{heap-size}\gets \attrib{A}{heap-size}-1$
    \li $\proc{Max-heapify}(A,1)$ \Comment reconstruit le tas
    \li \Return $max$
\end{codebox}}
\end{small}
\end{center}

\bigskip

\item Complexité: $O(1)$ et $O(\log n)$ respectivement (voir chapitre 3)
\end{itemize}

\end{frame}

\begin{frame}{Implémentation à l'aide d'un tas}

\begin{center}
\includegraphics[width=8cm]{Figures/04-fileprio.pdf}
\end{center}

\end{frame}

\begin{frame}{Implémentation à l'aide d'un tas: insertion}
\begin{itemize}
\item $\proc{Increase-Key}(S,x,k)$ augmente la valeur de la clé de $x$
  à $k$ (on suppose que $k\geq$ à la valeur courante de la clé de $x$).

\bigskip

\begin{center}
\begin{small}
\fcolorbox{white}{Lightgray}{
  \begin{codebox}
    \Procname{$\proc{Heap-Increase-Key}(A,i,key)$}
    \li \If $key<A[i]$
    \li \Then \Error ``new key is smaller than current key'' \End
    \li $A[i]\gets key$
    \li \While $i>1$ and $A[\proc{Parent}(i)]<A[i]$
    \li \Do $\id{swap}(A[i],A[\proc{Parent}(i)])$
    \li     $i\gets \proc{Parent}(i)$
    \End
\end{codebox}}
\end{small}
\end{center}

\bigskip

\item Complexité: $O(\log n)$ (la longueur de la branche de la racine à $i$ étant $O(\log n)$ pour un tas de taille $n$).
\end{itemize}
\note{Montrer un exemple sur le slide précédent: augmenter le poids de $A[9]$ de 4 à 15: on échange les noeuds 4 et 9, puis les noeuds 4 et 2.}
\end{frame}

\begin{frame}{Implémentation à l'aide d'un tas: insertion}
\begin{itemize}
\item Pour insérer un élément avec une clé $key$:
\begin{itemize}
\item l'insérer à la dernière position sur le tas avec une clé $-\infty$,
\item augmenter sa clé de $-\infty$ à $key$ en utilisant la procédure précédente
\end{itemize}

\bigskip

\begin{center}
\begin{small}
\fcolorbox{white}{Lightgray}{
  \begin{codebox}
    \Procname{$\proc{Heap-Insert}(A,key)$}
    \li $\attrib{A}{heap-size}\gets \attrib{A}{heap-size}+1$
    \li $A[\attrib{A}{heap-size}]\gets -\infty$
    \li $\proc{Heap-Increase-Key}(A,\attrib{A}{heap-size},key)$
\end{codebox}}
\end{small}
\end{center}

\bigskip

\item Complexité: $O(\log n)$.

\bigskip

\item[$\Rightarrow$] Implémentation d'une file à priorité par un tas: $O(\log n)$ pour l'extraction et l'insertion.
\end{itemize}
\note{}
\end{frame}

\begin{frame}{Ce qu'on a vu}

\begin{itemize}
\item Quelques structures de données classiques et différentes implémentations pour chacune d'elles
\begin{itemize}
\item Pile
\item Liste
\item Files simples, doubles et à priorité
\item Vecteurs
\end{itemize}
\item Structures de type liste liée
\end{itemize}

\end{frame}

\begin{frame}{Ce qu'on n'a pas vu}

\begin{itemize}
\item Structure de type séquence: hybride entre le vecteur et la
  liste
\item Notion d'itérateur
\item Tas binomial: alternative au tas binaire qui permet la fusion
  rapide de deux tas
\item Evolution dynamique de la taille d'un tas (analyse amortie)
\item \ldots
\end{itemize}

\end{frame}
