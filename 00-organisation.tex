
\begin{frame}{Contact}
\begin{itemize}
\item Chargé de cours:
\begin{itemize}
\item Pierre Geurts, \url{p.geurts@ulg.ac.be}, I141 Montefiore, 04/3664815
\end{itemize}
\item Assistants:
\begin{itemize}
\item Gilles Louppe, \url{g.louppe@ulg.ac.be}, GIGA-R (B34,+1), CHU, 04/3662766
%\item Julien Becker, \url{j.becker@ulg.ac.be}, GIGA-R (B34,+1), CHU, 04/3669805
\end{itemize}
\item Sites web du cours:
\begin{itemize}
\item Cours théorique:
{\small \url{http://www.montefiore.ulg.ac.be/~geurts/Cours/sda/2012/sda2012_2013.html}}
\item Répétitions et projets:
{\small \url{http://www.montefiore.ulg.ac.be/~glouppe/2012-2013/students.info0902.php}}
\end{itemize}
\end{itemize}
\end{frame}

\begin{frame}{Objectif du cours}

\begin{itemize}
\item Introduction à l'étude systématique des algorithmes et des
  structures de données
\item Vous fournir une boîte à outils contenant:
\begin{itemize}
\item Des structures de données permettant d'organiser et d'accéder efficacement
  aux données
\item Les algorithmes les plus populaires
\item Des méthodes génériques pour la modélisation, l'analyse et la résolution de problèmes algorithmiques
\end{itemize}
\item On insistera sur la généralité des algorithmes et structures de données et on les étudiera de manière formelle
\item Les projets visent à vous familiariser à la résolution de problèmes
\end{itemize}

\note{
Maintenant qu'on est débarassé de l'apprentissage, on va pouvoir se focaliser sur l'algorithmique. Vous avez appris à écrire des programmes, on va apprendre à écrire des algorithmes: procédure de résolution de problèmes.

Suite du cours d'introduction à l'informatique et précède le cours de techniques de programmation que vous aurez pour certains l'an prochain.
}
\end{frame}

\begin{frame}{Organisation du cours}
\begin{itemize}
\item Cours théoriques:
\begin{itemize}
\item Les vendredis de 13h30 à 15h30, S94, Bâtiment B4 (Europe).
\item 10-12 cours
%\item Transparents disponibles sur la page web du cours avant chaque cours
\end{itemize}
\item Répétitions:
\begin{itemize}
\item Certains vendredis de 15h30 à 17h30, S94, Bâtiment B4 (Europe)
\item Exercices portant sur la matière théorique + debriefing des projets
\end{itemize}
\item Projets (sujet à modifications):
\begin{itemize}
\item Trois projets tout au long de l'année, de difficulté croissante
\item Les deux premiers individuels, le troisième en binôme
\item En C
\end{itemize}

\bigskip
%\item Assistance au cours théorique n'est pas requise mais fortement conseillée.
\item Evaluation sur base des projets (30\%) et d'un examen écrit (70\%).
\end{itemize}
\note{
Le cours théorique sera totalement indépendant du langage de programmation. On utilisera ce qu'on appelle le pseudo-code

%\bigskip

Demander s'ils connaissant tous bien le C.

%\bigskip

Si certains points vus au cours théorique ne sont pas clairs, on peut
organiser une répétition mais il est préférable que vous posiez vos
questions directement au cours. Avant chaque cours, séance de
questions-réponses sur la séance précédente.

%\bigkskip

En plus de ça, il est recommandé d'essayer d'implémenter et de tester
les algorithmes et structures de données vues au cours.  }
\end{frame}


\begin{frame}{Notes de cours}
\begin{itemize}
\item Transparents disponibles sur la page web du cours.
\item Pas de livre de référence obligatoire mais le cours se base fortement sur l'ouvrage suivant:
{\small
\begin{itemize}
\item {\bf Introduction to algorithms, Cormen, Leiserson, Rivest, Stein, MIT press, Third edition, 2009.}
\begin{itemize}
\item \url{http://mitpress.mit.edu/algorithms/}
\end{itemize}
\end{itemize}
\item Autres références:
\begin{itemize}
\item Algorithms, Sedgewick and Wayne, Addison Wesley, Fourth edition, 2011.
\begin{itemize}
\item \url{http://algs4.cs.princeton.edu/home/}
\end{itemize}
\item Data structures and algorithms in Java, Goodrich and Tamassia, Fifth edition, 2010.
\begin{itemize}
\item \url{http://ww0.java4.datastructures.net/}
\end{itemize}
\item Algorithms, Dasgupta, Papadimitriou, and Vazirani, McGraw-Hill, 2006.
\begin{itemize}
\item \url{http://cseweb.ucsd.edu/users/dasgupta/book/index.html}
\item \url{http://www.cs.berkeley.edu/~vazirani/algorithms/all.pdf}
\end{itemize}
\end{itemize}
}
\end{itemize}
\note{Bouquin de référence: CLRS (pas obligatoire, mais vous avez peu de chance de regretter votre achat). Plus un ouvrage de réference qu'un ouvrage pédagogique. On ne verra qu'une toute petite partie. Une version française existe mais je ne l'ai pas et je me baserai sur la version anglaise
}
\end{frame}

\begin{frame}{Cours sur le web}
Ce cours s'inspire également de plusieurs cours disponibles sur le web:
\begin{itemize}
\item Antonio Carzaniga, Faculty of Informatics, University of Lugano
\begin{itemize}
\item \url{http://www.inf.usi.ch/carzaniga/edu/algo/index.html}
\end{itemize}
\item Marc Gaetano, Polytechnique, Nice-Sophia Antipolis
\begin{itemize}
\item \url{http://users.polytech.unice.fr/~gaetano/asd/}
\end{itemize}
\item Robert Sedgewick, Princeton University
\begin{itemize}
\item \url{http://www.cs.princeton.edu/courses/archive/spr10/cos226/lectures.html}
\end{itemize}
\item Charles Leiserson and Erik Demaine, MIT.
\begin{itemize}
\item \url{http://ocw.mit.edu/courses/electrical-engineering-and-computer-science/6-046j-introduction-to-algorithms-sma-5503-fall-2005/index.htm}
\end{itemize}
\item Le cours de 2009-2010 de Bernard Boigelot
\end{itemize}
\end{frame}
