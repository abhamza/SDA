
\begin{frame}{Introduction}

\begin{itemize}
\item Cours th�orique: totalement ind�pendant du langage de programmation (utilisation de pseudo-code)
\item TP: trois projets de difficult� et d'importance progressive
\item Tr�s fortement sugg�r� d'impl�menter et de tester les structures de donn�es vues tout au long du cours (bonne mani�re de s'assurer qu'on a bien compris les algorithmes).
\item Place dans le cursus: suite de INFO2009, pr�requis pour cours de Laurent Mathy.
\item assistance au cours th�orique n'est pas requise mais fortement conseill�e.
\item si certains points vus au cours th�orique ne sont pas clairs, on peut organiser une r�p�tition (Gilles Louppe) mais il est pr�f�rable que vous posiez vos questions directement au cours. Avant chaque cours, s�ance de questions-r�ponses sur la s�ance pr�c�dente.
\end{itemize}

\end{frame}

\begin{frame}{Ouvrages de r�f�rence}
Pas de livre de r�f�rence obligatoire mais certains conseill�s:
\begin{itemize}
\item CLRS: bible du domaine. Si vous ne devez en acheter qu'un...
% Bouquin de r�f�rence: CLRS (pas obligatoire, mais vous avez peu de chance de regretter votre achat). Plus un ouvrage de r�ference qu'un ouvrage p�dagogique. On ne verra qu'une toute petite partie. Une version fran�aise existe mais je ne l'ai pas et je me baserai sur la version anglaise
\item Algorithms: tr�s didactique, orient� java
\item Data structures and algorithms in Java
\end{itemize}
\end{frame}

\begin{frame}{Inspiration}
\begin{itemize}
\item Ce cours est bas� sur plusieurs cours disponibles sur internet:
\begin{itemize}
\item Antonio Carzaniga
\item Robert Sedgewick
\item Gaetano
\item Ulrich Klehmet
\item Wikipedia !
\end{itemize}
\end{itemize}
\end{frame}

\begin{frame}{Objectif du cours}

\begin{itemize}
\item Apprendre � r�soudre des probl�mes algorithmiques (faire du
  design d'algorithmes)
\item Fournir une bo�te � outil de base pour vous aider � cela
  (structures de donn�es)
\end{itemize}

\end{frame}
