
\begin{frame}{Introduction}

\begin{itemize}
\item Cours théorique: totalement indépendant du langage de programmation (utilisation de pseudo-code)
\item TP: trois projets de difficulté et d'importance progressive
\item Très fortement suggéré d'implémenter et de tester les structures de données vues tout au long du cours (bonne manière de s'assurer qu'on a bien compris les algorithmes).
\item Place dans le cursus: suite de INFO2009, prérequis pour cours de Laurent Mathy.
\item assistance au cours théorique n'est pas requise mais fortement conseillée.
\item si certains points vus au cours théorique ne sont pas clairs, on peut organiser une répétition (Gilles Louppe) mais il est préférable que vous posiez vos questions directement au cours. Avant chaque cours, séance de questions-réponses sur la séance précédente.
\end{itemize}

\end{frame}

\begin{frame}{Ouvrages de référence}
Pas de livre de référence obligatoire mais certains conseillés:
\begin{itemize}
\item CLRS: bible du domaine. Si vous ne devez en acheter qu'un...
% Bouquin de référence: CLRS (pas obligatoire, mais vous avez peu de chance de regretter votre achat). Plus un ouvrage de réference qu'un ouvrage pédagogique. On ne verra qu'une toute petite partie. Une version française existe mais je ne l'ai pas et je me baserai sur la version anglaise
\item Algorithms: très didactique, orienté java
\item Data structures and algorithms in Java
\end{itemize}
\end{frame}

\begin{frame}{Inspiration}
\begin{itemize}
\item Ce cours est basé sur plusieurs cours disponibles sur internet:
\begin{itemize}
\item Antonio Carzaniga
\item Robert Sedgewick
\item Gaetano
\item Ulrich Klehmet
\item Wikipedia !
\end{itemize}
\end{itemize}
\end{frame}

\begin{frame}{Objectif du cours}

\begin{itemize}
\item Apprendre à résoudre des problèmes algorithmiques (faire du
  design d'algorithmes)
\item Fournir une boîte à outil de base pour vous aider à cela
  (structures de données)
\end{itemize}

\end{frame}
