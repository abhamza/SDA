\documentclass[a4paper,10pt]{article}

% Packages
\usepackage[utf8]{inputenc}
\usepackage[frenchb]{babel}
\usepackage{graphicx}
\usepackage{url}
\usepackage{hyperref}
\usepackage{a4wide}
\usepackage{amsmath}
\usepackage{verbatim}
\usepackage{array}
\usepackage{../clrscode3epg}
\renewcommand{\labelenumi}{(\alph{enumi})}

% Style
\parskip=\smallskipamount

% En-têtes
\title{
    \textbf{Structures de données et algorithmes}\\
    Répétition 7: Résolution de problèmes (suite)
}

\author{Gilles \textsc{Louppe} -- \url{http://www.montefiore.ulg.ac.be/~glouppe}}
\date{4 mai 2012}

% Corps
\begin{document}
\maketitle

\section*{Exercice 1}

Soit une chaîne de caractère \texttt{S} au format ASCII.

\begin{enumerate}
\item Ce codage est-il optimal en termes d'espace mémoire utilisé? Illustrer par un exemple.
\item Comment réduire la mémoire utilisée?
\item Proposer un algorithme glouton pour compresser \texttt{S}.
\item Quel serait le codage de \texttt{S} pour l'exemple proposé en (a)?
\end{enumerate}

\section*{Exercice 2}

% http://people.csail.mit.edu/bdean/6.046/dp/

Soit une expression booléenne composée des symboles \texttt{true}, \texttt{false}, \texttt{and} et \texttt{or}. Proposer un algorithme qui détermine le nombre de façon de parenthèser l'expression de sorte qu'elle soit évaluée à \texttt{true}.

\section*{Exercice 3 (B. Boigelot, 2009)}

On considère un terrain de jeu rectangulaire possédant $n.m$ cases organisées en
$n$ lignes et $m$ colonnes. Des sommes d'argent sont initialement placées sur
ces cases. Un joueur traverse le terrain à partir du coin supérieur gauche
jusqu'au coin inférieur droit. A chaque étape de ce trajet, le joueur collecte
la somme d'argent placée sur la case où il se trouve, et a ensuite le choix de
se déplacer d'une case soit vers la droite, soit vers le bas (les déplacements
vers la gauche, vers le haut ou en diagonale sont interdits).

\begin{enumerate}
\item Ecrire un algorithme capable de calculer une suite de mouvements qui permet au joueur d'atteindre la case inférieure droite en ayant accumulé le plus d'argent possible.
\item Calculer la complexité en temps et en espace de l'algorithme proposé.
\end{enumerate}

\end{document}
