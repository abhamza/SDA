\documentclass[a4paper,10pt]{article}

% Packages
\usepackage[utf8]{inputenc}
\usepackage[frenchb]{babel}
\usepackage{graphicx}
\usepackage{url}
\usepackage{hyperref}
\usepackage{a4wide}
\usepackage{amsmath}
\usepackage{verbatim}
\usepackage{array}
\usepackage{../clrscode3epg}
\renewcommand{\labelenumi}{(\alph{enumi})}

% Style
\parskip=\smallskipamount

% En-têtes
\title{
    \textbf{Structures de données et algorithmes}\\
    Répétition 6: Arbres de recherche
}

\author{Gilles \textsc{Louppe} -- \url{http://www.montefiore.ulg.ac.be/~glouppe}}
\date{29 mars 2013}

% Corps
\begin{document}
\maketitle

\section*{Exercice 1}

Soit un arbre binaire de recherche \texttt{T} initialement vide.

\begin{enumerate}

\item Déterminez graphiquement l'arbre qui résulte de l'insertion des clés 30, 40, 24, 58, 48, 26, 11, 13, 35 et 36 (dans cet ordre).

\item Déterminez ensuite graphiquement l'arbre qui résulte de la suppression des clés 13, 58, et 30 (dans cet ordre).

\end{enumerate}

\section*{Exercice 2}

Le parcours préfixe d'un arbre binaire de recherche donne :
\begin{verbatim}
A B C - - D - - E - F - -
\end{verbatim}

En donner les parcours infixe et postfixe. Les lettres correspondent aux
noeuds internes et les tirets à des feuilles de l'arbre.

\section*{Exercice 3}

Soit un arbre binaire de recherche \texttt{T}.

\begin{enumerate}

\item Ecrire une méthode \texttt{getHeight(T)} qui détermine la hauteur de l'arbre \texttt{T}.
\item Ecrire une méthode \texttt{isComplete(T)} qui détermine si l'arbre \texttt{T} est complet ou non.
\item Ecrire une méthode \texttt{isBalanced(T)} qui détermine si l'arbre \texttt{T} est équilibré.
\item Ecrire une méthode \texttt{mirror(T)} qui retourne l'arbre miroir de l'arbre \texttt{T}.
\item Ecrire une méthode \texttt{printLowerThan(T, x)} qui imprime les valeurs contenues dans \texttt{T} inférieures ou égales à $x$.

\end{enumerate}

\textit{Bonus:} Pour chaque fonction, décrivez un algorithme a) récursif et b) non-récursif.

\section*{Exercice 4}

Prouver que le nombre de feuilles d'un arbre binaire est égal au nombre de noeuds de degré 2, plus 1.

\section*{Exercice 5}

Soit un ensemble de $N$ valeurs entières distinctes. Ecrivez une fonction calculant le nombre d'arbres binaires de recherche distincts qu'il est possible de construire à partir de ces $N$ valeurs.

\section*{Exercice 6}

Si insère un élement $z$ dans un arbre \texttt{T} (via \texttt{TreeInsert(T, z)}) et puis qu'on supprime cet élément (via \texttt{TreeDelete(T, z)}), l'arbre \texttt{T} qui en résulte est-il nécessairement identique à l'arbre de départ?

\section*{Exercice 7}

La suppression d'une valeur dans un arbre binaire de recherche peut-elle augmenter la hauteur de cet arbre?

\end{document}
