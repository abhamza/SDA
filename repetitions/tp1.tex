\documentclass[a4paper,10pt]{article}

% Packages
\usepackage[utf8]{inputenc}
\usepackage[frenchb]{babel}
\usepackage{graphicx}
\usepackage{url}
\usepackage{hyperref}
\usepackage{a4wide}
\usepackage{amsmath}
\usepackage{../clrscode3epg}
\renewcommand{\labelenumi}{(\alph{enumi})}

% Style
\parskip=\smallskipamount

% En-têtes
\title{
    \textbf{Structures de données et algorithmes}\\
    Répétition 1: Pseudo-code et récursion
}

\author{Gilles \textsc{Louppe} -- \url{http://www.montefiore.ulg.ac.be/~glouppe}}
\date{8 février 2013}

% Corps
\begin{document}
\maketitle

\section*{Exercice 1}

Que fait cette fonction?

\begin{codebox}
    \Procname{$\proc{Mystère}(A)$}
    \li \If $A.length < 2$
    \li \Then   \Return True
    \li \Else
    \li       \If $A[1] == A[A.length]$
    \li       \Then \Return $\proc{Mystère}(A[2..A.length-1])$
    \li       \Else
    \li             \Return False
              \End
        \End
    \End
\end{codebox}
\vspace{0.5cm}

\section*{Exercice 2}

\begin{enumerate}
\item Ecrire le pseudo-code d'une fonction récursive permettant de calculer le $n$-ième terme de la suite de Fibonacci
\begin{align*}
F_0 &= 0\text{, } F_1 = 1 \\
F_n &= F_{n-2} + F_{n-1} \text{ si } n > 1
\end{align*}
Réécrire ensuite cette fonction de façon itérative.

\item Ecrire le pseudo-code d'une fonction récursive permettant de calculer le $n$-ième terme de la suite de Padovan
\begin{align*}
P_{n+3} &= P_{n+1} + P_n \text{ avec } P_0 = P_1 = P_2 = 1
\end{align*}
Réécrire ensuite cette fonction de façon itérative.

\item Ecrire le pseudo-code d'une fonction itérative permettant de déterminer la valeur minimale des éléments d'un tableau. Réécrire ensuite cette fonction de façon récursive.

\item Ecrire le pseudo-code d'une fonction itérative permettant de déterminer la valeur moyenne des éléments d'un tableau. Réécrire ensuite cette fonction de façon récursive.

\item Ecrire le pseudo-code d'une fonction récursive permettant d'approximer le nombre d'Or défini par la fraction suivante:
\begin{align*}
\phi = 1 + \frac{1}{1 + \frac{1}{1 + \frac{1}{1 + ...}}}
\end{align*}
Réécrire ensuite cette fonction de façon itérative.

\end{enumerate}

\section*{Exercice 3}

Quelle est la sortie de l'appel de fonction $\proc{Ex3(6)}$?

\begin{codebox}
    \Procname{$\proc{Ex3}(n)$}
    \li \If $n <= 0$
    \li \Then   \Return
    \li \Else
    \li       \textbf{print} $n$
    \li       $\proc{Ex3}(n-2)$
    \li       $\proc{Ex3}(n-2)$
    \li       \textbf{print} $n$
        \End
    \End
\end{codebox}
\vspace{0.5cm}

\section*{Exercice 4}

\begin{itemize}
\item Ecrire le pseudo-code d'une fonction permettant d'imprimer à l'écran toutes les permutations d'un tableau $A$.
\item Ecrire le pseudo-code d'une fonction permettant d'imprimer à l'écran toutes les combinaisons d'éléments pris dans tableau $A$.
\end{itemize}

\section*{Exercice 5}

Ecrire le pseudo-code d'une fonction permettant de dessiner un triangle de Sierpinski de profondeur $n$.

\begin{figure}[h]
    \center
    \includegraphics[scale=0.25]{triangle.png}
    \caption{Triangle de Sierpinski}
\end{figure}

\section*{Exercice 6}

\begin{enumerate}

\item Déterminer la valeur de $M(n)$ pour toute valeur de $n$.

\begin{align*}
M(n) &= \left\{
    \begin{array}{l l}
        n - 10 & \text{si } n > 100 \\
        M(M(n+11)) & \text{si } n \leq 100\\
    \end{array} \right.
\end{align*}

\item Ecrire le pseudo-code d'une fonction récursive implémentant $M(n)$.
\item Réécrire cette fonction de façon itérative.

\end{enumerate}

\section*{Exercice 7}

On dispose d'une pile de crèpes bien chaudes qu'on souhaite trier de la plus
grande à la plus petite. Pour ce faire, la seule opération permise est
de retourner un paquet de crèpes pris à partir du sommet de la pile.
Décrivez une procédure pour trier la pile de crèpes.

\textit{Variante:} Chaque crèpe est brûlée d'un côté. Comment trier la pile de sorte
que le côté brûlé de chaque crèpe soit en dessous?

\textit{Note:} Les crèpes ne peuvent pas être mangées.


\end{document}
