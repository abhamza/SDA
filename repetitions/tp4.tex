\documentclass[a4paper,10pt]{article}

% Packages
\usepackage[utf8]{inputenc}
\usepackage[frenchb]{babel}
\usepackage{graphicx}
\usepackage{url}
\usepackage{hyperref}
\usepackage{a4wide}
\usepackage{amsmath}
\usepackage{verbatim}
\usepackage{array}
\usepackage{../clrscode3epg}
\renewcommand{\labelenumi}{(\alph{enumi})}

% Style
\parskip=\smallskipamount

% En-têtes
\title{
    \textbf{Structures de données et algorithmes}\\
    Répétition 4: Liste liée, pile et file
}

\author{Gilles \textsc{Louppe} -- \url{http://www.montefiore.ulg.ac.be/~glouppe}}
\date{23 mars 2012}

% Corps
\begin{document}
\maketitle

\section*{Exercice 1}

\begin{enumerate}

\item Soient deux noeuds \texttt{n} et \texttt{m} de listes simplement liées. Que provoquent les opérations suivantes sur la liste auquel \texttt{n} appartient (mais pas \texttt{m})?

\begin{itemize}

\item
\begin{verbatim}
n.next = n.next.next
\end{verbatim}

\item
\begin{verbatim}
m.next = n.next
n.next = m
\end{verbatim}

\item
\begin{verbatim}
n.next = m
m.next = n.next
\end{verbatim}

\end{itemize}

\item Soit \texttt{S} une liste simplement liée. Ecrire une fonction \texttt{AvantAvantDernier(S)} permettant de récupérer l'avant-avant dernier noeud de \texttt{S}.

\item Soit \texttt{S} une liste simplement liée. Ecrire une fonction \texttt{Reverse(S)} permettant d'inverser la liste \texttt{S}.

\end{enumerate}

\section*{Exercice 2}

\begin{enumerate}
\item Soit une pile \texttt{S}. Décrire la sortie des opérations suivantes. Quel est l'état de \texttt{S} en fin d'exécution?

\begin{verbatim}
push(S, 5)
push(S, 3)
pop(S)
push(S, 2)
push(S, 8)
pop(S)
pop(S)
push(S, 9)
push(S, 1)
pop(S)
push(S, 7)
push(S, 6)
pop(S)
pop(S)
push(S, 4)
pop(S)
pop(S)
\end{verbatim}

\item Soit une file \texttt{Q}. Décrire la sortie des opérations suivantes. Quel est l'état de \texttt{Q} en fin d'exécution?

\begin{verbatim}
enqueue(Q, 5)
enqueue(Q, 3)
dequeue(Q)
enqueue(Q, 2)
enqueue(Q, 8)
dequeue(Q)
dequeue(Q)
enqueue(Q, 9)
enqueue(Q, 1)
dequeue(Q)
enqueue(Q, 7)
enqueue(Q, 6)
dequeue(Q)
dequeue(Q)
enqueue(Q, 4)
dequeue(Q)
dequeue(Q)
\end{verbatim}

\end{enumerate}

\section*{Exercice 3}

Réécrire les fonctions \texttt{Enqueue(Q, x)} et \texttt{Dequeue(Q)} du cours de sorte à gérer les erreurs. Considérer l'implémentation par tableau et l'implémentation par liste liée.

\section*{Exercice 4}

Implémenter un buffer \texttt{B} d'éditeur de texte qui correspond à l'ADT suivant.

\begin{itemize}
\item \texttt{insert(B, c)}: insérér un caractère \texttt{c} à la position courante du curseur;
\item \texttt{delete(B)}: efface le caractère à la position courante du curseur et le retourne;
\item \texttt{left(B)}: déplace le curseur d'une position vers la gauche;
\item \texttt{right(B)}: déplace le curseur d'une position vers la droite;
\item \texttt{begin(B)}: place le curseur au début du buffer;
\item \texttt{end(B)}: déplace le curseur à la fin du buffer;
\item \texttt{size(B)}: retourne la taille du buffer;
\item \texttt{get(B, i)}: retourne le \texttt{i}-ème caractère du buffer.
\end{itemize}

\section*{Exercice 5}

Implémenter un algorithme non-récursif permettant de calculer la valeur d'une expression arithmétique en notation postfixe.

\end{document}
