\documentclass[a4paper,10pt]{article}

% Packages
\usepackage[utf8]{inputenc}
\usepackage[frenchb]{babel}
\usepackage{graphicx}
\usepackage{url}
\usepackage{hyperref}
\usepackage{a4wide}
\usepackage{amsmath}
\usepackage{verbatim}
\usepackage{array}
\usepackage{../clrscode3epg}
\renewcommand{\labelenumi}{(\alph{enumi})}

% Style
\parskip=\smallskipamount

% En-têtes
\title{
    \textbf{Structures de données et algorithmes}\\
    Répétition 8: Graphes
}

\author{Gilles \textsc{Louppe} -- \url{http://www.montefiore.ulg.ac.be/~glouppe}}
\date{11 mai 2012}

% Corps
\begin{document}
\maketitle

\section*{Exercice 1}

\begin{enumerate}
\item A quoi ressemble un graphe dont tous les sommets sont de degré 1 exactement?
\item A quoi ressemble un graphe dont tous les sommets sont de degré 2 exactement?
\item Si un graphe non orienté possède $n$ sommets, tous de degré $d$, combien comporte t-il d'arêtes?
\end{enumerate}

\section*{Exercice 2}

Soit un graphe $G$ dont $A$ est la matrice d'adjacence.

\begin{enumerate}
\item Caractériser le graphe $G'$ dont $A^T$ est la matrice d'adjacence.
\item Quelle interprétation peut-on faire du produit $AA$?
\item Quelle interprétation peut-on faire du produit $A^k$?
\end{enumerate}

\section*{Exercice 3}

Pour les cas suivants, est-il plus approprié d'utiliser un graphe implémenté par une liste d'adjacences ou par une matrice d'adjacence? Justifier.

\begin{enumerate}
\item Le graphe possède 10000 sommets et 20000 arêtes et on souhaite minimiser l'espace mémoire utilisé.
\item Le graphe possède 10000 sommets et 2000000 arêtes et on souhaite minimiser l'espace mémoire utilisé.
\item On souhaite déterminer le plus rapidement possible si deux sommets sont adjacents, peu importe l'espace mémoire requis.
\end{enumerate}

\section*{Exercice 4}

Illustrer graphiquement l'exécution de l'algorithme de Dijkstra sur le graphe de la Figure 1. On souhaite trouver tous les plus courts chemins à partir du sommet BWI.

\begin{figure}
    \center
    \includegraphics[scale=0.5]{graphe.png}
    \caption{Graphe}
\end{figure}

\section*{Exercice 5}

Dans certaines applications (les réseaux notamment), une architecture de graphes
est utilisée et un des noeuds joue souvent un rôle spécial par rapport aux
autres (e.g., un serveur de fichiers au sein d'un réseau d'ordinateurs). Pour
obtenir des performances optimales, on souhaiterait déterminer ce noeud comme le
"centre" du graphe. Pour cela, on définit, étant donné un graphe $G$ et un noeud
$v$, l'excentricité de $v$ comme la longueur du plus long des plus courts
chemins entre $v$ et tous les autres noeuds de $G$. Le centre de $G$ est défini
comme le noeud d'excentricité minimale.

\begin{enumerate}
\item Proposer un algorithme pour calculer le centre d'un graphe $G$.
\item Quelle est sa complexité si $G$ est implémenté par une liste d'adjacences?
\item  Quelle est sa complexité si $G$ est implémenté par une matrice d'adjacence?
\end{enumerate}

\end{document}
