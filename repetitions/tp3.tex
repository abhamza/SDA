\documentclass[a4paper,10pt]{article}

% Packages
\usepackage[utf8]{inputenc}
\usepackage[frenchb]{babel}
\usepackage{graphicx}
\usepackage{url}
\usepackage{hyperref}
\usepackage{a4wide}
\usepackage{amsmath}
\usepackage{verbatim}
\usepackage{array}
\usepackage{../clrscode3epg}
\renewcommand{\labelenumi}{(\alph{enumi})}

% Style
\parskip=\smallskipamount

% En-têtes
\title{
    \textbf{Structures de données et algorithmes}\\
    Répétition 3: Preuves et correction d'algorithmes
}

\author{Gilles \textsc{Louppe} -- \url{http://www.montefiore.ulg.ac.be/~glouppe}}
\date{9 mars 2012}

% Corps
\begin{document}
\maketitle

\section*{Exercice 1}

L'algorithme suivant est-il correct? Partiellement? Totalement?

\begin{codebox}
    \li \Comment $\{B \geq 0\}$
    \li $x\gets A$
    \li $y\gets B$
    \li $product\gets 0$
    \li \While $y > 0$
    \li \Do     $product\gets product + x$
    \li         $y = y - 1$
        \End
    \li \Comment $\{product = A \times B\}$
    \End
\end{codebox}
\vspace{10pt}

\section*{Exercice 2}

Déterminer la post-condition de l'algorithme suivant. Montrer ensuite que l'algorithme est totalement correct.

\begin{codebox}
    \li \Comment $\{m = A \geq 0\}$
    \li $r\gets 0$
    \li \While $(r+1)*(r+1)\leq m$
    \li \Do     $r\gets r+1$
        \End
    \li \Comment $\{Post-condition\}$
    \End
\end{codebox}
\vspace{10pt}

\section*{Exercice 3}

Montrer que l'algorithme suivant est totalement correct.

\begin{codebox}
    \li \Comment $\{N\geq 0\}$
    \li $sum\gets 0$
    \li $exp\gets 0$
    \li $term\gets 1$
    \li \While $exp < N$
    \li \Do     $sum\gets sum + term$
    \li         $exp = exp + 1$
    \li         $term = term*2$
        \End
    \li \Comment $\{sum = 2^N - 1\}$
\end{codebox}
\vspace{10pt}


\pagebreak
\section*{Exercice 4}

\begin{enumerate}
\item Que fait l'algorithme suivant?
\item Déterminer sa complexité.
\item Montrer qu'il est totalement correct.
\end{enumerate}

\vspace{10pt}
\begin{codebox}
    \Procname{$\proc{Power}(x, n)$}
    \li \If $n == 0$
    \li \Then   \Return $1$
    \li \ElseIf $odd(n)$
    \li \Then   \Return $x * \textsc{Power}(x, n \div 2)^2$
    \li \Else
    \li \Return $\textsc{Power}(x, n \div 2)^2$
        \End
\end{codebox}
\vspace{10pt}

\textit{Remarques}: La fonction $odd(n)$ détermine si $n$ est impair. L'opérateur $\div$ est l'opérateur de division entière.


% - Montrer que MERGE(A, p, q, r) est correct par un invariant

\end{document}
