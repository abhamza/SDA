\documentclass[a4paper,10pt]{article}

% Packages
\usepackage[utf8]{inputenc}
\usepackage[frenchb]{babel}
\usepackage{graphicx}
\usepackage{url}
\usepackage{hyperref}
\usepackage{a4wide}
\usepackage{amsmath}
\usepackage{verbatim}
\usepackage{array}
\usepackage{../clrscode3epg}
\renewcommand{\labelenumi}{(\alph{enumi})}

% Style
\parskip=\smallskipamount

% En-têtes
\title{
    \textbf{Structures de données et algorithmes}\\
    Répétition 5: Tas et arbres
}

\author{Gilles \textsc{Louppe} -- \url{http://www.montefiore.ulg.ac.be/~glouppe}}
\date{20 avril 2012}

% Corps
\begin{document}
\maketitle

\section*{Exercice 1}

Dessiner tous les tas-\textsc{min} possibles avec l'ensemble des clés $\{A, B, C, D, E\}$, où chaque clé n'apparaît qu'une seule fois.

\section*{Exercice 2}

Soit un tas-\textsc{min} \texttt{H}. Illustrer l'exécution des opérations suivantes:

\begin{verbatim}
heapInsert(H, 5)
heapInsert(H, 4)
heapInsert(H, 7)
heapInsert(H, 1)
heapExtractMin(H)
heapInsert(H, 3)
heapInsert(H, 6)
heapExtractMin(H)
heapExtractMin(H)
heapInsert(H, 8)
heapExtractMin(H)
heapInsert(H, 2)
heapExtractMin(H)
heapExtractMin(H)
\end{verbatim}

\section*{Exercice 3}

\begin{enumerate}

\item Comment implémenter une pile au moyen d'une file à priorité?

\item Comment implémenter une file au moyen d'une file à priorité?

\item Comment implémenter une file aléatoire au moyen d'une file à priorité?

\end{enumerate}

\section*{Exercice 4}

Ecrire un algorithme \texttt{printHeapInOrder(H)} permettant d'imprimer les clés
contenues dans le tas-\textsc{min} \texttt{H} selon l'ordre croissant. La
structure du tas \texttt{H} ne peut pas être modifiée. L'espace mémoire supplémentaire autorisé est $O(1)$.

\section*{Exercice 5}

Soit un arbre binaire de recherche \texttt{T} initialement vide.

\begin{enumerate}

\item Déterminez graphiquement l'arbre qui résulte de l'insertion des clés 30, 40, 24, 58, 48, 26, 11, 13, 35 et 36 (dans cet ordre).

\item Déterminez ensuite graphiquement l'arbre qui résulte de la suppression des clés 13, 58, et 30 (dans cet ordre).

\end{enumerate}

\section*{Exercice 6}

Le parcours préfixe d'un arbre binaire de recherche donne :
\begin{verbatim}
A B C - - D - - E - F - -
\end{verbatim}

En donner les parcours infixe et postfixe. Les lettres correspondent aux
noeuds internes et les tirets à des feuilles de l'arbre.

\section*{Exercice 7}

Soit un arbre binaire de recherche \texttt{T}.

\begin{enumerate}

\item Ecrire une méthode \texttt{getHeight(T)} qui détermine la hauteur de l'arbre \texttt{T}.

\item Ecrire une méthode \texttt{isComplete(T)} qui détermine si l'arbre \texttt{T} est complet ou non.

\item Ecrire une méthode \texttt{isBalanced(T)} qui détermine si l'arbre \texttt{T} est équilibré.

\item Ecrire une méthode \texttt{mirror(T)} qui retourne l'arbre miroir de l'arbre \texttt{T}.

\end{enumerate}

\section*{Exercice 8}

Prouver que le nombre de feuilles d'un arbre binaire est égal au nombre de noeuds de degré 2, plus 1.

\end{document}
