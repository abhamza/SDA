
\part{Introduction}

\section{Algorithms $+$ Data structures $=$ Programs (\footnotesize{Niklaus Wirth})}

\begin{frame}{Plan}

\tableofcontents

\note{la citation vient du titre d'un bouquin de 1976 dont le message principal était que les structures de données et les algorithmes étaient intimmement liés}
\end{frame}

\begin{frame}{Algorithmes}

\begin{itemize}
\item Un {\bf algorithme} est une suite {\em finie} et {\em non-ambiguë} d'opérations ou d'instructions permettant de résoudre un {\em problème}
\item Provient du nom du mathématicien persan {\em Al-Khawarizmi}
  ($\pm 820$), le père de l'algèbre
\item Un problème algorithmique est souvent formulé comme la
  transformation d'un ensemble de valeurs, {\bf d'entrée}, en un
  nouvel ensemble de valeurs, {\bf de sortie}.
\item Exemples d'algorithmes:
\begin{itemize}
\item Une recette de cuisine (ingrédients $\rightarrow$ plat préparé)
\item La recherche dans un dictionnaire (mot $\rightarrow$ définition)
\item La division entière (deux entiers $\rightarrow$ leur quotient)
\item Le tri d'une séquence (séquence $\rightarrow$ séquence ordonnée)

\end{itemize}
\end{itemize}

\note{dictionnaire: structure de donnée: le dictionnaire: mots écrits
  en français et surtout, classés par ordre alphabétique}
\end{frame}

\begin{frame}{Algorithmes}
\begin{itemize}
\item On étudiera essentiellement les algorithmes {\bf corrects}.
\begin{itemize}
\item Un algorithme est (totalement) {\em correct} lorsque pour chaque instance, il se termine en produisant la bonne sortie.
\item Il existe également des algorithmes {\em partiellement corrects} dont la terminaison n'est pas assurée mais qui fournissent la bonne sortie lorsqu'ils se terminent.
\item Il existe également des algorithmes {\em approximatifs} qui fournissent une sortie inexacte mais néanmoins proche de l'optimum.
\end{itemize}
\bigskip
\item Les algorithmes seront évalués en termes d'{\em utilisation de ressources}, essentiellement par rapport aux {\bf temps de calcul} mais aussi à l'utilisation de la {\bf mémoire}.
\end{itemize}
\end{frame}

\begin{frame}{Algorithmes}

Un algorithme peut être spécifié de différentes manières:
\begin{itemize}
\item en langage naturel,
\item graphiquement,
\item en pseudo-code,
\item par un programme écrit dans un langage informatique
\item ...
\end{itemize}
La seule condition est que la description soit précise.

\end{frame}

% objectif: présenter le pseudo-code

\begin{frame}{Exemple: le tri}
\begin{itemize}
\item Le problème de tri:
\begin{itemize}
\item Entrée: une séquence de $n$ nombres $\langle a_1,a_2,\ldots,a_n\rangle$
\item Sortie: une permutation de la séquence de départ $\langle a_1',a_2',\ldots,a_n'\rangle$ telle que $a_1'\leq a_2'\leq\ldots\leq a_n'$
\end{itemize}
\bigskip
\item Exemple:
\begin{itemize}
\item Entrée: $\langle 31,41,59,26,41,58\rangle$
\item Sortie: $\langle 26,31,41,41,58,59\rangle$
\end{itemize}
\end{itemize}

\end{frame}

\begin{frame}{Tri par insertion}

\hfill\includegraphics[height=3cm]{Figures/CLRS/Fig-2-1.pdf}

Description en langage naturel:

\bigskip

On parcourt la séquence de gauche à droite

\bigskip

Pour chaque élément $a_j$:
\begin{itemize}
\item On l'{\alert{insère}} à sa position dans une nouvelle séquence ordonnée contenant les éléments le précédant dans la séquence.
\end{itemize}
On s'arrête dès que le dernier élément a été inséré à sa place dans la séquence.

\note{Un peu l'inverse du tri par sélection: on recherche le minimum de la liste et on le place au sommet}
\end{frame}

\begin{frame}
\frametitle{Tri par insertion}
\centerline{\includegraphics[width=8cm]{Figures/01-insertionsort.pdf}}
\end{frame}

\begin{frame}[fragile]
\frametitle{Tri par insertion}

Description en C (sur des tableaux d'entiers):

%\fcolorbox{white}{Lightgray}{
{\small
\begin{verbatim}
void InsertionSort (int *a, int length) {
  int key, i;
  for(int j = 1; j < length; j++) {
    key = a[j];
    /* Insert a[j] into the sorted sequence a[0...j-1] */
    i = j-1;
    while (i>=0 && a[i]>key) {
      a[i+1] = a[i];
      i = i-1;
    }
    a[i+1] = key;
  }
}
\end{verbatim}
}

\end{frame}

\begin{frame}{Insertion sort}
Description en {\bf pseudo-code} (sur des tableaux d'entiers):

\bigskip

\fcolorbox{white}{Lightgray}{
\begin{codebox}
\Procname{$\proc{Insertion-Sort}(A)$}
\li \For $j \gets 2$ \To $\attrib{A}{length}$
\li     \Do
$\id{key} \gets A[j]$
\li \Comment Insert $A[j]$ into the sorted sequence
    $A[1 \twodots j-1]$.
\li $i \gets j-1$
\li \While $i > 0$ and $A[i] > \id{key}$
\li   \Do
        $A[i+1] \gets A[i]$
\li        $i \gets i-1$
    \End
\li $A[i+1] \gets \id{key}$
\End
\end{codebox}
}

\end{frame}

\begin{frame}{Pseudo-code}

Objectifs:
\begin{itemize}
\item Décrire les algorithmes de manière à ce qu'ils soient compris
par des humains.

~\\

\item Rendre la description indépendante de l'implémentation

~\\

\item S'affranchir de détails tels que la gestion d'erreurs, les déclarations de type, etc.

\end{itemize}

~\\

Très proche du C (langage procédural plutôt qu'orienté objet)

~\\

Peut contenir certaines instructions en langage naturel si nécessaire

% donner quelques règles de pseudo-code
\end{frame}


\begin{frame}{Pseudo-code}
Quelques règles
\begin{itemize}
\item Structures de blocs indiquées par l'indentation
\item Boucles ($\For$, $\While$, $\Repeat$) et conditions ($\If$, $\textbf{else}$, $\textbf{elseif}$) comme en C.
\item Le compteur de boucle garde sa valeur à la sortie de la boucle
\item En sortie d'un $\For$, le compteur a la valeur de la borne max+1.
\begin{columns}
\begin{column}{3cm}
\fcolorbox{white}{Lightgray}{\begin{codebox}
\zi \For $i \gets 1$ \To $Max$
\zi     \Do $Code$
\End
\end{codebox}}
\end{column}
$\Leftrightarrow$
\begin{column}{5cm}
\fcolorbox{white}{Lightgray}{\begin{codebox}
\zi $i\gets 1$
\zi \While $i \leq Max$
\zi     \Do $Code$
\zi     $i\gets i+1$
\End
\end{codebox}}
\end{column}
\end{columns}
\item Commentaires indiqués par $\Comment$
\item Affectation ($\gets$) et test d'égalité ($\isequal$) comme en C.
\end{itemize}
\end{frame}

\begin{frame}{Pseudo-code}

\begin{itemize}
\item Les variables ($i$, $j$ et $key$ par exemple) sont locales à la fonction.
\item $A[i]$ désigne l'élément $i$ du tableau $A$. $A[i..j]$ désigne
  un intervalle de valeurs dans un tableau. $A.length$ est la taille du tableau.
\item L'indexation des tableaux commence à 1.
\item Les types de données composés sont organisés en {\it objets}, qui sont composés d'attributs. On accède à la valeur de l'attribut $attr$ pour un objet $x$ par $\attrib{x}{attr}$.
\item Un variable représentant un tableau ou un objet est considérée comme un pointeur vers ce tableau ou cet objet.
\item Paramètres passés par valeur comme en C (mais tableaux et objets sont passés par pointeur).
\item ...
\end{itemize}
\note{y=x, x.f=3 implique que y.f=3 également}
\end{frame}

\begin{frame}{Trois questions récurrentes face à un algorithme}

\begin{enumerate}
\item Mon algorithme est-il correct, se termine-t-il ? %\textcolor{darkred}{oui}

\bigskip

\item Quelle est sa vitesse d'exécution ? %\textcolor{darkred}{$o(n^2)$}

\bigskip

\item Y-a-t'il moyen de faire mieux ? %\textcolor{darkred}{oui}

\end{enumerate}

\bigskip

Exemple du \textcolor{darkred}{tri par insertion}
\begin{enumerate}
\item Oui $\rightarrow$ technique des invariants (partie 2)
\item $O(n^2)$ $\rightarrow$ analyse de complexité (partie 2)
\item Oui $\rightarrow$ il existe un algorithme $O(n\log n)$ (partie 1)
\end{enumerate}

% commenter sur la facilité de répondre à ces questions
\note{On va passer en revue rapidement ces trois points pour l'insertion-sort. On reviendra sur ça la semaine prochaine}
\end{frame}

\begin{frame}{Correction de $\proc{Insertion-Sort}$}

\begin{center}
\fcolorbox{white}{Lightgray}{
\begin{codebox}
\Procname{$\proc{Insertion-Sort}(A)$}
\li \For $j \gets 2$ \To $\attrib{A}{length}$
\li     \Do
$\id{key} \gets A[j]$
\li $i \gets j-1$
\li \While $i > 0$ and $A[i] > \id{key}$
\li   \Do
        $A[i+1] \gets A[i]$
\li        $i \gets i-1$
    \End
\li $A[i+1] \gets \id{key}$
\End
\end{codebox}
}
\end{center}

\bigskip
\begin{itemize}
\item \alert{Invariant:} (pour la boucle externe) le sous-tableau $A[1\twodots j-1]$ contient les éléments du tableau original $A[1\twodots j-1]$ ordonnés.
\item On doit montrer que
\begin{itemize}
\item l'invariant est vrai avant la première itération %({\em initialisation})
\item l'invariant est vrai avant chaque itération suivante %({\em maintenance})
\item En sortie de boucle, l'invariant implique que l'algorithme est correct %({\em terminaison})
\end{itemize}
\end{itemize}
\note{Si c'est vrai à la terminaison, alors on aura montrer que le tableau est trié à la fin}
\end{frame}

\begin{frame}{Correction de $\proc{Insertion-Sort}$}
\begin{itemize}
\item Avant la première itération:
\begin{itemize}
\item $j=2 \Rightarrow A[1]$ est trivialement ordonné.
\end{itemize}
\bigskip
\item Avant la $j$ème itération:
\begin{itemize}
\item Informellement, la boucle interne déplace $A[j-1]$, $A[j-2]$, $A[j-3]\ldots$ d'une position vers la droite jusqu'à la bonne position pour $key$ ($A[j]$).
\end{itemize}
\bigskip
\item En sortie de boucle:
\begin{itemize}
\item A la sortie de boucle, $j=A.length+1$. L'invariant implique que $A[1\twodots A.length]$ est ordonné.
\end{itemize}
\end{itemize}
\end{frame}

\begin{frame}{Complexité de $\proc{Insertion-Sort}$}
\begin{center}
\fcolorbox{white}{Lightgray}{
\begin{codebox}
\Procname{$\proc{Insertion-Sort}(A)$}
\li \For $j \gets 2$ \To $\attrib{A}{length}$
\li     \Do
$\id{key} \gets A[j]$
\li $i \gets j-1$
\li \While $i > 0$ and $A[i] > \id{key}$
\li   \Do
        $A[i+1] \gets A[i]$
\li        $i \gets i-1$
    \End
\li $A[i+1] \gets \id{key}$
\End
\end{codebox}
}
\end{center}

\bigskip

\begin{itemize}
\item Nombre de comparaisons $T(n)$ pour trier un tableau de taille $n$?
\item Dans le pire des cas:
\begin{itemize}
\item La boucle $\For$ est exécutée $n-1$ fois ($n=A.length$).
\item La boucle $\While$ est exécutée $j-1$ fois
\end{itemize}
\end{itemize}

\note{Pire cas est lorsqu'on doit mettre l'élément à insérer en
  première position: on le compare au $j-1$, au $j-2$, ..., au 1er. Ce
  qui donnej- comparaisons.}
\end{frame}

\begin{frame}{Complexité de $\proc{Insertion-Sort}$}

\begin{itemize}
\item Le nombre de comparaisons est borné par:
$$T(n)\leq \sum_{j=2}^n (j-1)$$
\item Puisque $\sum_{i=1}^n i=n(n+1)/2$, on a:
$$T(n)\leq\frac{n(n-1)}{2}$$
\item Finalement, $T(n)=O(n^2)$
\end{itemize}

\bigskip

(borne inférieure ?)

\note{Lequel est le plus rapide: tri par sélection ou tri par insertion ? Le second, car le premier est toujours $O(n^2)$}
\end{frame}

\begin{frame}{Structures de données}

\begin{itemize}
\item Méthode pour stocker et organiser les données pour en faciliter
  l'accès et la modification
\item Une structure de données regroupe:
\begin{itemize}
\item un certain nombre de données à gérer, et
\item un ensemble d'opérations pouvant être appliquées à ces données
\end{itemize}
\item Dans la plupart des cas, il existe
\begin{itemize}
\item plusieurs manières de représenter les données et
\item différents algorithmes de manipulation.
\end{itemize}
\item On distingue généralement l'\alert{interface} des structures de
  leur \alert{implémentation}.
\end{itemize}

\end{frame}

\begin{frame}{Types de données abstraits}

\begin{itemize}
\item Un type de données abstrait (TDA) représente l'interface d'une structure de données.
\item Un TDA spécifie précisément:
\begin{itemize}
\item la nature et les propriétés des données
\item les modalités d'utilisation des opérations pouvant être effectuées
\end{itemize}
\item En général, un TDA admet différentes implémentations (plusieurs représentations possibles des données, plusieurs algorithmes pour les opérations).
\end{itemize}
\note{Super exemple est un dictionnaire: vous pouvez trier ou pas les mots. Si vous triez les mots, la recherche d'un mot sera facile. Sinon, il faudra parcourir le dictionnaire jusqu'à arriver au mot (leur demander comment il ferait d'ailleurs algorithmiquement dans les deux cas !!!) recherche binaire !!!}
\end{frame}

\begin{frame}{Exemple: file à priorités}

\begin{itemize}
\item Données gérées: des objets avec comme attributs:
\begin{itemize}
\item une clé, munie d'un opérateur de comparaison selon un ordre total
\item une valeur quelconque
\end{itemize}
\medskip
\item Opérations:
\begin{itemize}
\item Création d'une file vide
\item $\proc{Insert}(S,x)$: insère l'élément $x$ dans la file $S$.
%\item $\proc{Maximum}(S)$: renvoie l'élément de $S$ avec la clé la plus grande.
\item $\proc{Extract-Max}(S)$: retire et renvoie l'élément de $S$ avec
  la clé la plus grande.
\end{itemize}
\medskip
\item Il existe de nombreuses façons d'implémenter ce TDA:
\begin{itemize}
\item Tableau non trié;
\item Liste triée;
\item Structure de tas;
\item $\ldots$
\end{itemize}
Chacune mène à des complexités différentes des opérations $\proc{Insert}$ et $\proc{Extract-Max}$
\end{itemize}

\end{frame}

\begin{frame}{Structures de données et algorithmes en pratique}
% voir slides de sedgewick, intro-pas-mal.ppt

\begin{itemize}
\item La résolution de problème algorithmiques requiert presque
  toujours la combinaison de structures de données et d'algorithmes
  sophistiqués pour la gestion et la recherche dans ces structures.
\item D'autant plus vrai qu'on a à traiter des volumes de données importants.
\item Quelques exemples de problèmes réels:
\begin{itemize}
\item Routage dans les réseaux informatiques
\item Moteurs de recherche
\item Alignement de séquences ADN en bio-informatique
\end{itemize}
\end{itemize}
\centerline{\includegraphics[width=12cm]{Figures/01-exemplesproblemes.pdf}}
\note{
Routeur: Il faut stocker les chemins de manière distribuée. Trouver le plus cours chemin. Traiter les paquets rapidement.

~\\

Moteurs de recherche: Pour que google puisse répondre aussi rapidement, il faut créer des structures de données très sophistiquées (appelé des indexs)

~\\

Alignement de séquence: séquence du génome humain fait 3 milliards de paires de base. Aligner deux séquences ou faire une recherche dans une séquence est un problème assez difficile.
}

\end{frame}

\section{Introduction à la récursivité}

\begin{frame}{Plan}

\tableofcontents
\end{frame}


\begin{frame}{Algorithmes récursifs}

Un algorithme est {\bf récursif} s'il s'invoque lui-même
directement ou indirectement.

\bigskip

Motivation: Simplicité d'expression de certains algorithmes

\bigskip

Exemple: Fonction factorielle:

\[n!=\left\{\begin{array}{ll}
1 &\mbox{si }n=0\\
n \cdot (n-1)! &\mbox{si }n>0$$
\end{array}
\right.
\]

\begin{center}
\fcolorbox{white}{Lightgray}{
\begin{codebox}
\Procname{$\proc{Factorial}(n)$}
\li \If $n\isequal 0$
\li \Then \Return 1 \End
\li \Return $n \cdot \proc{Factorial}(n-1)$
\end{codebox}
}
\end{center}

\end{frame}

\begin{frame}{Algorithmes récursifs}

\begin{center}
\fcolorbox{white}{Lightgray}{
\begin{codebox}
\Procname{$\proc{Factorial}(n)$}
\li \If $n\isequal 0$
\li \Then \Return 1 \End
\li \Return $n \cdot \proc{Factorial}(n-1)$
\end{codebox}
}
\end{center}

\bigskip

Règles pour développer une solution récursive:
\bigskip
\begin{itemize}
\item On doit définir un cas de base ($n==0$)
\item On doit diminuer la ``taille'' du problème à chaque étape ($n\rightarrow n-1$)
\item Quand les appels récursifs se partagent la même structure de données, les sous-problèmes ne doivent pas se superposer (pour éviter les effets de bord)
\end{itemize}
\end{frame}

\begin{frame}{Exemple de récursion multiple}

Calcul du $n$ième nombre de Fibonacci:
\begin{eqnarray*}
F_0&=&0\\
F_1&=&1\\
\forall n\geq 2: F_n& = &F_{n-2}+F_{n-1}
\end{eqnarray*}

Algorithme:
\begin{center}
\fcolorbox{white}{Lightgray}{
\begin{codebox}
\Procname{$\proc{Fibonacci}(n)$}
\li \If $n \leq 1$
\li \Then \Return n \End
\li \Return $\proc{Fibonacci}(n-2)+\proc{Fibonacci}(n-1)$
\end{codebox}
}
\end{center}
% Prendre le début du cours de Carnagazi

\end{frame}

\begin{frame}{Exemple de récursion multiple}

\begin{center}
\fcolorbox{white}{Lightgray}{
\begin{codebox}
\Procname{$\proc{Fibonacci}(n)$}
\li \If $n \leq 1$
\li \Then \Return n \End
\li \Return $\proc{Fibonacci}(n-2)+\proc{Fibonacci}(n-1)$
\end{codebox}
}
\end{center}
% Prendre le début du cours de Carnagazi

\bigskip

\begin{enumerate}
\item L'algorithme est-il correct?
\item Quelle est sa vitesse d'exécution?
\item Y-a-t'il moyen de faire mieux?
\end{enumerate}

\end{frame}

\begin{frame}{Exemple de récursion multiple}

\begin{center}
\fcolorbox{white}{Lightgray}{
\begin{codebox}
\Procname{$\proc{Fibonacci}(n)$}
\li \If $n \leq 1$
\li \Then \Return n \End
\li \Return $\proc{Fibonacci}(n-2)+\proc{Fibonacci}(n-1)$
\end{codebox}
}
\end{center}

\bigskip

\begin{enumerate}
\item L'algorithme est correct?
\begin{itemize}
\item Clairement, l'algorithme est correct.
\item En général, la correction d'un algorithme récursif se démontre par induction.
\end{itemize}
\item Quelle est sa vitesse d'exécution?
\item Y-a-t'il moyen de faire mieux?
\end{enumerate}

\note{faire la démonstration au tableau:
Pour n<=1, l'algo renvoie n, ce qui est correct
Si l'algorithme est correct pour tout n<n', on doit montrer qu'il est correct pour n+1. C'est évident...
}
\end{frame}

\begin{frame}{Vitesse d'exécution}
\begin{itemize}
\item Nombre d'opérations pour calculer $\proc{Fibonacci}(n)$ en fonction de $n$
\item Empiriquement:

\begin{center}
\includegraphics[width=7.5cm]{Figures/cpu-fibonacci.pdf}\\
~\hfill\scriptsize(Carzaniga)
\end{center}

\item Toutes les implémentations atteignent leur limite, plus ou moins loin
\end{itemize}
\end{frame}

\begin{frame}{Trace d'exécution}
\centerline{\includegraphics[width=8cm]{Figures/trace-fibonacci.pdf}}
~\hfill\scriptsize(Boigelot)
\end{frame}

\begin{frame}{Complexité}
\begin{center}
\fcolorbox{white}{Lightgray}{
\begin{codebox}
\Procname{$\proc{Fibonacci}(n)$}
\li \If $n \leq 1$
\li \Then \Return n \End
\li \Return $\proc{Fibonacci}(n-2)+\proc{Fibonacci}(n-1)$
\end{codebox}
}
\end{center}

\bigskip

\begin{itemize}
\item Soit $T(n)$ le nombre d'opérations de base pour calculer $\proc{Fibonacci}(n)$:
\begin{eqnarray*}
T(0) & = & 2, T(1)=2\\
T(n) & = & T(n-1)+T(n-2)+2\\
\end{eqnarray*}
\item On a donc $T(n)\geq F_n$ (= le $n$ème nombre de Fibonacci).
\end{itemize}

\end{frame}

\begin{frame}{Complexité}

\begin{itemize}
\item Comment croît $F_n$ avec $n$ ?
$$T(n)\geq F_n=F_{n-1}+F_{n-2}$$
Puisque $F_n\geq F_{n-1}\geq F_{n-2}\geq\ldots$
$$F_n\geq 2 F_{n-2}\geq 2(2 F_{n-4})\geq 2(2(2 F_{n-6})) \geq 2^{\frac{n}{2}}$$
Et donc
$$T(n)\geq (\sqrt{2})^n \approx (1.4)^n$$
\item $T(n)$ croît \alert{exponentiellement} avec $n$

\bigskip

\item Peut-on faire mieux ?
\end{itemize}
\note{Dire que le problème vient du fait qu'on calcule plusieurs fois les mêmes choses

pprev va stocker la valeur de $F_{n-2}$ et prev va stocker la valeur de $F_{n-1}$.
}
\end{frame}

\begin{frame}{Solution itérative}

\begin{center}
\fcolorbox{white}{Lightgray}{
\begin{codebox}
\Procname{$\proc{Fibonacci-Iter}(n)$}
\li \If $n \leq 1$
\li \Then \Return n \End
\li \Else
\li \Then $pprev\gets 0$
\li $prev\gets 1$
\li \For $i\gets 2 \To n$
\li \Do $f\gets prev+pprev$
\li $pprev\gets prev$
\li $prev\gets f$\End
\li \Return f \End
\end{codebox}
}
\end{center}

%Complexité: $O(n)$

\end{frame}

\begin{frame}{Vitesse d'exécution}
Complexité: $O(n)$

\bigskip

\centerline{\includegraphics[width=8cm]{Figures/cpu-fibonacci-iter.pdf}}
~\hfill\scriptsize(Carzaniga)
\end{frame}

\begin{frame}{Tri par fusion}

Idée d'un tri basé sur la récursion:
\begin{itemize}
\item on sépare le tableau en deux sous-tableaux de la même taille
\item on trie (récursivement) chacun des sous-tableaux
\item on fusionne les deux sous-tableaux triés en maintenant l'ordre
\end{itemize}
Le cas de base correspond à un tableau d'un seul élément.

\bigskip

\begin{center}
\fcolorbox{white}{Lightgray}{%
    \begin{codebox}
      \Procname{$\proc{merge-sort}(A,p,r)$}
      \li \If $\id{p}<\id{r}$
      \li \Then $q \gets \lfloor \frac{p+r}{2} \rfloor$
      \li       $\proc{merge-sort}(A,p,q)$
      \li       $\proc{merge-sort}(A,q+1,r)$
      \li       $\proc{merge}(A,p,q,r)$ \End
    \end{codebox}}
\end{center}

\centerline{Appel initial: $\proc{merge-sort}(A,1,A.length)$}

\bigskip

Exemple d'application du principe général de ``\alert{diviser pour régner}''

\note{

Idée: on suppose qu'on dispose d'un algorithme pour trier un tableau, comment l'utiliser pour trier un tableau: on l'applique sur le tableau. Un autre manière de faire est de diviser le tableau en deux parties, trier ces deux parties et ensuite les fusionner. On applique la même idée sur les sous-tableaux. On s'arrête quand le tri est trivial: 1 seul élément.

Les contraintes mentionnées plus haut sont bien respectées: cas de base, division, pas de recouvrement.

Illustrer l'idée sur l'exemple: $[5,2,4,7,1,3,2,6]$

\includegraphics[width=5cm]{Figures/01-note-mergesort.pdf}

ce principe reviendra souvent dans ce cours


}

\end{frame}

\begin{frame}{Tri par fusion: illustration}

\centerline{\includegraphics[width=6cm]{Figures/CLRS/mergesort-power2.pdf}}

\note{\centerline{\includegraphics[width=6cm]{Figures/CLRS/mergesort-notpower2.pdf}}}

\end{frame}

\begin{frame}{Fonction $\proc{merge}$}

$\proc{Merge}(A,p,q,r)$:
\begin{itemize}
\item {\bf Entrée:} tableau $A$ et indice $p$, $q$, $r$ tels que:
\begin{itemize}
\item $p\leq q<r$ (pas de tableaux vides)
\item Les sous-tableaux $A[p\twodots q]$ et $A[q+1\twodots r]$ sont ordonnés
\end{itemize}
\item {\bf Sortie:} Les deux sous-tableaux sont fusionnés en un seul sous-tableau ordonné dans $A[p\twodots r]$
\end{itemize}

\bigskip

Idée:
\begin{itemize}
\item Utiliser un pointeur vers le début de chacun des sous-tableaux;
\item Déterminer le plus petit des deux éléments pointés;
\item Déplacer cet élément vers le tableau fusionné;
\item Avancer le pointeur correspondant
\end{itemize}

\note{Faire l'analogie avec deux jeux de cartes triés qu'on veut rassembler. prendre un jeu de carte.

\includegraphics[width=6cm]{Figures/01-note-merge.pdf}

}

\end{frame}

\begin{frame}{Fusion: algorithme}

\begin{center}\small
   \fcolorbox{white}{Lightgray}{
    \begin{codebox}
      \Procname{$\proc{Merge}(A,p,q,r)$}
      \li $n_1 = q-p+1$; $n2 = r-q$
      \li Soit $L[1..n_1+1]$ et $R[1..n_2+1]$ deux nouveaux tableaux
      \li \For $i=1$ \To $n_1$
      \li \Do $L[i]=A[p+i-1]$ \End
      \li \For $j=1$ \To $n_2$
      \li \Do $R[j]=A[q+j]$ \End
      \li $L[n_1+1]=\infty$; $R[n_2+1]=\infty$ \RComment \textcolor{red}{Sentinels}
      \li i=1;j=1
      \li \For $k\gets p$ \To $r$
      \li \Do \If $L[i]\leq R[j]$
      \li \Then $A[k]\gets L[i]$
      \li       $i\gets i+1$
      \li \Else
      \li $A[k]=R[j]$
      \li       $j\gets j+1$
          \End
    \end{codebox}
}
\end{center}
\note{Sentinel: pas toujours une bonne idée. Difficile d'avoir une valeur infinie.}
\end{frame}

\begin{frame}{Fusion: illustration}

\centerline{\includegraphics[width=8cm]{Figures/mergeillustration.pdf}}

\bigskip

Complexité: $O(n)$ (où $n=r-p+1$)

\end{frame}

\begin{frame}{Vitesse d'exécution}

Complexité de $\proc{merge-sort}$: $O(n\log n)$ (voir partie 2)

\bigskip

\centerline{\includegraphics[width=8cm]{Figures/compare-sort.pdf}}

\note{Dans un prochain cours, on verra que le tri par fusion est
  optimal en terme de temps de calcul (N log N)}

\end{frame}

\begin{frame}{Remarques}

\begin{itemize}
\item La fonction $\proc{merge}$ nécessite d'allouer deux tableaux $L$
  et $R$ (dont la taille est $O(n)$). Exercice (difficile): écrire une
  fonction $\proc{merge}$ qui ne nécessite pas d'allocation
  supplémentaire.
\item On pourrait réécrire $\proc{merge-sort}$ de manière itérative (au prix de la simplicité)
\item Version récursive du tri par insertion:
\begin{center}
\fcolorbox{white}{Lightgray}{
\begin{codebox}
\Procname{$\proc{Insertion-Sort-rec}(A,n)$}
\li \If $n>1$
\li \Then $\proc{Insertion-Sort-rec}(A,n-1)$
\li       $\proc{Merge}(A,1,n-1,n)$ \End
\end{codebox}
}
\end{center}
\end{itemize}
\note{Attirer l'attention sur le fait qu'il faut aussi tenir compte de
  la taille mémoire requise par un algorithme.

\bigskip

Dire que l'insertion sort est une version particulière du merge sort où au lieu de diviser en deux, on divise en 1-n-1 et n. C'est donc nettement moins efficace. On obtiendrait l'insertion-sort à partir du mergesort en changeant le q=p+r/2 en q=r-1.
}
\end{frame}

\begin{frame}{Note sur l'implémentation de la récursivité}
\begin{itemize}
\item Trace d'exécution de la factorielle
\centerline{\includegraphics[width=2cm]{Figures/trace-factorielle.pdf}}
\item Chaque appel récursif nécessite de mémoriser le \alert{contexte d'invocation}
\item L'espace mémoire utilisé est donc $O(n)$ ($n$ appels récursifs)
\end{itemize}
\note{!! Il est important de garder ça en mémoire: les algos récursifs ont un coût en terme d'espace mémoire}
\end{frame}

\begin{frame}{Récursivité terminale}

\begin{itemize}
\item Définition: Une procédure est \alert{récursive terminale} (``tail recursive'') si elle n'effectue plus aucune opération après s'être invoquée récursivement.

\bigskip

\item Avantages:
\begin{itemize}
\item Le contexte d'invocation ne doit pas être mémorisé et donc l'espace mémoire nécessaire est réduit
\item Les procédures récursives terminales peuvent facilement être converties en procédures itératives
\end{itemize}
\end{itemize}

\end{frame}

\begin{frame}{Version récursive terminale de la factorielle}

\begin{center}
\fcolorbox{white}{Lightgray}{
\begin{codebox}
\Procname{$\proc{Factorial2}(n)$}
\li \Return $\proc{Factorial2-rec}(n,2,1)$
\end{codebox}}
\end{center}

\begin{center}
\fcolorbox{white}{Lightgray}{
\begin{codebox}
\Procname{$\proc{Factorial2-rec}(n,i,f)$}
\li \If $i>n$
\li \Then \Return $f$\End
\li \Return $\proc{Factorial2-rec}(n,i+1,f\cdot i)$
\end{codebox}}
\end{center}

\bigskip

Espace mémoire utilisé: $O(1)$ (si la récursion terminale est implémentée efficacement)

\note{
le i est le numéro de l'itération et le f est l'accumulateur.
}

\end{frame}

\begin{frame}{Ce qu'on a vu}

\begin{itemize}
\item Définitions générales: algorithmes, structures de données, structures de données abstraites...
\item Analyse d'un algorithme itératif ($\proc{Insertion-Sort}$)
\item Notions de récursivité
\item Analyse d'un algorithme récursif ($\proc{Fibonacci}$)
\item Tri par fusion ($\proc{MergeSort}$)
\end{itemize}

\end{frame}
