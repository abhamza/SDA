\part{Résolution de problèmes}

% Rajouter les slides d'outline et un découpe des sections par exemple
% Mettre Exemple 1, 2, etc.
% maximum/maximal ?

\begin{frame}{Plan}

\tableofcontents[hideallsubsections]

\end{frame}

\section{Introduction}

\begin{frame}{Méthodes de résolution de problèmes}

Quelques approches génériques pour aborder la résolution d'un problème:
\begin{itemize}
\item \alert{Approche par force brute:} résoudre directement le problème, à partir de sa définition ou par une recherche exhaustive
\semitransp{\item Diviser pour régner: diviser le problème en sous-problèmes, les résoudre, fusionner les solutions pour obtenir une solution au problème original
\item Programmation dynamique: obtenir la solution optimale à un problème en combinant des solutions optimales à des sous-problèmes similaires plus petits et se chevauchant
\item Approche gloutonne: construire la solution incrémentalement, en optimisant de manière aveugle un critère local}
\end{itemize}

\end{frame}

\section{Approche par force brute}

\begin{frame}{Approche par force brute (brute-force)}
\begin{itemize}
\item Consiste à appliquer la solution la plus directe à un problème
\item Généralement obtenue en appliquant à la lettre la définition du problème
\item Exemple simple:
\begin{itemize}
\item Rechercher un élément dans un tableau (trié ou non) en le parcourant linéairement
\item Calculer $a^n$ en multipliant $a$ $n$ fois avec lui-même
\item Implémentation récursive naïve du calcul des nombres de Fibonacci
\item \ldots
\end{itemize}
\item Souvent pas très efficace en terme de temps de calcul mais facile à implémenter et fonctionnel
\end{itemize}

\end{frame}

\begin{frame}{Exemple: tri}

Approches par force brute pour le tri:
\begin{itemize}
\item Un tableau est trié (en ordre croissant) si tout élément est plus petit que l'élément à sa droite
\item $\Rightarrow$ tri à bulle: parcourir le tableau de gauche à droite en inversant toutes les paires d'éléments ne respectant pas cette définition
\item Complexité: $O(n^2)$
\item $\Rightarrow$ tri par sélection: trouver le minimum du tableau, l'échanger avec le premier élément, répéter pour trier le reste du tableau
\item Complexité: $\Theta(n^2)$
\end{itemize}
\end{frame}

\begin{frame}{Recherche exhaustive}
\begin{itemize}
\item Une solution par force brute au problème de la recherche d'un élément possédant une propriété particulière
\item Générer toutes les solutions possibles jusqu'à en obtenir une qui possède la propriété recherchée
\item Exemple pour le tri:
\begin{itemize}
\item Générer toutes les permutations du tableau de départ (une et une seule fois)
\item Vérifier si chaque tableau permuté est trié. S'arrêter si c'est le cas.
\item Complexité: $O(n!\cdot n)$
\end{itemize}
\item Généralement utilisable seulement pour des problèmes de petite taille
\item Dans la plupart des cas, il existe une meilleure solution
\item Dans certain cas, c'est la seule solution possible
\end{itemize}
\end{frame}

\begin{frame}{Problème du voyageur de commerce}
\begin{itemize}
\item Etant donné $n$ villes et les distances entre ces villes
\item Trouver le plus court chemin qui passe par toutes les villes
  exactement une fois avant de revenir à la ville de départ
\end{itemize}
\begin{columns}
\begin{column}{5cm}
\centerline{\includegraphics[width=3cm]{Figures/06-tsp.pdf}}
\end{column}
\begin{column}{5cm}
\footnotesize
\begin{tabular}{ll}
Tour & Coût\\
\hline
A-B-C-D-A & 17\\
A-B-D-C-A & 21\\
A-C-B-D-A & 20\\
A-C-D-B-A & 21\\
A-D-B-C-A & 20\\
A-D-C-B-A & 17\\
\end{tabular}
\end{column}
\end{columns}

\begin{itemize}
\item Recherche exhaustive: $O(n!)$
\item On n'a pas encore pu trouver un algorithme de complexité polynomiale (et il y a peu de chance qu'on y arrive)
\end{itemize}

\end{frame}

\begin{frame}{Force brute/recherche exhaustive}
Avantages:
\begin{itemize}
\item Simple et d'application très large
\item Un bon point de départ pour trouver de meilleurs algorithmes
\item Parfois, faire mieux n'en vaut pas la peine
\end{itemize}

\bigskip

Inconvénients:
\begin{itemize}
\item Produit rarement des solutions efficaces
\item Moins éléguant et créatif que les autres techniques
\end{itemize}

\bigskip

Dans ce qui suit, on commencera la plupart du temps par fournir la solution par force brute des problèmes, qu'on cherchera ensuite à résoudre par d'autres techniques

\end{frame}

% truc interessant:

\section{Diviser pour régner}

\begin{frame}{Méthodes de résolution de problèmes}

Quelques approches génériques pour aborder la résolution d'un problème:
\begin{itemize}
\item \alert{Approche par force brute:} résoudre directement le problème, à partir de sa définition ou par une recherche exhaustive
\item \alert{Diviser pour régner:} diviser le problème en sous-problèmes, les résoudre, fusionner les solutions pour obtenir une solution au problème original
\semitransp{
\item {Programmation dynamique:} obtenir la solution optimale à un problème en combinant des solutions optimales à des sous-problèmes similaires plus petits et se chevauchant
\item {Approche gloutonne:} construire la solution incrémentalement, en optimisant de manière aveugle un critère local}
\end{itemize}

\end{frame}

\begin{frame}{Plan}

\tableofcontents[currentsection,hideothersubsections]

\end{frame}

\begin{frame}{Approche diviser-pour-régner {\it (Divide and conquer)}}

Principe général:
\begin{itemize}
\item Si le problème est trivial, on le résoud directement
\item Sinon:
\begin{enumerate}
\item Diviser le problème en sous-problèmes de taille inférieure (Diviser)
\item Résoudre récursivement ces sous-problèmes (Régner)
\item Fusionner les solutions aux sous-problèmes pour produire une solution au problème original
\end{enumerate}
\end{itemize}

\end{frame}

\begin{frame}{Exemples déjà rencontrés}

\begin{itemize}
\item \alert{Merge sort:}
\begin{enumerate}
\item Diviser: Couper le tableau en deux sous-tableaux de même taille
\item Régner: Trier récursivement les deux sous-tableaux
\item Fusionner: fusionner les deux sous-tableaux
\end{enumerate}
Complexité: $\Theta(n\log n)$ (force brute: $\Theta(n^2)$)
\item \alert{Quicksort:}
\begin{enumerate}
\item Diviser: Partionner le tableau selon le pivot
\item Régner: Trier récursivement les deux sous-tableaux
\item Fusionner: /
\end{enumerate}
Complexité: $\Theta(n\log n)$ (force brute: $\Theta(n^2)$)
\item \alert{Recherche binaire} (dichotomique):
\begin{enumerate}
\item Diviser: Contrôler l'élement central du tableau
\item Régner: Chercher récursivement dans un des sous-tableaux
\item Fusionner: trivial
\end{enumerate}
Complexité: $O(\log n)$ (force brute: $O(n)$)
\end{itemize}

\end{frame}

\subsection{Exemple 1: calcul du minimum/maximum d'un tableau}

\begin{frame}{Exemple 1: Calcul du minimum/maximum d'un tableau}

\begin{itemize}
\item Approche par force brute pour trouver le minimum ou le maximum d'un tableau
\end{itemize}

\begin{columns}
\begin{column}{5cm}
\begin{center}
{\small
\fcolorbox{white}{Lightgray}{%
      \begin{codebox}
        \Procname{$\proc{Min}(A)$}
        \li $min\gets A[1]$
        \li \For $i\gets 2$ \To $\attrib{A}{length}$
        \li \Do \If $min>A[i]$
        \li \Then $min\gets A[i]$\End\End
        \li \Return $min$
      \end{codebox}}
}
\end{center}
\end{column}
\begin{column}{5cm}
\begin{center}
{\small
\fcolorbox{white}{Lightgray}{%
      \begin{codebox}
        \Procname{$\proc{Max}(A)$}
        \li $max\gets A[1]$
        \li \For $i\gets 2$ \To $\attrib{A}{length}$
        \li \Do \If $max<A[i]$
        \li \Then $max\gets A[i]$\End\End
        \li \Return $max$
      \end{codebox}}
}
\end{center}
\end{column}
\end{columns}

\bigskip

\begin{itemize}
\item Complexité: $\Theta(n)$ ($n-1$ comparaisons)
\item Peut-on faire mieux ?
\begin{itemize}
\item Non, pas en notation asymptotique (le problème est $\Theta(n)$)
\item Par contre, on peut diminuer le nombre total de comparaisons pour calculer à la fois le minimum et le maximum
\end{itemize}
\end{itemize}

\end{frame}

\begin{frame}{Calcul simultané du minimum et du maximum}

\begin{itemize}
\item Approche diviser-pour-régner pour le calcul simultané du minimum et du maximum
\end{itemize}

\begin{center}
{\small
\fcolorbox{white}{Lightgray}{%
      \begin{codebox}
        \Procname{$\proc{Max-Min}(A,p,r)$}
        \li \If $\id{r}-\id{p}\leq 1$
        \li \Then \If $A[p]<A[r]$
        \li \Then \Return $(A[r],A[p])$
        \li \Else \Return $(A[p],A[r])$\End\End
        \li $q\gets \lfloor\frac{p+r}{2}\rfloor$
        \li $(max1,min1)=\proc{Max-Min}(A,p,q)$
        \li $(max2,min2)=\proc{Max-Min}(A,q+1,r)$
        \li \Return $(\proc{max}(max1,max2),\proc{min}(min1,min2))$
      \end{codebox}}
}
\end{center}

\centerline{Appel initial: $\proc{Max-Min}(A,1,A.length)$}

\bigskip

\begin{itemize}
\item Correct ? Oui (preuve par induction)
\item Complexité ?
\end{itemize}

\end{frame}

\begin{frame}{Analyse de complexité}

\begin{itemize}
\item En supposant que $n$ est une puissance de 2, le nombre de
  comparaisons $T(n)$ est donné par: {\small
\[
T(n) = \left\{
\begin{array}{ll}
 1 & \mbox{si }n=2\\
2 T(n/2)+2 & \mbox{sinon}
\end{array}
\right.
\]}
qui se résoud en:
{\footnotesize
\begin{eqnarray*}
T(n) & = & 2 T(n/2)+2\\
%& = & 2(2 T(n/4)+2)+2\\
& = & 4 T(n/4)+4+2\\
& = & 8 T(n/4)+8+4+2\\
& = & 2^i T(n/2^i)+\sum_{j=1}^i 2^j\\
& = & 2^{\log_2(n-1)} T(2)+\sum_{j=1}^{\log_2(n)-1} 2^j\\
& = & 3/2 n -2
\end{eqnarray*}}
\item C'est-à-dire 25\% de comparaisons en moins que les méthodes séparées
\end{itemize}

\end{frame}

\subsection{Exemple 2: Recherche de pics}

\begin{frame}{Exemple 2: Recherche de pics}

\centerline{\includegraphics[width=6cm]{Figures/06-peakfinding.pdf}}

\bigskip

\begin{itemize}
\item Soit un tableau $A[1\twodots \attrib{A}{length}]$. On supposera
  que $A[0]=A[\attrib{A}{length}+1]=-\infty$.
\item Définition: $A[i]$ est un \alert{pic} s'il n'est pas plus petit que ses voisins:
$$A[i-1]\leq A[i]\geq A[i+1]$$
($A[i]$ est un maximum local)
\item \alert{But:} trouver un pic dans le tableau (n'importe lequel)
\item Note: il en existe toujours un
\end{itemize}

\end{frame}

\begin{frame}{Approche par force brute}
\begin{itemize}
\item Tester toutes les positions séquentiellement:

\bigskip

\begin{center}
{\small
\fcolorbox{white}{Lightgray}{%
      \begin{codebox}
        \Procname{$\proc{Peak1d}(A)$}
        \li \For $i\gets 1$ \To $\attrib{A}{length}$
        \li \Do \If $A[i-1]\leq A[i]\geq A[i+1]$
        \li \Then \Return $i$\End\End
      \end{codebox}}
}
\end{center}

\bigskip

\item Complexité: $O(n)$ dans le pire cas
\end{itemize}

\end{frame}

\begin{frame}{Approche par force brute 2}
\begin{itemize}
\item Le maximum global du tableau est un maximum local et donc un pic

\bigskip

\begin{center}
{\small
\fcolorbox{white}{Lightgray}{%
      \begin{codebox}
        \Procname{$\proc{Peak1d}(A)$}
        \li $m\gets A[0]$
        \li \For $i\gets 1$ \To $\attrib{A}{length}$
        \li \Do \If $A[i]>A[m]$
        \li \Then $m\gets i$\End\End
        \li \Return m
      \end{codebox}}
}
\end{center}

\bigskip

\item Complexité: $\Theta(n)$ dans tous les cas
\end{itemize}

\end{frame}

\begin{frame}{Une meilleure idée}

Approche diviser-pour-régner:
\begin{itemize}
\item Sonder un élément $A[i]$ et ses voisins $A[i-1]$ et $A[i+1]$
\item Si c'est un pic: renvoyer $i$
\item Sinon:
\begin{itemize}
\item les valeurs doivent croître au moins d'un côté
$$A[i-1]>A[i]\mbox{ ou }A[i]<A[i+1]$$
%\item Il doit y avoir un pic de ce côté
\item Si $A[i-1]>A[i]$, on cherche le pic dans $A[1\twodots i-1]$
\item Si $A[i+1]>A[i]$, on cherche le pic dans $A[i+1\twodots \attrib{A}{length}]$
\end{itemize}

\bigskip

\centerline{\includegraphics[width=7cm]{Figures/06-peakfinding-idea.pdf}}

\item A quel position $i$ faut-il sonder ?
\end{itemize}

\note{montrer graphiquement\\
Il faut sonder au milieu pour accélerer les calculs}
\end{frame}

\begin{frame}{Algorithme}

\begin{center}
{\small
\fcolorbox{white}{Lightgray}{%
      \begin{codebox}
        \Procname{$\proc{Peak1d}(A,p,r)$}
        \li $q\gets \lfloor\frac{p+r}{2}\rfloor$
        \li \If $A[q-1]\leq A[q]\geq A[q+1]$
        \li \Then \Return $q$
        \li \ElseIf $A[q-1]>A[q]$
        \li \Then \Return $\proc{Peak1d}(A,i,q-1)$
        \li \ElseIf $A[q]<A[q+1]$
        \li \Then \Return $\proc{Peak1d}(A,q+1,r)$\End
      \end{codebox}}
}
\end{center}

\centerline{Appel initial: $\proc{Peak1d}(A,1,A.length)$}

\end{frame}

\begin{frame}{Analyse}
\begin{itemize}
\item Correction: oui
\begin{itemize}
\item On doit prouver qu'il y aura un pic du côté choisi
\item Preuve par l'absurde:
\begin{itemize}
\item Supposons que $A[q+1]>A[q]$ et qu'il n'y ait pas de pic dans $A[q+1\twodots r]$
\item On doit avoir $A[q+2]>A[q+1]$ (sinon $A[q+1]$ serait un pic)
\item On doit avoir $A[q+3]>A[q+2]$ (sinon $A[q+1]$ serait un pic)
\item \ldots
\item On doit avoir $A[r]>A[r-1]$ (sinon $A[r-1]$ serait un pic)
\item Comme $A[r]>A[r+1]=-\infty$, $A[r]$ est un pic, ce qui contredit l'hypothèse
\end{itemize}
\end{itemize}

\bigskip

\item Complexité:
\begin{itemize}
\item Dans le pire cas, on a $T(n)=T(n/2)+c_1$ et $T(1)=c_2$ (idem recherche binaire)
\item $\Rightarrow T(n)=O(\log n)$
\end{itemize}
\end{itemize}
\end{frame}

\begin{frame}{Extension à un tableau 2D}

\begin{columns}
\begin{column}{5cm}
\begin{itemize}
\item Soit une matrice $n\times n$ de nombres
\item Trouver un élement plus grand ou égal à ses 4 voisins (max)
\end{itemize}

\bigskip

\centerline{\includegraphics[width=2cm]{Figures/06-2dpeak-constraints.pdf}}
\end{column}
\begin{column}{5cm}
\centerline{\includegraphics[width=5cm]{Figures/06-2dpeak.pdf}}
~\hfill{\scriptsize(Demaine \& Leiserson)}
\end{column}
\end{columns}

\bigskip

\begin{itemize}
\item Approche par force brute: $O(n^2)$
\item Recherche du maximum: $\Theta(n^2)$
\end{itemize}

\end{frame}

\begin{frame}{Approche diviser-pour-régner}

\begin{columns}
\begin{column}{7cm}
\begin{itemize}
\item Chercher le maximum global dans la colonne \alert{centrale}
\item Si c'est un pic, le renvoyer
\item Sinon appeler la fonction récursivement sur les colonnes à
  gauche (resp. droite) si le voisin à gauche (resp. droite) est plus grand
\end{itemize}

\end{column}
\begin{column}{4cm}
\centerline{\includegraphics[width=4cm]{Figures/06-2dpeak-dc.pdf}}
~\hfill{\scriptsize(Demaine \& Leiserson)}
\end{column}
\end{columns}

\bigskip

\end{frame}

\begin{frame}{Analyse: correction}

\begin{itemize}
\item On doit prouver qu'il y a bien un pic du côté choisi
\item Preuve par l'absurde:
\begin{itemize}
\item Supposons qu'il n'y ait pas de pic
\item Soient $A[i,j]$ le maximum de la colonne centrale et $A[i,k]$ le voisin le plus grand ($k=j-1$ ou $k=j+1$)
\item $A[i,k]$ doit avoir un voisin $A[p_1,q_1]$ avec une valeur plus élevée (sinon, ce serait un pic)
\item $A[p_1,q_1]$ doit avoir un voisin $A[p_2,q_2]$ avec une valeur plus élevée (sinon, ce serait un pic)
\item \ldots
\item Le voisin doit toujours rester du même côté de la colonne centrale (puisque $A[i,k]>A[i,j]$ et $A[i,j]$ est la maximum de la colonne $j$)
\item A un certain point, on va manquer de points
\item Il doit donc y avoir un pic
\end{itemize}
\end{itemize}

\end{frame}

\begin{frame}{Analyse: complexité}

\begin{itemize}
\item $O(n)$ pour trouver le maximum d'une colonne
\item $O(\log n)$ itérations
\item $O(n\log n)$ au total
\bigskip

\item Peut-on faire mieux ? Oui, il est possible de proposer un algorithme en $O(n)$ (pas vu dans ce cours)
\end{itemize}

\end{frame}

\subsection{Exemple 3: sous-séquence de somme maximale}

\begin{frame}{Exemple 3: Achat/vente d'actions}

\centerline{\includegraphics[width=8cm]{Figures/06-stockprice.pdf}}

\bigskip

\begin{itemize}
\item Soit le prix d'une action au cours de $n$ jours consécutifs (prix à la fermeture)
\item On aimerait déterminer rétrospectivement:
\begin{itemize}
\item à quel moment, on aurait dû acheter et
\item à quel moment, on aurait dû vendre
\end{itemize}
de manière à maximiser notre gain
\end{itemize}

\end{frame}

\begin{frame}{Exemple 3: Achat/vente d'actions}

Première stratégie:
\begin{itemize}
\item Acheter au prix minimum, vendre au prix maximum
\item Pas correct: Le prix maximum ne suit pas nécessairement le prix minimum
\end{itemize}

\bigskip

Deuxième stratégie:
\begin{itemize}
\item Soit acheter au prix minimum et vendre au prix le plus élevé qui suit
\item Soit vendre au prix maximum et acheter au prix le plus bas qui précède
\item Pas correct:
\end{itemize}
\vspace{-0.5cm}
~\hfill\includegraphics[width=4cm, height=2cm]{Figures/06-stockprice-2.pdf}

\bigskip
\vspace{-0.5cm}
Troisième stratégie:
\begin{itemize}
\item Tester toutes les paires (force brute)
\item Correct ? Complexité ?
\end{itemize}
\end{frame}

\begin{frame}{Achat/vente d'actions: transformation}

\centerline{\includegraphics[width=10cm]{Figures/06-stockprice-3.pdf}}

\bigskip

\begin{itemize}
\item Transformation du problème:
\begin{itemize}
\item Calculer le tableau $A[i]=\mbox{(prix du jour i)-(prix du jour i-1)}$ (de taille $A.length=n$ en supposant qu'on démarre avec un prix au jour 0)
\item Déterminer la sous-séquence  non vide contiguë de somme maximale dans $A$
\item Soit $A[i\twodots j]$ cette sous-séquence. Il aurait fallu acheter juste avant le jour $i$ (juste après le jour $i-1$) et vendu juste après le jour $j$.
\end{itemize}
\item Exemple sur la tableau ci-dessus: $A[8\twodots 11]$ est la sous-séquence maximale de somme 43 $\Rightarrow$ acheter juste avant le jour 8 et vendre juste après le jour 11.
\item Si on peut trouver la sous-séquence maximale dans un tableau, on aura une solution à notre problème d'achat/vente d'actions
%\item Est-ce que la transformation est utile ?
\end{itemize}
\end{frame}

\begin{frame}{Approche par force brute}
\centerline{\includegraphics[width=8cm]{Figures/06-maxsubarray.pdf}}

\bigskip

\begin{itemize}
\item Implémentation naïve:
\begin{itemize}
\item On génère tous les sous-tableaux 
\item On calcule la somme des éléments de chaque sous-tableau
\item On renvoie les bornes du (d'un) sous-tableau de somme maximale
\end{itemize}
\item Complexité: $\Theta(n^2)$ sous-tableaux et $O(n)$ pour le calcul
  de la somme d'un sous-tableau $\Rightarrow$ $O(n^3)$
\item On peut l'implémenter en $\Theta(n^2)$
\end{itemize}

\end{frame}

\begin{frame}{Approche par force brute}

\begin{center}
{\small
\fcolorbox{white}{Lightgray}{%
      \begin{codebox}
        \Procname{$\proc{Max-subarray-brute-force}(A)$}
        \li $n=A.length$
        \li $\id{max-so-far}\gets -\infty$
        \li \For $i\gets 1$ \To n
        \li \Do $sum=0$
        \li \For $h\gets l$ \To n
        \li \Do $sum\gets sum+A[h]$
        \li \If $sum>\id{max-so-far}$
        \li \Then $\id{max-so-far}=sum$
        \li $low=l$
        \li $high=h$\End\End\End
        \li \Return $(low,high)$
      \end{codebox}}
}
\end{center}

Complexité: $\Theta(n^2)$\\

Peut-on faire mieux ?

\end{frame}

\begin{frame}{Approche diviser-pour-régner}

\begin{itemize}
\item Nouveau problème:
\begin{itemize}
\item trouver un sous-tableau maximal dans $A[low\twodots high]$
\item fonction $\proc{maximum-subarray}(A,low,high)$
\end{itemize}
\item Diviser:
\begin{itemize}
\item diviser le sous-tableau en deux sous-tableau de tailles aussi proches que possible
\item choisir $mid=\lfloor (low+high)/2 \rfloor$
\end{itemize}
\item Régner:
\begin{itemize}
\item trouver récursivement les sous-tableaux maximaux dans ces deux sous-tableaux
\item appeler $\proc{maximum-subarray}(A,low,mid)$ et $\proc{maximum-subarray}(A,mid+1,high)$
\end{itemize}
\item Fusionner: ?
\end{itemize}

\end{frame}

\begin{frame}{Approche diviser-pour-régner}

\centerline{\includegraphics[width=7cm]{Figures/06-maxsubarray-cross.pdf}}

\bigskip

\begin{itemize}
\item Fusionner:
\begin{itemize}
\item Rechercher un sous-tableau maximum qui traverse la jonction
\item Choisir la meilleure solution parmi les 3
\end{itemize}
\item $\proc{max-crossing-subarray}(A,low,mid,high)$
\begin{itemize}
\item Force brute: $\Theta(n^2)$ (car n/2 choix pour l'extrémité gauche, n/2 choix pour l'extrémité droite)
\item Meilleure solution: on recherche indépendamment les extrémités gauche et droite
\end{itemize}
\end{itemize}

\end{frame}

\begin{frame}{$\proc{max-crossing-subarray}$}

{\footnotesize
\fcolorbox{white}{Lightgray}{%
      \begin{codebox}
        \Procname{$\proc{Max-crossing-subarray}(A,low,mid,high)$}
        \li $\id{left-sum}=-\infty$
        \li $sum=0$
        \li \For $i\gets mid$ \Downto $low$
        \li \Do $sum=sum+A[i]$
        \li \If $sum>\id{left-sum}$
        \li \Then $\id{left-sum}=sum$
        \li $\id{max-left}=i$\End\End
        \li $\id{right-sum}=-\infty$
        \li $sum=0$
        \li \For $j\gets mid+1$ \To $high$
        \li \Do $sum=sum+A[j]$
        \li \If $sum>\id{right-sum}$
        \li \Then $\id{right-sum}=sum$
        \li $\id{max-right}=j$\End\End
        \li \Return $(\id{max-left},\id{max-right},\id{left-sum}+\id{right-sum})$
      \end{codebox}}
}\\
\vspace{-0.5cm}
Complexité: $\Theta(n)$~\hfill\includegraphics[width=6cm]{Figures/06-maxsubarray-cross-algo.pdf}
\end{frame}

\begin{frame}{$\proc{Max-subarray}$}

\begin{center}
{\footnotesize
\fcolorbox{white}{Lightgray}{%
      \begin{codebox}
        \Procname{$\proc{Max-subarray}(A)$}
        \li \If $high\isequal low$
        \li \Then \Return $(low,hig,A[low])$
        \li \Else $mid=\lfloor (low+high)/2 \rfloor$
        \li $(\id{left-low},\id{left-high},\id{left-sum})=\proc{Max-subarray}(A,low,mid)$
        \li $(\id{right-low},\id{right-high},\id{right-sum})=\proc{Max-subarray}(A,mid+1,high)$
        \li $(\id{cross-low},\id{cross-high},\id{cross-sum})=$
        \li \Indent $\proc{Max-crossing-subarray}(A,low,mid,high)$
        \li \If $\id{left-sum}\geq \id{right-sum}$ and $\id{left-sum}\geq \id{cross-sum}$
        \li \Then \Return $(\id{left-low},\id{left-high},\id{left-sum})$
        \li \ElseIf $\id{right-sum}\geq \id{right-sum}$ and $\id{right-sum}\geq \id{cross-sum}$
        \li \Then \Return $(\id{right-low},\id{right-high},\id{right-sum})$
        \li \Else \Return $(\id{cross-low},\id{cross-high},\id{cross-sum})$
      \end{codebox}}
}
\end{center}

\end{frame}

\begin{frame}{Analyse}

\begin{itemize}
\item Si on suppose que $n$ est un multiple de 2, le nombre d'opérations $T(n)$ est donné par:
\[
T(n) = \left\{
\begin{array}{ll}
 c_1 & \mbox{si }n=1\\
2 T(n/2)+c_2 n & \mbox{sinon}
\end{array}
\right.
\]
\item Même complexité que le tri par fusion $\Rightarrow$ $\Theta(n\log n)$
\item Peut-on faire mieux ? On verra plus loin que oui
\end{itemize}

\end{frame}

\begin{frame}{Divide pour régner: conclusions}

\begin{itemize}
\item Mène à des algorithmes très efficaces
\item Pas toujours applicable mais quand même très utile

\bigskip

\item Applications:
\begin{itemize}
\item Tris optimaux
\item Recherche binaire
\item Problème de sélection
\item Trouver la paire de points les plus proches
\item Recherche de l'enveloppe convexe (conver-hull)
\item Multiplication de matrice (méthode de Strassens)
\item \ldots
\end{itemize}
\end{itemize}

\end{frame}

\section{Programmation dynamique}

\begin{frame}{Méthodes de résolution de problèmes}

Quelques approches génériques pour aborder la résolution d'un problème:
\begin{itemize}
\item \alert{Approche par force brute:} résoudre directement le problème, à partir de sa définition ou par une recherche exhaustive
\item \alert{Diviser pour régner:} diviser le problème en sous-problèmes, les résoudre, fusionner les solutions pour obtenir une solution au problème original
\item \alert{Programmation dynamique:} obtenir la solution optimale à un problème en combinant des solutions optimales à des sous-problèmes similaires plus petits et se chevauchant
\semitransp{\item {Approche gloutonne:} construire la solution incrémentalement, en optimisant de manière aveugle un critère local}
\end{itemize}

\end{frame}

\begin{frame}{Plan}

\tableofcontents[currentsection,hideothersubsections]

\end{frame}

\subsection{Exemple 1: découpage de tiges d'acier}

\begin{frame}{Exemple 1: découpage de tiges d'acier}

\bigskip

\centerline{\includegraphics[width=4cm]{Figures/06-steelrod.pdf}}

\bigskip

\begin{itemize}
\item Soit une tige d'acier qu'on découpe pour la vendre morceau par morceau
\item La découpe ne peut se faire que par nombre entier de centimètres
\item Le prix de vente d'une tige dépend (non linéairement) de sa longueur
\item On veut déterminer le revenu maximum qu'on peut attendre de la vente d'une tige de $n$ centimètre
\item Problème algorithmique:
\begin{itemize}
\item Entrée: une longueur $n>0$ et une table de prix $p_i$, pour $i=1,2,\ldots,n$
\item Sortie: le revenu maximum qu'on peut obtenir pour des tiges de longueur $n$
\end{itemize}
\end{itemize}
\end{frame}

\begin{frame}{Illustration}
\begin{itemize}
\item Soit la table de prix:
\bigskip

\begin{center}\small
\begin{tabular}{l|llllllllll}
Longueur $i$ & 1 & 2 & 3 & 4 & 5 & 6 & 7 & 8 & 9 & 10\\
\hline
Prix $p_i$ & 1 & 5 & 8 & 9 & 10 & 17 & 17 & 20 & 24 & 30\\
\end{tabular}
\end{center}

\bigskip

\item Découpes possibles d'une tige de longueur $n=4$

\bigskip

\centerline{\includegraphics[width=10cm]{Figures/06-rodcutting.pdf}}

\bigskip

\item Meilleur revenu: découpage en 2 tiges de 2 centimètres, revenu de 10
\end{itemize}
\end{frame}

\begin{frame}{Approche par force brute}

\begin{itemize}
\item Enumérer toutes les découpes, calculer leur revenu, déterminer le revenu maximum
\item Complexité: exponentielle en $n$:
\begin{itemize}
\item Il y a $2^{n-1}$ manières de découper une tige de longueur $n$ (on peut couper ou non après chacun des $n-1$ premiers centimètres)
\item Plusieurs découpes sont équivalentes (1+1+2 et 1+2+1 par
  exemple) mais même en prenant cela en compte, le nombre de découpes
  reste exponentiel
\end{itemize}
\item Infaisable pour $n$ un peu grand
\end{itemize}

\note{dessiner un arbre avec toutes les possibilités. Dire que le nombre de feuille est $2^{n-1}$.}
\end{frame}

\begin{frame}{Idée}
\begin{itemize}
\item Soit $r_i$ le revenu maximum pour une tige de longueur $i$
\item Déterminons $r_i$ pour notre exemple:
\begin{center}
\footnotesize
\begin{tabular}{l|cl}
i & $r_i$ & solution optimale\\
\hline
1 & 1 & 1 (pas de découpe)\\
2 & 5 & 2 (pas de découpe)\\
3 & 8 & 3 (pas de découpe)\\
4 & 10 & 2+2\\
5 & 13 & 2+3\\
6 & 17 & 6 (pas de découpe)\\
7 & 18 & 1+6 ou 2+2+3\\
8 & 22 & 2+6
\ldots\\
\end{tabular}
\end{center}
\end{itemize}

\note{Dans le cas $n=4$, on doit comparer: 4, 1+3, 2+2, 3+1, 1+1+2, 1+2+1, 2+1+1, 1+1+1+1\\

Le max de 1+3, 1+2+1, 1+1+2, 1+1+1+1 est égal au prix de 1 plus $r_3$}
\end{frame}

\begin{frame}{Formulation récursive de $r_n$: version naïve}
\begin{itemize}
\item $r_n$ peut être calculé comme le maximum de:
\begin{itemize}
\item $p_n$: le prix sans découpe
\item $r_1+r_{n-1}$: le revenu max pour une tige de 1 et une tige de $n-1$
\item $r_2+r_{n-2}$: le revenu max  pour une tige de 2 et une tige de $n-2$
\item \ldots
\item $r_{n-1}+r_1$.
\end{itemize}
\item C'est-à-dire $$r_n=\max(p_n,r_1+r_{n-1},r_2+r_{n-2},\ldots,r_{n-1}+r_1)$$
\end{itemize}
\end{frame}

\begin{frame}{Formulation récursive de $r_n$: version simplifiée}

\begin{itemize}
\item Toute solution optimale a un découpe la plus à gauche
\item On peut calculer $r_n$ en considérant toutes les tailles pour la première découpe et en combinant avec le découpage optimal pour la partie à droite
\item Pour chaque cas, on n'a donc qu'à résoudre un seul sous-problème (au lieu de deux), celui du découpage de la partie droite
\item En supposant $r_0=0$, on obtient ainsi:
$$r_n=\max_{1\leq i \leq n} (p_i+r_{n-1})$$
\end{itemize}

\note{Faire un dessin au tableau avec la partie gauche de toutes les tailles}

\end{frame}

\begin{frame}{Implémentation récursive directe}

\begin{itemize}
\item La formule récursive peut être implémentée directement

\bigskip

\begin{center}
{\footnotesize
\fcolorbox{white}{Lightgray}{%
      \begin{codebox}
        \Procname{$\proc{Cut-rod}(p,n)$}
        \li \If $n\isequal 0$
        \li \Then \Return 0
        \li $q=-\infty$
        \li \For $i\gets 1$\To $n$
        \li \Do $q\gets \max(q,p[i]+\proc{Cut-Rod}(p,n-i))$\End
        \li \Return $q$
      \end{codebox}}
}
\end{center}
(p est un tableau de taille $n$ contenant les prix des tiges de tailles 1 à $n$)
\bigskip
\item Complexité ?
\end{itemize}

\end{frame}

\begin{frame}{Implémentation récursive directe: analyse}

\begin{itemize}
\item L'algorithme est extrêmement inefficace à cause des appels récursifs redondants
\item Exemple: arbre des appels récursifs pour le calcul de $r_4$

\centerline{\includegraphics[width=5cm]{Figures/06-rodcutting-recursiontree.pdf}}

%% \item Le sous-problème de taille 3 est résolu 1 fois, le sous-problème de taille 2 est résolu 2 fois, le sous-problème de taille 1 est résolu 4 fois, le sous-problème de taille 0 est résolu 8 fois
\item En général, le nombre de n\oe uds $T(n)$ de l'arbre est $2^n$.\\
{\small Preuve par induction:
\begin{itemize}
\item Cas de base: $T(1)=1$
\item Cas inductif: $$T(n)=1+\sum_{j=0}^{n-1} T(j)=1+\sum_{j=0}^{n-1} 2^j=1+2^n-1=2^n$$
\end{itemize}}
\item Complexité de l'algorithme est exponentielle en $n$
\end{itemize}
\note{Compter sur le graphe: nombre de 3, nombre de 1, nombre de 2}
\end{frame}

\begin{frame}{Solution par programmation dynamique}

\begin{itemize}
\item Solution: plutôt que de résoudre les mêmes sous-problèmes plusieurs fois,
  s'arranger pour ne les résoudre chacun qu'une seule fois
\item Comment ? En sauvegardant les solutions dans une table et en se
  référant à la table à chaque demande de résolution d'un
  sous-problème déjà rencontré
\item On échange des temps de calcul contre de la mémoire
\item Permet de transformer une solution en temps exponentiel en une solution en temps polynomial
\item Deux implémentations possibles:
\begin{itemize}
\item descendante (top-down) avec \alert{mémoization}
\item ascendante (bottom-up)
\end{itemize}

\end{itemize}

\end{frame}

\begin{frame}{Approche descendante avec mémoization}

\begin{center}
{\footnotesize
\fcolorbox{white}{Lightgray}{%
      \begin{codebox}
        \Procname{$\proc{Memoized-Cut-rod}(p,n)$}
        \li Let $r[0\twodots n]$ be a new array
        \li \For $i\gets 1$ \To $n$
        \li \Do $r[i]\gets -\infty$\End
        \li \Return $\proc{memoized-cut-rod-aux}(p,n,r)$
      \end{codebox}}

\fcolorbox{white}{Lightgray}{%
      \begin{codebox}
        \Procname{$\proc{Memoized-Cut-rod-aux}(p,n,r)$}
        \li \If $r[n]\geq 0$
        \li \Then \Return $r[n]$\End
        \li \If $n\isequal 0$
        \li \Then $q=0$
        \li \Else $q=-\infty$
        \li \For $i\gets 1$ \To $n$
        \li \Do $q=\max(q,p[i]+\proc{memoized-cut-rod-aux}(p,n-i,r))$\End\End
        \li $r[n]=q$
        \li \Return $q$
      \end{codebox}}
}
\end{center}

(Attention: suppose que le tableau est passé par pointeur)

\note{Dire qu'ils doivent bien réfléchir pour être sûr de comprendre\\

\bigskip

{\color{red} Montrer sur l'arbre de récursion comme ça fonctionne !!!}
}
\end{frame}

\begin{frame}{Approche ascendante}

Principe: résoudre les sous-problèmes par taille en commençant d'abord par les plus petits

\bigskip

\begin{center}
{\footnotesize
\fcolorbox{white}{Lightgray}{%
      \begin{codebox}
        \Procname{$\proc{Bottom-up-Cut-rod}(p,n)$}
        \li Let $r[0\twodots n]$ be a new array
        \li $r[0]=0$
        \li \For $j\gets 1$ \To $n$
        \li \Do $q=-\infty$
        \li \For  $i\gets 1$ \To $j$
        \li \Do $q=\max(q,p[i]+r[j-i])$\End
        \li $r[j]=q$\End
        \li \Return $r[n]$
      \end{codebox}}
}
\end{center}

\end{frame}

\begin{frame}{Programmation dynamique: analyse}

\begin{itemize}
\item Solution ascendante est clairement $\Theta(n^2)$ (deux boucles imbriquées)
\item Solution descendante est également $\Theta(n^2)$
\begin{itemize}
\item Chaque sous-problème est résolu une et une seule fois
\item La résolution d'un sous-problème passe par une boucle à $n$ itérations
\end{itemize}
\item Graphes des sous-problèmes:

\centerline{\includegraphics[width=1.5cm]{Figures/06-subproblems-graph.pdf}}

(une flèche de $x$ à $y$ indique que la résolution de $x$ dépend de la résolution de $y$)
\end{itemize}

\note{On voit bien que c'est $n^2$ sur ce graphe}

\end{frame}

\begin{frame}{Reconstruction de la solution}
\begin{itemize}
\item Fonction $\proc{Bottom-up-Cut-rod}$ calcule le revenu maximum mais ne donne pas directement la découpe correspondant à ce revenu
\item On peut étendre l'approche ascendante pour enregistrer également la solution dans une autre table
\end{itemize}

\begin{center}
{\footnotesize
\fcolorbox{white}{Lightgray}{%
      \begin{codebox}
        \Procname{$\proc{Extended-Bottom-up-Cut-rod}(p,n)$}
        \li Let $r[0\twodots n]$ and $s[0\twodots n]$ be new arrays
        \li $r[0]=0$
        \li \For $j\gets 1$ \To $n$
        \li \Do $q=-\infty$
        \li \For  $i\gets 1$ \To $j$
        \li \Do \If $q<p[i]+r[j-i]$
        \li \Then $q=p[i]+r[j-i]$
        \li $s[j]=i$\End\End
        \li $r[j]=q$\End
        \li \Return $r$ and $s$
      \end{codebox}}
}
\end{center}

\begin{itemize}
\item $s[j]$ contient la coupure la plus à gauche d'une solution optimale au problème de taille $i$
\end{itemize}
\end{frame}

\begin{frame}{Reconstruction de la solution}
\begin{itemize}
\item Pour afficher la solution, on doit ``remonter'' dans $s$
\bigskip

\begin{center}
{\footnotesize
\fcolorbox{white}{Lightgray}{%
      \begin{codebox}
        \Procname{$\proc{Print-cut-rod-solution}(p,n)$}
        \li $(r,s)=\proc{Extended-bottom-up-cut-rod}(p,n)$
        \li \While $n>0$
        \li \Do $\proc{print}$ $s[n]$
        \li $n=n-s[n]$\End
      \end{codebox}}
}
\end{center}

\bigskip

\item Exemple:
\bigskip
\begin{center}\small
\begin{tabular}{c|llllllllll}
$i$ & 0 & 1 & 2 & 3 & 4 & 5 & 6 & 7 & 8 \\
\hline
$p[i]$ & 0 & 1 & 5 & 8 & 9 & 10 & 17 & 17 & 20 \\
$r[i]$ & 0 &  1 & 5 & 8 & 10 & 13 & 17 & 18 & 22\\
$s[i]$ & 0 & 1 & 2 & 3 & 2 & 2 & 6 & 1 & 2\\
\end{tabular}
\end{center}
\bigskip
$$\proc{print-cut-rod-solution}(p,8) \Rightarrow \mbox{"2 6"}$$

\end{itemize}
\end{frame}

\begin{frame}{Programmation dynamique: généralités}
\begin{itemize}
\item La programmation dynamique s'applique aux problèmes d'\alert{optimisation} qui peuvent se décomposer en sous-problèmes de même nature, et qui possèdent les deux propriétés suivantes:
\begin{itemize}
\item \alert{Sous-structure optimale:} on peut calculer la solution d'un problème de taille $n$ à partir de la solution de sous-problèmes de taille inférieure
\item \alert{Chevauchement des sous-problèmes:} Certains sous-problèmes distincts partagent une partie de leurs sous-problèmes
\end{itemize}
\item Implémentation directe récursive donne une solution de complexité exponentielle
\item Sauvegarde des solutions aux sous-problèmes donne une complexité linéaire dans le nombre d'arcs et de sommets du graphe des sous-problèmes
\end{itemize}
\end{frame}

\subsection{Exemple 2: Fibonacci}

\begin{frame}{Exemple 2: Fibonacci}

\begin{itemize}
\item La fonction $\proc{Fibonacci-Iter}$ vue au début du cours est un
  exemple de programmation dynamique (ascendante)
\end{itemize}

\begin{columns}
\begin{column}{5cm}
\begin{center}\footnotesize
\fcolorbox{white}{Lightgray}{
\begin{codebox}
\Procname{$\proc{Fibonacci-Iter}(n)$}
\li \If $n \leq 1$
\li \Then \Return n \End
\li \Else
\li \Then $pprev\gets 0$
\li $prev\gets 1$
\li \For $i\gets 2 \To n$
\li \Do $f\gets prev+pprev$
\li $pprev\gets prev$
\li $prev\gets f$\End
\li \Return f \End
\end{codebox}
}
\end{center}
\end{column}
\begin{column}{5cm}
\centerline{\includegraphics[width=0.8cm]{Figures/06-subproblem-graph-fibonacci.pdf}}
\end{column}
\end{columns}

\begin{itemize}
\item On peut se contenter de ne stocker que les deux dernières valeurs
\item Complexité $\Theta(n)$ (graphe contient $n+1$ n\oe uds et $2n-2$ arcs)
\end{itemize}
{\it (Exercice: écrivez la version descendante avec mémoization)}

\end{frame}

\begin{frame}{Interlude: Fibonacci en $\Theta(\log n)$}
\begin{itemize}
\item Peut-on faire mieux que $\Theta(n)$ pour Fibonacci ? Oui !
\item Propriété: {\small $$\begin{pmatrix} F_{n+1} &F_n\\F_n&F_{n-1}\\\end{pmatrix}=\begin{pmatrix} 1&1\\1&0\\\end{pmatrix}^n$$}
\item Preuve par induction:
\begin{itemize}
\item Cas de base ($n=1$): ok puisque $F_0=0$, $F_1=1$, et $F_2=1$
\item Cas inductif ($n\geq 2$):
{\small
\begin{eqnarray*}
\begin{pmatrix} F_{n+1} &F_n\\F_n&F_{n-1}\\\end{pmatrix} & = & \begin{pmatrix} F_{n} &F_{n-1}\\F_{n-1}&F_{n-2}\\\end{pmatrix} \cdot \begin{pmatrix} 1&1\\1&0\\\end{pmatrix}\\
& =&  \begin{pmatrix} 1&1\\1&0\\\end{pmatrix}^{n-1} \cdot \begin{pmatrix} 1&1\\1&0\\\end{pmatrix}\\
& = & \begin{pmatrix} 1&1\\1&0\\\end{pmatrix}^{n}
\end{eqnarray*}}\qed
\end{itemize}
\end{itemize}

\end{frame}

\begin{frame}{Interlude: Fibonacci en $\Theta(\log n)$}
\begin{itemize}
\item Approche par force brute pour le calcul de $\begin{pmatrix} 1&1\\1&0\\\end{pmatrix}^n$: $\Theta(n)$
\item Idée: utiliser le diviser-pour-régner pour le calcul de $a^n$
\[
a^n = \left\{
\begin{array}{ll}
a^{n/2} \cdot a^{n/2}  & \mbox{si }n\mbox{ est pair}\\
a^{(n-1)/2}\cdot a^{(n-1)/2} \cdot a & \mbox{si }n\mbox{ est impair}
\end{array}
\right.
\]
\item Complexité: $\Theta(\log n)$ (comme la recherche binaire)
\end{itemize}

\bigskip

{\it (Exercice: implémenter l'algorithme)}
\end{frame}

\subsection{Exemple 3: sous-séquence de somme maximale}

\begin{frame}{Exemple 3: sous-séquence maximale}
%(ou la revenge de la programmation dynamique)

\bigskip

%Algorithme de Kadane pour la recherche d'un sous-séquence de somme maximale dans un tableau

\begin{columns}
\begin{column}{7cm}
\begin{center}
{\footnotesize
\fcolorbox{white}{Lightgray}{%
      \begin{codebox}
        \Procname{$\proc{Max-subarray-linear}(A)$}
        \li Let $m[0\twodots n]$ be a new array
        \li $\id{max-so-far}\gets -\infty$
        \li $m[0]=-\infty$
        \li \For $i\gets 1$ \To $A.length$
        \li \Do \If $m[i-1]>0$
        \li \Then $m[i]=m[i-1]+A[i]$
        \li \Else $m[i]=A[i]$\End\End
        \li \Return $m[A.length]$
      \end{codebox}}
}
\end{center}
\end{column}
\begin{column}{3cm}
\centerline{\includegraphics[width=0.7cm]{Figures/06-subproblems-graph-max.pdf}}
\end{column}
\end{columns}

\bigskip

\begin{itemize}
\item Complexité: $\Theta(n)$ (diviser pour régner: $\Theta(n\log n)$)
\item $m[i]$ est la somme de la sous-séquence maximale qui se termine en $i$
\item L'algorithme calcule $m[i]$ à partir de $m[i-1]$
\item Forme de programmation dynamique ascendante (très simple)
\end{itemize}

{\it\small (Exercice: ajouter le calcul des bornes d'un sous-tableau solution, remplacer le tableau $m$ par une seule variable)}

\end{frame}

%%%%% fin premier cours sur méthodes de résolution

\begin{frame}{Etapes clés de la programmation dynamique}

\begin{itemize}
\item Charactérise l'optimalité
\item Recursively define an optimal solution - analyze the problem in a top-down fashion to determine how subproblems relate to the original
\item Solve the subproblems - start with a base case and solve the subproblems in a bottom-up manner to find the optimal value
\item Reconstruct the optimal solution - (optionally) determine the solution that produces the optimal value
\end{itemize}

\end{frame}

\subsection{Exemple 4: la plus longue sous-séquence commune}

\begin{frame}{Plus longue sous-séquence commune}

(Mettre le problème et leur demander d'y réfléchir)

\end{frame}

\subsection{Exemple 5: le problème 0-1 du sac-à-dos}

\begin{frame}{Sac à dos}

\end{frame}

\begin{frame}{Discussion}

Mettre une discussion comme dans les slides de CLRS sur la
programmation dynamique en général

voir slide Kevin Wayne, Tardos, Kleinberg pour exemple d'algorithmes célèbres

\end{frame}

\section{Algorithmes gloutons}

\begin{frame}{Méthodes de résolution de problèmes}

\end{frame}

\begin{frame}{Algorithme glouton}

Change de pieces de Benard Boigelot\\

Activity scheduling\\

Code Huffman\\

fractionnal Knapsack

\end{frame}

\begin{frame}{Synthèse}

Mettre les algos qu'on a vu ici. Proposer d'autres exercices ?

\end{frame}
